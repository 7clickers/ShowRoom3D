\section{Resoconto}
\subsection{Introduzione}
Continuazione del meeting lasciato in sospeso (vedi VI\_24-02-23).
Da notare che la riunione dovrebbe essere avvenuta il 27-02-23 ma per impegni imprevisti abbiamo deciso di spostarla al 28-02-23.
In questa riunione abbiamo trattato i punti lasciati in sospeso dal meeting precedente ed abbiamo discusso insieme sul materiale che
gli analisti hanno corretto dopo il meeting con il professor Riccardo Cardin.
I punti affrontati pertanto sono stati i seguenti:
\begin{itemize}
    \item Discussione con il responsabile di progetto sul \textit{Piano di Progetto}: sono state determinate le attività da svolgere in modo chiaro;
    \item Determinazione delle possibili issues\textsubscript{g}:sono state create le issue\textsubscript{g} da risolvere per il completamento delle attività;
    \item Ripristinare l'ordine nella project board\textsubscript{g}: sono state chiuse le issue\textsubscript{g}\textsuperscript{g}s completate ed organizzate quelle da completare con checkbox e label esplicative.
    \item Valutare meglio i tempi per ripresentarsi alla revisione di avanzamento: abbiamo deciso di confermare le date fissate in precedenza.
    \end{itemize}
Inoltre sono stati trattati i seguenti punti non previsti nella riunione precedente:
\begin{itemize}
\item Assegnazione di una scadenza alle label che indicano la priorità delle issue\textsubscript{g};
\item Creazione di un file per l'identificazione dei termini plurali che sarà utilizzato dallo script senza influire sul \textit{Glossario};
\end{itemize}
\subsection{Assegnazione di una scadenza alle labels}
Abbiamo deciso di assegnare alle label di priorità bassa una scadenza di 15 giorni, alle label di priorità media una scadenza di 7-10 giorni e alle 
label di priorità alta una scadenza di 3-4 giorni.
\subsection{Creazione di un file per l'identificazione dei termini plurali}
Il file che verrà creato sarà utilizzato dallo script che inserisce i pedici all'interno dei documenti per riconoscere anche i termini scritti al 
plurale. Ci siamo accorti che capita spesso che un termine non venga indicato come appartenente al \textit{Glossario} quando viene scritto nella 
sua forma plurale.
