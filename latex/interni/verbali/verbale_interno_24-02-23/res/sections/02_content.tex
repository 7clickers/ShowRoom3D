\section{Resoconto}
\subsection{Introduzione}

Dal meeting riguardante la prima revisione di avanzamento con il Prof. Riccardo Cardin (vedi VE\_17-02-23) ci sono state segnalate 
alcune mancanze presenti nel documento \textit{Analisi dei Requisiti}.
In questo \textit{Verbale} vengono esaminate le mancanze del documento che ci sono state segnalate via mail dal professore e vengono rivisti dei punti del documento \textit{Norme di Progetto}
che fino ad oggi hanno creato difficoltà nel modo di lavorare in particolare ci siamo resi conto della difficoltà a chiudere in breve tempo le Pull Request
accumulando così molto materiale da revisionare prima di integrarlo con il branch principale.
Infine abbiamo fatto la rotazione dei ruoli.
 Dato che la riunione è durata più del previsto abbiamo deciso di spezzarla in due per concludere i punti che non siamo riusciti a discutere nel tempo limite che ci eravamo prefissati.
La riunione che si terrà lunedì 27 febbraio 2023 sarà la continuazione degli argomenti lasciati in sospeso in questo \textit{Verbale}.
I punti che verranno trattati in seguito sono:
\begin{itemize}
    \item Analisi dei punti sui casi d'uso;
    \item Discussione sulle \textit{Norme di Progetto};
    \item Rotazione dei ruoli;
    \item Dichiarazione dei punti lasciati in sospeso.
\end{itemize}

\subsection{Analisi dei punti sui casi d'uso}
Sono stati analizzati i seguenti punti riguardanti i casi d'uso:
\begin{itemize}
    \item Suddivisione in dettaglio dei casi d'uso esistenti in modo da analizzare più in profondità il progetto e ricavarne quindi nuovi requisiti;
    \item Sistemare il diagramma dei casi d'uso con i nuovi requisiti emersi;
    \item Sistemare il file testuale dei casi d'uso con i nuovi requisiti emersi;
    \item Aggiunta dei termini di \textit{Glossario} al documento \textit{Analisi dei Requisiti} e aggiunta la descrizione del modo in cui essi vengono indicati;
    \item Ordinare i casi di uso in modo che seguendo l'ordine di lettura del documento venga mano a mano capito il contesto di applicazione dichiarando per prime le funzionalità più significative;
    \item Inserire tutte le immagini relative ai casi d'uso ovvero anche quelle che riguardano relazioni tra l'attore\textsubscript{g} e il singolo caso d'uso;
    \item Eliminazione del colore dal diagramma dei casi d'uso per indicare le funzionalità facoltative;
    \item Inserire tra i requisiti qualitativi la stesura del manuale;
    \item Indicare le versioni dei browser che devono supportare l'app;
    \item Inserire una tabella che indichi il tracciamento dei casi d'uso in modo da riconoscere se tutte le funzionalità abbiano individuato almeno un requisito.
\end{itemize}
\subsection{Discussione sulle \textit{Norme di Progetto}}
 I punti che sono stati trattati sulle \textit{Norme di Progetto} riguardano la stesura del registro delle modifiche e come trattare la chiusura delle Pull Requests.
\\\\
\textbf{Registro delle modifiche:}
\\
Abbiamo notato che la colonna descrizione del registro delle modifiche risultava troppo dettagliata e non portava a nessun vantaggio significativo nel tracciare i cambiamenti all'interno del documento.
Abbiamo quindi deciso di indicare nella colonna descrizione solamente le sezioni che hanno subito modifiche o le nuove sezioni aggiunte.
Sarà poi compito di chi è interessato alla lettura del documento andare a vedere quali sono state in dettaglio le parti modificate indicate nel registro delle modifiche.
\\\\
\textbf{Chiusura delle Pull Requests:}
Questo è stato il punto principale trattato per quanto riguarda le \textit{Norme di Progetto}.
Ci siamo accorti che è indispensabile fissare un limite dopo il quale una Pull Request deve essere chiusa.
In questo modo la mole di lavoro da verificare prima di un'integrazione con il branch principale diventa più gestibile e i verificatori vengono coinvolti maggiormente
diminuendo così il tempo tra una verifica\textsubscript{g} e l'altra.
I dettagli della discussione verranno aggiunti nelle \textit{Norme di Progetto}. In ogni caso questo metodo di lavoro sarà sperimentato e non sarà una versione definitiva.
\\
\subsection{Rotazione dei ruoli}
La rotazione dei ruoli ha prodotto\textsubscript{g} come risultato la seguente suddivisione dei ruoli:
\begin{itemize}
    \item Giacomo Mason: Responsabile di progetto;
	\item Gabriele Mantoan: Analista;
	\item Mirko Stella: Verificatore;
	\item Marco Brigo: Analista;
	\item Tommaso Allegretti: Verificatore;
	\item Elena Pandolfo: Progettista;
	\item Rino Sincic: Progettista.
\end{itemize}
\subsection{Dichiarazione dei punti lasciati in sospeso}
 I punti che verranno discussi nella riunione straordinaria di lunedì 27 febbraio 2023 sono:
 \begin{itemize}
    \item Discussione con il responsabile di progetto sul \textit{Piano di Progetto};
    \item Determinazione delle possibili issue\textsubscript{g}s;
    \item Ripristinare l'ordine nella project board\textsubscript{g};
    \item Valutare meglio i tempi per ripresentarsi alla revisione di avanzamento.
 \end{itemize}
