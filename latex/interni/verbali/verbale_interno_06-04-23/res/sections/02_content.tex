\section{Resoconto}
\subsection{Introduzione}
In questa riunione abbiamo deciso come modificare il nostro modello di sviluppo passando ad un modello Agile, come suggeritoci dal Prof.Tullio Vardanega. Sono state poi discusse modifiche da apportare alla stesura degli Sprint nel Piano di Progetto.
\subsection{Argomenti di discussione}
\begin{itemize}
    \item Modello di sviluppo Agile;
    \item Modifiche nel \textit{Piano di Progetto};
    \item Contattare il proponente e il Prof.Cardin.
\end{itemize}
\subsection{Modello di sviluppo Agile}
Dai consigli dati dal Prof.Tullio Vardanega, abbiamo deciso di suddividere i periodi successivi tramite gli Sprint\textsubscript{g} comunemente utilizzati nella metodologia di sviluppo Agile.
Gli Sprint durano una settimana e per ognuno di essi bisogna svolgere le tasks, che sono state identificate in riunione dal gruppo e che compongono il backlog di questo ultimo periodo.
Per mantenere l'ordine nelle tabelle, le task sono state associate a delle etichette denominate attività. 
Sono state identificate 5 Milestone rappresentanti 5 Sprint.
Ad ogni issue (che rappresenta una task) verrà quindi associata la Milestone che rappresenta lo Sprint in cui è contenuta. 

\subsection{Modifiche nel \textit{Piano di Progetto}}
Dato il cambio di modello di sviluppo, abbiamo abbozzato un template per ogni Sprint, creando la tabella delle task svolte in quello Sprint e inserendo il suo diagramma di Gantt.

\subsection{Contattare il proponente e il Prof.Cardin}
Durante la fase di progettazione è nata la necessità di redigere un diagramma delle classi e uno di sequenza. Questi diagrammi verranno poi mostrati prima al Prof.Cardin, concordando un meeting nei prossimi giorni, e successivamente al proponente.

\subsection{Issues rilevate dalla riunione}
\begin{itemize}
    \item Stesura del verbale 06-04-2023;
    \item Aggiornare \textit{Piano di Progetto};
    \begin{itemize}
        \item Creare una sezione per il nuovo sprint;
        \item Inserire il consuntivo dello sprint;
        \item Inserire il preventivo dello sprint.
    \end{itemize}
    \item Individuazione pattern architetturali;
    \item Apprendimento librerie;
    \item Aggiornamento metriche e misure \textit{Piano di Qualifica};
    \item Diario di Bordo per la settimana di lavoro;
\end{itemize}