\section{Motivo della Riunione}
Discutere della data di iscrizione a ricevimento per la requirment and technology baseline e dello stato di avanzamento dei documenti necessari ad conseguimento di essa.
\section{Resoconto}
\begin{itemize}
	\item \textit{Analisi dei Requisiti}: il testo è pronto e deve essere caricato nella repository\textsubscript{g}, mentre i diagrammi UML\textsubscript{g} devo ancora essere corretti in base alle considerazioni fatte assieme al docente Cardin (\textit{Verbale} esterno 25-01-23);
	
	\item PoC\textsubscript{g}: è a buon punto, ma non funziona ancora sul browser Safari. Inoltre va fatto uno studio che riguardi eventuali librerie che gestiscono la fisica dell'applicazione ed un eventuale package builder che potrebbe rendere più semplice la gestione dei file e forse anche migliorare le prestazioni globali tramite alcune compressioni. Nel prossimo periodo i programmatori si concentreranno nell'ispezionare le possibili scelte su queste componenti, per capire se inserirle nel proof of concept\textsubscript{g};
	
	\item \textit{Norme di Progetto}: sono a buon punto, ma ci sono ancora norme minori da aggiornare o revisionare;
	
	\item \textit{Piano di Progetto}: la pianificazione dell'ultimo periodo era rimasta in sospeso per la mancanza di una data di chiusura del periodo, che coincide col raggiungimento della rtb. É stato quindi deciso che la data di chiusura del periodo sarà il 6 febbraio, ovvero la data in cui chiederemo di avere il ricevimento coi docenti;
	
	\item \textit{Piano di Qualifica}: è pronto, i membri del gruppo continueranno a cecare misure di valutazione per arrichirlo eventualmente;
	
	\item \textit{Glossario}: vanno aggiunti ancora dei vocaboli;

	\item \textit{Lettera di Presentazione}: deve ancora essere redatta.
\end{itemize}

\subsection{Data di consegna}
Il lavoro rimanente sembra essere completabile nel giro di poco più di una settimana, quindi la data di consegna è stata fissata per il 6 febbraio. 
\newline Nei prossimi giorni verrà inviata la richiesta al professore, dopo che i programmatori avranno completato l'ispezione delle tecnologie prima descritte, in modo da essere sicuri che il tempo sia sufficiente.
