\section{Resoconto}
\subsection{Introduzione}
Durante la riunione sono stati rilevati i miglioramenti da apportare ai processi attualmente 
utilizzati seguendo ogni punto della valutazione ricevuta dal professor Tullio Vardanega dopo la revisione
di avanzamento RTB.
Inoltre i progettisti in carica hanno preso l'onere di pensare ad una possibile architettura del sistema 
circoscritta a pochi componenti in modo da procedere per piccoli passi.
\subsection{Argomenti di discussione}
\begin{itemize}
    \item Registro delle modifiche e numeri di versione;
    \item Riferimenti a risorse esterne all'interno dei documenti;
    \item \textit{Piano di Progetto};
    \item \textit{Piano di Qualifica};
    \item Script glossario;
    \item Progettazione;
    \item Rotazione dei ruoli.
\end{itemize}
\subsection{Registro delle modifiche e numeri di versione}
Per rendere ogni riga del registro delle modifiche corrispondente ad un cambiamento andato a buon fine 
(verificato) abbiamo ritenuto opportuno passare dalla rappresentazione dei numeri di versione dei documenti 
da tre a due cifre.
La prima cifra viene incrementata ad ogni verifica andata a buon fine mentre la seconda cifra all'approvazione
da parte del responsabile di progetto.
Inoltre abbiamo deciso che solo i verificatore aggiungeranno righe al registro delle modifiche al termine della verifica.
La riga verrà creata indicando chi ha partecipato alle modifiche e alle verifiche e i ruoli di ciascun componente.
Stiamo ancora pensando se inserire un riferimento al verbale che tratta le modifiche fatte motivandole.
\subsection{Riferimenti a risorse esterne all'interno dei documenti}
Dalla RTB ci è stato segnalato che tutti i riferimenti a documenti esterni o risorse web vanno accompagnate con 
il loro numero di versione al momento in cui si fa riferimento o alla data.
\subsection{\textit{Piano di Progetto}}
Abbiamo abbandonato il metodo incrementale considerandolo non adatto al nostro progetto per passare ad un metodo 
agile che definiremo più nel dettaglio.
Inoltre alla fine di ogni periodo pianificato verrà fatta una discussione sui risultati ottenuti e sulle azioni 
correttive da mettere in pratica per migliorare le pianificazioni successive.
\subsection{\textit{Piano di Qualifica}}
Spostare la sezione dei test come ultima e mettere la tabella che contiene i test come ultimo punto di tale sezione.
Inoltre ogni volta che si verifica un artefatto il \textit{Piano di Qualifica} deve essere aggiornato calcolando il valore del 
test fatto e si devono discutere le azioni correttive per migliorare la qualità del prodotto.
\subsection{Script glossario}
Modificare lo script in modo da correggerlo (non inserire g nei subsection e section,ma solo all’interno dei paragrafi,
non mettere g all’interno delle label e nei path)
\subsection{Progettazione}
I progettisti in carica hanno il compito di studiare Redux e creare un primo schema UML che rappresenti una possibile 
architettura iniziale.
Lo schema dovrà indicare la relazione tra alcuni componenti del sistema scelti per fare una prima progettazione esplorativa
in modo da discuterne successivamente con il gruppo.
\subsection{Rotazione dei ruoli}
I ruoli decisi sono i seguenti:
\begin{itemize}
    \item Marco Brigo: Progettista;
    \item Rino Sincic: Responsabile di progetto;
    \item Elena Pandolfo: Amministratore;
    \item Mirko Stella: Progettista;
    \item Giacomo Mason: Verificatore;
    \item Gabriele Mantoan: Verificatore;
    \item Tommaso Allegretti: Progettista.
\end{itemize}
\subsection{Issues rilevate dalla riunione}
\begin{itemize}
    \item Stesura del verbale 29-03-2023;
    \item Aggiornare norme di progetto;
    \begin{itemize}
        \item Registro delle modifiche;
        \item Norme su nomi dei documenti e risorse esterne;
        \item Aggiungere l'introduzione all'inizio dei documenti come nel \textit{Piano di Qualifica}. 
    \end{itemize}
    \item Sistemare lo script glossario;
    \item Aggiornare \textit{Piano di Progetto}.
    \begin{itemize}
        \item Creare una sezione che descriva il metodo agile che utilizzeremo.
    \end{itemize}
\end{itemize}