\section{Resoconto}
\subsection{Introduzione}
Durante questo incontro abbiamo fatto il punto dello sviluppo dell'analisi dei requisiti\textsubscript{g}, del piano di progetto e delle norme di progetto e abbiamo pianificato i prossimi obbiettivi per arrivare alla requirement and technology baseline. Inoltre abbiamo terminato il precedente sprint e ne abbiamo iniziato uno nuovo, cambiando quindi i ruoli di ogni componente

\subsection{Considerazioni sull'Issue tracking system\textsubscript{g}}
Abbiamo deciso di abbandonare l'utilizzo di Jira\textsubscript{g} a supporto di Github\textsubscript{g} in quanto l'automazione del sistema non sembra essere facilmente concretizzabile e rischia di creare una discrepanza del tracciamento delle issue\textsubscript{g} nei due sistemi

\subsection{Inizio esplorazione dei framework}
Interrogandoci sulla possibile aggiunta di nuove tecnologie da inserire nella requirement e technology baseline, abbiamo considerato i consigli del docente Cardin e abbiamo deciso di iniziare a esplorare l'utilizzo di framework per lo sviluppo in Javascript. Al momento Tommaso e Rino si occuperanno di dare un occhiata a Angular, React o Svelte.


\subsection{Documenti da aggiornare}
Abbiamo deciso di proseguire nello sviluppo di alcuni file nel seguente modo:
\begin{itemize}
	\item \textit{analisi dei requisiti\textsubscript{g}}: il documento è a buon punto, ma va ancora rifinito e verificato. Come metodologia di verifica, oltre ad un controllo dei verificatori, abbiamo deciso di chiedere l'opinione del proponente; il file gli verrà inviato entro venerdì 25 novembre alle 12:00.
	\item \textit{piano di progetto}: il documento va aggiornato inserendo un diagramma di Gaant, pianificazione dello sprint appena concluso, resoconto dello sprint appena concluso, e pianificazione del nuovo sprint
	\item \textit{norme di progetto}: va eliminata la sezione dell'utilizzo di Jira e va aggiunta tra le norme di verifica, una sezione che dica che i verbali vanno approvati subito dopo la loro verifica
\end{itemize}

\subsection{Nuovo sprint}
Abbiamo concluso lo sprint precedente ed iniziato un nuovo sprint, quindi abbiamo anche cambiato i ruoli per le prossime 2 settimane:
\begin{itemize}
\item \textit{Responsabile}: Elena
\item \textit{Amministratore}: Rino
\item \textit{Analista}: Giacomo, Gabriele, Marco
\item \textit{Verificatore}: Mirko, Tommaso
\end{itemize}

\subsection{Obbiettivi per il prossimo periodo}
I prossimi obbiettivi da raggiungere sono:
\begin{itemize}
\item Aggiornare il piano di progetto
\item Aggiornare il glossario
\item Concludere l'analisi dei requisiti (chiedendo prima l'opinione del proponente)
\item Capire i dettagli della consegna per la requirment and technology baseline (scadenze e materiale da presentare)
\end{itemize}

