\section{Resoconto}
\subsection{Introduzione}

In questo verbale vengono esaminate le mancanze nella documentazione che ci sono state segnalate dal professore Vardanega e vengono rivisti dei punti dei documenti Piano di Qualifica, Lettera di Presentazione e Norme di Progetto.
Abbiamo deciso di utilizzare la libreria front-end React per il nostro progetto poiché sembrava una buona soluzione e il suo grande ecosistema fornisce ottime librerie per l'implementazione di three.js
Infine abbiamo fatto la rotazione dei ruoli.

\subsection{Correzioni documentazione}
Avendo analizzato le mancanze nella documentazione segnalate, abbiamo fissato i compiti da svolgere:
\begin{itemize}
    \item Aggiungere una descrizione, prevenzione dei rischi e loro mitigazione nel consuntivo;
    \item Sistemare le date dei file in modo che il formato sia anno mese giorno;
    \item Aggiungere come abbiamo calcolato il risultato dei test fatti nel documento Piano di Qualifica;
    \item Rifare la Lettera di Presentazione aggiungendo anche il preventivo e la data di consegna aggiornata;
    \item Controllare le norme riguardanti il formato delle date nel documento Norme di Progetto;
    \item Evidenziare che i verificatori fanno solo i verificatori nel documento Norme di Progetto;
\end{itemize}
\subsection{Framework}
Dopo aver ricercato il framework Vue e la libreria React, abbiamo discusso dei pro e dei contro di ciascuno e siamo giunti alla conclusione che React è una scelta migliore per il nostro progetto perché è più maturo e il suo vasto ecosistema offre grandi librerie che si integrano bene con three.js.
\subsection{Rotazione dei ruoli}
La rotazione dei ruoli ha prodotto come risultato la seguente suddivisione dei ruoli:
\begin{itemize}
        \item Rino Sincic: Responsabile di progetto;
	\item Gabriele Mantoan: Verificatore;
	\item Giacomo Mason: Verificatore;
	\item Marco Brigo: Progettista;
	\item Tommaso Allegretti: Progettista;
	\item Mirko Stella: Progettista;
	\item Elena Pandolfo: Amministratore.
\end{itemize}
