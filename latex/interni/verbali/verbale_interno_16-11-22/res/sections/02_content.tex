\section{Resoconto}
\subsection{Introduzione}
In questo incontro ci siamo trovati per organizzare: gli argomenti da discutere col proponente\textsubscript{g} nel meeting nati dalla riunione degli analisti e non solo, l'ITS Jira\textsubscript{g} e alcune norme per i verificatori.   

\subsection{Organizzazione Jira\textsubscript{g}}
Come primo argomento affrontato abbiamo parlato dell'ITS Jira\textsubscript{g} e come organizzarci con esso. Il responsabile di progetto ha dedicato una piccola parte della riunione per spiegare in maniera sintetica le basi di Jira\textsubscript{g} e decidendo di valutarlo nei prossimi giorni.

\subsection{Resoconto riunione analisti}
Gli attuali analisti si sono riuniti in un canale Discord\textsubscript{g} apposito e hanno discusso dei vari casi d'uso e requisiti da soddisfare nel progetto.
Ne sono uscite molteplici idee tra cui pensieri sulla decisione dell'ambiente , sulla decisione e sistemazione dei vari asset e su eventuali dettagli implementativi. E' stato deciso di creare un ambiente suddiviso o unico con possibili locazioni quali: un acquario, un deserto, un vulcano, una montagna , una nave , monumenti vari, un giardino oppure un parco naturalistico. L'ambientazione scelta dovrà inoltre essere suggestiva per quello che si mostra, inerente quindi agli asset scelti.
Si è poi pensato di implementare una piccola lista degli oggetti visti durante il percorso, la possibilità di teletrasportarsi senza far perdere il senso di esplorazione e l'eventuale implementazione di combinazioni di tasti per facilitare la navigazione verso la lista o verso il carrello.  
Questi argomenti saranno poi discussi nel meeting successivo con l'azienda.

\subsection{Norme per i verificatori}
Per terminare la riunione abbiamo discusso di normative per i verificatori. E' stato deciso che:
\begin{itemize}
	\item l'ultimo verificatore cambia il changelog, mettendo un verificato effettuando un commit.
	\item discusso del ciclo di vita\textsubscript{g} delle issue\textsubscript{g}
	\item quando si crea una pull request si devono assegnare i verificatori
\end{itemize} 
 

\subsection{Obiettivi prossimo periodo}
Gli obiettivi in programma per il prossimo periodo sono:
\begin{itemize}
\item Terminare l'\textit{Analisi dei Requisiti} e continuare con la stesura della documentazione
\item Documentarsi sulle tecnologie da adottare nel progetto
\item Decidere definitivamente l'ambiente e gli asset da utilizzare
\item Passare definitivamente a Jira\textsubscript{g} dalla Projects Board di GitHub\textsubscript{g}
\item Ottimizzare al meglio l'automazione tra Jira\textsubscript{g} e le issue\textsubscript{g} di GitHub\textsubscript{g}
\end{itemize}
