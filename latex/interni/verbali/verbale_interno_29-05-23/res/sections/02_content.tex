\section{Resoconto}
\subsection{Introduzione}
In questa riunione abbiamo fatto il punto della situazione, per capire cosa mancasse da sistemare nella documentazione e nel codice. 
Abbiamo iniziato anche a impostare le slide e dividere le parti per la presentazione finale della PB.
\subsection{Argomenti di discussione}
\begin{itemize}
    \item Documentazione;
    \item Codice;
    \item Presentazione.
\end{itemize}
\subsection{Documentazione}
La riunione è iniziata discutendo su cosa mancasse da fare di documentazione. A questo proposito è nato il bisogno di 
dover:
\begin{itemize}
	\item Scrivere la \textit{Lettera di Presentazione};
	\item Completare il \textit{Piano di Progetto} con una tabella riassuntiva finale di preventivo e consuntivo;
	\item Completare la stesura del \textit{Manuale Utente} aggiungendo le ultime modifiche;
	\item Aggiornare il cruscotto nel \textit{Piano di Qualifica} inserendo le ultime misure rilevate;
	\item Approvare i \textit{Verbali} e i documenti che necessitano di approvazione.
\end{itemize}

\subsection{Codice}
Per quanto riguarda il codice l'unica cosa ritenuta necessaria da fare prima della consegna è l'inserimento di più prodotti all'interno delle stanze.

\subsection{Presentazione}
Abbiamo iniziato ad impostare le slide per la presentazione finale, dividendoci il lavoro e assegnando a ciascun componente del gruppo 
una parte da scrivere in un foglio Google condiviso, per poi assemblare il tutto.

\subsection{Issues rilevate dalla riunione}
\begin{itemize}
    \item Stesura del verbale 29-05-2023;
    \item Aggiornare \textit{Piano di Progetto};    
    \item Aggiornamento metriche e misure \textit{Piano di Qualifica};
    \item \textit{Diario di Bordo} per la settimana di lavoro;
    \item Aggiornamento \textit{Manuale Utente};
    \item Stesura \textit{Lettera di Presentazione};
    \item Approvazione documenti;
    \item Inserimento prodotti nelle stanze;
    \item Creazione presentazione finale.
\end{itemize}