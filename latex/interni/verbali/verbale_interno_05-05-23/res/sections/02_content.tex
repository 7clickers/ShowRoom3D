\section{Resoconto}
\subsection{Introduzione}
In questa riunione abbiamo fatto il punto del lavoro sul progetto.
Abbiamo deciso di concentrarci sui requisiti obbligatori, in modo da velocizzare il lavoro.
Abbiamo discusso inoltre del Testing, di aggiornamenti o inserimenti nella Documentazione e infine programmato dei meeting.
\subsection{Argomenti di discussione}
\begin{itemize}
    \item Testing;
    \item Documentazione;
    \item Meeting futuri.
\end{itemize}
\subsection{Testing}
La riunione è iniziata con la suddivisione dei test di unità e di integrazione da effettuare ad ogni membro. I test verranno effettuati su ogni componente sviluppato fino ad ora.\\
Tommaso e Giacomo hanno poi spiegato al gruppo come utilizzare un framework di test JavaScript chiamato Jest\textsubscript{g}, come creare un test, come leggere i risultati da ogni test.
\subsection{Documentazione}
Successivamente alla definizione dei test è nato il bisogno di dover: 
\begin{itemize}
	\item Aggiornare la sezione delle norme di testing;
	\item Inserire una sezione di versionamento del codice, dove ogni integrazione di feature incrementa la versione;
	\item Descrivere il way of working per il branching del codice: creando un nuovo branch per ogni test di unità e per di test di integrazione si passerà tutto su un branch temporaneo per poi inserirlo nel principale.
\end{itemize}
Oltre all'aggiornamento del \textit{Piano di Progetto} è stata assegnata la stesura dei seguenti documenti:
\begin{itemize}
	\item Specifica tecnica, dove bisogna impostare la struttura del documento per quando l’architettura sarà ben consolidata;
	\item Manuale utente, per cercare di spiegare il funzionamento attuale all'utente. 
\end{itemize}

\subsection{Meeting futuri}
Infine abbiamo programmato due meeting.\\
Abbiamo deciso di mandare un'email al proponente per fissare un meeting di chiarimento su come e cosa implementare nell'ambiente, per soddisfare in maniera precisa le sue richieste.\\
Il secondo meeting da effettuare è invece con il prof.Cardin per ricevere un feedback sulla struttura del nostro diagramma delle classi.
\subsection{Issues rilevate dalla riunione}
\begin{itemize}
    \item Stesura del verbale 05-05-2023;
    \item Aggiornare \textit{Piano di Progetto};    
    \item Aggiornamento metriche e misure \textit{Piano di Qualifica};
    \item Diario di Bordo per la settimana di lavoro;
    \item Aggiornamento sezione norme di Testing;
    \item Inserire sezione versionamento del codice;
    \item Descrizione way of working del branching del codice su GitHub;
    \item Stesura Specifica Tecnica;
    \item Stesura Manuale Utente;
    \item Issue varie su i test da effettuare.
\end{itemize}