\section{Resoconto}
\subsection{Introduzione}
In questa riunione abbiamo discusso su come migliorare la progettazione dopo l'incontro con il prof. Cardin e su come procedere con 
la codifica e l'aggiornamento del documento \textit{Norme di Progetto}.
\subsection{Argomenti di discussione}
\begin{itemize}
    \item Progettazione;
    \item Codifica;
    \item \textit{Norme di Progetto}.
\end{itemize}
\subsection{Progettazione}
La riunione è iniziata discutendo su come continuare con la progettazione dopo i feedback ricevuti dalla riunione con il prof. Cardin. Si è deciso di alternare periodi di progettazione a 
periodi di codifica, e quindi di codificare prima quanto già progettato, per poi continuare con la progettazione delle parti rimanenti e l'eventuale modifica di ciò che si riterrà necessario.

\subsection{Codifica}
Si è deciso di iniziare a creare i vari componenti progettati e implementare l'architettura di Redux. In particolare i programmatori si occuperanno di:
\begin{itemize}
    \item Impostare la scena di base;
    \item Creare il componente Player, impostando i movimenti;
    \item Creare il componente Cart e con la relativa CartSlice;
    \item Creare il componente Product con la relativa ProductSlice;
\end{itemize}
\subsection{\textit{Norme di Progetto}}
Per lavorare al meglio abbiamo deciso che prima di iniziare la codifica è necessario scrivere le norme sulla codifica e il testing, aggiornando il 
documento \textit{Norme di Progetto}.

\subsection{Issues rilevate dalla riunione}
\begin{itemize}
    \item Stesura del verbale 19-04-2023;
    \item Scrivere norme di codifica;    
    \item Scrivere dorme di testing;
    \item \textit{Diario di Bordo} per la settimana di lavoro;
    \item Impostazione scena di base;
    \item Codifica componente Player;
    \item Codifica componente Cart;
    \item Codifica componente Product;
    \item Codifica CartSlice;
    \item Codifica ProductSlice.
\end{itemize}