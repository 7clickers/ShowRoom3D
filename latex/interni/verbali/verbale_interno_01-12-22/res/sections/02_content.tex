\section{Resoconto}
\subsection{Introduzione}
Abbiamo sfruttato questo incontro per aggiornare tutti i membri del gruppo sui progressi fatti rispetto alla settimana precedente e programmare i prossimi step da compiere e le scelte da fare.

\subsection{Avanzamento \textit{Piano di Progetto} e \textit{Analisi dei Requisiti}}
I membri dedicati al \textit{Piano di Progetto} hanno dimostrato i progressi compiuti ed evidenziato gli obiettivi per il futuro. In particolare, è stato discusso tra tutti i membri in che modo esporre all'interno del documento il diagramma di Gantt\textsubscript{g}.\\Allo stesso modo, gli incaricati dello svolgimento dell'\textit{Analisi dei Requisiti}, hanno mostrato al resto del gruppo l'avanzamento del documento. In particolare è nato un dubbio per quanto riguarda l'inclusione dei requisiti opzionali del capitolato all'interno dell'analisi, si è giunti alla conclusione di chiedere chiarimenti al professor Cardin.  


\subsection{Aggiornamento Milestone\textsubscript{g} e \textit{Norme di Progetto}}
Sono stati discussi alcuni cambiamenti marginali per quanto riguarda le \textit{Norme di Progetto} e sono state create delle nuove Milestone\textsubscript{g} (creazione Proof of Concept\textsubscript{g}, aggiornamento \textit{Piano di Progetto}, aggiornamento \textit{Norme di Progetto}).

\subsection{Discussione tecnologie}
Sono stati incaricati Elena Pandolfo e Rino Sincic per preparare una breve spiegazione riguardante Three.js, ossia la libreria\textsubscript{g} Javascript che andremo a usare per lo sviluppo del progetto.
