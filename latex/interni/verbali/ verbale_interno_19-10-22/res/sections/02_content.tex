\section{Resoconto}
\subsection{Introduzione}
Questa settimana ci siamo concentrati nel definire le informazioni principali e gli strumenti da utilizzare per poter lavorare in team in maniera ottimale. Successivamente ci siamo focalizzati nella scelta dei capitolati di interesse. 

\subsection{Nome del Gruppo}
Dopo un breve colloquio abbiamo deciso di procedere per votazione, ci siamo trovati d'accordo, scegliendo in modo unanime \textit{Seven Clickers} come nome del gruppo.  In seguito Mirko Stella ha proposto un motto \textit{"You wish We click"} da accompagnare al nome del team, anche quest'ultimo approvato da parte di tutti.

\subsection{Logo Seven Clickers}
Il logo è stato prodotto da Mirko Stella e approvato fin da subito da parte di tutto il gruppo.  Abbiamo deciso di crearne due versioni, una compresa di motto più adatta ai titoli, ed una senza più consona alle intestazioni dei vari documenti. 

\subsection{Organizzazione incontri}
Abbiamo ritenuto opportuno fissare degli incontri settimanali al fine di aggiornarsi e discutere sull'andamento del progetto. La scelta è ricaduta su mercoledì,  in aggiunta è stato proposto di tenere venerdì mattina come incontro opzionale in caso di necessità.  Per questi incontri abbiamo deciso di utilizzare la piattaforma Zoom per le chiamate, data la familiarità e la conoscenza da parte di tutti i componenti del team.  Per la condivisione di documenti invece,  abbiamo optato per Discord.

\subsection{E-mail progetto}
Come riferimento di posta elettronica abbiamo scelto :\textit{7clickersgroup@gmail.com}

\subsection{Sistema di versionamento}
Per il sistema di versionamento abbiamo deciso di utilizzare \textit{Git}.  Il gruppo ha ritenuto opportuno scegliere come piattaforma \textit{GitHub}, e di conseguenza creare una repository dedicata al progetto, in cui sarà contenuta tutta la documentazione ed il software relativo a quest'ultimo.

\subsection{Redazione documenti}
Per la stesura dei documenti abbiamo scelto di utilizzare \textit{Latex}, in modo tale da dare un'impronta professionale alla documentazione.  Inoltre abbiamo optato per creare un template contenente lo scheletro per la produzione degli elaborati,  in modo tale da velocizzare la stesura.

\subsection{Scelta Capitolati}
Durante un meeting \textit{Zoom} nel quale eravamo tutti presenti, abbiamo valutato e discusso i vari capitolati,  in modo tale da considerare quali fossero i più adatti al nostro gruppo.  Dopo un attenta riflessione siamo giunti alla conclusione che il Capitolato 6 è quello che ci rappresenta  di più,  non escludendo però C1 e C5, anche quest'ultimi di nostro interesse.

\subsection{Obiettivi prossimo periodo}
Gli obiettivi in programma per il prossimo periodo sono:
\begin{itemize}
\item Organizzare incontro con proponente
\item Familiarizzare con le tecnologie necessarie per il progetto
\end{itemize}
