\documentclass[a4paper]{article}
\usepackage[normalem]{ulem}

% impostazioni generali
%Tutti gli usepackage vanno qui
\usepackage{geometry}
\usepackage[italian]{babel}
\usepackage[utf8]{inputenc}
\usepackage{tabularx}
\usepackage{longtable}
\usepackage{hyperref}
\usepackage{enumitem}
\usepackage{array} 
\usepackage{booktabs}
\newcolumntype{M}[1]{>{\centering\arraybackslash}m{#1}}
\usepackage[toc]{appendix}

\hypersetup{
	colorlinks=true,
	linkcolor=blue,
	filecolor=magenta,
	urlcolor=blue,
}
% Numerazione figure
\let\counterwithout\relax
\let\counterwithin\relax
\usepackage{chngcntr}

% distanziare elenco delle figure e delle tabelle
\usepackage{tocbasic}
\DeclareTOCStyleEntry[numwidth=3.5em]{tocline}{figure}% for figure entries
\DeclareTOCStyleEntry[numwidth=3.5em]{tocline}{table}% for table entries


%\counterwithout{table}{section}
%\counterwithout{figure}{section}
\captionsetup[table]{font=small,skip=5pt} 

\usepackage[bottom]{footmisc}
\usepackage{fancyhdr}
\setcounter{secnumdepth}{4}
\usepackage{amsmath, amssymb}
\usepackage{array}
\usepackage{graphicx}

\usepackage{ifthen}

\usepackage{float}
\restylefloat{table}

\usepackage{layouts}
\usepackage{url}
\usepackage{comment}
\usepackage{eurosym}

\usepackage{lastpage}
\usepackage{layouts}
\usepackage{eurosym}

\geometry{a4paper,top=3cm,bottom=4cm,left=2.5cm,right=2.5cm}

%Comandi di impaginazione uguale per tutti i documenti
\pagestyle{fancy}
\lhead{\includegraphics[scale=0.1]{../../../template/images/logo_no_motto.jpeg}}
%Titolo del documento
\rhead{\doctitle{}}
%\rfoot{\thepage}
\cfoot{Pagina \thepage\ di \pageref{LastPage}}
\setlength{\headheight}{35pt}
\setcounter{tocdepth}{5}
\setcounter{secnumdepth}{5}
\renewcommand{\footrulewidth}{0.4pt}

% multirow per tabelle
\usepackage{multirow}

% Permette tabelle su più pagine
%\usepackage{longtable}


% colore di sfondo per le celle
\usepackage[table]{xcolor}

%COMANDI TABELLE
\newcommand{\rowcolorhead}{\rowcolor[HTML]{007c95}}
\newcommand{\captionline}{\rowcolor[HTML]{FFFFFF}} %comando per le caption delle tabelle
\newcommand{\cellcolorhead}{\cellcolor[HTML]{007c95}}
\newcommand{\hlinetable}{\arrayrulecolor[HTML]{007c95}\hline}

%intestazione
% check for missing commands
\newcommand{\headertitle}[1]{\textbf{\color{white}#1}} %titolo colonna
\definecolor{pari}{HTML}{b1dae3}
\definecolor{dispari}{HTML}{d7f2f7}

% comandi glossario
\newcommand{\glo}{$_{G}$}
\newcommand{\glosp}{$_{G}$ }


%label custom
\makeatletter
\newcommand{\uclabel}[2]{%
	\protected@write \@auxout {}{\string \newlabel {#1}{{#2}{\thepage}{#2}{#1}{}} }%
	\hypertarget{#1}{#2}
}
\makeatother

%riportare pezzi di codice
\definecolor{codegray}{gray}{0.9}
\newcommand{\code}[1]{\colorbox{codegray}{\texttt{#1}}}



% dati relativi alla prima pagina
% Configurazione della pagina iniziale
\newcommand{\doctitle}{Glossario}
\newcommand{\docdate}{21 Marzo 2023}
\newcommand{\rev}{2.0.0}
\newcommand{\stato}{Approvato}
\newcommand{\uso}{Esterno}
\newcommand{\approv}{Mirko Stella}
\newcommand{\red}{Giacomo Mason \\ & Mirko Stella}
\newcommand{\ver}{Giacomo Mason \\ & Gabriele Mantoan}
\newcommand{\dest}{\textit{Seven Clickers}
									  \\ Prof. Vardanega Tullio 
									   \\ Prof. Cardin Riccardo}
\newcommand{\describedoc}{Glossario del gruppo \textit{Seven Clickers}}


 % editare questo

\makeindex

%comando per far andare a capo i paragrafi
\makeatletter
\renewcommand\paragraph{
\@startsection {paragraph}{4}{0mm}{-\baselineskip}{.5\baselineskip}{\normalfont \normalsize \bfseries }}
\makeatother

\begin{document}
\counterwithin{table}{section}

% Prima pagina
\thispagestyle{empty}
\renewcommand{\arraystretch}{1.3}

\begin{titlepage}
	\begin{center}
		
	\includegraphics[scale = 0.40]{../../../template/images/logo.jpeg}
	\\[1cm]
	\href{mailto:7clickersgroup@gmail.com}		      	
	{\large{\textit{7clickersgroup@gmail.com} } }\\[2.5cm]
	\Huge \textbf{\doctitle} \\[1cm]
	 \large
			 \begin{tabular}{r|l}
                        \textbf{Versione} & \rev{} \\
                        \textbf{Stato} & \stato{} \\
                        \textbf{Uso} & \uso{} \\                         
                        \textbf{Approvazione\textsubscript{g}} & \approv{} \\                      
                        \textbf{Redazione} & \red{} \\ 
                        \textbf{Verifica\textsubscript{g}} &  \ver{} \\                         
                        \textbf{Distribuzione} & \parbox[t]{5cm}{ \dest{} }
                \end{tabular} 
                \\[3.3cm]
                \large \textbf{Descrizione} \\ \describedoc{} 
     \end{center}
\end{titlepage}

% Diario delle modifiche
\section*{Registro delle modifiche}

\newcommand{\changelogTable}[1]{
	 

\renewcommand{\arraystretch}{1.5}
\rowcolors{2}{pari}{dispari}
\begin{longtable}{ 
		>{\centering}M{0.07\textwidth} 
		>{\centering}M{0.13\textwidth}
		>{\centering}M{0.20\textwidth}
		>{\centering}M{0.17\textwidth} 
		>{\centering\arraybackslash}M{0.30\textwidth} 
		 }
	\rowcolorhead
	\headertitle{Vers.} &
	\centering \headertitle{Data} &	
	\headertitle{Autore} &
	\headertitle{Ruolo} & 
	\headertitle{Descrizione} 
	\endfirsthead	
	\endhead
	
	#1

\end{longtable}
\vspace{-2em}

}



\changelogTable{
	0.1.0 & 07-01-23 & Marco Brigo \\ Elena Pandolfo & Verificatori & Verifica\textsubscript{g} documento \\
	0.0.1 & 05-01-23 & Mirko Stella & Responsabile & Stesura documento \\
} % editare questo
\pagebreak

% Indice
{
    \hypersetup{linkcolor=black}
    \tableofcontents
}
\pagebreak

% Contenuto
\section{Introduzione}
\subsection{Scopo del documento}
\textit{Norme di Progetto} é il documento che definisce le regole e gli standard che devono essere seguiti durante il ciclo di vita del prodotto.
\subsection{Scopo del prodotto}
Il prodotto in questione nasce dalla necessità dell'azienda SanMarco Informatica di fornire una soluzione agli sprechi
derivati dall'adozione di uno ShowRoom tradizionale proponendo uno ShowRoom3D che sia ugualmente o ancora più immersivo.
\section{Processi Primari}
I processi primari sono un insieme di procedure che devono essere seguite  durante il ciclo di vita del software.
Nel nostro caso le fasi del ciclo di vita del software non comprendono l'installazione e la manutenzione ma ci limiteremo a terminare lo sviluppo.
Infatti il progetto in questione é da considerarsi un progetto didattico.
\input{res/sections/processi_primari/fornitura.tex}
\subsection{Sviluppo}

\subsubsection{Scopo}
Lo scopo del processo\textsubscript{g} di sviluppo è quello di stabilire le attività necessarie allo sviluppo del prodotto\textsubscript{g} software.

\subsubsection{Descrizione}
Nelle seguenti sezioni verranno descritte più nel dettaglio le principali attività, relative al processo\textsubscript{g} di sviluppo:
\begin{itemize} 
    \item \textit{Analisi dei Requisiti};
	\item Progettazione;
\end{itemize}
\subsubsection{\textit{Analisi dei Requisiti}}
L’\textit{Analisi dei Requisiti} è un’attività fondamentale che precede la progettazione. 
Si concretizza nel documento \textit{Analisi dei Requisiti}. Lo scopo di quest’attività è:
    \begin{itemize}
        \item Definire le funzionalità che il prodotto\textsubscript{g} andrà ad offrire;
        \item Porre le basi per la fase di progettazione del software;
        \item Fare una stima della mole di lavoro;
        \item Facilitare la verifica\textsubscript{g}. 
    \end{itemize}

\paragraph{Struttura del documento} Il documento è strutturato nel modo seguente:
    \begin{itemize}
        \item \textbf{Introduzione}: contiene una breve descrizione del documento e specifica gli attori dei casi d’uso;
		\item \textbf{Casi d’uso}: vengono specificati i casi d’uso individuati;
		\item \textbf{Requisiti}: sono classificati i requisiti che il prodotto\textsubscript{g} finale dovrà soddisfare.
    \end{itemize}

\paragraph{Casi d’uso} 

I casi d’uso sono identificati seguendo la seguente struttura:

\begin{center}\textbf{UC [Numero caso d’uso].[Numero sottocaso d’uso]-[Titolo caso d’uso]}\end{center}

\noindent Vengono poi indicati:
    \begin{itemize}
        \item \textbf{Diagramma UML\textsubscript{g}}: viene mostrato il diagramma solo per i casi più complessi, in cui è ritenuto necessario per una maggiore comprensione. Per i casi d’uso facoltativi il diagramma è colorato di azzurro (non standard UML\textsubscript{g}) per una maggiore differenziazione;
	    \item \textbf{Attore\textsubscript{g} primario}: utente esterno al sistema che svolge il caso d’uso;
	    \item \textbf{Descrizione}: breve descrizione del caso d’uso;
	    \item \textbf{Precondizioni}: indica la condizione del sistema prima del verificarsi del caso d’uso;
	    \item \textbf{Postcondizioni}: indica la condizione del sistema dopo che si è verificato il caso d’uso;
	    \item \textbf{Scenario principale}: descrive l’interazione tra l’attore\textsubscript{g} primario e il sistema;
	    \item \textbf{Estensioni}: casi d’uso alternativi che possono verificarsi al posto di quello a cui sono collegati. 
    \end{itemize}

\paragraph{Requisiti} 
I requisiti sono classificati secondo la seguente struttura: 

\begin{center}\textbf{R.[Tipologia][Numero seriale]}\end{center}

\noindent con:
    \begin{itemize}
        \item \textbf{Tipologia}: 
            \begin{itemize}
                \item \textbf{F}: funzionale;
                \item \textbf{Q}: qualitativo;
                \item \textbf{D}: di dominio;
                \item \textbf{P}: prestazionale.
            \end{itemize}
	    \item \textbf{Descrizione}: breve descrizione del requisito;
	    \item \textbf{Classificazione}: un requisito può essere facoltativo o obbligatorio;
	    \item \textbf{Fonti}: possono essere: 
            \begin{itemize}
                \item Casi d’uso;
                \item Capitolato;
                \item Decisione interna.
            \end{itemize}
    \end{itemize}

\subsubsection{Progettazione} 
L’attività di progettazione è assegnata ai progettisti. Seguendo quanto indicato 
nell’\textit{Analisi dei Requisiti}, l’obiettivo è quello di progettare l’architettura del sistema, 
prima con la creazione di un Proof of Concept\textsubscript{g} per la Requirments and Tecnology Baseline e poi andando 
nel dettaglio per la Product Baseline.

\paragraph{Requirments and Tecnology Baseline} 
In questa fase vengono fissati i requisiti che il gruppo si impegna a soddisfare, in accordo con il 
proponente\textsubscript{g}; si studiano le tecnologie, i framework\textsubscript{g} e le librerie utili alla realizzazione del 
prodotto\textsubscript{g} finale e si crea di conseguenza un Proof of Concept\textsubscript{g}. I materiali da mostrare sono:
    \begin{itemize}
        \item Tecnologie, framework\textsubscript{g}, librerie utilizzate, motivando la scelta;
        \item Proof of Concept\textsubscript{g};
        \item \textit{Analisi dei Requisiti}.
    \end{itemize}
La documentazione da mostrare:
    \begin{itemize}
        \item \textit{Piano di Progetto};
        \item \textit{Piano di Qualifica};
        \item \textit{Norme di Progetto};
        \item \textit{Verbali} interni ed esterni.
    \end{itemize}
\subsubsection{Metriche}
Per mantenere la qualità nel processo\textsubscript{g} di Sviluppo abbiamo deciso di adottare la seguente metrica.
\begin{itemize}
    \item \textbf{MPC09: Requirements Stability Index}.
\end{itemize}


\pagebreak
\section{Processi di Supporto}
Sono tutti quei processi che supportano i processi primari in modo da renderli piú efficienti ed efficaci.
\subsection{Documentazione}
\subsubsection{Scopo}
Il processo\textsubscript{g} di documentazione ha come scopo quello di fornire un insieme di documenti che descrivono in dettaglio il progetto software, comprese le sue funzionalità, i requisiti, l'architettura, il design, i processi di sviluppo e i risultati. 
Questa documentazione serve come riferimento per lo sviluppo del software e aiuta a garantire la coerenza e la completezza del progetto. 
La documentazione del progetto software è un elemento essenziale e deve essere costantemente aggiornata.
\subsubsection{Descrizione}
Nella seguente sezione sono raggruppate tutte le norme necessarie alla stesura, alla verifica\textsubscript{g} e 
all’approvazione\textsubscript{g} della documentazione, in modo tale da mantenere una coerenza nella forma e nella 
struttura dei documenti prodotti dal gruppo.
\subsubsection{Ciclo di vita di un documento}
\begin{itemize}
    \item \textbf{Creazione}: il documento viene creato su un nuovo branch, utilizzando un template uguale per tutti i documenti. 
    Viene aggiunta l’intestazione e il registro delle modifiche;
    \item \textbf{Stesura}: si procede con la stesura delle varie sezioni, tracciando i cambiamenti nel registro delle modifiche; 
    \item \textbf{Verifica}: ogni sezione viene verificata dai verificatori (per maggiori specifiche guardare la 
    sezione x.x.x Verifica della documentazione);
    \item \textbf{Approvazione}: una volta che tutte le sezioni sono state verificate si procede 
    con l’approvazione del documento, che viene effettuata dal responsabile nel seguente modo:
    \begin{itemize}
        \item Viene aperta una Pull Request di approvazione;
        \item Il responsabile rilegge il documento: se riscontra ulteriori problematiche, segnala ai verificatori le eventuali modifiche da apportare;
        \item Se sono richieste delle modifiche i verificatori si occupano di apportarle;
        \item Il responsabile crea una nuova riga e compila i campi nelle colonne corrispondenti del registro delle modifiche 
        inserendo: l'ultima versione secondo le norme di versionamento, la data, il proprio nome, il suo ruolo e la voce "Approvazione" nell'ultima colonna.
    \end{itemize}
    Per i \textit{Verbali} si effettua questa pratica non appena il file prodotto viene 
    verificato, mentre per tutti gli altri documenti prima di una consegna.
\end{itemize}
\subsubsection{Struttura generale}
Ogni documento deve presentare le seguenti sezioni nell'ordine in cui vengono presentate:
\begin{itemize} 
    \item \textbf{Intestazione}:
    contiene:
    \begin{itemize} 
        \item Logo compreso di motto;
        \item Indirizzo email di gruppo; 
        \item Titolo;
        \item Tabella contenente le informazioni generali:
        \begin{itemize}
            \item Versione;
            \item Stato;
            \item Uso;
            \item Approvazione\textsubscript{g}: indica il responsabile di progetto che ha approvato il documento; 
            \item Redazione: elenco dei collaboratori che hanno partecipato alla stesura del documento;
            \item Verifica\textsubscript{g}: elenco dei verificatori che hanno verificato il documento;
            \item Distribuzione: elenco delle persone o organizzazioni a cui è destinato il documento.
        \end{itemize}
        \item Breve descrizione del documento .
    \end{itemize}
    \item \textbf{Registro delle modifiche}:
    tabella che identifica ogni versione del documento indicandone:
    \begin{itemize} 
        \item Versione;
        \item Data;
        \item Autore;
        \item Ruolo;
        \item Descrizione: la descrizione deve essere breve. Nel caso di aggiunte o modifiche si deve indicare 
        il nome della sezione che è stata aggiunta o modificata. 
        I nomi delle sezioni vanno riportati uguali a come sono scritti nell'indice, 
        senza virgolette.
    \end{itemize}
    \item \textbf{Indice}:
    elenco ordinato dei titoli dei capitoli, ovvero delle varie parti di cui si compone il documento;
    
    \item \textbf{Elenco delle figure}: sezione che fornisce l'elenco delle immagini presenti nel documento. 
    Se il documento non presenta immagini, questa sezione sarà omessa di conseguenza;  
    \item \textbf{Elenco delle tabelle}: sezione che fornisce l'elenco delle tabelle presenti nel documento. 
    Se il documento non presenta tabelle, questa sezione sarà omessa di conseguenza; 
    \item \textbf{Contenuto}:
    varia a seconda del tipo di documento.
\end{itemize}
\subsubsection{Convenzioni}
Le convenzioni di seguito riportate vengono applicate a tutti i documenti.
Esse rendono i documenti stilati omogenei tra loro contribuendo a rendere il progetto professionale.

\paragraph{Date}
Le date devono rispettare il seguente formato: \textbf{yyyy-mm-dd}
All'interno delle tabelle il formato deve essere il seguente: \textbf{dd-mm-yy}
\paragraph{Nomi di persona}
All'interno dei documenti i nomi di persona rispetteranno l'ordine nome seguito dal cognome della persona menzionata.

\paragraph{Elenchi puntati}
Gli elenchi puntati devono rispettare le seguenti regole:
\begin{itemize} 
    \item Ogni elemento dell'elenco deve iniziare con la lettera maiuscola;
    \item Ogni elemento dell'elenco deve terminare con ";" ad eccezione dell'ultimo elemento
    che deve terminare con "."; 
    \item Dopo i due punti la frase deve iniziare con la lettera minuscola.
\end{itemize}

\paragraph{Stile del testo}
\begin{itemize} 
    \item \textbf{Grassetto}: stile utilizzato per i titoli delle sezioni e per i primi termini degli elenchi puntati;
    \item \textbf{Corsivo}: viene utilizzato per citare il nome dei documenti, ad esempio \textit{Piano di Progetto}. 
\end{itemize}

\paragraph{Immagini}
Le immagini sono raccolte nella cartella “images”, e sono inserite sempre con una didascalia descrittiva posizionata sotto l’immagine.

\paragraph{Tabelle} 
Le tabelle sono provviste di didascalia descrittiva posizionata sotto alla tabella. La tabella contenente il registro delle modifiche è l’unica che fa da eccezione a questa regola.

\paragraph{\textit{Glossario}}
All'interno della documentazione si possono trovare dei termini che possono risultare ambigui a seconda del contesto, o non conosciuti dagli utilizzatori.\\\
Per ovviare ad incomprensioni si è deciso di stilare un elenco di termini di interesse accompagnati da una descrizione del loro significato.\\
I termini presenti all'interno dei documenti che necessitano di una descrizione vengono indicati con il pedice 'g' come nell'esempio seguente: termine.
É quindi possibile consultare il \textit{Glossario} per reperire tale descrizione.
\\
Ogni componente del gruppo all'inserimento di un termine ritenuto ambiguo deve preoccuparsi di aggiornare il \textit{Glossario}.

Per aggiornare il \textit{Glossario} si devono inserire i nuovi termini nel file .tex nella cartella corrispondente all'iniziale del termine situata al percorso
\textbf{\textit{latex\textsubscript{g}/esterni/doc\_esterna/\textit{Glossario}/res/sections/alphabet/}}.
É stato creato uno script che scansiona un documento di interesse per inserire in automatico il pedice sui termini contenuti nel \textit{Glossario}.

Il componente del gruppo che inserisce all'interno del \textit{Glossario} un nuovo termine deve aggiungere nel file .tex il segnaposto \%parola\% dopo la subsection (senza spazi) che racchiude il termine per permetterne il riconoscimento da parte dello script.
\\
Il \textit{Glossario} ordina i termini in ordine alfabetico in modo da permetterne una facile e veloce ricerca.
\subsubsection{Strumenti per la stesura}
\begin{itemize} 
    \item \textbf{LaTeX}: è un linguaggio di marcatura per la preparazione di testi, basato sul 				  		  programma di composizione tipografica TEX.\\
	Nel branch documentation  si possono trovare i file .pdf prodotti e la cartella “latex”. La cartella latex contiene tre cartelle interne:
	\begin{itemize}
	\item \textbf{esterni e interni}: contengono file .tex di documentazione esterna ed interna come ad esempio i \textit{Verbali} o altra documentazione esterna/interna:
	\begin{itemize}
		\item La cartella config contiene i file .tex con le parti fisse dei documenti (intestazione, registro delle modifiche, tracciamento dei temi affrontati) che vengono modificati con i dati del documento specifico;
		\item La cartella res/sections contiene i file .tex con il contenuto vero e proprio (sezioni del documento) che viene redatto in maniera libera dal redattore;
		\item Un file col nome del documento pdf con estensione .tex che viene compilato per produrre il file pdf.
	\end{itemize}
	 \item \textbf{template}: contiene file .tex di base utilizzati secondo necessità per comporre i documenti:
	 \begin{itemize}
	 \item changelox.tex è il file di template che serve per scrivere il registro delle modifiche;
	 \item package.tex è il file che contiene tutti gli usepackage\textsubscript{g};
	 \item titlepage.tex è il file di template che contiene la configurazione della pagina iniziale di ogni documento;  
	 \item tracking.tex è il file di template che contiene il tracciamento dei temi affrontati nel documento.
	 \end{itemize}
	\end{itemize}
\end{itemize}

\subsubsection{Documentazione interna}
La documentazione interna comprende tutti i documenti che contengono informazioni utili principalmente per il gruppo, e che vengono quindi consultati di conseguenza. 
    \input{res/sections/documentazione/documentazione_prodotta/diari_di_bordo.tex}
    \input{res/sections/documentazione/documentazione_prodotta/verbali.tex}
    \input{res/sections/documentazione/documentazione_prodotta/studio_di_fattibilita.tex}
    \input{res/sections/documentazione/documentazione_prodotta/norme_di_progetto.tex}
    \subsubsection{Documentazione esterna}
La documentazione esterna comprende tutti i documenti che interessano anche al proponente e al committente.
    \input{res/sections/documentazione/documentazione_prodotta/lettera_di_presentazione.tex}
    \input{res/sections/documentazione/documentazione_prodotta/piano_di_progetto.tex}
    \input{res/sections/documentazione/documentazione_prodotta/piano_di_qualifica.tex}
    \input{res/sections/documentazione/documentazione_prodotta/glossario.tex}








\subsubsection{Metriche}
Per perseguire la qualità sulla documentazione prodotta si è deciso di adottare le seguenti metriche:
\paragraph{Indice di Gulpease}
Si tratta dell'indice di leggibilità di un testo tarato sulla lingua italiana.
I risultati sono compresi tra 0 e 100, dove il valore "100" indica la leggibilità più alta e "0" la leggibilità più bassa. Ai seguenti valori si associano i seguenti significati:
\begin{itemize}
\item inferiore a 80 sono difficili da leggere per chi ha la licenza elementare
\item inferiore a 60 sono difficili da leggere per chi ha la licenza media
\item inferiore a 40 sono difficili da leggere per chi ha un diploma superiore
\end{itemize}
Viene adottata la seguente formula per calcolarlo:
\begin{equation}
89+\frac{300*(\text{numero delle frasi})-10*(\text{numero delle lettere})}{\text{numero delle parole}}
\end{equation}\\
Questi sono i valori da noi ritenuti opportuni:\\
\textit{Valore minimo}:$$ \geq 50 $$ 
\textit{Valore ottimo}:$$ \geq 80 $$\\
\subsection{Gestione della configurazione}
\subsubsection{Scopo}
La gestione della configurazione è un processo\textsubscript{g} che mira a gestire e controllare i cambiamenti apportati a un prodotto\textsubscript{g} software o a un sistema durante il suo ciclo di vita\textsubscript{g}. 
La gestione della configurazione per la documentazione descrive come vengono identificate, controllate, tracciate e gestite le versioni di un documento.
\subsubsection{Descrizione}
In questa sezione sono riportate tutte le norme relative alla configurazione degli strumenti utilizzati per il tracciamento e l’organizzazione dei documenti e del codice.
\subsubsection{Versionamento}
Il numero di versione permette di capire lo stato in cui si trova un documento.
Un documento può trovarsi nei seguenti stati:
\begin{itemize} 
    \item \textbf{Approvato}: il documento è verificato ed approvato dal responsabile;
    \item \textbf{Verificato}: il documento risulta verificato ma non ancora visionato dal responsabile;
    \item \textbf{In Sviluppo}: sono presenti delle modifiche che non sono state verificate.
\end{itemize}
Il numero di versione ha il formato \textbf{X.Y.Z} dove:
\begin{itemize} 
    \item \textbf{X}: indica una versione approvata dal responsabile, la numerazione parte da 0
    e la prima versione approvata è la 1.0.0;
    \item \textbf{Y}: indica una versione verificata dal verificatore, la numerazione inizia da 0 e si azzera ad ogni incremento di X. La prima versione
    verificata è la 0.1.0;
    \item \textbf{Z}: indica una versione in fase di modifica da parte dei redattori che ne incrementano il numero ad ogni modifica,
    la numerazione parte da 1 e si azzera ad ogni incremento di X o Y. La prima versione modificata è la 0.0.1.
\end{itemize}

\paragraph{Sistemi software utilizzati}
Per il versionamento il gruppo ha deciso di utilizzare un repository\textsubscript{g} GitHub\textsubscript{g}, un servizio di hosting per progetti software che implementa uno strumento di controllo versione distribuito Git\textsubscript{g}.
Per informazioni più approfondite sulla struttura e la gestione del repository\textsubscript{g} si veda sezione \fullref{GitHub\textsubscript{g}}.
\subsection{Accertamento della qualità}

\subsubsection{Scopo}
Lo scopo del processo di accertamento della qualità è quello di garantire e rispettare i requisiti di qualità preposti.
\subsubsection{Descrizione}
Questo processo è direttamente collegato al documento di \textit{Piano di Qualifica} in cui il gruppo si impegna a mantenere, con l'adozione di metriche concordate e discusse, la qualità nei processi e nei prodotti.
\subsubsection{Attività per il controllo di qualità}
Queste sono le attività che ogni membro deve rispettare per il mantenimento della qualità.
Ogni membro del gruppo deve:
\begin{itemize}
\item Comprendere le attività da svolgere e gli obiettivi da raggiungere;
\item Aver compreso le metriche inserite nel documento di \textit{Piano di Qualifica} e rispettare le normative e gli standard definiti al suo interno;
\item Mantenere le normative inserite nelle \textit{Norme di Progetto};
\item Riconoscere nel documento su cui si sta lavorando valori di cui tenere conto per un'analisi successiva;
\item Utilizzare il sistema di tracciamento delle issue\textsubscript{g} fornito da Github\textsubscript{g} come descritto nel documento di \textit{Norme di Progetto};
\item Attuare miglioramento continuo, ponendosi obiettivi incrementali.
\end{itemize}
\subsubsection{Denominazione degli obiettivi di qualità nel \textit{Piano di Qualifica}}
Per la denominazione delle metriche di prodotto è stato adottata la seguente scrittura:
\begin{center}\textbf{MPD[Numero]}\end{center}
Per la denominazione delle metriche di processo è stato adottata la seguente scrittura:
\begin{center}\textbf{MPC[Numero]}\end{center}
Viene inoltre inserito il nome della metrica, il valore minimo e il valore ottimo.\\
La descrizione dettagliata di ogni metrica la si può trovare all'interno delle \textit{Norme di Progetto} nel processo a loro associato.
\subsubsection{Metriche} 
Per tracciare la soddisfazione dei requisiti è stata concordata una semplice metrica.
\begin{itemize}
    \item \textbf{MPC11: Metriche soddisfatte}.
\end{itemize}

\input{res/sections/processi_supporto/verifica\textsubscript{g}/verifica_main.tex}
\input{res/sections/processi_supporto/validazione\textsubscript{g}.tex}

\pagebreak
\section{Processi Organizzativi}
I processi organizzativi software sono un insieme di procedure che mirano a gestire i processi e il loro miglioramento, l'organizzazione degli strumenti 
di supporto e la gestione del personale.
\subsection{Gestione Organizzativa}
\subsubsection{Ruoli}
I componenti del gruppo si suddivideranno nei seguenti ruoli per periodi di circa 2-3 settimane (dipendentemente dalle esigenze del periodo) e al termine del periodo i ruoli verranno risuddivisi. 
Visto che nelle varie fasi di sviluppo del progetto le attività da svolgere variano, non sempre sarà necessario coprire tutti i ruoli.\\
Inoltre sarà necessario tenere traccia delle ore che ogni componente dedica al progetto ed il ruolo associato a quelle ore, in modo da andare a rispettare la tabella degli impegni individuali.\\
Per questo tracciamento verrà utilizzato un foglio Excel in cui ogni componente del gruppo segnerà le ore di lavoro settimanalmente.
I ruoli e le loro competenze sono i seguenti:

\paragraph{Responsabile}
Deve avere la visione d'insieme del progetto e coordinare i membri, inoltre si occupa di rappresentare il gruppo con le interazione esterne (proponente, committente ecc...). Le sue competenze specifiche sono:
\begin{itemize}
	\item Ad ogni iterazione\textsubscript{g} c'è un solo responsabile;
	\item Presentare il \textit{Diario di Bordo} in aula;
	\item Redarre l'ordine del giorno prima di ogni meeting interno del gruppo;
	\item Suddivide le attività del gruppo in singole issue (ma non le assegna ai membri del gruppo);
	\item In fase di release\textsubscript{g} si occupa di approvare\textsubscript{g} tutti i documenti che necessitano approvazione.
\end{itemize}

\paragraph{Analista}
Si occupa di trasformare i bisogni del proponente nelle aspettative che il gruppo deve soddisfare per sviluppare un prodotto professionale. Le sue competenze specifiche sono:
\begin{itemize}
	\item Interrogare il proponente riguardo allo scopo del prodotto e le funzionalità che deve avere;
	\item Studiare le risposte del proponente per identificare i requisiti\textsubscript{g} e redarre l' \textit{Analisi dei Requisiti}.
\end{itemize}

\paragraph{Amministratore}
Si occupa del funzionamento, mantenimento e sviluppo degli strumenti tecnologici usati dal gruppo. Le sue competenze specifiche sono:
\begin{itemize}
	\item Ad ogni iterazione\textsubscript{g} basta un solo amministratore;
	\item Gestione delle segnalazioni e problemi dei membri del gruppo riguardanti problemi e malfunzionamenti con gli strumenti tecnologici;
	\item Valuta l'utilizzo di nuove tecnologie e ne fa uno studio preliminare per poter presentare al gruppo i pro e i contro del suo utilizzo.
\end{itemize}

\paragraph{Progettista}
Si occupa di scegliere la modalità migliore per soddisfare le aspettative del committente che gli analisti hanno ricavato dall'analisi dei requisiti\textsubscript{g}. Le sue competenze specifiche sono:
\begin{itemize}
	\item Scegliere eventuali pattern architetturali da implementare;
	\item Sviluppare lo schema UML\textsubscript{g} delle classi\textsubscript{g}.
\end{itemize}

\paragraph{Programmatore}
Si occupa di implementare le scelte e i modelli fatti dal progettista. Le sue competenze specifiche sono:
\begin{itemize}
	\item Scrivere il codice atto a implementare lo schema delle classi;
	\item Scrivere eventuali test;
	\item Scrivere la documentazione per la comprensione del codice che scrive.
\end{itemize}

\paragraph{Verificatore}
Si occupa di controllare che ogni file che viene caricato in un branch protetto della repository\textsubscript{g} sia conforme alle norme di progetto. Le sue competenze specifiche sono:
\begin{itemize}
	\item Controllare i file modificati o aggiunti durante una pull request tra un ramo non protetto e un ramo protetto siano conformi alle norme di progetto e cercano errori di altra natura (ortografici, sintattici, logici, build ecc...).
\end{itemize}
\subsubsection{Metriche}
Abbiamo considerato questa metrica come rilevante per mantenere la qualità.
\paragraph{Rischi non previsti}
Il valore è identificato con un numero intero e indica il numero di rischi non preventivati.
Questi sono i valori da noi ritenuti opportuni:\\
\textit{Valore minimo}: 10\\
\textit{Valore ottimo}: 5\\
\subsection{Gestione Infrastrutture}
\subsubsection{GitHub\textsubscript{g}}
\label{GitHub} 
Servizio di hosting per progetti software che implementa uno strumento di controllo versione distribuito Git\textsubscript{g}.
Oltre alla copia in remoto del repository\textsubscript{g} di progetto ogni componente del gruppo ha una propria copia in locale.\\
Per ottenere una copia del repository\textsubscript{g} ogni componente ha scaricato lo strumento Git\textsubscript{g} ed eseguendo il 
comando 'git clone' da git\textsubscript{g} bash viene creata una cartella collegata alla repository\textsubscript{g} di progetto.\\
Non sono state imposte modalità specifiche sull'interazione con il repository\textsubscript{g} remoto in modo da non sconvolgere le abitudini di lavoro di 
ciascun componente.\\
I componenti del gruppo abituati ad interagire con GitHub\textsubscript{g} da interfaccia grafica possono continuare a farne uso.
\paragraph{Repository\textsubscript{g}}
Il repository\textsubscript{g} si può trovare all'indirizzo \textbf{\textit{https://github.com/7clickers/ShowRoom3D}} ed è pubblico. 
I collaboratori sono i componenti del gruppo SevenClickers che utilizzano il proprio account GitHub\textsubscript{g} personale per collaborare al progetto.
\paragraph{Branching}
I branch si dividono in: 
\begin{itemize}
	\item \textbf{Branches protetti}:
    \begin{itemize} 
        \item \textbf{main}: contiene le versioni di release\textsubscript{g} del software;
        \item \textbf{documentation}: contiene i template latex\textsubscript{g} e rispettivi pdf della documentazione. I documenti presenti 
		in documentation sono stati approvati dal responsabile o almeno verificati dai verificatori.
    \end{itemize}
	\item \textbf{Branches liberi}: viene creato un branch\textsubscript{g} libero per ogni documento che si sta modificando. Il nome del branch deve essere:
	\begin{itemize}	
		\item Uguale al nome del documento;
		\item Scritto tutto in minuscolo;
		\item Gli spazi tra le parole devono essere sostituiti con il trattino basso.
	\end{itemize}
	
	In ogni branch\textsubscript{g} libero possono lavorare uno o più componenti del gruppo, a condizione che ognuno lavori su parti diverse 
	del documento a cui fa riferimento il branch\textsubscript{g}. Ad ogni componente che lavora sul branch\textsubscript{g}  
	viene assegnata una issue da risolvere. Una volta che tutte le issue relative al branch\textsubscript{g} sono state risolte ed è avvenuta 
	la verifica, si fa il merge e il branch\textsubscript{g} viene eliminato.
\end{itemize}

\paragraph{Commits}
\label{Commits}
È preferibile che ogni commit abbia una singola responsabilità per cambiamento.\\
I commits non possono essere effettuati direttamente sui branch protetti ma per integrare delle aggiunte o modifiche sarà necessario aprire una Pull Request.
All'approvazione di una Pull Request tutti i commit relativi al merge verranno raggruppati in un unico commit che rispetti la struttura sintattica descritta in seguito.

I commit dovranno essere accompagnati da una descrizione solo se ritenuta indispensabile alla comprensione del commit stesso.

I messaggi di commit sui \textbf{\uppercase{branch protetti}} dovranno seguire la seguente struttura sintattica:\\

\textbf{$<$label$>$$<$\#n\_issue$>$$<$testo$>$}\\\\
dove:\\\\
\begin{itemize}
\item \textbf{label} può assumere i seguenti valori:
	\begin{itemize}
	\item \textbf{feat}: indica che è stata implementata una nuova funzionalità;
	\item \textbf{fix}: indica che è stato risolto un bug;
	\item \textbf{update}: indica che è stata apportata una modifica che non sia fix o feat;
	\item \textbf{test}: qualsiasi cosa legata ai test;
	\item \textbf{docs}: qualsiasi cosa legata alla documentazione.
	\end{itemize}
\item \textbf{n\_issue}: indica il numero della issue a cui fa riferimento il commit (se non fa riferimento a nessuna issue viene omesso);
\item \textbf{testo}: indica con quale branch è stato effettuato il merge e deve rispettare la forma: merge from $<$nome branch da integrare$>$ to $<$nome branch corrente$>$;
\item \textbf{descrizione}: se aggiunta ad un commit deve rispondere alle domande WHAT?, WHY?, HOW? ovvero
cosa è cambiato, perché sono stati fatti i cambiamenti,in che modo sono stati fatti i cambiamenti.\\
\end{itemize}
I messaggi di commit sui \textbf{\uppercase{branch liberi}} dovranno seguire la struttura sintattica dei branch protetti ad eccezione del testo.
Il testo dei commit sui branch liberi non è soggetto a restrizioni particolari a patto che indichi in maniera intuitiva i cambiamenti fatti
in modo che possano essere compresi anche dagli altri collaboratori.



\paragraph{Pull Requests}
\label{Pull_Requests}
Per effettuare un merge su un branch protetto si deve aprire da GitHub\textsubscript{g} una Pull Request.
La Pull Request permette di verificare il lavoro svolto prima di integrarlo con il branch desiderato.\\
Alla creazione di una Pull Request bisogna associare:
\begin{itemize}
\item I verificatori in carica hanno il compito di trovare eventuali errori o mancanze e fornire un feedback riguardante il contenuto direttamente su GitHub\textsubscript{g} richiedendo
una review con un review comment sulla parte specifica da revisionare o con un commento generico.\\
Non sarà possibile effettuare il merge finchè tutti i commenti di revisione non saranno stati risolti e la Pull Request approvata da due verificatori;
\item L’issue associata nell’opzione “Development” che verrà chiusa alla risoluzione della Pull Request;
\item La Projects Board di cui fa parte;
\item Gli assegnatari che hanno il compito di apportare le modifiche necessarie in fase di verifica;
\item Le labels associate.
\end{itemize}
Per i commit relativi alle Pull Requests seguire le regole descritte nella sezione \fullref{Commits} per i branch protetti.

\paragraph{Milestone\textsubscript{g}}
Una milestone indica un traguardo intermedio significativo per il progetto.
Ad essa possono venire assegnate delle issues per verificarne il raggiungimento.
Ogni milestone ha una scadenza che viene discussa e fissata da tutto il gruppo.
Oltre alle 2 milestone + 1 milestone falcoltativa fissate dal committente il gruppo ne creerá ulteriori per scandire piú nel dettaglio i passi che ci porteranno 
ad ottenere i risultati prefissati.
Una delle milestone create dal gruppo durante il completamento della tecnology baseline relativa alla prima milestone imposta dal committente (Requirements and Tecnology Baseline)
riguarda l'implementazione di un PoC (Proof of concept).

\paragraph{Projects Board\textsubscript{g}}
Vengono utilizzate due project board per tracciare le issues della repo.
Una project board principale utilizzata da tutti i membri del gruppo e una project board per le approvazioni utilizzata solo dal responsabile di progetto per approvare i file che richiedono l'approvazione prima di una consegna.

\begin{itemize}
	\item La projectboard\textsubscript{g} principale è suddivisa in queste sezioni:
	\begin{itemize}
		\item \textbf{Todo}: issue\textsubscript{g} che non sono ancora state iniziate o che non sono ancora state assegnate;
		\item \textbf{In Progress}: issue\textsubscript{g} che sono state assegnate e a cui almeno un membro a cui è stata assegnata ha iniziato a lavorarci;
		\item \textbf{Pull Request}: issue\textsubscript{g} che è in fase di integrazione e necessita della verifica dei verificatori. Corrisponde all'inizio di una pull request;
		\item \textbf{Done}: issue\textsubscript{g} che sono state chiuse e che sono state verificate (se necessitano di verifica\textsubscript{g});
		\item \textbf{Approved}: issue\textsubscript{g} con label "Da Approvare" che hanno ottenuto l'approvazione\textsubscript{g} del responsabile subito dopo la verifica;
		Per tutte le issues che richiedono un'approvazione solo al momento della consegna è stata creata la project board dedicata alle approvazioni.
	\end{itemize}
	Inoltre nella project board principale vengono registrate delle issue che non richiedono verifica, approvazione o neanche integrazione, con lo scopo di monitorare meglio il lavoro di ogni membro del team.\\
Queste issue verranno chiuse e archiviate manualmente una volta che avranno terminato la loro utilità, un esempio può essere la seguente issue:\\
\textbf{diario di bordo 21-11-22}; questa issue non necessità verifica, approvazione o integrazione perchè non è di interesse caricare il file nella repo, però è utile tracciare lo svolgimento della issue.

	\item La projectboard\textsubscript{g} riservata alle approvazioni è suddivisa in queste sezioni:
	\begin{itemize}
		\item \textbf{Todo}: issue\textsubscript{g} che non sono ancora state iniziate dal responsabile;
		\item \textbf{In Progress}: issue\textsubscript{g}  che sono state prese in carico per essere approvate dal responsabile;
		\item \textbf{Pull Request}: issue\textsubscript{g} che è in fase di integrazione e necessita della verifica dei verificatori. Corrisponde all'inizio di una pull request;
		in questo caso i verificatori dovranno solo controllare che l'intestazione e il registro delle modifiche siano stati compilati correttamente dal responsabile di progetto in quanto tutto il resto del contenuto è già stato verificato in precedenza;
		\item \textbf{Approved}: issue\textsubscript{g} che sono state approvate dal responsabile.
	\end{itemize}
\end{itemize}
Le issue che si trovano nella project board approvazioni sono state create come continuazione ad issues che hanno in precedenza attraversato il work flow della project board principale fino allo stato "Done".
In questo modo il Responsabile di progetto dovrà approvare solamente file che si trovano nei branches protetti.
Vengono ricreate in questa project board per  mantenerne la tracciabilità e permetterne una più facile reperibilità futura per l'approvazione.
Questo permette anche di pulire la project board principale da tutte quelle issues che rimarrebbero inattive per molto tempo.
\\\\
Esempio di work flow issues:
\begin{enumerate}
	\item Viene creata l'issue ed inserita all'interno della sezione "To Do" della project board principale;
	\item Quando viene presa in carico l'issue viene spostata nella sezione "In Progress" fino al suo completamento;
	\item L'incaricato alla risoluzione dell'issue apre una pull request a cui assegna l'issue in modo che  venga chiusa in automatico al momento dell'integrazione con il branch di destinazione;
	\item Vengono fatte le verifiche opportune dai verificatori e le conseguenti modifiche prima di accettare la pull request;
	\item Dopo l'accettazione la pull request viene postata insieme alle issues associate nella colonna "Done";
	\item A questo punto si possono presentare due casi: 
	\begin{enumerate}
		\item I file relativi alla pull request richiedono approvazione immediata.
		Viene archiviata la pull request (e le issue associate) e creata una issue a cui viene aggiunta la label "Da approvare" con il nome del file da approvare.
		Il responsabile creerà un branch per approvare le issues con approvazione immediata aprirà una pull request per integrare nel branch di destinazione l'approvazione;
		\item I file relativi alla pull request richiedono l'approvazione prima di una consegna.
		Il responsabile archivia la pull request e ne crea una issue con il nome dei file da approvare nella project board dedicata alle approvazioni.
		NB: se nella project board Approvazioni è già presente un'issue di approvazione per il file da approvare non viene creata una nuova issue di approvazione ma si tiene la precedente.
		L'issue segue il work flow della project board approvazioni.
	\end{enumerate}
\end{enumerate}
\paragraph{Issue Tracking System\textsubscript{g}}
Il gruppo utilizza l'issue tracking system\textsubscript{g} di Github\textsubscript{g} per tenere traccia delle issue\textsubscript{g}. 
Le issues\textsubscript{g} verranno determinate dal responsabile, ma la loro assegnazione verrà effettuata dai membri del gruppo, in base alla priorità delle issues\textsubscript{g}, i loro ruoli e alle loro disponibilità temporali.\\
Per marchiare le issues secondo criteri di interesse (come ambito o prioritá) vengono utilizzate le labels.
\\\\\textbf{Labels:}
\begin{itemize}
	\item \textbf{Da approvare}: indica una issue o pull request che necessita di approvazione;
	\item \textbf{Da verificare}: indica una issue o pull request che necessita di verifica;
	\item \textbf{Documentazione}: indica una issue o pull request riguardante la documentazione;
	\item \textbf{P=Bassa}: indica una issue o pull request a priorità bassa, con una scadenza di massimo 15 giorni;
	\item \textbf{P=Media}: indica una issue o pull request a priorità media, con una scadenza di massimo 7 giorni;
	\item \textbf{P=Alta}: indica una issue o pull request a priorità alta, con una scadenza di massimo 3 giorni;
	\item \textbf{Presentazione}: indica una issue riguardante la redazione di una presentazione in classe o al proponente.
\end{itemize}

Nel caso un membro del gruppo dovesse rendersi conto che l'issue\textsubscript{g} che sta svolgendo potrebbe essere suddiviso in ulteriori issue\textsubscript{g}, dovrà rivolgersi al responsabile, che è l'unico che può aggiungere, modificare o eliminare le issue\textsubscript{g}.\\
Per raggruppare più issue si marca ciascuna con la label di raggruppamento.
Il nome di una label di raggruppamento viene preceduto da una G maiuscola (che sta per gruppo) seguito dal nome dell'attività da svolgere. esempio: "G Piano di progetto"
Al completamento di tutte le issues di un gruppo si potrà scegliere se eliminare la label o tenerla per raggruppare issues future dello stesso gruppo.
indica un gruppo di labels relative al documento \textit{Piano di Progetto}.
\\\\
\textbf{Utilizzo delle checkbox:}
\\\\
Una issue può essere suddivisa in più attività tramite delle checkbox in base alla grandezza o
alla necessità di suddividere il lavoro. Al completamento di un'attività si spunta la checkbox
corrispondente in modo da avvisare il gruppo sullo stato di avanzamento di quella issue.

\subsubsection{Discord}
Strumento utilizzato per la comunicazione tra i componenti del gruppo. Sono creati diversi canali con diverse funzioni:
\begin{itemize}
	\item \textbf{Testuali}: si condividono informazioni in formato testuale;
	\item \textbf{Vocali}: si effettuano videochiamate brevi, che non necessitano di essere verbalizzate;
	\item \textbf{Risorse}: si condivide materiale da consultazione (ed esempio documenti, link, ecc...). 
\end{itemize}
I canali sono creati secondo le necessità del periodo, in accordo con tutti i componenti del gruppo.
Su Discord è anche possibile controllare in ogni momento quale ruolo è assegnato ad ogni membro del gruppo.

\subsubsection{Metriche}
Il processo\textsubscript{g} di gestione delle infrastrutture non utilizza metriche qualitative particolari.

\pagebreak
\section{Standard di qualità ISO/IEC 9126}
Questa sezione descrive in dettaglio lo standard ISO/IEC 9126 utilizzato dal gruppo. Le norme di questo standard descrivono un modello di qualità del software, definiscono le caratteristiche che lo determinano e propongono metriche per la misurazione.
Le norme constano di quattro parti:
\begin{itemize}
\item Parte 1: Metriche per la qualità esterna
\item Parte 2: Metriche per la qualità interna
\item Parte 3: Modello della qualità del software
\item Parte 4: Metriche per la qualità in uso
\end{itemize}

\subsection{Qualità esterne}
Le metriche esterne misurano i comportamenti del prodotto software rilevabili dai test, dall'operatività, dall'osservazione durante la sua esecuzione. L'esecuzione del prodotto software è fatta in base al suo utilizzo in funzione degli obiettivi di business stabiliti ed in un contesto organizzativo e tecnico rilevante.

\subsection{Qualità interne}
Le metriche interne si applicano alla parte non eseguibile del software (come le specifiche tecniche o il codice sorgente) durante le fasi di progettazione e codifica. Durante le fasi di sviluppo del software, quindi, i prodotti intermedi sono valutati tramite metriche interne che misurano le proprietà intrinseche del prodotto\textsubscript{g}. 
Le metriche interne permettono ad utenti e sviluppatori di individuare eventuali problemi che potrebbero inibire la qualità finale del prodotto\textsubscript{g}.
\subsection{Modello di qualità}
Il modello definisce le caratteristiche di qualità. A ciascuna di esse sono associate sotto-caratteristiche. La tabella successiva descrive le caratteristiche e le sue sotto-caratteristiche del software proposti dal modello. 
Il modello presenta le seguenti caratteristiche:

\subsubsection{Funzionalità}
Si intende quella capacità di un prodotto software di determinare le esigenze richieste tramite delle funzioni, sotto precise condizioni.
Le sue sotto-caratteristiche sono rispettivamente:
\begin{itemize}
\item Appropriatezza: rappresenta la capacità di fornire un appropriato insieme di funzioni che permettano agli utenti di svolgere determinate task e di raggiungere gli obiettivi prefissati;
\item Accuratezza: rappresenta la capacità del software di fornire i risultati o gli effetti attesi con il livello di precisione richiesta;
\item Interoperabilità: rappresenta la capacità del software di interagire con uno o più sistemi specificati;
\item Conformità: rappresenta la capacità del software di aderire a standard, convenzioni e regolamenti di carattere legale o prescrizioni simili che abbiano attinenza con la funzionalità;
\item Sicurezza: rappresenta la capacità del software di proteggere le informazioni ed i dati in modo che, persone o sistemi non autorizzati, non possano accedervi e quindi non possano leggerli o modificarli.
\end{itemize}

\subsubsection{Affidabilità}
Si intende quella capacità di un prodotto software di mantenere il livello di prestazione quando utilizzato sotto certe condizioni specifiche.
Le sue sotto-caratteristiche sono rispettivamente:
\begin{itemize}
\item Maturità: rappresenta la capacità del software di evitare che si verifichino errori o siano prodotti risultati non corretti in fase di esecuzione;
\item Tolleranza agli errori: rappresenta la capacità del software  di mantenere il livello di prestazioni in caso di errori nel software o di violazione delle interfacce specificate;
\item Recuperabilità: rappresenta la capacità del software di ripristinare il livello di prestazioni e di recuperare i dati direttamente coinvolti in caso di errori o malfunzionamenti;
\item Aderenza: rappresenta la capacità del software di aderire a standard, convenzioni e regole relative all'affidabilità.
\end{itemize}

\subsubsection{Efficienza}
Si intende quella capacità di un prodotto software di realizzare le funzioni richieste nel minor tempo possibile ed utilizzando nel miglior modo le risorse necessarie.
Le sue sotto-caratteristiche sono rispettivamente:
\begin{itemize}
\item Comportamento rispetto al tempo: rappresenta la capacità del software di fornire appropriati tempi di risposta, tempi di elaborazione e quantità di lavoro eseguendo le funzionalità previste sotto determinate condizioni di utilizzo;
\item Utilizzo delle risorse: rappresenta la capacità del software di utilizzare un appropriato numero e tipo di risorse in maniera adeguata;
\item Conformità: rappresenta la capacità del software di aderire a standard e convenzioni relative all'efficienza.
\end{itemize}

\subsubsection{Usabilità}
Si intende quella capacità di un prodotto software di essere comprensibile e appreso dall'utente, quando usato sotto condizioni specificate.
Le sue sotto-caratteristiche sono rispettivamente:
\begin{itemize}
\item Comprensibilità: rappresenta la capacità del software di permettere all'utente di capire le sue funzionalità e come poterla utilizzare con successo per svolgere particolari task in determinate condizioni di utilizzo;
\item Apprendibilità: rappresenta la capacità del software di permettere all'utente di imparare l'applicazione;
\item Operabilità: rappresenta la capacità del software di permettere all'utente di utilizzarlo e di controllarlo;
\item Attrattività: rappresenta la capacità del software di risultare "attraente" per l'utente;
\item Conformità: rappresenta la capacità del software di aderire a standard,convenzioni o regole relative all'usabilità.
\end{itemize}

\subsubsection{Manutenibilità}
Si intende quella capacità di un prodotto software di essere modificato, includendo correzioni, miglioramenti o adattamenti.
Le sue sotto-caratteristiche sono rispettivamente:
\begin{itemize}
\item Analizzabilità: rappresenta la capacità del software di poter effettuare la diagnosi sul software ed individuare le cause di errori o malfunzionamenti;
\item Modificabilità: rappresenta la capacità del software di consentire lo sviluppo di modifiche al software originale. Si tratta quindi di modifiche al codice o alla progettazione ed alla sua documentazione;
\item Stabilità: rappresenta la capacità del software di evitare effetti non desiderati a seguito di modifiche al software;
\item Testabilità: rappresenta la capacità del software di consentire la verifica e validazione del software modificato, cioè di eseguire i test.
\end{itemize}

\subsubsection{Portabilità}
Si intende quella capacità di un prodotto software di poter essere trasportato da un ambiente di lavoro ad un altro. 
Le sue sotto-caratteristiche sono rispettivamente:
\begin{itemize}
\item Adattabilità: rappresenta la capacità del software di essere adattato a differenti ambienti senza richiedere azioni specifiche diverse da quelle previste dal software per tali attività;
\item Installabilità: rappresenta la capacità del software di essere installato in un determinato ambiente;
\item Conformità: rappresenta la capacità del software di aderire a standard,convenzioni o regole relative alla portabilità;
\item Sostituibilità: rappresenta la capacità del software di sostituire un altro software specifico indipendente, per lo stesso scopo e nello stesso ambiente.
\end{itemize}
\subsection{Qualità in uso}
Le metriche delle qualità in uso misurano il grado con cui il prodotto software permette agli utenti di svolgere le proprie attività con efficacia, produttività, sicurezza e soddisfazione nel contesto operativo previsto.
\begin{itemize}
\item Efficacia: la capacità del software di permettere all'utente di svolgere le funzioni specificate con accuratezza e completezza;
\item Produttività: la capacità di far utilizzare all'utente un quantitativo adeguato di risorse in relazione al caso d'uso;
\item Soddisfazione: la capacità del software di soddisfare l'utente;
\item Sicurezza: la capacità del prodotto software di avere livelli accettabili di rischio per dati e persone.
\end{itemize}

\pagebreak
\section{Standard di qualità ISO/IEC 12207:1995}
Questa sezione descrive in dettaglio lo standard ISO/IEC 12207:1995 scelto dal gruppo per garantire la qualità dei processi del ciclo di vita di un software. I processi dello standard contengono attività e compiti che devono essere applicati durante l'acquisizione di un sistema software. Esso contiene tre tipologie di processi che sono: 
\begin{itemize}
\item Processi primari
\item Processi di supporto
\item Processi organizzativi
\end{itemize}

\subsection{Processi primari}
Questi processi sono essenziali in quanto un progetto è detto tale se e solo se è attivo almeno un processo\textsubscript{g} primario.
I processi di supporto definiti dallo standard sono i seguenti.
\subsubsection{Acquisizione}
Il processo\textsubscript{g} di acquisizione contiene le attività ed i compiti dell'acquirente.
Le attività del processo\textsubscript{g} sono le seguenti:
\begin{itemize}
\item Iniziazione
\item Preparazione della richiesta di proposta;
\item Preparazione e aggiornamento del contratto;
\item Monitoraggio dei fornitori;
\item Accettazione e completamento.
\end{itemize}

\subsubsection{Fornitura}
Il processo\textsubscript{g} di fornitura contiene le attività e le mansioni del fornitore. Il processo\textsubscript{g} può essere.
Le attività del processo\textsubscript{g} sono le seguenti:
\begin{itemize}
\item Iniziazione;
\item Preparazione alla risposta;
\item Contratto;
\item Pianificazione;
\item Esecuzione e controllo;
\item Revisione e valutazione;
\item Rilascio e completamento.
\end{itemize}

\subsubsection{Sviluppo}
Il processo\textsubscript{g} di sviluppo contiene le attività e i compiti dello sviluppatore.
Il processo\textsubscript{g} contiene le attività per l'analisi dei requisiti, la progettazione, la codifica, l'integrazione, il test, l'installazione e l'accettazione relative al sistema se previsto dal contratto.
Le attività del processo\textsubscript{g} sono le seguenti:
\begin{itemize}
\item Implementazione dei processi;
\item Analisi dei requisiti di sistema;
\item Progettazione architettura di sistema;
\item Analisi requisiti software;
\item Progettazione architettura software;
\item Progettazione dettagliata del software;
\item Codifica e test del software;
\item Integrazione software;
\item Test di qualificazione del software;
\item Integrazione del sistema;
\item Test di qualificazione del sistema; 
\item Installazione software;
\item Supporto per l'accettazione del software.
\end{itemize}

\subsubsection{Gestione operativa}
Il processo\textsubscript{g} di gestione operativa contiene le attività e i compiti dell'operatore. Il processo\textsubscript{g} riguarda il funzionamento del prodotto\textsubscript{g} software e il supporto operativo agli utenti. 
Le attività del processo\textsubscript{g} sono le seguenti:
\begin{itemize}
\item Implementazione dei processi;
\item Collaudo operativo;
\item Operazione di sistema;
\item Supporto all'utente.
\end{itemize}

\subsubsection{Manutenzione}
Il processo\textsubscript{g} di manutenzione contiene le attività e i compiti del manutentore. Questo processo\textsubscript{g} viene attivato quando il prodotto\textsubscript{g} software subisce modifiche al codice e alla documentazione associata a causa di un problema o necessità di miglioramento o adattamento.
Le attività del processo\textsubscript{g} sono le seguenti:
\begin{itemize}
\item Implementazione di processi;
\item Analisi dei problemi e delle modifiche;
\item Implementanzione della modifica;
\item Revisione/Accettazione della manuntenzione;
\item Migrazione;
\item Ritiro del software.
\end{itemize}



\subsection{Processi di supporto}
I processi di supporto aiutano le attività di tutti gli altri processi dell'organizzazione a garantire il successo e la qualità del progetto. Questi processi possono essere attivati da un processo primario o da un altro processo di supporto.
Le attività e i compiti in un processo di supporto sono di responsabilità dell'organizzazione che esegue tale processo. Questa organizzazione garantisce che il processo sia in atto e funzionante.
I processi di supporto definiti dallo standard sono i seguenti.
\subsubsection{Documentazione}
Il processo di documentazione è un processo per la registrazione delle informazioni prodotte da un processo o attività del ciclo di vita. Il processo contiene l'insieme delle attività che pianificano, progettano, sviluppano, producono, modificano, distribuiscono e mantengono quei documenti necessari a tutti gli interessati.
Le attività del processo sono le seguenti:
\begin{itemize}
\item Implementazione di processo;
\item Produzione;
\item Progettazione e sviluppo;
\item Manutenzione.
\end{itemize}
\subsubsection{Gestione della configurazione}
Il processo di gestione della configurazione è un processo di applicazione di procedure amministrative e tecniche durante tutto il ciclo di vita del software per mantenere l'integrità di tutti i componenti della configurazione e di renderli accessibili a chi ne ha diritto.
Le attività del processo sono le seguenti:
\begin{itemize}
\item Implementazione del processo;
\item Identificazione della configurazione;
\item Controllo della configurazione;
\item Valutazione dello stato di configurazione;
\item Gestione del rilascio e distribuzione.
\end{itemize}
\subsubsection{Accertamento della qualità}
Il processo di accertamento della qualità è un processo per fornire un'adeguata garanzia che i prodotti ed i processi software nel ciclo di vita del progetto siano conformi ai requisiti specificati e aderiscano ai piani stabiliti. Per essere imparziale, la garanzia della qualità deve avere libertà organizzativa e autorità da parte delle persone direttamente responsabili dello sviluppo del prodotto software o dell'esecuzione del processo nel progetto.
Le attività del processo sono le seguenti:
\begin{itemize}
\item Implementazione del processo;
\item Accertamento di prodotto;
\item Accertamento di processo;
\item Accertamento della qualità di sistema.
\end{itemize}
\subsubsection{Verifica}
Il processo di verifica è un processo per determinare se i prodotti software di un'attività soddisfano i requisiti o le condizioni loro imposti nelle attività precedenti.
Le attività del processo sono le seguenti:
\begin{itemize}
\item Implementanzione del processo;
\item Verifica.
\end{itemize}
\subsubsection{Validazione}
Il processo di validazione è un processo per determinare se i requisiti e il sistema finale soddisfano l'uso specifico previsto.
Le attività del processo sono le seguenti:
\begin{itemize}
\item Implementazione del processo;
\item Validazione.
\end{itemize}

\subsubsection{Revisione congiunta}
Il processo di revisione congiunta è un processo per valutare lo stato ed i prodotti di un'attività di un progetto in modo appropriato. Le revisioni congiunte avvengono sia a livello di gestione del progetto che a livello tecnico.
\begin{itemize}
\item Implementazione del processo;
\item Revisioni della gestione del progetto;
\item Revisioni tecniche.
\end{itemize}

\subsubsection{Audit}
Il processo di Audit ha lo scopo di determinare in maniera indipendente la conformità di prodotti e processi selezionati ai requisiti, piani e accordi.
Le attività del processo sono le seguenti:
\begin{itemize}
\item Implementazione del processo;
\item Audit.
\end{itemize}

\subsubsection{Risoluzione dei problemi}
Il Processo di risoluzione dei problemi è un processo per l'analisi e la risoluzione dei problemi, indipendentemente dalla loro natura o origine, che vengono scoperti durante l'esecuzione di processi di sviluppo, funzionamento, manutenzione o altri. L'obiettivo è fornire un mezzo tempestivo, responsabile e documentato per garantire che tutti i problemi scoperti siano analizzati e risolti e che le tendenze siano riconosciute.
\begin{itemize}
\item Implementazione del processo;
\item Risoluzione del problema.
\end{itemize}


\subsection{Processi organizzativi}
Le attività e i compiti in un processo organizzativo sono di responsabilità dell'organizzazione che utilizza il processo. L'organizzazione garantisce che il processo sia attivo e funzionante.
I processi organizzativi definiti nello standard sono i seguenti.

\subsubsection{Gestione organizzativa}
Il processo di gestione organizzativa ha lo scopo di organizzare e monitorare i processi per il raggiungimento dei loro obiettivi. Il processo è stabilito da una organizzazione per assicurare la consistente applicazione di pratiche per l'uso dall'organizzazione e nei progetti.
Le attività del processo sono le seguenti:
\begin{itemize}
\item Inizio e definizione dello scopo
\item Pianificazione 
\item Esecuzione e controllo
\item Revisione e valutazione
\item Chiusura
\end{itemize}
\subsubsection{Gestione delle infrastrutture}
Il processo di gestione delle infrastrutture è un processo per stabilire e mantenere l'infrastruttura necessaria per qualsiasi altro processo. L'infrastruttura può includere hardware, software, strumenti, tecniche, standard e strutture per lo sviluppo, il funzionamento o la manutenzione.
Le attività del processo sono le seguenti:
\begin{itemize}
\item Implementazione del processo
\item Realizzazione dell'infrastruttura
\item Manutenzione dell'infrastruttura
\end{itemize}

\subsubsection{Miglioramento del processo}
Il processo di miglioramento è un processo per stabilire, valutare, misurare, controllare e migliorare un processo del ciclo di vita del software.
Le attività del processo sono le seguenti:
\begin{itemize}
\item Stabilimento del processo
\item Valutazione del processo
\item Miglioramento del processo
\end{itemize}

\subsubsection{Formazione del personale}
Il processo di formazione del personale è un processo per fornire e mantenere personale qualificato. L'acquisizione, la fornitura, lo sviluppo, il funzionamento o la manutenzione di prodotti software dipende in gran parte da personale esperto e qualificato.
Le attività del processo sono le seguenti:
\begin{itemize}
\item Implementazione del processo
\item Sviluppo materiale didattico
\item Realizzazione piano formativo
\end{itemize}
\pagebreak

\end{document}
