\documentclass[a4paper]{article}
\usepackage[normalem]{ulem}

% impostazioni generali
%Tutti gli usepackage vanno qui
\usepackage{geometry}
\usepackage[italian]{babel}
\usepackage[utf8]{inputenc}
\usepackage{tabularx}
\usepackage{longtable}
\usepackage{hyperref}
\usepackage{enumitem}
\usepackage{array} 
\usepackage{booktabs}
\newcolumntype{M}[1]{>{\centering\arraybackslash}m{#1}}
\usepackage[toc]{appendix}

\hypersetup{
	colorlinks=true,
	linkcolor=blue,
	filecolor=magenta,
	urlcolor=blue,
}
% Numerazione figure
\let\counterwithout\relax
\let\counterwithin\relax
\usepackage{chngcntr}

% distanziare elenco delle figure e delle tabelle
\usepackage{tocbasic}
\DeclareTOCStyleEntry[numwidth=3.5em]{tocline}{figure}% for figure entries
\DeclareTOCStyleEntry[numwidth=3.5em]{tocline}{table}% for table entries


%\counterwithout{table}{section}
%\counterwithout{figure}{section}
\captionsetup[table]{font=small,skip=5pt} 

\usepackage[bottom]{footmisc}
\usepackage{fancyhdr}
\setcounter{secnumdepth}{4}
\usepackage{amsmath, amssymb}
\usepackage{array}
\usepackage{graphicx}

\usepackage{ifthen}

\usepackage{float}
\restylefloat{table}

\usepackage{layouts}
\usepackage{url}
\usepackage{comment}
\usepackage{eurosym}

\usepackage{lastpage}
\usepackage{layouts}
\usepackage{eurosym}

\geometry{a4paper,top=3cm,bottom=4cm,left=2.5cm,right=2.5cm}

%Comandi di impaginazione uguale per tutti i documenti
\pagestyle{fancy}
\lhead{\includegraphics[scale=0.1]{../../../template/images/logo_no_motto.jpeg}}
%Titolo del documento
\rhead{\doctitle{}}
%\rfoot{\thepage}
\cfoot{Pagina \thepage\ di \pageref{LastPage}}
\setlength{\headheight}{35pt}
\setcounter{tocdepth}{5}
\setcounter{secnumdepth}{5}
\renewcommand{\footrulewidth}{0.4pt}

% multirow per tabelle
\usepackage{multirow}

% Permette tabelle su più pagine
%\usepackage{longtable}


% colore di sfondo per le celle
\usepackage[table]{xcolor}

%COMANDI TABELLE
\newcommand{\rowcolorhead}{\rowcolor[HTML]{007c95}}
\newcommand{\captionline}{\rowcolor[HTML]{FFFFFF}} %comando per le caption delle tabelle
\newcommand{\cellcolorhead}{\cellcolor[HTML]{007c95}}
\newcommand{\hlinetable}{\arrayrulecolor[HTML]{007c95}\hline}

%intestazione
% check for missing commands
\newcommand{\headertitle}[1]{\textbf{\color{white}#1}} %titolo colonna
\definecolor{pari}{HTML}{b1dae3}
\definecolor{dispari}{HTML}{d7f2f7}

% comandi glossario
\newcommand{\glo}{$_{G}$}
\newcommand{\glosp}{$_{G}$ }


%label custom
\makeatletter
\newcommand{\uclabel}[2]{%
	\protected@write \@auxout {}{\string \newlabel {#1}{{#2}{\thepage}{#2}{#1}{}} }%
	\hypertarget{#1}{#2}
}
\makeatother

%riportare pezzi di codice
\definecolor{codegray}{gray}{0.9}
\newcommand{\code}[1]{\colorbox{codegray}{\texttt{#1}}}



% dati relativi alla prima pagina
% Configurazione della pagina iniziale
\newcommand{\doctitle}{Glossario}
\newcommand{\docdate}{21 Marzo 2023}
\newcommand{\rev}{2.0.0}
\newcommand{\stato}{Approvato}
\newcommand{\uso}{Esterno}
\newcommand{\approv}{Mirko Stella}
\newcommand{\red}{Giacomo Mason \\ & Mirko Stella}
\newcommand{\ver}{Giacomo Mason \\ & Gabriele Mantoan}
\newcommand{\dest}{\textit{Seven Clickers}
									  \\ Prof. Vardanega Tullio 
									   \\ Prof. Cardin Riccardo}
\newcommand{\describedoc}{Glossario del gruppo \textit{Seven Clickers}}


 % editare questo

\makeindex

\begin{document}
\counterwithin{table}{section}

% Prima pagina
\thispagestyle{empty}
\renewcommand{\arraystretch}{1.3}

\begin{titlepage}
	\begin{center}
		
	\includegraphics[scale = 0.40]{../../../template/images/logo.jpeg}
	\\[1cm]
	\href{mailto:7clickersgroup@gmail.com}		      	
	{\large{\textit{7clickersgroup@gmail.com} } }\\[2.5cm]
	\Huge \textbf{\doctitle} \\[1cm]
	 \large
			 \begin{tabular}{r|l}
                        \textbf{Versione} & \rev{} \\
                        \textbf{Stato} & \stato{} \\
                        \textbf{Uso} & \uso{} \\                         
                        \textbf{Approvazione\textsubscript{g}} & \approv{} \\                      
                        \textbf{Redazione} & \red{} \\ 
                        \textbf{Verifica\textsubscript{g}} &  \ver{} \\                         
                        \textbf{Distribuzione} & \parbox[t]{5cm}{ \dest{} }
                \end{tabular} 
                \\[3.3cm]
                \large \textbf{Descrizione} \\ \describedoc{} 
     \end{center}
\end{titlepage}

% Diario delle modifiche
\section*{Registro delle modifiche}

\newcommand{\changelogTable}[1]{
	 

\renewcommand{\arraystretch}{1.5}
\rowcolors{2}{pari}{dispari}
\begin{longtable}{ 
		>{\centering}M{0.07\textwidth} 
		>{\centering}M{0.13\textwidth}
		>{\centering}M{0.20\textwidth}
		>{\centering}M{0.17\textwidth} 
		>{\centering\arraybackslash}M{0.30\textwidth} 
		 }
	\rowcolorhead
	\headertitle{Vers.} &
	\centering \headertitle{Data} &	
	\headertitle{Autore} &
	\headertitle{Ruolo} & 
	\headertitle{Descrizione} 
	\endfirsthead	
	\endhead
	
	#1

\end{longtable}
\vspace{-2em}

}



\changelogTable{
	0.1.0 & 07-01-23 & Marco Brigo \\ Elena Pandolfo & Verificatori & Verifica\textsubscript{g} documento \\
	0.0.1 & 05-01-23 & Mirko Stella & Responsabile & Stesura documento \\
} % editare questo
\pagebreak

% Indice
{
    \hypersetup{linkcolor=black}
    \tableofcontents
}
\pagebreak

% Contenuto
\section{Introduzione}
\subsection{Scopo del documento}
Lo scopo del documento è quello di stabilire le regole che ogni componente del gruppo SevenClickers
deve rispettare per mantenere un ambiente di lavoro che mira a massimizzare
l'economicità dei processi durante il ciclo di vita del prodotto ShowRoom3D.\\
Le norme verranno inserite in modo incrementale per regolamentare le attività di progetto imminenti rimandando quelle meno urgenti a quando 
se ne presenterà la necessità.\\
Inoltre tali norme potranno subire modifiche nel tempo in modo da garantire un miglioramento continuo della qualità del lavoro svolto.
Il responsabile di progetto ha il compito di comunicare l'aggiunta di una nuova norma o la modifica di una già esistente a tutti i componenti
del gruppo e di assicurarsi che siano comprese a pieno.
\subsection{Scopo del prodotto}
Il prodotto in questione nasce dalla necessità dell'azienda SanMarco Informatica di fornire una soluzione agli sprechi
derivati dall'adozione di uno ShowRoom tradizionale proponendo uno ShowRoom3D che sia 
ugualmente o ancora più immersivo.
\pagebreak

\subsection{Documentazione}
\subsubsection{Scopo}
Il processo\textsubscript{g} di documentazione ha come scopo quello di fornire un insieme di documenti che descrivono in dettaglio il progetto software, comprese le sue funzionalità, i requisiti, l'architettura, il design, i processi di sviluppo e i risultati. 
Questa documentazione serve come riferimento per lo sviluppo del software e aiuta a garantire la coerenza e la completezza del progetto. 
La documentazione del progetto software è un elemento essenziale e deve essere costantemente aggiornata.
\subsubsection{Descrizione}
Nella seguente sezione sono raggruppate tutte le norme necessarie alla stesura, alla verifica\textsubscript{g} e 
all’approvazione\textsubscript{g} della documentazione, in modo tale da mantenere una coerenza nella forma e nella 
struttura dei documenti prodotti dal gruppo.
\subsubsection{Ciclo di vita di un documento}
\begin{itemize}
    \item \textbf{Creazione}: il documento viene creato su un nuovo branch, utilizzando un template uguale per tutti i documenti. 
    Viene aggiunta l’intestazione e il registro delle modifiche;
    \item \textbf{Stesura}: si procede con la stesura delle varie sezioni, tracciando i cambiamenti nel registro delle modifiche; 
    \item \textbf{Verifica}: ogni sezione viene verificata dai verificatori (per maggiori specifiche guardare la 
    sezione x.x.x Verifica della documentazione);
    \item \textbf{Approvazione}: una volta che tutte le sezioni sono state verificate si procede 
    con l’approvazione del documento, che viene effettuata dal responsabile nel seguente modo:
    \begin{itemize}
        \item Viene aperta una Pull Request di approvazione;
        \item Il responsabile rilegge il documento: se riscontra ulteriori problematiche, segnala ai verificatori le eventuali modifiche da apportare;
        \item Se sono richieste delle modifiche i verificatori si occupano di apportarle;
        \item Il responsabile crea una nuova riga e compila i campi nelle colonne corrispondenti del registro delle modifiche 
        inserendo: l'ultima versione secondo le norme di versionamento, la data, il proprio nome, il suo ruolo e la voce "Approvazione" nell'ultima colonna.
    \end{itemize}
    Per i \textit{Verbali} si effettua questa pratica non appena il file prodotto viene 
    verificato, mentre per tutti gli altri documenti prima di una consegna.
\end{itemize}
\subsubsection{Struttura generale}
Ogni documento deve presentare le seguenti sezioni nell'ordine in cui vengono presentate:
\begin{itemize} 
    \item \textbf{Intestazione}:
    contiene:
    \begin{itemize} 
        \item Logo compreso di motto;
        \item Indirizzo email di gruppo; 
        \item Titolo;
        \item Tabella contenente le informazioni generali:
        \begin{itemize}
            \item Versione;
            \item Stato;
            \item Uso;
            \item Approvazione\textsubscript{g}: indica il responsabile di progetto che ha approvato il documento; 
            \item Redazione: elenco dei collaboratori che hanno partecipato alla stesura del documento;
            \item Verifica\textsubscript{g}: elenco dei verificatori che hanno verificato il documento;
            \item Distribuzione: elenco delle persone o organizzazioni a cui è destinato il documento.
        \end{itemize}
        \item Breve descrizione del documento .
    \end{itemize}
    \item \textbf{Registro delle modifiche}:
    tabella che identifica ogni versione del documento indicandone:
    \begin{itemize} 
        \item Versione;
        \item Data;
        \item Autore;
        \item Ruolo;
        \item Descrizione: la descrizione deve essere breve. Nel caso di aggiunte o modifiche si deve indicare 
        il nome della sezione che è stata aggiunta o modificata. 
        I nomi delle sezioni vanno riportati uguali a come sono scritti nell'indice, 
        senza virgolette.
    \end{itemize}
    \item \textbf{Indice}:
    elenco ordinato dei titoli dei capitoli, ovvero delle varie parti di cui si compone il documento;
    
    \item \textbf{Elenco delle figure}: sezione che fornisce l'elenco delle immagini presenti nel documento. 
    Se il documento non presenta immagini, questa sezione sarà omessa di conseguenza;  
    \item \textbf{Elenco delle tabelle}: sezione che fornisce l'elenco delle tabelle presenti nel documento. 
    Se il documento non presenta tabelle, questa sezione sarà omessa di conseguenza; 
    \item \textbf{Contenuto}:
    varia a seconda del tipo di documento.
\end{itemize}
\subsubsection{Convenzioni}
Le convenzioni di seguito riportate vengono applicate a tutti i documenti.
Esse rendono i documenti stilati omogenei tra loro contribuendo a rendere il progetto professionale.

\paragraph{Date}
Le date devono rispettare il seguente formato: \textbf{yyyy-mm-dd}
All'interno delle tabelle il formato deve essere il seguente: \textbf{dd-mm-yy}
\paragraph{Nomi di persona}
All'interno dei documenti i nomi di persona rispetteranno l'ordine nome seguito dal cognome della persona menzionata.

\paragraph{Elenchi puntati}
Gli elenchi puntati devono rispettare le seguenti regole:
\begin{itemize} 
    \item Ogni elemento dell'elenco deve iniziare con la lettera maiuscola;
    \item Ogni elemento dell'elenco deve terminare con ";" ad eccezione dell'ultimo elemento
    che deve terminare con "."; 
    \item Dopo i due punti la frase deve iniziare con la lettera minuscola.
\end{itemize}

\paragraph{Stile del testo}
\begin{itemize} 
    \item \textbf{Grassetto}: stile utilizzato per i titoli delle sezioni e per i primi termini degli elenchi puntati;
    \item \textbf{Corsivo}: viene utilizzato per citare il nome dei documenti, ad esempio \textit{Piano di Progetto}. 
\end{itemize}

\paragraph{Immagini}
Le immagini sono raccolte nella cartella “images”, e sono inserite sempre con una didascalia descrittiva posizionata sotto l’immagine.

\paragraph{Tabelle} 
Le tabelle sono provviste di didascalia descrittiva posizionata sotto alla tabella. La tabella contenente il registro delle modifiche è l’unica che fa da eccezione a questa regola.

\paragraph{\textit{Glossario}}
All'interno della documentazione si possono trovare dei termini che possono risultare ambigui a seconda del contesto, o non conosciuti dagli utilizzatori.\\\
Per ovviare ad incomprensioni si è deciso di stilare un elenco di termini di interesse accompagnati da una descrizione del loro significato.\\
I termini presenti all'interno dei documenti che necessitano di una descrizione vengono indicati con il pedice 'g' come nell'esempio seguente: termine.
É quindi possibile consultare il \textit{Glossario} per reperire tale descrizione.
\\
Ogni componente del gruppo all'inserimento di un termine ritenuto ambiguo deve preoccuparsi di aggiornare il \textit{Glossario}.

Per aggiornare il \textit{Glossario} si devono inserire i nuovi termini nel file .tex nella cartella corrispondente all'iniziale del termine situata al percorso
\textbf{\textit{latex\textsubscript{g}/esterni/doc\_esterna/\textit{Glossario}/res/sections/alphabet/}}.
É stato creato uno script che scansiona un documento di interesse per inserire in automatico il pedice sui termini contenuti nel \textit{Glossario}.

Il componente del gruppo che inserisce all'interno del \textit{Glossario} un nuovo termine deve aggiungere nel file .tex il segnaposto \%parola\% dopo la subsection (senza spazi) che racchiude il termine per permetterne il riconoscimento da parte dello script.
\\
Il \textit{Glossario} ordina i termini in ordine alfabetico in modo da permetterne una facile e veloce ricerca.
\subsubsection{Strumenti per la stesura}
\begin{itemize} 
    \item \textbf{LaTeX}: è un linguaggio di marcatura per la preparazione di testi, basato sul 				  		  programma di composizione tipografica TEX.\\
	Nel branch documentation  si possono trovare i file .pdf prodotti e la cartella “latex”. La cartella latex contiene tre cartelle interne:
	\begin{itemize}
	\item \textbf{esterni e interni}: contengono file .tex di documentazione esterna ed interna come ad esempio i \textit{Verbali} o altra documentazione esterna/interna:
	\begin{itemize}
		\item La cartella config contiene i file .tex con le parti fisse dei documenti (intestazione, registro delle modifiche, tracciamento dei temi affrontati) che vengono modificati con i dati del documento specifico;
		\item La cartella res/sections contiene i file .tex con il contenuto vero e proprio (sezioni del documento) che viene redatto in maniera libera dal redattore;
		\item Un file col nome del documento pdf con estensione .tex che viene compilato per produrre il file pdf.
	\end{itemize}
	 \item \textbf{template}: contiene file .tex di base utilizzati secondo necessità per comporre i documenti:
	 \begin{itemize}
	 \item changelox.tex è il file di template che serve per scrivere il registro delle modifiche;
	 \item package.tex è il file che contiene tutti gli usepackage\textsubscript{g};
	 \item titlepage.tex è il file di template che contiene la configurazione della pagina iniziale di ogni documento;  
	 \item tracking.tex è il file di template che contiene il tracciamento dei temi affrontati nel documento.
	 \end{itemize}
	\end{itemize}
\end{itemize}

\subsubsection{Documentazione interna}
La documentazione interna comprende tutti i documenti che contengono informazioni utili principalmente per il gruppo, e che vengono quindi consultati di conseguenza. 
    \paragraph{\textit{Diario di Bordo}}
Il \textit{Diario di Bordo} è un documento informale ad uso esterno e permette di interagire con il Professore (committente) in modo da aggiornarlo sullo stato di avanzamento del progetto settimanalmente ed è usato 
anche per chiedere chiarimenti sui temi in cui riscontriamo dubbi o domande.
Fino al termine delle lezioni il \textit{Diario di Bordo} veniva redatto settimanalmente dal responsibile che presentava in classe prima dello svolgimento 
della lezione.
Dal termine delle lezioni il \textit{Diario di Bordo} viene redatto settimanalmente dal responsabile e caricato in una cartella denominata diari\_di\_bordo presente nel drive di gruppo.
\\\\
La cartella diari\_di\_bordo si trova al link: \href{https://drive.google.com/drive/u/1/folders/1a0kAZuOSsDEp_AaQUb6OQJ4XQyZ--RCN}{\\https:\/\/drive.google.com\/drive\/u\/1\/folders\/1a0kAZuOSsDEp\_AaQUb6OQJ4XQyZ\-\-RCN}
\\\\
È stato fornito l'accesso in lettura alla cartella al Professore che viene notificato tramite mail (deve contenere il link al \textit{Diario di Bordo} di interesse) al caricamento di un \textit{Diario di Bordo}.
Il formato dei diari di bordo condivisi nella cartella è .pdf.
Inoltre tutti i diari di bordo una volta redatti vengono inviati anche al canale discord\textsubscript{g} "diari-di-bordo-file" in modo da avere una duplice copia a fronte di 
qualsiasi imprevisto dato che i diari di bordo non vengono inseriti all'interno del repository di gruppo.
\\\\
\textbf{Struttura \textit{Diario di Bordo}} 
\\\\
La struttura del \textit{Diario di Bordo} non rispetta quella indicata nella sezione 3.1.1.2 relativa alla struttura generale della documentazione.
Sono state fissate delle regole per la stesura dei diari di bordo per fornire delle presentazioni il più possibile simili tra loro 
senza troppi vincoli superflui per tali documenti.
\\\\
Il \textbf{font} utilizzato per la stesura del documento è Helvetica Neue con dimensione 22pt.
Il nome del file per ogni diario di bordo deve rispettare la codifica \textbf{AAAA/MM/GG} in modo da permettere l'ordinamento
cronologico utilizzando i filtri di ricerca di Google Drive.
\\\\
Il \textit{Diario di Bordo} si divide in 4 slides che possono essere raggruppate in 3 se l'elenco degli obiettivi raggiunti e quelli futuri ci stanno
in una singola slide.
Le slides che compongono il \textit{Diario di Bordo} sono:
\begin{itemize}
    \item Intestazione; 
    \item Obiettivi raggiunti;
    \item Obiettivi futuri;
    \item Domande.
\end{itemize}
Descrizione slides:
\begin{itemize}
    \item \textbf{Intestazione}: presenta nella parte centrale il logo del gruppo compreso di slogan, in basso centrale la data relativa al \textit{Diario di Bordo} e l'anno accademico;
    \item \textbf{Obiettivi raggiunti}: presenta in alto centrale il titolo "Obiettivi raggiunti", a sinistra un elenco puntato con gli obiettivi raggiunti
    della settimana, in alto a destra il logo del gruppo compreso di slogan ed in basso a destra la data del \textit{Diario di Bordo};
    \item \textbf{Obiettivi futuri}: presenta in alto centrale il titolo "Obiettivi futuri", a sinistra un elenco puntato con gli obiettivi raggiunti
    della settimana ed in alto a destra il logo del gruppo compreso di slogan;
    \item \textbf{Domande}: presenta in alto centrale il titolo "Domande", a sinistra un elenco puntato con le domande da porre al Professore, in alto a destra il logo del gruppo compreso di slogan ed in basso 
    a destra la data del \textit{Diario di Bordo}.
\end{itemize}
    \paragraph{\textit{Verbale}}
Il \textit{Verbale} ha lo scopo di rendicontare ciò che viene detto durante una riunione.
Rispetta la struttura generale.
In aggiunta presenta:
\begin{itemize} 
    \item \textbf{Informazioni Generali}
    contiene:
    \begin{itemize} 
        \item Luogo;
        \item Data;
        \item Ora;
        \item Partecipanti.
    \end{itemize}
\item \textbf{Tabella tracciamento temi affrontati}:
tabella che riassume i punti salienti della riunione indicandone:
    \begin{itemize} 
        \item Codice: ha il formato V\textbf{X} \textbf{Y}.\textbf{Z} dove X indica la tipologia di \textit{Verbale}, Y indica il numero di \textit{Verbale} (incrementale rispetto agli altri verbali);
        e Z indica il numero dell'argomento trattato (incrementale rispetto agli altri argomenti del \textit{Verbale});
        \item Descrizione: breve descrizione di uno specifico argomento trattato.
    \end{itemize}
\end{itemize}
    \paragraph{\textit{Studio di Fattibilità}}
Lo \textit{Studio di Fattibilità} ha lo scopo di valutare i pro ed i contro dei capitolati proposti in modo da scegliere il più vantaggioso al quale candidarsi.
La scelta del capitolato viene fatta sulla base di diverse considerazioni.
Vengono valutati:
\begin{itemize}
    \item \textbf{Risorse}: in termini di budget, tempo e personale;
    \item \textbf{Previe Conoscenze}: un capitolato potrebbe rivelarsi più o meno difficile da realizzare a seconda delle conoscenze del contesto applicativo; 
    \item \textbf{Guadagno}: un capitolato potrebbe venire scartato per scarso guadagno netto al termine del lavoro commissionato (non è il nostro caso ma 
andrebbe tenuto in considerazione in ambito lavorativo);
    \item \textbf{Aspetti psicologici}: avendo la possibilità di scegliere tra capitolati (fatto che capita raramente in un reale ambito lavorativo) di pari interesse siamo portati a scegliere il contesto applicativo che ci entusiasma maggiormente.
    Questo aspetto potrebbe avere un impatto positivo sulla produttività del gruppo.
\end{itemize}

Il documento tiene in considerazione tutti i capitolati proposti per non scartare un capitolato per le sensazioni soggettive dei componenti del gruppo.
Non tenendo in considerazione ogni capitolato, infatti, si potrebbe scartare un capitolato di forte interesse senza rendersene conto.
    \documentclass[a4paper]{article}
\usepackage[normalem]{ulem}
\usepackage[font=small,labelfont=bf]{caption}

% impostazioni generali
\input{../../../template/package.tex}

% dati relativi alla prima pagina
\input{config/titlepage_input.tex} % editare questo

\makeindex

%comando per far andare a capo i paragrafi
\makeatletter
\renewcommand\paragraph{
\@startsection {paragraph}{4}{0mm}{-\baselineskip}{.5\baselineskip}{\normalfont \normalsize \bfseries }}
\makeatother

\newcommand*{\fullref}[1]{\hyperref[{#1}]{\ref*{#1} \nameref*{#1}}}

\begin{document}
\counterwithin{table}{section}

% Prima pagina
\input{../../../template/titlepage.tex}

% Diario delle modifiche
\input{../../../template/changelog.tex}
\input{config/changelog_input.tex} % editare questo
\pagebreak

% Indice
{
    \hypersetup{linkcolor=black}
    \tableofcontents
    \listoffigures %elenco figure
}
\pagebreak

% Contenuto
\section{Introduzione}
\subsection{Scopo del documento}
\textit{Norme di Progetto} è il documento che definisce le regole e gli standard che devono essere seguiti durante il ciclo di vita del prodotto.
\subsection{Scopo del prodotto}
Il prodotto in questione nasce dalla necessità dell'azienda SanMarco Informatica di fornire una soluzione agli sprechi
derivati dall'adozione di uno ShowRoom tradizionale proponendo uno ShowRoom3D che sia ugualmente o ancora più immersivo.
\section{Processi Primari}
I processi primari devono essere seguiti durante il ciclo di vita del software.
Nel nostro il ciclo di vita del software non comprende l'installazione e la manutenzione ma termina con lo sviluppo.
Il progetto in questione è da considerarsi un progetto didattico.
\input{res/sections/processi_primari/processi_primari_main.tex}
\pagebreak
\section{Processi di Supporto}
I processi di supporto supportano quelli primari in modo da renderli più efficienti ed efficaci.
\input{res/sections/processi_supporto/processi_supporto_main.tex}
\pagebreak
\section{Processi Organizzativi}
I processi organizzativi mirano a gestire i processi e il loro miglioramento, l'organizzazione degli strumenti 
di supporto e la gestione del personale.
\input{res/sections/processi_organizzativi/processi_organizzativi_main.tex}
\pagebreak
\input{res/sections/isoiec_9126/isoiec_9126_main.tex}
\pagebreak
\input{res/sections/isoiec_12207_1995/isoiec_12207_1995_main.tex}
\pagebreak
\input{res/sections/metriche/metriche_main.tex}
\pagebreak

\end{document}

    \subsubsection{Documentazione esterna}
La documentazione esterna comprende tutti i documenti che interessano anche al proponente e al committente.
    \documentclass[10pt]{article}

\usepackage{geometry}
\usepackage{fancyhdr,graphicx}
\usepackage{hyperref}
\usepackage{eurosym}

\geometry{a4paper,top=2.5cm,bottom=2.5cm,left=2cm,right=2cm}

\fancypagestyle{firstpage}{%
  \fancyhf{}% Clear header/footer
  \renewcommand{\headrulewidth}{0pt}%
}

\fancypagestyle{otherpages}{%
  \fancyhf{}% Clear header/footer
  \renewcommand{\headrulewidth}{1pt}%
}

\AtBeginDocument{\thispagestyle{firstpage}}
\pagestyle{otherpages}

\setlength{\parindent}{0pt}
\setlength{\parskip}{1ex}

\begin{document}

\noindent\begin{minipage}{0.5\textwidth}% adapt widths of minipages to your needs
\includegraphics[width=9cm]{images/logo.jpeg}
\end{minipage}%
\hfill%
\begin{minipage}{4cm}
\includegraphics[width=2.3cm]{images/uni.png}
\end{minipage}

\bigskip\bigskip

\begin{tabular}{ @{} l  }
  Gruppo \textit{Seven Clickers} \\ 
  E-mail: \textit{\href{mailto:7clickersgroup@gmail.com}{7clickersgroup@gmail.com} }\\ 
  Corso di Ingegneria del Software AA 2022/2023 \\
  17 Marzo 2023
\end{tabular}

\bigskip
\hfill
\begin{tabular}{ l @{} }
Prof. Vardanega Tullio\\
Prof. Cardin Riccardo\\
Università degli Studi di Padova\\
Dipartimento di Matematica\\
Via Trieste, 63\\
35121 Padova
\end{tabular}

\bigskip

Egregio Prof. Vardanega Tullio,\\
Egregio Prof. Cardin Riccardo,\\

\bigskip

Con la presente il gruppo \textit{Seven Clickers} intende comunicarVi la partecipazione al secondo passaggio della revisione di avanzamento RTB,
al fine di esporvi l’avanzamento dello sviluppo del progetto, denominato:

\begin{center}
  \textbf{ShowRoom3D}
\end{center}

proposto dall’azienda \textbf{Sanmarco Informatica}.


Si allegano i seguenti documenti di interesse:
\begin{itemize}
  \item \textit{\textit{Studio di Fattibilità}}
  \item \textit{Glossario} vx.x.x
  \item \textit{Analisi dei Requisiti} v2.0.0
  \item \textit{Norme di Progetto} v1.0.0
  \item \textit{Piano di Progetto} v2.0.0
  \item \textit{Piano di Qualifica} v1.0.0
\end{itemize}

Inoltre sono allegati anche i verbali esterni ed interni:

\begin{itemize}
  \item \textit{Verbale} esterno del 25-10-2022
  \item \textit{Verbale} esterno del 17-11-2022
  \item \textit{Verbale} esterno del 11-01-2023
  \item \textit{Verbale} esterno del 18-01-2023
  \item \textit{Verbale} esterno del 17-02-2023
  \item \textit{Verbale} interno del 19-10-2022
  \item \textit{Verbale} interno del 25-10-2022
  \item \textit{Verbale} interno del 26-10-2022
  \item \textit{Verbale} interno del 04-11-2022
  \item \textit{Verbale} interno del 09-11-2022
  \item \textit{Verbale} interno del 16-11-2022
  \item \textit{Verbale} interno del 23-11-2022
  \item \textit{Verbale} interno del 01-12-2022
  \item \textit{Verbale} interno del 07-12-2022
  \item \textit{Verbale} interno del 14-12-2022
  \item \textit{Verbale} interno del 04-01-2023
  \item \textit{Verbale} interno del 25-01-2023
  \item \textit{Verbale} interno del 01-02-2023
  \item \textit{Verbale} interno del 08-02-2023
  \item \textit{Verbale} interno del 24-02-2023
  \item \textit{Verbale} interno del 28-02-2023
\end{itemize}

Il gruppo stima di consegnare il prodotto\textsubscript{g} entro il 03-05-2023 con un preventivo di 13975\euro{}  come
specificato nel documento \textit{Piano di Progetto} v2.0.0.


Cordiali saluti,

\vspace{15pt}

\hfill
\begin{tabular}{ l @{} }
Rino Sincic\\
\textit{Responsabile di Progetto}\\
\includegraphics[width=2.3cm]{images/Rino_Sincic_firma.png}
\end{tabular}

\end{document}
    \paragraph{\textit{Piano di Progetto}}

Il documento piano di progetto ha lo scopo di aiutare il gruppo nella gestione delle risorse a disposizione per portare a termine il progetto entro la data decisa.
Il documento ha anche la funzione di monitorare l'avanzamento del progetto in modo da poter applicare miglioramenti continui e azioni correttive
basandosi sull'esperienza ottenuta da pianificazioni precedenti.
\\\\
\textbf{Struttura piano di progetto}
\\\\
La struttura del piano di progetto segue la struttura generale.
In aggiunta il contenuto del documento si compone di:
\begin{enumerate}
    \item Analisi dei rischi;
    \item Pianificazione;
    \item Preventivo.
\end{enumerate}
\textbf{NOTA:} ogni tabella o immagine all'interno del documento verrá indicata anche nell'indice del documento stesso in modo che sia rintracciabile con 
facilitá.
\\\\
\textbf{Descrizione delle sezioni:}
\\\\
\textbf{Analisi dei rischi:} vengono riportati in forma tabellare i rischi a cui si va incontro aggiudicandosi il capitolato.
Ogni rischio fa riferimento ad una categoria di rischi precisa e viene indicato con un nome,una probabilitá che si verifichio,
un grado che indica l'impatto negativo che puó
comportare es una breve descrizione su come affrontare il rischio nel momento in cui dovesse presentarsi.
\\\\
\textbf{Pianificazione:} la sezione dedicata alle pianificazioni ha lo scopo di indicare l'inizio e la fine di un periodo di pianificazione e 
di fornire la suddivisione e l'organizzazione delle attivitá all'interno di tale periodo.
Viene fornito un elenco con descrizione delle attivitá che si andranno a svolgere e viene diviso il periodo di pianificazione in sottoperiodi 
ciascuno con un inizio,una fine ed indicando cosa verrá fatto.
Infine é presente un diagramma di Gantt che illustra graficamente l'ordine temporale delle attivitá da svolgere tenendo conto di 
eventuali margini temporali dovuti ad imprevisti vari.
\\\\
\textbf{Preventivo:} include una sezione iniziale in cui si vanno ad indicare in forma tabellare le risore umane disponibili all'inizio del progetto e 
come queste risorse andranno suddivise. Successivamente viene fornito,sempre in forma tabellare,il costo totale calcolato in base alle ore 
di impegno dei componenti del gruppo ed al ruolo che dovranno ricoprire (ogni ruolo ha un costo orario).
Le tabelle riportate produrranno come risultato la conclusione del preventivo che comprende il costo totale calcolato e la data di fine progetto.
É presente una sezione (dettaglio periodi) che prevede i preventivi di ciascun periodo di pianificazione.
In questa sezione sono fornite le tabelle con le ore che ciascun componente del gruppo dovrá svolgere con relativi ruoli associati ed una tabella con i costi
derivati da tali ruoli.
Nel caso del nostro progetto viene prevista una rotazione dei ruoli a scopo didattico perció ogni componente userá le sue ore di impegno in modo diverso 
a seconda della rotazione dei ruoli.
    \paragraph{\textit{Piano di Qualifica}}
Il \textit{Piano di Qualifica} è un documento che contiene le descrizione dei processi e i controlli necessari per garantire la qualità del prodotto. 
Il \textit{Piano di Qualifica} include: definizioni di processo, standard, metodologie e obiettivi di qualità; requisiti per la 
qualifica del software; linee guida per la formazione e la documentazione del progetto; test di qualifica e di validazione.
\\\\
\textbf{Struttura \textit{Piano di Qualifica}}
\\\\
La struttura del \textit{Piano di Qualifica} segue la struttura generale.
In aggiunta il contenuto del documento si compone di:
\begin{enumerate}
    \item Introduzione;
    \item Qualitá del processo;
    \item Qualitá del prodotto.
\end{enumerate}
\textbf{Introduzione:} contiene una breve descrizione dello scopo del documento e del progetto.
Successivamente é presente una sezione in cui vengono riportati i riferimenti informativi 
con i relativi link alle risorse che abbiamo utilizzato per garantire la qualitá del prodotto.
\\\\
\textbf{Qualitá del processo:} Viene indicato lo standard utilizzato per garantire la qualitá di processo.
Per ogni tipo di processo (primari,supporto,organizzativi) vengono indicati in forma tabellare il nome del processo,la descrizione e le metriche utilizzate
indicate utilizzando un codice identificativo univoco per ognuna di esse.
Sucessivamente viene riportata una tabella che fornisce per ogni metrica i nomi ed i valori significativi previsti.
I dettagli su come vengono applicate tali metriche vengono poi descritti nelle \textit{Norme di Progetto}.
\\\\
\textbf{Qualitá del prodotto:} Viene indicato lo standard utilizzato per garantire la qualitá di prodotto.
Trattandosi di un prodotto software si fa riferimento ad uno standard dedicato alle applicazioni software.
Vengono indicati in forma tabellare gli obiettivi,la descrizione e il codice che contraddistingue le metriche utilizzate per soddisfare gli obiettivi.
Successivamente viene riportata una tabella che per ogni codice ne descrive il significato ed i valori minimi ed ottimi per cui la metrica é rispettata.
I dettagli su come vengono applicate tali metriche vengono poi descritti nelle \textit{Norme di Progetto}.
    \paragraph{\textit{Glossario}}
All'interno della documentazione si possono trovare dei termini che possono risultare ambigui a seconda del contesto, o non conosciuti dagli utilizzatori.\\\
Per ovviare ad incomprensioni si è deciso di stilare un elenco di termini di interesse accompagnati da una descrizione del loro significato.\\
I termini presenti all'interno dei documenti che necessitano di una descrizione vengono indicati con il pedice 'g' come nell'esempio seguente: termine.
É quindi possibile consultare il \textit{Glossario} per reperire tale descrizione.
\\
Ogni componente del gruppo all'inserimento di un termine ritenuto ambiguo deve preoccuparsi di aggiornare il \textit{Glossario}.

Per aggiornare il \textit{Glossario} si devono inserire i nuovi termini nel file .tex nella cartella corrispondente all'iniziale del termine situata al percorso
\textbf{\textit{latex/esterni/doc\_esterna/Glossario/res/sections/alphabet/}}.
É stato creato uno script che scansiona un documento di interesse per inserire in automatico il pedice sui termini contenuti nel \textit{Glossario}.

Il componente del gruppo che inserisce all'interno del \textit{Glossario} un nuovo termine deve aggiungere nel file .tex il segnaposto \%parola\% dopo la subsection (senza spazi) che racchiude il termine per permetterne il riconoscimento da parte dello script.
\\
Il \textit{Glossario} ordina i termini in ordine alfabetico in modo da permetterne una facile e veloce ricerca.








\subsubsection{Metriche}
Per perseguire la qualità sulla documentazione prodotta si è deciso di adottare le seguenti metriche:
\paragraph{Indice di Gulpease}
Si tratta dell'indice di leggibilità di un testo tarato sulla lingua italiana.
I risultati sono compresi tra 0 e 100, dove il valore "100" indica la leggibilità più alta e "0" la leggibilità più bassa. Ai seguenti valori si associano i seguenti significati:
\begin{itemize}
\item inferiore a 80 sono difficili da leggere per chi ha la licenza elementare
\item inferiore a 60 sono difficili da leggere per chi ha la licenza media
\item inferiore a 40 sono difficili da leggere per chi ha un diploma superiore
\end{itemize}
Viene adottata la seguente formula per calcolarlo:
\begin{equation}
89+\frac{300*(\text{numero delle frasi})-10*(\text{numero delle lettere})}{\text{numero delle parole}}
\end{equation}\\
Questi sono i valori da noi ritenuti opportuni:\\
\textit{Valore minimo}:$$ \geq 50 $$ 
\textit{Valore ottimo}:$$ \geq 80 $$\\
\pagebreak
\section{Strumenti collaborativi}
\subsection{GitHub\textsubscript{g}}
Servizio di hosting per progetti software che implementa uno strumento di controllo versione distribuito Git\textsubscript{g}.
Oltre alla copia in remoto del repository\textsubscript{g} di progetto ogni componente del gruppo ha una propria copia in locale.\\
Per ottenere una copia del repository\textsubscript{g} ogni componente ha scaricato lo strumento Git\textsubscript{g} ed eseguendo il 
comando 'git clone' da git\textsubscript{g} bash viene creata una cartella collegata alla repository\textsubscript{g} di progetto.\\
Non sono state imposte modalità specifiche sull'interazione con il repository\textsubscript{g} remoto in modo da non sconvolgere le abitudini di lavoro di 
ciascun componente.\\
I componenti del gruppo abituati ad interagire con GitHub\textsubscript{g} da interfaccia grafica possono continuare a farne uso.
\subsubsection{Repository\textsubscript{g}}
Il repository\textsubscript{g} si può trovare all'indirizzo \textbf{\textit{https://github.com/7clickers/ShowRoom3D}} ed è pubblico. 
I collaboratori sono i componenti del gruppo SevenClickers che utilizzano il proprio account GitHub\textsubscript{g} personale per collaborare al progetto.
\subsubsection{Branching}
\textbf{Branches protetti}:
    \begin{itemize} 
        \item main: contiene le versioni di release\textsubscript{g} del software
        \item documentation: contiene i template latex\textsubscript{g} e rispettivi pdf della documentazione
    \end{itemize}
\textit{documentation}: 
I documenti presenti in documentation sono stati approvati dal Responsabile di progetto o almeno verificati dai verificatori.\\
Per integrare delle modifiche da un branch protetto ad uno libero si utilizza un branch d’appoggio creato in locale partendo dall'ultimo commit di documentation e facendone il merge con il branch che necessita delle integrazioni. In seguito il branch di appoggio verrà eliminato.\\
\textbf{Branches liberi}:
Vengono utilizzati per creare nuove funzionalità e gli sviluppatori possono effettuare i commit senza l'approvazione degli altri componenti del gruppo
in quanto ciascun componente sviluppa su un solo branch alla volta salvo casi eccezionali.
Un branch\textsubscript{g} libero avrà il nome del documento che si sviluppa su quel branch\textsubscript{g}, oppure della feature\textsubscript{g} che va ad implementare.\\
Non appena i/il file nel branch sono stati verificati ed il merge è stato fatto, il branch libero verrà eliminato.
I branch di approvazione saranno chiamati con la sintassi appr\_nomefile1\_nomefile2\_nomefile3.... a seconda dei file che verranno approvati durante il ciclo di vita del branch.
\subsubsection{Commits}
È preferibile che ogni commit abbia una singola responsabilità per cambiamento.\\
I commits non possono essere effettuati direttamente sui branch protetti ma per integrare delle aggiunte o modifiche sarà necessario aprire una Pull Request.
All'approvazione di una Pull Request tutti i commit relativi al merge verranno raggruppati in un unico commit che rispetti la struttura sintattica descritta in seguito.

I commit dovranno essere accompagnati da una descrizione solo se ritenuta indispensabile alla comprensione del commit stesso.

I messaggi di commit sui \textbf{\uppercase{branch protetti}} dovranno seguire la seguente struttura sintattica:\\

\textbf{$<$label$>$$<$\#n\_issue$>$$<$testo$>$}\\\\
dove:\\\\
\textbf{label}: può assumere i seguenti valori
\begin{itemize}
\item feat: indica che è stata implementata una nuova funzionalità
\item fix: indica che è stato risolto un bug
\item update: indica che è stata apportata una modifica che non sia fix o feat
\item test: qualsiasi cosa legata ai test
\item docs: qualsiasi cosa legata alla documentazione
\end{itemize}
\textbf{n\_issue}: indica il numero della issue a cui fa riferimento il commit (se non fa riferimento a nessuna issue viene omesso).\\\\
\textbf{testo}:indica con quale branch è stato effettuato il merge e deve rispettare la forma: merge from $<$nome branch da integrare$>$ to $<$nome branch corrente$>$ \\\\
\textbf{descrizione}: se aggiunta ad un commit deve rispondere alle domande WHAT?,WHY?,HOW? ovvero
cosa è cambiato,perchè sono stati fatti i cambiamenti,in che modo sono stati fatti i cambiamenti.\\

I messaggi di commit sui \textbf{\uppercase{branch liberi}} dovranno seguire la struttura sintattica dei branch protetti ad eccezione del testo.
Il testo dei commit sui branch liberi non è soggetto a restrizioni particolari a patto che indichi in maniera intuitiva i cambiamenti fatti
in modo che possano essere compresi anche dagli altri collaboratori.



\subsubsection{Pull Requests}
Per effettuare un merge su un branch protetto si deve aprire da GitHub\textsubscript{g} una Pull Request.
La Pull Request permette di verificare il lavoro svolto prima di integrarlo con il branch desiderato.\\
Alla creazione di una Pull Request bisogna associare:
\begin{itemize}
\item i verificatori in carica hanno il compito di trovare eventuali errori o mancanze e fornire un feedback riguardante il contenuto direttamente su GitHub\textsubscript{g} richiedendo
una review con un review comment sulla parte specifica da revisionare o con un commento generico.\\
Non sarà possibile effettuare il merge finchè tutti i commenti di revisione non saranno stati risolti e la Pull Request approvata da due verificatori
\item l’issue associata nell’opzione “Development” che verrà chiusa alla risoluzione della Pull Request
\item la Projects Board di cui fa parte
\item gli assegnatari che hanno il compito di apportare le modifiche necessarie in fase di verifica
\item le labels associate
\end{itemize}
Per i commit relativi alle Pull Requests seguire le regole descritte nella sottosezione Commits per i branch protetti.

\subsubsection{Milestone\textsubscript{g}}
Una milestone indica un traguardo intermedio significativo per il progetto.
Ad essa possono venire assegnate delle issues per verificarne il raggiungimento.
Ogni milestone ha una scadenza che viene discussa e fissata da tutto il gruppo.
Oltre alle 2 milestone + 1 milestone falcoltativa fissate dal committente il gruppo ne creerá ulteriori per scandire piú nel dettaglio i passi che ci porteranno 
ad ottenere i risultati prefissati.
Una delle milestone create dal gruppo durante il completamento della tecnology baseline relativa alla prima milestone imposta dal committente (Requirements and Tecnology Baseline)
riguarda l'implementazione di un PoC (Proof of concept).

\subsubsection{Projects Board\textsubscript{g}}
Vengono utilizzate due project board per tracciare le issues della repo.
Una project board principale utilizzata da tutti i membri del gruppo e una project board per le approvazioni utilizzata solo dal responsabile di progetto per approvare i file che richiedono l'approvazione prima di una consegna.

\begin{itemize}
	\item La projectboard\textsubscript{g} principale è suddivisa in queste sezioni:
	\begin{itemize}
		\item \textbf{Todo}: Issue\textsubscript{g} che non sono ancora state iniziate o che non sono ancora state assegnate
		\item \textbf{In Progress}: Issue\textsubscript{g} che sono state assegnate e a cui almeno un membro a cui è stata assegnata ha iniziato a lavorarci
		\item \textbf{Pull Request}: Issue\textsubscript{g} che è in fase di integrazione e necessita della verifica dei verificatori. Corrisponde all'inizio di una pull request
		\item \textbf{Done}: Issue\textsubscript{g} che sono state chiuse e che sono state verificate (se necessitano di verifica\textsubscript{g})
		\item \textbf{Approved}: Issue\textsubscript{g} con label "Da Approvare" che hanno ottenuto l'approvazione\textsubscript{g} del responsabile subito dopo la verifica.
		Per tutte le issues che richiedono un'approvazione solo al momento della consegna è stata creata la project board dedicata alle approvazioni.
	\end{itemize}
	Inoltre nella project board principale vengono registrate delle issue che non richiedono verifica, approvazione o neanche integrazione, con lo scopo di monitorare meglio il lavoro di ogni membro del team.\\
Queste issue verranno chiuse e archiviate manualmente una volta che avranno terminato la loro utilità, un esempio può essere la seguente issue:\\
\textbf{diario di bordo 21-11-22}; questa issue non necessità verifica, approvazione o integrazione perchè non è di interesse caricare il file nella repo, però è utile tracciare lo svolgimento della issue.

	\item La projectboard\textsubscript{g} riservata alle approvazioni è suddivisa in queste sezioni:
	\begin{itemize}
		\item \textbf{Todo}: Issue\textsubscript{g} che non sono ancora state iniziate dal responsabile di progetto
		\item \textbf{In Progress}: Issue\textsubscript{g}  che sono state prese in carico per essere approvate dal responsabile di progetto
		\item \textbf{Pull Request}: Issue\textsubscript{g} che è in fase di integrazione e necessita della verifica dei verificatori. Corrisponde all'inizio di una pull request
		in questo caso i verificatori dovranno solo controllare che l'intestazione e il registro delle modifiche siano stati compilati correttamente dal responsabile di progetto in quanto tutto il resto del contenuto è già stato verificato in precedenza
		\item \textbf{Approved}: Issue\textsubscript{g} che sono state approvate dal responsabile di progetto
	\end{itemize}
\end{itemize}
Le issue che si trovano nella project board approvazioni sono state create come continuazione ad issues che hanno in precedenza attraversato il work flow della project board principale fino allo stato "Done".
In questo modo il Responsabile di progetto dovrà approvare solamente file che si trovano nei branches protetti.
Vengono ricreate in questa project board per  mantenerne la tracciabilità e permetterne una più facile reperibilità futura per l'approvazione.
Questo permette anche di pulire la project board principale da tutte quelle issues che rimarrebbero inattive per molto tempo.
\\\\
Esempio di work flow issues:
\begin{enumerate}
	\item Viene creata l'issue ed inserita all'interno della sezione "To Do" della project board principale.
	\item Quando viene presa in carico l'issue viene spostata nella sezione "In Progress" fino al suo completamento.
	\item L'incaricato alla risoluzione dell'issue apre una pull request a cui assegna l'issue in modo che  venga chiusa in automatico al momento dell'integrazione con il branch di destinazione.
	\item Vengono fatte le verifiche opportune dai verificatori e le conseguenti modifiche prima di accettare la pull request.
	\item Dopo l'accettazione la pull request viene postata insieme alle issues associate nella colonna "Done".
	\item A questo punto si possono presentare due casi: 
	\begin{enumerate}
		\item I file relativi alla pull request richiedono approvazione immediata.
		Viene archiviata la pull request (e le issue associate) e creata una issue a cui viene aggiunta la label "Da approvare" con il nome del file da approvare.
		Il responsabile creerá un branch per approvare le issues con approvazione immediata aprirá una pull request per integrare nel branch di destinazione l'approvazione.
		\item I file relativi alla pull request richiedono l'approvazione prima di una consegna.
		Il responsabile archivia la pull request e ne crea una issue con il nome dei file da approvare nella project board dedicata alle approvazioni.
		NB:Se nella project board Approvazioni é giá presente un'issue di approvazione per il file da approvare non viene creata una nuova issue di approvazione ma si tiene la precedente.
		L'issue segue il work flow della project board approvazioni.
	\end{enumerate}
\end{enumerate}
\subsubsection{Issue Tracking System\textsubscript{g}}
Il gruppo utilizza l'issue tracking system\textsubscript{g} di Github\textsubscript{g} per tenere traccia delle issue\textsubscript{g}. 
Le issues\textsubscript{g} verranno determinate dal responsabile, ma la loro assegnazione verrà effettuata dai membri del gruppo, in base alla priorità delle issues\textsubscript{g}, i loro ruoli e alle loro disponibilità temporali.\\
Per marchiare le issues secondo criteri di interesse (come ambito o prioritá) vengono utilizzate le labels.
\\\\\textbf{Labels:}
\begin{itemize}
	\item Da approvare: indica una issue o pull request che necessita di approvazione
	\item Da verificare: indica una issue o pull request che necessita di verifica
	\item Documentazione: indica una issue o pull request riguardante la documentazione
	\item P=Bassa: indica una issue o pull request a prioritá bassa (scadenza lontana o non limita il lavoro altrui)
	\item P=Media: indica una issue o pull request a prioritá media (scadenza di almeno due settimane o non limita il lavoro altrui)
	\item P=Alta indica una issue o pull request a prioritá alta (scadenza breve o limita il lavoro altrui)
	\item Presentazione: indica una issue riguardante la redazione di una presentazione in classe o al proponente
\end{itemize}

Nel caso un membro del gruppo dovesse rendersi conto che l'issue\textsubscript{g} che sta svolgendo potrebbe essere suddiviso in ulteriori issue\textsubscript{g}, dovrà rivolgersi al responsabile, che è l'unico che può aggiungere, modificare o eliminare le issue\textsubscript{g}.\\
Per raggruppare piú issue si marca ciascuna con la label di raggruppamento.
Il nome di una label di raggruppamento viene preceduto da una G maiuscola (che sta per gruppo) seguito dal nome dell'attivitá da svolgere. esempio: "G Piano di progetto"
indica un gruppo di labels relative al documento "Piano di progetto".
Al completamento di tutte le issues di un gruppo si potrá scegliere se eliminare la label o tenerla per raggruppare issues future dello stesso gruppo.
\\\\
\textbf{Utilizzo delle checkbox:}
\\\\
Una issue può essere suddivisa in più attività tramite delle checkbox in base alla grandezza o
alla necessità di suddividere il lavoro. Al completamento di un'attività si spunta la checkbox
corrispondente in modo da avvisare il gruppo sullo stato di avanzamento di quella issue.
\paragraph{\textit{Glossario}}
All'interno della documentazione si possono trovare dei termini che possono risultare ambigui a seconda del contesto, o non conosciuti dagli utilizzatori.\\\
Per ovviare ad incomprensioni si è deciso di stilare un elenco di termini di interesse accompagnati da una descrizione del loro significato.\\
I termini presenti all'interno dei documenti che necessitano di una descrizione vengono indicati con il pedice 'g' come nell'esempio seguente: termine.
É quindi possibile consultare il \textit{Glossario} per reperire tale descrizione.
\\
Ogni componente del gruppo all'inserimento di un termine ritenuto ambiguo deve preoccuparsi di aggiornare il \textit{Glossario}.

Per aggiornare il \textit{Glossario} si devono inserire i nuovi termini nel file .tex nella cartella corrispondente all'iniziale del termine situata al percorso
\textbf{\textit{latex/esterni/doc\_esterna/Glossario/res/sections/alphabet/}}.
É stato creato uno script che scansiona un documento di interesse per inserire in automatico il pedice sui termini contenuti nel \textit{Glossario}.

Il componente del gruppo che inserisce all'interno del \textit{Glossario} un nuovo termine deve aggiungere nel file .tex il segnaposto \%parola\% dopo la subsection (senza spazi) che racchiude il termine per permetterne il riconoscimento da parte dello script.
\\
Il \textit{Glossario} ordina i termini in ordine alfabetico in modo da permetterne una facile e veloce ricerca.
\pagebreak
\section{Organizzazione del gruppo}
\subsubsection{Ruoli}
I componenti del gruppo si suddivideranno nei seguenti ruoli per periodi di circa 2-3 settimane (dipendentemente dalle esigenze del periodo) e al termine del periodo i ruoli verranno risuddivisi. 
Visto che nelle varie fasi di sviluppo del progetto le attività da svolgere variano, non sempre sarà necessario coprire tutti i ruoli.\\
Inoltre sarà necessario tenere traccia delle ore che ogni componente dedica al progetto ed il ruolo associato a quelle ore, in modo da andare a rispettare la tabella degli impegni individuali.\\
Per questo tracciamento verrà utilizzato un foglio Excel in cui ogni componente del gruppo segnerà le ore di lavoro settimanalmente.
I ruoli e le loro competenze sono i seguenti:

\paragraph{Responsabile}
Deve avere la visione d'insieme del progetto e coordinare i membri, inoltre si occupa di rappresentare il gruppo con le interazione esterne (proponente, committente ecc...). Le sue competenze specifiche sono:
\begin{itemize}
	\item Ad ogni iterazione\textsubscript{g} c'è un solo responsabile;
	\item Presentare il \textit{Diario di Bordo} in aula;
	\item Redarre l'ordine del giorno prima di ogni meeting interno del gruppo;
	\item Suddivide le attività del gruppo in singole issue (ma non le assegna ai membri del gruppo);
	\item In fase di release\textsubscript{g} si occupa di approvare\textsubscript{g} tutti i documenti che necessitano approvazione.
\end{itemize}

\paragraph{Analista}
Si occupa di trasformare i bisogni del proponente nelle aspettative che il gruppo deve soddisfare per sviluppare un prodotto professionale. Le sue competenze specifiche sono:
\begin{itemize}
	\item Interrogare il proponente riguardo allo scopo del prodotto e le funzionalità che deve avere;
	\item Studiare le risposte del proponente per identificare i requisiti\textsubscript{g} e redarre l' \textit{Analisi dei Requisiti}.
\end{itemize}

\paragraph{Amministratore}
Si occupa del funzionamento, mantenimento e sviluppo degli strumenti tecnologici usati dal gruppo. Le sue competenze specifiche sono:
\begin{itemize}
	\item Ad ogni iterazione\textsubscript{g} basta un solo amministratore;
	\item Gestione delle segnalazioni e problemi dei membri del gruppo riguardanti problemi e malfunzionamenti con gli strumenti tecnologici;
	\item Valuta l'utilizzo di nuove tecnologie e ne fa uno studio preliminare per poter presentare al gruppo i pro e i contro del suo utilizzo.
\end{itemize}

\paragraph{Progettista}
Si occupa di scegliere la modalità migliore per soddisfare le aspettative del committente che gli analisti hanno ricavato dall'analisi dei requisiti\textsubscript{g}. Le sue competenze specifiche sono:
\begin{itemize}
	\item Scegliere eventuali pattern architetturali da implementare;
	\item Sviluppare lo schema UML\textsubscript{g} delle classi\textsubscript{g}.
\end{itemize}

\paragraph{Programmatore}
Si occupa di implementare le scelte e i modelli fatti dal progettista. Le sue competenze specifiche sono:
\begin{itemize}
	\item Scrivere il codice atto a implementare lo schema delle classi;
	\item Scrivere eventuali test;
	\item Scrivere la documentazione per la comprensione del codice che scrive.
\end{itemize}

\paragraph{Verificatore}
Si occupa di controllare che ogni file che viene caricato in un branch protetto della repository\textsubscript{g} sia conforme alle norme di progetto. Le sue competenze specifiche sono:
\begin{itemize}
	\item Controllare i file modificati o aggiunti durante una pull request tra un ramo non protetto e un ramo protetto siano conformi alle norme di progetto e cercano errori di altra natura (ortografici, sintattici, logici, build ecc...).
\end{itemize}
\pagebreak

\end{document}
