\paragraph{\textit{Lettera di Presentazione}}
Una \textit{Lettera di Presentazione} serve a manifestare la volontà da parte del gruppo di prendere un impegno con il committente\textsubscript{g}.
L'impegno può essere la candidatura per un capitolato di interesse o per una revisione di avanzamento.
Dopo aver ricevuto una \textit{Lettera di Presentazione} sta al committente\textsubscript{g} la decisione di accettare o rifiutare l'impegno che ne consegue.
\\\\
\textbf{Struttura \textit{Lettera di Presentazione}}
\\\\
Una \textit{\textit{Lettera di Presentazione}} è formata dai seguenti campi:
\begin{itemize}
    \item Header;
    \item Mittente e data;
    \item Destinatari;
    \item Contenuto;
    \item Riferimenti a documenti;
    \item Conclusioni e saluti;
    \item Nome, cognome, firma responsabile.
\end{itemize}

\noindent Descrizione dei vari campi:
\begin{itemize}
\item \textbf{Header}: contiene il logo del gruppo (con slogan) a sinistra ed il logo dell'Università di Padova a destra;
\item \textbf{Mittente e data}: contiene il nome,email del gruppo seguito dal nome del corso e dalla data di invio della lettera al destinatario;
\item \textbf{Destinatari}: contiene i destinatari della lettera compresi di titoli e luogo della loro sede;
\item \textbf{Contenuto}: contiene il contenuto della lettera scritto in modo formale dichiarando lo scopo della lettera;
\item \textbf{Riferimenti a documenti}: contiene un elenco puntato di tutti i documenti (e loro versioni) a cui si vuole sottoporre l'attenzione dei destinatari. 
\item \textbf{Conclusioni}: contiene le considerazioni finali seguite dai saluti rivolti al destinatario;
\item \textbf{Nome, cognome, firma responsabile}: contiene il nome, cognome, titolo e firma digitale del responsabile.
\end{itemize}
Di seguito viene riportata un'immagine che illustra visivamente i campi appena descritti:
\begin{figure}[htbp]
    \centering
    \fbox{\includegraphics[scale=0.68]{../../../template/images/esempio\_lettera\_di\_presentazione.png}}
    \caption{Esempio di \textit{Lettera di Presentazione}}
\end{figure}
\pagebreak