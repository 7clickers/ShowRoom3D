\paragraph{\textit{Norme di Progetto}}
\textbf{Struttura \textit{Norme di Progetto}}
\\\\
La struttura di \textit{Norme di Progetto} segue la struttura generale.
In aggiunta il contenuto del documento si compone di:
\begin{itemize}
    \item Introduzione;
    \item Processi primari;
    \item Processi di supporto;
    \item Processi organizzativi;
    \item Standard di qualità ISO/IEC 9126;
    \item Standard di qualità ISO/IEC 12207:1995.
\end{itemize}
Descrizione dei campi che compongono il documento:
\begin{itemize}
    \item \textbf{Introduzione:} comprende una breve descrizione dello scopo del documento e del prodotto\textsubscript{g};
    \item \textbf{Processi primari:} comprende la definizione di processi primari.
    I processi primari inclusi nel nostro progetto sono:
    \begin{itemize}
        \item Fornitura: comprende la definizione e le regole di fornitura;
        \item Sviluppo: comprende la definizione e le regole di sviluppo.
    \end{itemize}
    \item \textbf{Processi di supporto:} comprende la definizione di processi di supporto.
    I processi di supporto inclusi nel nostro progetto sono:
    \begin{itemize}
        \item Documentazione: comprende una descrizione dell'attività di documentazione seguito dalle sezioni che descrivono le norme necessarie alla stesura, alla verifica\textsubscript{g} e 
        alla validazione\textsubscript{g} della documentazione. Si riportano anche le sezioni dedicate ai documenti interni ed esterni nelle quali si descrive la struttura che deve avere ogni documento;
        \item Gestione della configurazione: comprende la definizione e le regole di configurazione; 
        \item Verifica\textsubscript{g}: comprende la definizione e le regole di verifica\textsubscript{g};
        \item Validazione\textsubscript{g}: comprende la definizione e le regole di validazione\textsubscript{g}.
    \end{itemize}
    \item \textbf{Processi organizzativi:} comprende la definizione di processi organizzativi.
    I processi organizzativi inclusi nel nostro progetto sono:
    \begin{itemize}
        \item Gestione organizzativa: riguarda l'organizzazione all'interno del gruppo ovvero la ripartizione dei compiti e le attivitá che ogni figura professionale é tenuta a svolgere fino al 
        completamento del progetto;
        \item Gestione infrastrutture: cruciale per garantire la disponibilità, l'affidabilità e la sicurezza delle risorse necessarie per lo sviluppo e l'esecuzione del software.
    \end{itemize}
    \item \textbf{Standard di qualità ISO/IEC 9126:} comprende la descrizione dello standard e le sue metriche utilizzate;
    \item \textbf{Standard di qualità ISO/IEC 12207:1995:} comprende la descrizione dello standard e le sue metriche utilizzate.
\end{itemize}

