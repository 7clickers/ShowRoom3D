\paragraph{\textit{Piano di Qualifica}}
Il \textit{Piano di Qualifica} è un documento che contiene le descrizione dei processi e i controlli necessari per garantire la qualità del prodotto. 
Il \textit{Piano di Qualifica} include: definizioni di processo, standard, metodologie e obiettivi di qualità; requisiti per la 
qualifica del software; linee guida per la formazione e la documentazione del progetto; test di qualifica e di validazione.
\\\\
\textbf{Struttura \textit{Piano di Qualifica}}
\\\\
La struttura del \textit{Piano di Qualifica} segue la struttura generale.
In aggiunta il contenuto del documento si compone di:
\begin{itemize}
    \item Introduzione;
    \item Qualità del processo;
    \item Qualità del prodotto.
\end{itemize}
\noindent Descrizione delle varie sezioni:
\begin{itemize}
\item \textbf{Introduzione:} contiene una breve descrizione dello scopo del documento e del progetto.
Successivamente è presente una sezione in cui vengono riportati i riferimenti informativi 
con i relativi link alle risorse che abbiamo utilizzato per garantire la qualità del prodotto.

\item \textbf{Qualità del processo:} viene indicato lo standard utilizzato per garantire la qualità di processo.
Per ogni tipo di processo (primari, supporto, organizzativi) vengono indicati in forma tabellare il nome del processo, la descrizione e le metriche utilizzate
indicate con un codice identificativo univoco per ognuna di esse.
Sucessivamente viene riportata una tabella che fornisce per ogni metrica i nomi ed i valori significativi previsti.
I dettagli su come vengono applicate tali metriche vengono poi descritti nelle \textit{Norme di Progetto}.

\item \textbf{Qualità del prodotto:} viene indicato lo standard utilizzato per garantire la qualità di prodotto.
Trattandosi di un prodotto software si fa riferimento ad uno standard dedicato alle applicazioni software.
Vengono indicati in forma tabellare gli obiettivi, la descrizione e il codice che contraddistingue le metriche utilizzate per soddisfare gli obiettivi.
Successivamente viene riportata una tabella che per ogni codice ne descrive il significato ed i valori minimi ed ottimi per cui la metrica è rispettata.
I dettagli su come vengono applicate tali metriche vengono poi descritti nelle \textit{Norme di Progetto}.
\end{itemize}