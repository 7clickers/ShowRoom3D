\subsubsection{Convenzioni}
Le convenzioni di seguito riportate vengono applicate a tutti i documenti.
Esse rendono i documenti stilati omogenei tra loro contribuendo a rendere il progetto professionale.

\paragraph{Date}
Le date devono rispettare il seguente formato: \textbf{dd-mm-yyyy}
All'interno delle tabelle il formato deve essere il seguente: \textbf{dd-mm-yy}
\paragraph{Nomi di persona}
All'interno dei documenti i nomi di persona rispetteranno l'ordine nome seguito dal cognome della persona menzionata.

\paragraph{Elenchi puntati}
Gli elenchi puntati devono rispettare le seguenti regole:
\begin{itemize} 
    \item Ogni elemento dell'elenco deve iniziare con la lettera maiuscola;
    \item Ogni elemento dell'elenco deve terminare con ";" ad eccezione dell'ultimo elemento
    che deve terminare con "."; 
    \item Dopo i due punti la frase deve iniziare con la lettera minuscola.
\end{itemize}

\paragraph{Stile del testo}
\begin{itemize} 
    \item \textbf{Grassetto}: stile utilizzato per i titoli delle sezioni e per i primi termini degli elenchi puntati;
    \item \textbf{Corsivo}: viene utilizzato per citare il nome dei documenti, ad esempio \textit{Piano di Progetto}. 
\end{itemize}

\paragraph{Immagini}
Le immagini sono raccolte nella cartella “”, e sono inserite sempre con una didascalia descrittiva posizionata sotto l’immagine.

\paragraph{Tabelle} 
Le tabelle sono provviste di didascalia descrittiva posizionata sotto alla tabella. La tabella contenente il registro delle modifiche è l’unica che fa da eccezione a questa regola.

\paragraph{\textit{Glossario}}
All'interno della documentazione si possono trovare dei termini che possono risultare ambigui a seconda del contesto, o non conosciuti dagli utilizzatori.\\\
Per ovviare ad incomprensioni si è deciso di stilare un elenco di termini di interesse accompagnati da una descrizione del loro significato.\\
I termini presenti all'interno dei documenti che necessitano di una descrizione vengono indicati con il pedice 'g' come nell'esempio seguente: termine\textsubscript{g}.
É quindi possibile consultare il \textit{Glossario} per reperire tale descrizione.
\\
Ogni componente del gruppo all'inserimento di un termine ritenuto ambiguo deve preoccuparsi di aggiornare il \textit{Glossario}.

Per aggiornare il \textit{Glossario} si devono inserire i nuovi termini nel file .tex nella cartella corrispondente all'iniziale del termine situata al percorso
\textbf{\textit{latex/esterni/doc\_esterna/Glossario/res/sections/alphabet/}}.
É stato creato uno script che scansiona un documento di interesse per inserire in automatico il pedice sui termini contenuti nel \textit{Glossario}.

Il componente del gruppo che inserisce all'interno del \textit{Glossario} un nuovo termine deve aggiungere nel file .tex il segnaposto \%parola\% dopo la subsection (senza spazi) che racchiude il termine per permetterne il riconoscimento da parte dello script.
\\
Il \textit{Glossario} ordina i termini in ordine alfabetico in modo da permetterne una facile e veloce ricerca.