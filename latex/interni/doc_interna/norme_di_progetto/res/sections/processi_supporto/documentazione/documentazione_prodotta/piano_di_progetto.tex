\paragraph{\textit{Piano di Progetto}}

Il documento \textit{\textit{Piano di Progetto}} ha lo scopo di aiutare il gruppo nella gestione delle risorse a disposizione per portare a termine il progetto entro la data decisa.
Il documento ha anche la funzione di monitorare l'avanzamento del progetto in modo da poter applicare miglioramenti continui e azioni correttive
basandosi sull'esperienza ottenuta da pianificazioni precedenti.
\\\\
\textbf{Struttura \textit{\textit{Piano di Progetto}}}
\\\\
La struttura del \textit{\textit{Piano di Progetto}} segue la struttura generale.
In aggiunta il contenuto del documento si compone di:
\begin{enumerate}
    \item Analisi dei rischi;
    \item Pianificazione;
    \item Preventivo.
\end{enumerate}
\textbf{NOTA:} ogni tabella o immagine all'interno del documento verrá indicata anche nell'indice del documento stesso in modo che sia rintracciabile con 
facilitá.
\\\\
\textbf{Descrizione delle sezioni:}
\\\\
\textbf{Analisi dei rischi:} vengono riportati in forma tabellare i rischi a cui si va incontro aggiudicandosi il capitolato.
Ogni rischio fa riferimento ad una categoria di rischi precisa e viene indicato con un nome,una probabilitá che si verifichio,
un grado che indica l'impatto negativo che puó
comportare es una breve descrizione su come affrontare il rischio nel momento in cui dovesse presentarsi.
\\\\
\textbf{Pianificazione:} la sezione dedicata alle pianificazioni ha lo scopo di indicare l'inizio e la fine di un periodo di pianificazione e 
di fornire la suddivisione e l'organizzazione delle attivitá all'interno di tale periodo.
Viene fornito un elenco con descrizione delle attivitá che si andranno a svolgere e viene diviso il periodo di pianificazione in sottoperiodi 
ciascuno con un inizio,una fine ed indicando cosa verrá fatto.
Infine é presente un diagramma di Gantt che illustra graficamente l'ordine temporale delle attivitá da svolgere tenendo conto di 
eventuali margini temporali dovuti ad imprevisti vari.
\\\\
\textbf{Preventivo:} include una sezione iniziale in cui si vanno ad indicare in forma tabellare le risore umane disponibili all'inizio del progetto e 
come queste risorse andranno suddivise. Successivamente viene fornito,sempre in forma tabellare,il costo totale calcolato in base alle ore 
di impegno dei componenti del gruppo ed al ruolo che dovranno ricoprire (ogni ruolo ha un costo orario).
Le tabelle riportate produrranno come risultato la conclusione del preventivo che comprende il costo totale calcolato e la data di fine progetto.
É presente una sezione (dettaglio periodi) che prevede i preventivi di ciascun periodo di pianificazione.
In questa sezione sono fornite le tabelle con le ore che ciascun componente del gruppo dovrá svolgere con relativi ruoli associati ed una tabella con i costi
derivati da tali ruoli.
Nel caso del nostro progetto viene prevista una rotazione dei ruoli a scopo didattico perció ogni componente userá le sue ore di impegno in modo diverso 
a seconda della rotazione dei ruoli.