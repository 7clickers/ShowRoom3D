\paragraph{\textit{Verbale}}
Il \textit{Verbale} ha lo scopo di rendicontare ciò che viene detto durante una riunione.
Rispetta la struttura generale.
In aggiunta presenta:
\begin{itemize} 
    \item \textbf{Informazioni Generali}
    contiene:
    \begin{itemize} 
        \item Luogo;
        \item Data;
        \item Ora;
        \item Partecipanti.
    \end{itemize}
\item \textbf{Tabella tracciamento temi affrontati}:
tabella che riassume i punti salienti della riunione indicandone:
    \begin{itemize} 
        \item Codice: ha il formato V\textbf{X} \textbf{Y}.\textbf{Z} dove X indica la tipologia di \textit{Verbale}, Y indica il numero di \textit{Verbale} (incrementale rispetto agli altri verbali);
        e Z indica il numero dell'argomento trattato (incrementale rispetto agli altri argomenti del \textit{Verbale});
        \item Descrizione: breve descrizione di uno specifico argomento trattato.
    \end{itemize}
\end{itemize}