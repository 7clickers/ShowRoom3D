\subsubsection{Struttura generale}
Ogni documento deve presentare le seguenti sezioni nell'ordine in cui vengono presentate:
\begin{itemize} 
    \item \textbf{Intestazione}:
    contiene:
    \begin{itemize} 
        \item Logo compreso di motto;
        \item Indirizzo email di gruppo; 
        \item Titolo;
        \item Tabella contenente le informazioni generali:
        \begin{itemize}
            \item Versione;
            \item Stato;
            \item Uso;
            \item Approvazione: indica il responsabile di progetto che ha approvato il documento; 
            \item Redazione: elenco dei collaboratori che hanno partecipato alla stesura del documento;
            \item Verifica: elenco dei verificatori che hanno verificato il documento;
            \item Distribuzione: elenco delle persone o organizzazioni a cui è destinato il documento.
        \end{itemize}
        \item Breve descrizione del documento .
    \end{itemize}
    \item \textbf{Registro delle modifiche}:
    tabella che identifica ogni versione del documento indicandone:
    \begin{itemize} 
        \item Versione;
        \item Data;
        \item Autore;
        \item Ruolo;
        \item Descrizione: la descrizione deve essere breve. Nel caso di aggiunte o modifiche si deve indicare 
        il nome della sezione che è stata aggiunta o modificata. 
        I nomi delle sezioni vanno riportati uguali a come sono scritti nell'indice, 
        senza virgolette.
    \end{itemize}
    \item \textbf{Indice}:
    elenco ordinato dei titoli dei capitoli, ovvero delle varie parti di cui si compone il documento;
    
    \item \textbf{Elenco delle figure}: sezione che fornisce l'elenco delle immagini presenti nel documento. 
    Se il documento non presenta immagini, questa sezione sarà omessa di conseguenza;  
    \item \textbf{Elenco delle tabelle}: sezione che fornisce l'elenco delle tabelle presenti nel documento. 
    Se il documento non presenta tabelle, questa sezione sarà omessa di conseguenza; 
    \item \textbf{Contenuto}:
    varia a seconda del tipo di documento.
\end{itemize}