\subsubsection{Ciclo di vita di un documento}
\begin{itemize}
    \item \textbf{Creazione}: il documento viene creato su un nuovo branch, utilizzando un template uguale per tutti i documenti. 
    Viene aggiunta l’intestazione e il registro delle modifiche;
    \item \textbf{Stesura}: si procede con la stesura delle varie sezioni, tracciando i cambiamenti nel registro delle modifiche; 
    \item \textbf{Verifica}: ogni sezione viene verificata dai verificatori (per maggiori specifiche guardare la 
    sezione \fullref{verifica_documentazione}); 
    \item \textbf{Approvazione}: una volta che tutte le sezioni sono state verificate si procede 
    con l’approvazione del documento, che viene effettuata dal responsabile nel seguente modo:
    \begin{itemize}
        \item Viene aperta una Pull Request di approvazione;
        \item Il responsabile rilegge il documento: se riscontra ulteriori problematiche, segnala ai verificatori le eventuali modifiche da apportare;
        \item Se sono richieste delle modifiche i verificatori si occupano di apportarle;
        \item Il responsabile crea una nuova riga e compila i campi nelle colonne corrispondenti del registro delle modifiche 
        inserendo: l'ultima versione secondo le norme di versionamento, la data, il proprio nome, il suo ruolo e la voce "Approvazione" nell'ultima colonna.
    \end{itemize}
    Per i \textit{Verbali} si effettua questa pratica non appena il file prodotto viene 
    verificato, mentre per tutti gli altri documenti prima di una consegna.
\end{itemize}