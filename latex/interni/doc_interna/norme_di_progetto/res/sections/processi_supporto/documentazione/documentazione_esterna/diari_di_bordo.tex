\paragraph{\textit{Diario di Bordo}}
Il \textit{Diario di Bordo} è un documento informale ad uso esterno e permette di interagire con il committente in modo da aggiornarlo sullo stato di avanzamento del progetto settimanalmente ed è usato 
anche per chiedere chiarimenti sui temi in cui riscontriamo dubbi o domande.
Fino al termine delle lezioni il \textit{Diario di Bordo} veniva redatto settimanalmente dal responsibile che presentava in classe prima dello svolgimento 
della lezione.
Dal termine delle lezioni il \textit{Diario di Bordo} viene redatto settimanalmente dal responsabile e caricato in una cartella denominata diari\_di\_bordo presente nel drive di gruppo.
\\\\
La cartella diari\_di\_bordo si trova al link: \href{https://drive.google.com/drive/u/1/folders/1a0kAZuOSsDEp_AaQUb6OQJ4XQyZ--RCN}{\\https:\/\/drive.google.com\/drive\/u\/1\/folders\/1a0kAZuOSsDEp\_AaQUb6OQJ4XQyZ\-\-RCN}
\\\\
È stato fornito l'accesso in lettura alla cartella al Professore che viene notificato tramite mail (deve contenere il link al \textit{Diario di Bordo} di interesse) al caricamento di un \textit{Diario di Bordo}.
Il formato dei diari di bordo condivisi nella cartella è .pdf.
Inoltre tutti i diari di bordo una volta redatti vengono inviati anche al canale discord\textsubscript{g} "diari-di-bordo-file" in modo da avere una duplice copia a fronte di 
qualsiasi imprevisto dato che i diari di bordo non vengono inseriti all'interno del repository di gruppo.
\\\\
\textbf{Struttura \textit{Diario di Bordo}} 
\\\\
La struttura del \textit{Diario di Bordo} non rispetta quella indicata nella sezione 3.1.1.2 relativa alla struttura generale della documentazione.
Sono state fissate delle regole per la stesura dei diari di bordo per fornire delle presentazioni il più possibile simili tra loro 
senza troppi vincoli superflui per tali documenti.
\\\\
Il \textbf{font} utilizzato per la stesura del documento è Helvetica Neue con dimensione 22pt.
Il nome del file per ogni diario di bordo deve rispettare la codifica \textbf{diario\_di\_bordo\_AAAA-MM-GG} in modo da permettere l'ordinamento
cronologico utilizzando i filtri di ricerca di Google Drive.
\\\\
Il \textit{Diario di Bordo} si divide in 4 slides che possono essere raggruppate in 3 se l'elenco degli obiettivi raggiunti e quelli futuri ci stanno
in una singola slide.
Le slides che compongono il \textit{Diario di Bordo} sono:
\begin{itemize}
    \item Intestazione; 
    \item Obiettivi raggiunti;
    \item Obiettivi futuri;
    \item Domande.
\end{itemize}
Descrizione slides:
\begin{itemize}
    \item \textbf{Intestazione}: presenta nella parte centrale il logo del gruppo compreso di slogan, in basso centrale la data relativa al \textit{Diario di Bordo} e l'anno accademico;
    \item \textbf{Obiettivi raggiunti}: presenta in alto centrale il titolo "Obiettivi raggiunti", a sinistra un elenco puntato con gli obiettivi raggiunti
    della settimana, in alto a destra il logo del gruppo compreso di slogan ed in basso a destra la data del \textit{Diario di Bordo};
    \item \textbf{Obiettivi futuri}: presenta in alto centrale il titolo "Obiettivi futuri", a sinistra un elenco puntato con gli obiettivi raggiunti
    della settimana ed in alto a destra il logo del gruppo compreso di slogan;
    \item \textbf{Domande}: presenta in alto centrale il titolo "Domande", a sinistra un elenco puntato con le domande da porre al Professore, in alto a destra il logo del gruppo compreso di slogan ed in basso 
    a destra la data del \textit{Diario di Bordo}.
\end{itemize}