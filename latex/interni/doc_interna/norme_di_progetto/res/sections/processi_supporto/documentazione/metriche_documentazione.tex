\subsubsection{Metriche}
Per perseguire la qualità sulla documentazione prodotta si è deciso di adottare le seguenti metriche:
\paragraph{Indice di Gulpease}
Si tratta dell'indice di leggibilità di un testo tarato sulla lingua italiana.
I risultati sono compresi tra 0 e 100, dove il valore "100" indica la leggibilità più alta e "0" la leggibilità più bassa. Ai seguenti valori si associano i seguenti significati:
\begin{itemize}
\item inferiore a 80 sono difficili da leggere per chi ha la licenza elementare
\item inferiore a 60 sono difficili da leggere per chi ha la licenza media
\item inferiore a 40 sono difficili da leggere per chi ha un diploma superiore
\end{itemize}
Viene adottata la seguente formula per calcolarlo:
\begin{equation*}
89+\frac{300*(\text{numero delle frasi})-10*(\text{numero delle lettere})}{\text{numero delle parole}}
\end{equation*}\\
Questi sono i valori da noi ritenuti opportuni:\\
\textit{Valore minimo}: $ \ge 50 $\\ 
\textit{Valore ottimo}: $ \ge 80 $\\