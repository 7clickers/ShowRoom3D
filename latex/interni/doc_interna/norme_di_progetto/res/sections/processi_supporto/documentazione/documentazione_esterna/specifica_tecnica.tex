\paragraph{\textit{Specifica Tecnica}}
La Specifica Tecnica è il documento in cui viene trattata in maniera descrittiva e dettagliata la struttura architetturale dell'applicazione sviluppata. 
La specifica tecnica include: l'elenco dei componenti, il design pattern architetturale determinato dalle tecnologie adottate, l'architettura logica, l'architettura di deployment, idiomi e pattern di livello più basso e altri aspetti di design.
\\\\
\textbf{Struttura della \textit{Specifica Tecnica}}
\\\\
La struttura della Specifica Tecnica segue la struttura generale.
In aggiunta il contenuto del documento si compone di:
\begin{itemize}
	\item Introduzione
    \item Elenco delle componenti;
    \item Design pattern architetturale determinato dalle tecnologie adottate;
    \item Architettura logica;
	\item Architettura di deployment;
	\item Idiomi e pattern di livello più basso;
	\item Altri aspetti di design.
\end{itemize}
\noindent Descrizione delle varie sezioni:
\begin{itemize}
\item \textbf{Introduzione:} contiene una breve descrizione dello scopo del documento e del progetto.
Successivamente è presente una sezione in cui vengono riportati i riferimenti informativi 
con i relativi link alle risorse che abbiamo utilizzato per garantire la qualità del prodotto\textsubscript{g}.

\item \textbf{Elenco delle componenti:} elenca tutti i componenti architetturali dell'applicazione e li raggruppa in base alla loro tipologia:
\begin{itemize}
 	\item Slices;
 	\item  Initaial states;
 	\item Actions; 
 	\item Model components; 
 	\item UI React components; 
 	\item 3D React components.
\end{itemize}

\item \textbf{Design pattern architetturale determinato dalle tecnologie adottate:} contiene una spiegazione dettagliata delle tecnologie che abbiamo scelto di usare per definire l'architettura del codice e il loro funzionamento.
Comprende quindi una descrizione del funzionamento di Redux analizzandone le singole compo	nenti (Store, RootReducer, Slice, Reducer, Actions, InitialState, Selector) e della libreria Rect-three-fiber;

\item \textbf{ Architettura logica:} in questa sezione la struttura architetturale viene descritta con l'aiuto di diagrammi UML. Ogni diagramma si concentra solo su una specifica parte dell'applicazione raggruppando quindi le componenti che interessano una feature precisa per dare una visione complessiva ed a più alto livello del funzionamento.
A seguito di ogni diagramma viene descritto lo scopo della feature rappresentata e poi vengono elencate le componenti che lo compongo assieme alle loro connessioni ed interazioni con gli altri elementi;

\item \textbf{Architettura di deployment:} in questa sezione viene descritta, con l'utilizzo di diagrammi UML, come avviene l'istanziazione degli oggetti rappresentati nella sezione precedente;

\item \textbf{Idiomi e pattern di livello più basso:} in questa sezione vengono evidenziati specifici pattern architetturali che si possono identificare analizzando l'architettura logica. Per ogni pattern individuato nell'architettura dell'applicazione viene mostrato un diagramma che rappresenta tutte le classi che compongono quel pattern specifica, inoltre viene descritto testualmente di che tipo di pattern si sta parlando e che proprietà garantisce al sistema o alle classi interessate.

\item \textbf{Altri aspetti di design:}

\end{itemize}