\paragraph{\textit{Piano di Progetto}}

Il documento \textit{\textit{Piano di Progetto}} ha lo scopo di aiutare il gruppo nella gestione delle risorse a disposizione per portare a termine il progetto entro la data decisa.
Il documento ha anche la funzione di monitorare l'avanzamento del progetto in modo da poter applicare miglioramenti continui e azioni correttive
basandosi sull'esperienza ottenuta da pianificazioni precedenti.
\\\\
\textbf{Struttura \textit{\textit{Piano di Progetto}}}
\\\\
La struttura del \textit{\textit{Piano di Progetto}} segue la struttura generale.
In aggiunta il contenuto del documento si compone di:
\begin{itemize}
    \item Analisi dei rischi;
    \item Pianificazione;
    \item Preventivo.
\end{itemize}
Descrizione delle sezioni:
\begin{itemize}
\item \textbf{Analisi dei rischi:} vengono riportati in forma tabellare i rischi a cui si va incontro aggiudicandosi il capitolato.
Ogni rischio fa riferimento ad una categoria di rischi precisa e viene indicato con:
\begin{itemize}
    \item Nome;
    \item Descrizione;
    \item Probabilità che si verifichi;
    \item Grado di pericolosità;
    \item Precauzioni da prendere;
    \item Piano di contingenza.
\end{itemize}
\item \textbf{Pianificazione:} la sezione dedicata alle pianificazioni ha lo scopo di indicare l'inizio e la fine di un periodo di pianificazione e 
di fornire la suddivisione e l'organizzazione delle attività all'interno di tale periodo. Per ogni periodo vengono forniti:
\begin{itemize}
    \item Attività che si andranno a svolgere nel periodo;
    \item Sottoperiodi in cui viene diviso il periodo principale, ciascuno con un inizio, una fine ed indicando cosa verrà fatto;
    \item Diagramma di Gantt\textsubscript{g} che illustra graficamente l'ordine temporale delle attività da svolgere tenendo conto di 
    eventuali margini temporali dovuti ad imprevisti vari.
\end{itemize}

\item \textbf{Preventivo:} include: 
\begin{itemize}
    \item Sezione iniziale in cui si vanno ad indicare in forma tabellare le risore umane disponibili all'inizio del progetto e 
    come queste risorse andranno suddivise;
    \item Costo totale calcolato in base alle ore di impegno dei componenti del gruppo ed al ruolo che dovranno ricoprire (ogni ruolo ha un costo orario);
    \item Dettaglio periodi, sezione che prevede i preventivi di ciascun periodo di pianificazione.
    In questa sezione sono fornite le tabelle con le ore che ciascun componente del gruppo dovrà svolgere con relativi ruoli associati ed una tabella con i costi
    derivati da tali ruoli.
\end{itemize}
Nel caso del nostro progetto viene prevista una rotazione dei ruoli a scopo didattico, perciò ogni componente userà le sue ore di impegno in modo diverso 
a seconda della rotazione dei ruoli.
\end{itemize}