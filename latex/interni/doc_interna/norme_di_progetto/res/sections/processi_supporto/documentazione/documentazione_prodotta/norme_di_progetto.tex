\paragraph{\textit{Norme di Progetto}}
\textbf{Struttura \textit{Norme di Progetto}}
\\\\
La struttura di \textit{Norme di Progetto} segue la struttura generale.
In aggiunta il contenuto del documento si compone di:
\begin{enumerate}
    \item Introduzione;
    \item Processi primari;
    \item Processi di supporto;
    \item Processi organizzativi;
    \item Standard di qualità ISO/IEC 9126;
    \item Standard di qualità ISO/IEC 12207:1995;
\end{enumerate}
\textbf{Introduzione:} comprende una breve descrizione dello scopo del documento e del prodotto.
\\\\
\textbf{Processi primari:}
Comprende la definizione di processi primari. 
\\\\
I processi primari inclusi nel nostro progetto sono:
\begin{enumerate}
    \item Fornitura;
    \item Sviluppo;
        \begin{enumerate}
            \item Analisi dei requisiti:;
        \end{enumerate}
    \item Progettazione;
    \item Codifica.
\end{enumerate}
\textbf{Processi di supporto:} 
Comprende la definizione di processi di supporto.
\\\\
I processi di supporto inclusi nel nostro progetto sono:
\begin{enumerate}
    \item Documentazione: comprende una descrizione del processo di documentazione seguito dalle sezioni che descrivono le norme necessarie alla stesura, alla verifica e 
    alla validazione della documentazione. Si riportano anche le sezioni dedicate ai documenti interni ed esterni nelle quali si descrive la struttura che deve avere ogni documento.
    \item Gestione della configurazione: 
    \item Verifica: La verifica della documentazione del progetto software è un processo che mira a verificare l'accuratezza e la completezza della documentazione del progetto;
    \item Validazione: processo che mira a confermare che la documentazione rappresenti effettivamente il progetto software e che soddisfi i requisiti e le specifiche del progetto;
\end{enumerate}
\textbf{Processi organizzativi:}
Comprende la definizione di processi organizzativi.
\\\\
I processi organizzativi inclusi nel nostro progetto sono:
\begin{enumerate}
    \item Gestione organizzativa: riguarda l'organizzazione all'interno del gruppo ovvero la ripartizione dei compiti e le attivitá che ogni figura professionale é tenuta a svolgere fino al 
    completamento del progetto;
    \item Gestione infrastrutture: cruciale per garantire la disponibilità, l'affidabilità e la sicurezza delle risorse necessarie per lo sviluppo e l'esecuzione del software.
\end{enumerate}
\textbf{Standard di qualità ISO/IEC 9126:} La norma ISO/IEC 9126 definisce uno standard di qualità per il software. 
Si basa su sei criteri di qualità: affidabilità, usabilità, efficienza, mantenibilità, conformità e funzionalità. 
Ogni criterio è suddiviso in sottocriteri che possono aiutare gli sviluppatori a valutare e migliorare la qualità del loro software. 
Il documento offre anche consigli generali su come misurare e monitorare questi criteri di qualità.
\\\\
\textbf{Standard di qualità ISO/IEC 12207:1995:} ISO/IEC 12207:1995 è uno standard internazionale rilasciato dall'International Organization 
for Standardization (ISO) e dall'International Electrotechnical Commission (IEC) nel 1995. 
Lo standard definisce un processo di sviluppo del software che può essere utilizzato da tutti i team di sviluppo software per garantire 
un prodotto di qualità. Il processo di sviluppo può essere applicato a qualsiasi tipo di progetto software, indipendentemente dal 
linguaggio di programmazione, dalle dimensioni, dai requisiti o dall'ambiente. 
Lo standard fornisce una struttura coerente per l'organizzazione, la pianificazione e l'implementazione di un progetto, 
inoltre, definisce i processi necessari per assicurare che il prodotto consegnato soddisfi i requisiti di qualità stabiliti dal cliente.