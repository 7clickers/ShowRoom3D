\subsection{Documentazione}
Il processo di documentazione ha come scopo quello di fornire un insieme di documenti che descrivono in dettaglio il progetto software, comprese le sue funzionalità, i requisiti, l'architettura, il design, i processi di sviluppo e i risultati. 
Questa documentazione serve come riferimento per lo sviluppo del software e aiuta a garantire la coerenza e la completezza del progetto. 
La documentazione del progetto software è un elemento essenziale e deve essere costantemente aggiornata.
\subsubsection{Ciclo di vita di un documento}
\begin{itemize}
    \item \textbf{Creazione}: il documento viene creato su un nuovo branch, utilizzando un template uguale per tutti i documenti. 
    Viene aggiunta l’intestazione e il registro delle modifiche;
    \item \textbf{Stesura}: si procede con la stesura delle varie sezioni, tracciando i cambiamenti nel registro delle modifiche; 
    \item \textbf{Verifica}: ogni sezione viene verificata dai verificatori (per maggiori specifiche guardare la 
    sezione x.x.x Verifica della documentazione);
    \item \textbf{Approvazione}: una volta che tutte le sezioni sono state verificate si procede 
    con l’approvazione del documento, che viene effettuata dal responsabile nel seguente modo:
    \begin{itemize}
        \item Viene aperta una Pull Request di approvazione;
        \item Il responsabile rilegge il documento: se riscontra ulteriori problematiche, segnala ai verificatori le eventuali modifiche da apportare;
        \item Se sono richieste delle modifiche i verificatori si occupano di apportarle;
        \item Il responsabile crea una nuova riga e compila i campi nelle colonne corrispondenti del registro delle modifiche 
        inserendo: l'ultima versione secondo le norme di versionamento, la data, il proprio nome, il suo ruolo e la voce "Approvazione" nell'ultima colonna.
    \end{itemize}
    Per i \textit{Verbali} si effettua questa pratica non appena il file prodotto viene 
    verificato, mentre per tutti gli altri documenti prima di una consegna.
\end{itemize}
\subsubsection{Struttura generale}
Ogni documento deve presentare le seguenti sezioni nell'ordine in cui vengono presentate:
\begin{itemize} 
    \item \textbf{Intestazione}:
    contiene:
    \begin{itemize} 
        \item Logo compreso di motto;
        \item Indirizzo email di gruppo; 
        \item Titolo;
        \item Tabella contenente le informazioni generali:
        \begin{itemize}
            \item Versione;
            \item Stato;
            \item Uso;
            \item Approvazione\textsubscript{g}: indica il responsabile di progetto che ha approvato il documento; 
            \item Redazione: elenco dei collaboratori che hanno partecipato alla stesura del documento;
            \item Verifica\textsubscript{g}: elenco dei verificatori che hanno verificato il documento;
            \item Distribuzione: elenco delle persone o organizzazioni a cui è destinato il documento.
        \end{itemize}
        \item Breve descrizione del documento .
    \end{itemize}
    \item \textbf{Registro delle modifiche}:
    tabella che identifica ogni versione del documento indicandone:
    \begin{itemize} 
        \item Versione;
        \item Data;
        \item Autore;
        \item Ruolo;
        \item Descrizione: la descrizione deve essere breve. Nel caso di aggiunte o modifiche si deve indicare 
        il nome della sezione che è stata aggiunta o modificata. 
        I nomi delle sezioni vanno riportati uguali a come sono scritti nell'indice, 
        senza virgolette.
    \end{itemize}
    \item \textbf{Indice}:
    elenco ordinato dei titoli dei capitoli, ovvero delle varie parti di cui si compone il documento;
    
    \item \textbf{Elenco delle figure}: sezione che fornisce l'elenco delle immagini presenti nel documento. 
    Se il documento non presenta immagini, questa sezione sarà omessa di conseguenza;  
    \item \textbf{Elenco delle tabelle}: sezione che fornisce l'elenco delle tabelle presenti nel documento. 
    Se il documento non presenta tabelle, questa sezione sarà omessa di conseguenza; 
    \item \textbf{Contenuto}:
    varia a seconda del tipo di documento.
\end{itemize}
\subsubsection{Convenzioni}
Le convenzioni di seguito riportate vengono applicate a tutti i documenti.
Esse rendono i documenti stilati omogenei tra loro contribuendo a rendere il progetto professionale.

\paragraph{Date}
Le date devono rispettare il seguente formato: \textbf{yyyy-mm-dd}
All'interno delle tabelle il formato deve essere il seguente: \textbf{dd-mm-yy}
\paragraph{Nomi di persona}
All'interno dei documenti i nomi di persona rispetteranno l'ordine nome seguito dal cognome della persona menzionata.

\paragraph{Elenchi puntati}
Gli elenchi puntati devono rispettare le seguenti regole:
\begin{itemize} 
    \item Ogni elemento dell'elenco deve iniziare con la lettera maiuscola;
    \item Ogni elemento dell'elenco deve terminare con ";" ad eccezione dell'ultimo elemento
    che deve terminare con "."; 
    \item Dopo i due punti la frase deve iniziare con la lettera minuscola.
\end{itemize}

\paragraph{Stile del testo}
\begin{itemize} 
    \item \textbf{Grassetto}: stile utilizzato per i titoli delle sezioni e per i primi termini degli elenchi puntati;
    \item \textbf{Corsivo}: viene utilizzato per citare il nome dei documenti, ad esempio \textit{Piano di Progetto}. 
\end{itemize}

\paragraph{Immagini}
Le immagini sono raccolte nella cartella “images”, e sono inserite sempre con una didascalia descrittiva posizionata sotto l’immagine.

\paragraph{Tabelle} 
Le tabelle sono provviste di didascalia descrittiva posizionata sotto alla tabella. La tabella contenente il registro delle modifiche è l’unica che fa da eccezione a questa regola.

\paragraph{\textit{Glossario}}
All'interno della documentazione si possono trovare dei termini che possono risultare ambigui a seconda del contesto, o non conosciuti dagli utilizzatori.\\\
Per ovviare ad incomprensioni si è deciso di stilare un elenco di termini di interesse accompagnati da una descrizione del loro significato.\\
I termini presenti all'interno dei documenti che necessitano di una descrizione vengono indicati con il pedice 'g' come nell'esempio seguente: termine.
É quindi possibile consultare il \textit{Glossario} per reperire tale descrizione.
\\
Ogni componente del gruppo all'inserimento di un termine ritenuto ambiguo deve preoccuparsi di aggiornare il \textit{Glossario}.

Per aggiornare il \textit{Glossario} si devono inserire i nuovi termini nel file .tex nella cartella corrispondente all'iniziale del termine situata al percorso
\textbf{\textit{latex\textsubscript{g}/esterni/doc\_esterna/\textit{Glossario}/res/sections/alphabet/}}.
É stato creato uno script che scansiona un documento di interesse per inserire in automatico il pedice sui termini contenuti nel \textit{Glossario}.

Il componente del gruppo che inserisce all'interno del \textit{Glossario} un nuovo termine deve aggiungere nel file .tex il segnaposto \%parola\% dopo la subsection (senza spazi) che racchiude il termine per permetterne il riconoscimento da parte dello script.
\\
Il \textit{Glossario} ordina i termini in ordine alfabetico in modo da permetterne una facile e veloce ricerca.
\subsubsection{Strumenti per la stesura}
\begin{itemize} 
    \item \textbf{LaTeX}: è un linguaggio di marcatura per la preparazione di testi, basato sul 				  		  programma di composizione tipografica TEX.\\
	Nel branch documentation  si possono trovare i file .pdf prodotti e la cartella “latex”. La cartella latex contiene tre cartelle interne:
	\begin{itemize}
	\item \textbf{esterni e interni}: contengono file .tex di documentazione esterna ed interna come ad esempio i \textit{Verbali} o altra documentazione esterna/interna:
	\begin{itemize}
		\item La cartella config contiene i file .tex con le parti fisse dei documenti (intestazione, registro delle modifiche, tracciamento dei temi affrontati) che vengono modificati con i dati del documento specifico;
		\item La cartella res/sections contiene i file .tex con il contenuto vero e proprio (sezioni del documento) che viene redatto in maniera libera dal redattore;
		\item Un file col nome del documento pdf con estensione .tex che viene compilato per produrre il file pdf.
	\end{itemize}
	 \item \textbf{template}: contiene file .tex di base utilizzati secondo necessità per comporre i documenti:
	 \begin{itemize}
	 \item changelox.tex è il file di template che serve per scrivere il registro delle modifiche;
	 \item package.tex è il file che contiene tutti gli usepackage\textsubscript{g};
	 \item titlepage.tex è il file di template che contiene la configurazione della pagina iniziale di ogni documento;  
	 \item tracking.tex è il file di template che contiene il tracciamento dei temi affrontati nel documento.
	 \end{itemize}
	\end{itemize}
\end{itemize}

\subsubsection{Documentazione interna}
La documentazione interna comprende tutti i documenti che contengono informazioni utili principalmente per il gruppo, e che vengono quindi consultati di conseguenza. 
    \paragraph{\textit{Diario di Bordo}}
Il \textit{Diario di Bordo} è un documento informale ad uso esterno e permette di interagire con il Professore (committente) in modo da aggiornarlo sullo stato di avanzamento del progetto settimanalmente ed è usato 
anche per chiedere chiarimenti sui temi in cui riscontriamo dubbi o domande.
Fino al termine delle lezioni il \textit{Diario di Bordo} veniva redatto settimanalmente dal responsibile che presentava in classe prima dello svolgimento 
della lezione.
Dal termine delle lezioni il \textit{Diario di Bordo} viene redatto settimanalmente dal responsabile e caricato in una cartella denominata diari\_di\_bordo presente nel drive di gruppo.
\\\\
La cartella diari\_di\_bordo si trova al link: \href{https://drive.google.com/drive/u/1/folders/1a0kAZuOSsDEp_AaQUb6OQJ4XQyZ--RCN}{\\https:\/\/drive.google.com\/drive\/u\/1\/folders\/1a0kAZuOSsDEp\_AaQUb6OQJ4XQyZ\-\-RCN}
\\\\
È stato fornito l'accesso in lettura alla cartella al Professore che viene notificato tramite mail (deve contenere il link al \textit{Diario di Bordo} di interesse) al caricamento di un \textit{Diario di Bordo}.
Il formato dei diari di bordo condivisi nella cartella è .pdf.
Inoltre tutti i diari di bordo una volta redatti vengono inviati anche al canale discord\textsubscript{g} "diari-di-bordo-file" in modo da avere una duplice copia a fronte di 
qualsiasi imprevisto dato che i diari di bordo non vengono inseriti all'interno del repository di gruppo.
\\\\
\textbf{Struttura \textit{Diario di Bordo}} 
\\\\
La struttura del \textit{Diario di Bordo} non rispetta quella indicata nella sezione 3.1.1.2 relativa alla struttura generale della documentazione.
Sono state fissate delle regole per la stesura dei diari di bordo per fornire delle presentazioni il più possibile simili tra loro 
senza troppi vincoli superflui per tali documenti.
\\\\
Il \textbf{font} utilizzato per la stesura del documento è Helvetica Neue con dimensione 22pt.
Il nome del file per ogni diario di bordo deve rispettare la codifica \textbf{AAAA/MM/GG} in modo da permettere l'ordinamento
cronologico utilizzando i filtri di ricerca di Google Drive.
\\\\
Il \textit{Diario di Bordo} si divide in 4 slides che possono essere raggruppate in 3 se l'elenco degli obiettivi raggiunti e quelli futuri ci stanno
in una singola slide.
Le slides che compongono il \textit{Diario di Bordo} sono:
\begin{itemize}
    \item Intestazione; 
    \item Obiettivi raggiunti;
    \item Obiettivi futuri;
    \item Domande.
\end{itemize}
Descrizione slides:
\begin{itemize}
    \item \textbf{Intestazione}: presenta nella parte centrale il logo del gruppo compreso di slogan, in basso centrale la data relativa al \textit{Diario di Bordo} e l'anno accademico;
    \item \textbf{Obiettivi raggiunti}: presenta in alto centrale il titolo "Obiettivi raggiunti", a sinistra un elenco puntato con gli obiettivi raggiunti
    della settimana, in alto a destra il logo del gruppo compreso di slogan ed in basso a destra la data del \textit{Diario di Bordo};
    \item \textbf{Obiettivi futuri}: presenta in alto centrale il titolo "Obiettivi futuri", a sinistra un elenco puntato con gli obiettivi raggiunti
    della settimana ed in alto a destra il logo del gruppo compreso di slogan;
    \item \textbf{Domande}: presenta in alto centrale il titolo "Domande", a sinistra un elenco puntato con le domande da porre al Professore, in alto a destra il logo del gruppo compreso di slogan ed in basso 
    a destra la data del \textit{Diario di Bordo}.
\end{itemize}
    \paragraph{\textit{Verbale}}
Il \textit{Verbale} ha lo scopo di rendicontare ciò che viene detto durante una riunione.
Rispetta la struttura generale.
In aggiunta presenta:
\begin{itemize} 
    \item \textbf{Informazioni Generali}
    contiene:
    \begin{itemize} 
        \item Luogo;
        \item Data;
        \item Ora;
        \item Partecipanti.
    \end{itemize}
\item \textbf{Tabella tracciamento temi affrontati}:
tabella che riassume i punti salienti della riunione indicandone:
    \begin{itemize} 
        \item Codice: ha il formato V\textbf{X} \textbf{Y}.\textbf{Z} dove X indica la tipologia di \textit{Verbale}, Y indica il numero di \textit{Verbale} (incrementale rispetto agli altri verbali);
        e Z indica il numero dell'argomento trattato (incrementale rispetto agli altri argomenti del \textit{Verbale});
        \item Descrizione: breve descrizione di uno specifico argomento trattato.
    \end{itemize}
\end{itemize}
    \paragraph{\textit{Studio di Fattibilità}}
Lo \textit{Studio di Fattibilità} ha lo scopo di valutare i pro ed i contro dei capitolati proposti in modo da scegliere il più vantaggioso al quale candidarsi.
La scelta del capitolato viene fatta sulla base di diverse considerazioni.
Vengono valutati:
\begin{itemize}
    \item \textbf{Risorse}: in termini di budget, tempo e personale;
    \item \textbf{Previe Conoscenze}: un capitolato potrebbe rivelarsi più o meno difficile da realizzare a seconda delle conoscenze del contesto applicativo; 
    \item \textbf{Guadagno}: un capitolato potrebbe venire scartato per scarso guadagno netto al termine del lavoro commissionato (non è il nostro caso ma 
andrebbe tenuto in considerazione in ambito lavorativo);
    \item \textbf{Aspetti psicologici}: avendo la possibilità di scegliere tra capitolati (fatto che capita raramente in un reale ambito lavorativo) di pari interesse siamo portati a scegliere il contesto applicativo che ci entusiasma maggiormente.
    Questo aspetto potrebbe avere un impatto positivo sulla produttività del gruppo.
\end{itemize}

Il documento tiene in considerazione tutti i capitolati proposti per non scartare un capitolato per le sensazioni soggettive dei componenti del gruppo.
Non tenendo in considerazione ogni capitolato, infatti, si potrebbe scartare un capitolato di forte interesse senza rendersene conto.
    \documentclass[a4paper]{article}
\usepackage[normalem]{ulem}
\usepackage[font=small,labelfont=bf]{caption}

% impostazioni generali
%Tutti gli usepackage vanno qui
\usepackage{geometry}
\usepackage[italian]{babel}
\usepackage[utf8]{inputenc}
\usepackage{tabularx}
\usepackage{longtable}
\usepackage{hyperref}
\usepackage{enumitem}
\usepackage{array} 
\usepackage{booktabs}
\newcolumntype{M}[1]{>{\centering\arraybackslash}m{#1}}
\usepackage[toc]{appendix}

\hypersetup{
	colorlinks=true,
	linkcolor=blue,
	filecolor=magenta,
	urlcolor=blue,
}
% Numerazione figure
\let\counterwithout\relax
\let\counterwithin\relax
\usepackage{chngcntr}

% distanziare elenco delle figure e delle tabelle
\usepackage{tocbasic}
\DeclareTOCStyleEntry[numwidth=3.5em]{tocline}{figure}% for figure entries
\DeclareTOCStyleEntry[numwidth=3.5em]{tocline}{table}% for table entries


%\counterwithout{table}{section}
%\counterwithout{figure}{section}
\captionsetup[table]{font=small,skip=5pt} 

\usepackage[bottom]{footmisc}
\usepackage{fancyhdr}
\setcounter{secnumdepth}{4}
\usepackage{amsmath, amssymb}
\usepackage{array}
\usepackage{graphicx}

\usepackage{ifthen}

\usepackage{float}
\restylefloat{table}

\usepackage{layouts}
\usepackage{url}
\usepackage{comment}
\usepackage{eurosym}

\usepackage{lastpage}
\usepackage{layouts}
\usepackage{eurosym}

\geometry{a4paper,top=3cm,bottom=4cm,left=2.5cm,right=2.5cm}

%Comandi di impaginazione uguale per tutti i documenti
\pagestyle{fancy}
\lhead{\includegraphics[scale=0.1]{../../../template/images/logo_no_motto.jpeg}}
%Titolo del documento
\rhead{\doctitle{}}
%\rfoot{\thepage}
\cfoot{Pagina \thepage\ di \pageref{LastPage}}
\setlength{\headheight}{35pt}
\setcounter{tocdepth}{5}
\setcounter{secnumdepth}{5}
\renewcommand{\footrulewidth}{0.4pt}

% multirow per tabelle
\usepackage{multirow}

% Permette tabelle su più pagine
%\usepackage{longtable}


% colore di sfondo per le celle
\usepackage[table]{xcolor}

%COMANDI TABELLE
\newcommand{\rowcolorhead}{\rowcolor[HTML]{007c95}}
\newcommand{\captionline}{\rowcolor[HTML]{FFFFFF}} %comando per le caption delle tabelle
\newcommand{\cellcolorhead}{\cellcolor[HTML]{007c95}}
\newcommand{\hlinetable}{\arrayrulecolor[HTML]{007c95}\hline}

%intestazione
% check for missing commands
\newcommand{\headertitle}[1]{\textbf{\color{white}#1}} %titolo colonna
\definecolor{pari}{HTML}{b1dae3}
\definecolor{dispari}{HTML}{d7f2f7}

% comandi glossario
\newcommand{\glo}{$_{G}$}
\newcommand{\glosp}{$_{G}$ }


%label custom
\makeatletter
\newcommand{\uclabel}[2]{%
	\protected@write \@auxout {}{\string \newlabel {#1}{{#2}{\thepage}{#2}{#1}{}} }%
	\hypertarget{#1}{#2}
}
\makeatother

%riportare pezzi di codice
\definecolor{codegray}{gray}{0.9}
\newcommand{\code}[1]{\colorbox{codegray}{\texttt{#1}}}



% dati relativi alla prima pagina
% Configurazione della pagina iniziale
\newcommand{\doctitle}{Glossario}
\newcommand{\docdate}{21 Marzo 2023}
\newcommand{\rev}{2.0.0}
\newcommand{\stato}{Approvato}
\newcommand{\uso}{Esterno}
\newcommand{\approv}{Mirko Stella}
\newcommand{\red}{Giacomo Mason \\ & Mirko Stella}
\newcommand{\ver}{Giacomo Mason \\ & Gabriele Mantoan}
\newcommand{\dest}{\textit{Seven Clickers}
									  \\ Prof. Vardanega Tullio 
									   \\ Prof. Cardin Riccardo}
\newcommand{\describedoc}{Glossario del gruppo \textit{Seven Clickers}}


 % editare questo

\makeindex

%comando per far andare a capo i paragrafi
\makeatletter
\renewcommand\paragraph{
\@startsection {paragraph}{4}{0mm}{-\baselineskip}{.5\baselineskip}{\normalfont \normalsize \bfseries }}
\makeatother

\newcommand*{\fullref}[1]{\hyperref[{#1}]{\ref*{#1} \nameref*{#1}}}

\begin{document}
\counterwithin{table}{section}

% Prima pagina
\thispagestyle{empty}
\renewcommand{\arraystretch}{1.3}

\begin{titlepage}
	\begin{center}
		
	\includegraphics[scale = 0.40]{../../../template/images/logo.jpeg}
	\\[1cm]
	\href{mailto:7clickersgroup@gmail.com}		      	
	{\large{\textit{7clickersgroup@gmail.com} } }\\[2.5cm]
	\Huge \textbf{\doctitle} \\[1cm]
	 \large
			 \begin{tabular}{r|l}
                        \textbf{Versione} & \rev{} \\
                        \textbf{Stato} & \stato{} \\
                        \textbf{Uso} & \uso{} \\                         
                        \textbf{Approvazione\textsubscript{g}} & \approv{} \\                      
                        \textbf{Redazione} & \red{} \\ 
                        \textbf{Verifica\textsubscript{g}} &  \ver{} \\                         
                        \textbf{Distribuzione} & \parbox[t]{5cm}{ \dest{} }
                \end{tabular} 
                \\[3.3cm]
                \large \textbf{Descrizione} \\ \describedoc{} 
     \end{center}
\end{titlepage}

% Diario delle modifiche
\section*{Registro delle modifiche}

\newcommand{\changelogTable}[1]{
	 

\renewcommand{\arraystretch}{1.5}
\rowcolors{2}{pari}{dispari}
\begin{longtable}{ 
		>{\centering}M{0.07\textwidth} 
		>{\centering}M{0.13\textwidth}
		>{\centering}M{0.20\textwidth}
		>{\centering}M{0.17\textwidth} 
		>{\centering\arraybackslash}M{0.30\textwidth} 
		 }
	\rowcolorhead
	\headertitle{Vers.} &
	\centering \headertitle{Data} &	
	\headertitle{Autore} &
	\headertitle{Ruolo} & 
	\headertitle{Descrizione} 
	\endfirsthead	
	\endhead
	
	#1

\end{longtable}
\vspace{-2em}

}



\changelogTable{
	0.1.0 & 07-01-23 & Marco Brigo \\ Elena Pandolfo & Verificatori & Verifica\textsubscript{g} documento \\
	0.0.1 & 05-01-23 & Mirko Stella & Responsabile & Stesura documento \\
} % editare questo
\pagebreak

% Indice
{
    \hypersetup{linkcolor=black}
    \tableofcontents
    \listoffigures %elenco figure
}
\pagebreak

% Contenuto
\section{Introduzione}
\subsection{Scopo del documento}
\textit{Norme di Progetto} è il documento che definisce le regole e gli standard che devono essere seguiti durante il ciclo di vita del prodotto.
\subsection{Scopo del prodotto}
Il prodotto in questione nasce dalla necessità dell'azienda SanMarco Informatica di fornire una soluzione agli sprechi
derivati dall'adozione di uno ShowRoom tradizionale proponendo uno ShowRoom3D che sia ugualmente o ancora più immersivo.
\section{Processi Primari}
I processi primari devono essere seguiti durante il ciclo di vita del software.
Nel nostro il ciclo di vita del software non comprende l'installazione e la manutenzione ma termina con lo sviluppo.
Il progetto in questione è da considerarsi un progetto didattico.
\input{res/sections/processi_primari/fornitura.tex}
\input{res/sections/processi_primari/sviluppo/sviluppo_main.tex}
\pagebreak
\section{Processi di Supporto}
I processi di supporto supportano quelli primari in modo da renderli più efficienti ed efficaci.
\input{res/sections/processi_supporto/documentazione/documentazione_main.tex}
\input{res/sections/processi_supporto/gestione_configurazione.tex}
\input{res/sections/processi_supporto/gestione_qualita.tex}
\input{res/sections/processi_supporto/verifica\textsubscript{g}/verifica_main.tex}
\input{res/sections/processi_supporto/validazione\textsubscript{g}.tex}

\pagebreak
\section{Processi Organizzativi}
I processi organizzativi mirano a gestire i processi e il loro miglioramento, l'organizzazione degli strumenti 
di supporto e la gestione del personale.
\input{res/sections/processi_organizzativi/gestione_organizzativa/gestione_organizzativa_main.tex}
\input{res/sections/processi_organizzativi/gestione_infrastrutture/gestione_infrastrutture_main.tex}

\pagebreak
\section{Standard di qualità ISO/IEC 9126}
Questa sezione descrive in dettaglio lo standard ISO/IEC 9126 utilizzato dal gruppo. Le norme di questo standard descrivono un modello di qualità del software, definiscono le caratteristiche che lo determinano e propongono metriche per la misurazione.
Le norme constano di quattro parti:
\begin{itemize}
\item Parte 1: Metriche per la qualità esterna
\item Parte 2: Metriche per la qualità interna
\item Parte 3: Modello della qualità del software
\item Parte 4: Metriche per la qualità in uso
\end{itemize}

\input{res/sections/isoiec_9126/qualità_esterne.tex}
\input{res/sections/isoiec_9126/qualità_interne.tex}
\input{res/sections/isoiec_9126/modello_qualità.tex}
\input{res/sections/isoiec_9126/qualità_in_uso.tex}

\pagebreak
\section{Standard di qualità ISO/IEC 12207:1995}
Questa sezione descrive in dettaglio lo standard ISO/IEC 12207:1995 scelto dal gruppo per garantire la qualità dei processi del ciclo di vita di un software. I processi dello standard contengono attività e compiti che devono essere applicati durante l'acquisizione di un sistema software. Esso contiene tre tipologie di processi che sono: 
\begin{itemize}
\item Processi primari
\item Processi di supporto
\item Processi organizzativi
\end{itemize}

\input{res/sections/isoiec_12207_1995/processi_primari.tex}
\input{res/sections/isoiec_12207_1995/processi_supporto.tex}
\input{res/sections/isoiec_12207_1995/processi_organizzativi.tex}
\pagebreak
\section{Metriche di qualità}

\input{res/sections/metriche/metriche_processo.tex}
\input{res/sections/metriche/metriche_prodotto.tex}

\pagebreak

\end{document}

    \subsubsection{Documentazione esterna}
La documentazione esterna comprende tutti i documenti che interessano anche al proponente e al committente.
    \documentclass[10pt]{article}

\usepackage{geometry}
\usepackage{fancyhdr,graphicx}
\usepackage{hyperref}
\usepackage{eurosym}

\geometry{a4paper,top=2.5cm,bottom=2.5cm,left=2cm,right=2cm}

\fancypagestyle{firstpage}{%
  \fancyhf{}% Clear header/footer
  \renewcommand{\headrulewidth}{0pt}%
}

\fancypagestyle{otherpages}{%
  \fancyhf{}% Clear header/footer
  \renewcommand{\headrulewidth}{1pt}%
}

\AtBeginDocument{\thispagestyle{firstpage}}
\pagestyle{otherpages}

\setlength{\parindent}{0pt}
\setlength{\parskip}{1ex}

\begin{document}

\noindent\begin{minipage}{0.5\textwidth}% adapt widths of minipages to your needs
\includegraphics[width=9cm]{images/logo.jpeg}
\end{minipage}%
\hfill%
\begin{minipage}{4cm}
\includegraphics[width=2.3cm]{images/uni.png}
\end{minipage}

\bigskip\bigskip

\begin{tabular}{ @{} l  }
  Gruppo \textit{Seven Clickers} \\ 
  E-mail: \textit{\href{mailto:7clickersgroup@gmail.com}{7clickersgroup@gmail.com} }\\ 
  Corso di Ingegneria del Software AA 2022/2023 \\
  17 Marzo 2023
\end{tabular}

\bigskip
\hfill
\begin{tabular}{ l @{} }
Prof. Vardanega Tullio\\
Prof. Cardin Riccardo\\
Università degli Studi di Padova\\
Dipartimento di Matematica\\
Via Trieste, 63\\
35121 Padova
\end{tabular}

\bigskip

Egregio Prof. Vardanega Tullio,\\
Egregio Prof. Cardin Riccardo,\\

\bigskip

Con la presente il gruppo \textit{Seven Clickers} intende comunicarVi la partecipazione al secondo passaggio della revisione di avanzamento RTB,
al fine di esporvi l’avanzamento dello sviluppo del progetto, denominato:

\begin{center}
  \textbf{ShowRoom3D}
\end{center}

proposto dall’azienda \textbf{Sanmarco Informatica}.


Si allegano i seguenti documenti di interesse:
\begin{itemize}
  \item \textit{\textit{Studio di Fattibilità}}
  \item \textit{Glossario} vx.x.x
  \item \textit{Analisi dei Requisiti} v2.0.0
  \item \textit{Norme di Progetto} v1.0.0
  \item \textit{Piano di Progetto} v2.0.0
  \item \textit{Piano di Qualifica} v1.0.0
\end{itemize}

Inoltre sono allegati anche i verbali esterni ed interni:

\begin{itemize}
  \item \textit{Verbale} esterno del 25-10-2022
  \item \textit{Verbale} esterno del 17-11-2022
  \item \textit{Verbale} esterno del 11-01-2023
  \item \textit{Verbale} esterno del 18-01-2023
  \item \textit{Verbale} esterno del 17-02-2023
  \item \textit{Verbale} interno del 19-10-2022
  \item \textit{Verbale} interno del 25-10-2022
  \item \textit{Verbale} interno del 26-10-2022
  \item \textit{Verbale} interno del 04-11-2022
  \item \textit{Verbale} interno del 09-11-2022
  \item \textit{Verbale} interno del 16-11-2022
  \item \textit{Verbale} interno del 23-11-2022
  \item \textit{Verbale} interno del 01-12-2022
  \item \textit{Verbale} interno del 07-12-2022
  \item \textit{Verbale} interno del 14-12-2022
  \item \textit{Verbale} interno del 04-01-2023
  \item \textit{Verbale} interno del 25-01-2023
  \item \textit{Verbale} interno del 01-02-2023
  \item \textit{Verbale} interno del 08-02-2023
  \item \textit{Verbale} interno del 24-02-2023
  \item \textit{Verbale} interno del 28-02-2023
\end{itemize}

Il gruppo stima di consegnare il prodotto\textsubscript{g} entro il 03-05-2023 con un preventivo di 13975\euro{}  come
specificato nel documento \textit{Piano di Progetto} v2.0.0.


Cordiali saluti,

\vspace{15pt}

\hfill
\begin{tabular}{ l @{} }
Rino Sincic\\
\textit{Responsabile di Progetto}\\
\includegraphics[width=2.3cm]{images/Rino_Sincic_firma.png}
\end{tabular}

\end{document}
    \paragraph{\textit{Piano di Progetto}}

Il documento piano di progetto ha lo scopo di aiutare il gruppo nella gestione delle risorse a disposizione per portare a termine il progetto entro la data decisa.
Il documento ha anche la funzione di monitorare l'avanzamento del progetto in modo da poter applicare miglioramenti continui e azioni correttive
basandosi sull'esperienza ottenuta da pianificazioni precedenti.
\\\\
\textbf{Struttura piano di progetto}
\\\\
La struttura del piano di progetto segue la struttura generale.
In aggiunta il contenuto del documento si compone di:
\begin{enumerate}
    \item Analisi dei rischi;
    \item Pianificazione;
    \item Preventivo.
\end{enumerate}
\textbf{NOTA:} ogni tabella o immagine all'interno del documento verrá indicata anche nell'indice del documento stesso in modo che sia rintracciabile con 
facilitá.
\\\\
\textbf{Descrizione delle sezioni:}
\\\\
\textbf{Analisi dei rischi:} vengono riportati in forma tabellare i rischi a cui si va incontro aggiudicandosi il capitolato.
Ogni rischio fa riferimento ad una categoria di rischi precisa e viene indicato con un nome,una probabilitá che si verifichio,
un grado che indica l'impatto negativo che puó
comportare es una breve descrizione su come affrontare il rischio nel momento in cui dovesse presentarsi.
\\\\
\textbf{Pianificazione:} la sezione dedicata alle pianificazioni ha lo scopo di indicare l'inizio e la fine di un periodo di pianificazione e 
di fornire la suddivisione e l'organizzazione delle attivitá all'interno di tale periodo.
Viene fornito un elenco con descrizione delle attivitá che si andranno a svolgere e viene diviso il periodo di pianificazione in sottoperiodi 
ciascuno con un inizio,una fine ed indicando cosa verrá fatto.
Infine é presente un diagramma di Gantt che illustra graficamente l'ordine temporale delle attivitá da svolgere tenendo conto di 
eventuali margini temporali dovuti ad imprevisti vari.
\\\\
\textbf{Preventivo:} include una sezione iniziale in cui si vanno ad indicare in forma tabellare le risore umane disponibili all'inizio del progetto e 
come queste risorse andranno suddivise. Successivamente viene fornito,sempre in forma tabellare,il costo totale calcolato in base alle ore 
di impegno dei componenti del gruppo ed al ruolo che dovranno ricoprire (ogni ruolo ha un costo orario).
Le tabelle riportate produrranno come risultato la conclusione del preventivo che comprende il costo totale calcolato e la data di fine progetto.
É presente una sezione (dettaglio periodi) che prevede i preventivi di ciascun periodo di pianificazione.
In questa sezione sono fornite le tabelle con le ore che ciascun componente del gruppo dovrá svolgere con relativi ruoli associati ed una tabella con i costi
derivati da tali ruoli.
Nel caso del nostro progetto viene prevista una rotazione dei ruoli a scopo didattico perció ogni componente userá le sue ore di impegno in modo diverso 
a seconda della rotazione dei ruoli.
    \paragraph{\textit{Piano di Qualifica}}
Il \textit{Piano di Qualifica} è un documento che contiene le descrizione dei processi e i controlli necessari per garantire la qualità del prodotto. 
Il \textit{Piano di Qualifica} include: definizioni di processo, standard, metodologie e obiettivi di qualità; requisiti per la 
qualifica del software; linee guida per la formazione e la documentazione del progetto; test di qualifica e di validazione.
\\\\
\textbf{Struttura \textit{Piano di Qualifica}}
\\\\
La struttura del \textit{Piano di Qualifica} segue la struttura generale.
In aggiunta il contenuto del documento si compone di:
\begin{enumerate}
    \item Introduzione;
    \item Qualitá del processo;
    \item Qualitá del prodotto.
\end{enumerate}
\textbf{Introduzione:} contiene una breve descrizione dello scopo del documento e del progetto.
Successivamente é presente una sezione in cui vengono riportati i riferimenti informativi 
con i relativi link alle risorse che abbiamo utilizzato per garantire la qualitá del prodotto.
\\\\
\textbf{Qualitá del processo:} Viene indicato lo standard utilizzato per garantire la qualitá di processo.
Per ogni tipo di processo (primari,supporto,organizzativi) vengono indicati in forma tabellare il nome del processo,la descrizione e le metriche utilizzate
indicate utilizzando un codice identificativo univoco per ognuna di esse.
Sucessivamente viene riportata una tabella che fornisce per ogni metrica i nomi ed i valori significativi previsti.
I dettagli su come vengono applicate tali metriche vengono poi descritti nelle \textit{Norme di Progetto}.
\\\\
\textbf{Qualitá del prodotto:} Viene indicato lo standard utilizzato per garantire la qualitá di prodotto.
Trattandosi di un prodotto software si fa riferimento ad uno standard dedicato alle applicazioni software.
Vengono indicati in forma tabellare gli obiettivi,la descrizione e il codice che contraddistingue le metriche utilizzate per soddisfare gli obiettivi.
Successivamente viene riportata una tabella che per ogni codice ne descrive il significato ed i valori minimi ed ottimi per cui la metrica é rispettata.
I dettagli su come vengono applicate tali metriche vengono poi descritti nelle \textit{Norme di Progetto}.
    \paragraph{\textit{Glossario}}
All'interno della documentazione si possono trovare dei termini che possono risultare ambigui a seconda del contesto, o non conosciuti dagli utilizzatori.\\\
Per ovviare ad incomprensioni si è deciso di stilare un elenco di termini di interesse accompagnati da una descrizione del loro significato.\\
I termini presenti all'interno dei documenti che necessitano di una descrizione vengono indicati con il pedice 'g' come nell'esempio seguente: termine.
É quindi possibile consultare il \textit{Glossario} per reperire tale descrizione.
\\
Ogni componente del gruppo all'inserimento di un termine ritenuto ambiguo deve preoccuparsi di aggiornare il \textit{Glossario}.

Per aggiornare il \textit{Glossario} si devono inserire i nuovi termini nel file .tex nella cartella corrispondente all'iniziale del termine situata al percorso
\textbf{\textit{latex/esterni/doc\_esterna/Glossario/res/sections/alphabet/}}.
É stato creato uno script che scansiona un documento di interesse per inserire in automatico il pedice sui termini contenuti nel \textit{Glossario}.

Il componente del gruppo che inserisce all'interno del \textit{Glossario} un nuovo termine deve aggiungere nel file .tex il segnaposto \%parola\% dopo la subsection (senza spazi) che racchiude il termine per permetterne il riconoscimento da parte dello script.
\\
Il \textit{Glossario} ordina i termini in ordine alfabetico in modo da permetterne una facile e veloce ricerca.








\subsubsection{Metriche}
Per perseguire la qualità sulla documentazione prodotta si è deciso di adottare le seguenti metriche:
\paragraph{Indice di Gulpease}
Si tratta dell'indice di leggibilità di un testo tarato sulla lingua italiana.
I risultati sono compresi tra 0 e 100, dove il valore "100" indica la leggibilità più alta e "0" la leggibilità più bassa. Ai seguenti valori si associano i seguenti significati:
\begin{itemize}
\item inferiore a 80 sono difficili da leggere per chi ha la licenza elementare
\item inferiore a 60 sono difficili da leggere per chi ha la licenza media
\item inferiore a 40 sono difficili da leggere per chi ha un diploma superiore
\end{itemize}
Viene adottata la seguente formula per calcolarlo:
\begin{equation}
89+\frac{300*(\text{numero delle frasi})-10*(\text{numero delle lettere})}{\text{numero delle parole}}
\end{equation}\\
Questi sono i valori da noi ritenuti opportuni:\\
\textit{Valore minimo}:$$ \geq 50 $$ 
\textit{Valore ottimo}:$$ \geq 80 $$\\
\subsection{Gestione della configurazione}
\subsubsection{Scopo}
La gestione della configurazione è un processo\textsubscript{g} che mira a gestire e controllare i cambiamenti apportati a un prodotto\textsubscript{g} software o a un sistema durante il suo ciclo di vita\textsubscript{g}. 
La gestione della configurazione per la documentazione descrive come vengono identificate, controllate, tracciate e gestite le versioni di un documento.
\subsubsection{Descrizione}
In questa sezione sono riportate tutte le norme relative alla configurazione degli strumenti utilizzati per il tracciamento e l’organizzazione dei documenti e del codice.
\subsubsection{Versionamento}
Il numero di versione permette di capire lo stato in cui si trova un documento.
Un documento può trovarsi nei seguenti stati:
\begin{itemize} 
    \item \textbf{Approvato}: il documento è verificato ed approvato dal responsabile;
    \item \textbf{Verificato}: il documento risulta verificato ma non ancora visionato dal responsabile;
    \item \textbf{In Sviluppo}: sono presenti delle modifiche che non sono state verificate.
\end{itemize}
Il numero di versione ha il formato \textbf{X.Y.Z} dove:
\begin{itemize} 
    \item \textbf{X}: indica una versione approvata dal responsabile, la numerazione parte da 0
    e la prima versione approvata è la 1.0.0;
    \item \textbf{Y}: indica una versione verificata dal verificatore, la numerazione inizia da 0 e si azzera ad ogni incremento di X. La prima versione
    verificata è la 0.1.0;
    \item \textbf{Z}: indica una versione in fase di modifica da parte dei redattori che ne incrementano il numero ad ogni modifica,
    la numerazione parte da 1 e si azzera ad ogni incremento di X o Y. La prima versione modificata è la 0.0.1.
\end{itemize}

\paragraph{Sistemi software utilizzati}
Per il versionamento il gruppo ha deciso di utilizzare un repository\textsubscript{g} GitHub\textsubscript{g}, un servizio di hosting per progetti software che implementa uno strumento di controllo versione distribuito Git\textsubscript{g}.
Per informazioni più approfondite sulla struttura e la gestione del repository\textsubscript{g} si veda sezione \fullref{GitHub\textsubscript{g}}.
\subsection{Verifica\textsubscript{g}}
\subsubsection{Scopo}
Lo scopo del processo\textsubscript{g} di verifica\textsubscript{g} è quello di accertare che non siano stati commessi 
errori nello svolgimento delle attività prefissate. Questo processo\textsubscript{g} viene applicato 
costantemente sia durante la stesura della documentazione, che durante lo sviluppo 
del software. 

\subsubsection{Descrizione}
In questa sezione sono descritte le norme necessarie ad applicare il processo\textsubscript{g} di 
verifica\textsubscript{g} in modo efficace.
\begin{itemize}
    \item \textbf{Analisi statica}: questo tipo di analisi va a valutare la 
    documentazione o il software senza bisogno di esecuzione. Si controlla che il 
    prodotto\textsubscript{g} sia corretto rispetto a determinate regole, che soddisfi determinate 
    caratteristiche, che non abbia errori.\\ Vengono inoltre adottati dal gruppo i seguenti metodi di lettura:
    \begin{itemize}
    \item \textbf{Walkthrough}: per attuare questo metodo si eseguono letture ad ampio spettro alla ricerca di errori. Non appena il verificatore trova un errore, sarà compito suo e del redattore di discutere a una possibile soluzione;
    \item \textbf{Inspection}: metodo utile a eseguire una lettura mirata del prodotto in esame, rilevando errori specifici. La ricerca si focalizza quindi su errori noti redigendo una checklist, cioè una lista di controllo che verrà utilizzata dal verificatore.
    \end{itemize} 
    \item \textbf{Analisi dinamica}: questo tipo di analisi invece richiede esecuzione, e, per 
    questo, può essere applicata solo al codice. Si tratta di verificare se il comportamento del 
    prodotto\textsubscript{g} software durante l’esecuzione è corretto, soddisfi i requisiti preposti.
\end{itemize} 
\subsubsection{Verifica della documentazione}
La verifica viene svolta da due verificatori prima del merge con il branch documentation.
Consiste nell'esaminare i file prodotti da chi ne ha fatto la stesura e segnalarne la non validità o 
la presenza di errori nei concetti esposti.\\
Un verificatore dovrà verificare il documento a partire dalle modifiche fatte dopo l'ultima versione verificata.
Le modifiche da verificare quindi possono essere dedotte dal registro dei cambiamenti presente in ogni documento.
Una volta controllato il documento, il primo verificatore segnalerà eventuali errori e
successivamente dovrà spuntare come approvata la Pull Request nella sezione dedicata su GitHub\textsubscript{g}. \\
A questo punto se un secondo verificatore noterà la necessità di qualche altro cambiamento da apportare, chi dovrà apportare le modifiche farà una pull in locale per
allineare il proprio branch con quello in remoto e continuare con il proprio lavoro.\\
Dopo il push delle modifiche se i file risultano corretti anche dal secondo verificatore esso aggiungerà i nomi dei verificatori all'intestazione modificando il file titlepage\_input.tex
 e creerà la riga nel registro delle modifiche, nel file changelog\_input.tex,  inserendo la nuova versione secondo le regole di versionamento e scrivendo nella colonna Autore sia il suo nome,
che quello del primo verificatore, e in descrizione "Verifica".
Dopo aver eseguito un commit sul ramo da integrare approva la Pull Request e conferma il merge secondo le norme di progetto descritte nella sezione dedicata.
\subsubsection{Verifica del codice}
L'attività di analisi dinamica del codice si concretizza con lo sviluppo e l'utillizzo dei test. I test servono per assicurare un certo comportamento delle componenti oppure dell'intera applicazione. Esistono più tipi di test ed ogni tipologia va a controllare aspetti diversi del software, nello specifico:
\begin{itemize}
    \item \textit{Test di unità}:  è il tipo di test che verifica se i singoli moduli funzionano correttamente. L'obiettivo principale del test unitario è identificare, analizzare e correggere i difetti in ciascuna unità isolandola dal sistema. É compito del programmatore che implementa un modulo di scrivere il test di unità corrispondente. Inoltre i test di unità devono rispettare la struttura AAA che prevede di dividere il test in queste 3 parti:
	\begin{itemize}
		\item \textit{Arrange}: si prepara l'oggetto e i prerequisiti del test;
		\item \textit{Act}: si svolge l'effettiva componente da testare;
		\item \textit{Arrange}: si verifica che i risultati ottenuti nella fase di Act corrispondano ai risultati attesi.
	\end{itemize}
    \item \textit{Test di integrazione}: è il tipo di test che verifica che i moduli già testati singolarmente si comportino correttamente anche quando interagiscono tra di loro;
    \item \textit{Test di regressione}: è il tipo di test che verifica che dopo l'aggiunta di una nuova feature il codice complessivo non perda di qualità, ovvero che le componenti già integrate continuino a comportarsi nel modo atteso;
    \item \textit{Test di sistema}:  è il tipo di test che verifica l'intero comportamento del sistema. In particolare controlla che i requisiti necessari vengano soddisfatti;
    \item \textit{Test di accettazione}:  è il tipo di test che verifica che l'applicazione nel suo complesso soddisfi pienamente i requisiti dal punto di vista strettamente funzionale. Viene effettuato alla fine dello sviluppo dell'applicazione, quando ha già superato tutti gli altri test, e ne precede il rilascio.
\end{itemize}
Ogni test è caraterrizato dei seguenti parametri:
\begin{itemize}
	\item \textbf{ambiente}: sistema hardware e software sul quale verrà eseguito il test;
	\item \textbf{stato iniziale}: stato iniziale dal quale il test viene eseguito;
	\item \textbf{input}: dati in ingresso immessi;
	\item \textbf{output}: dati in uscita attesi; 
	\item\textbf{istruzioni aggiuntive}: ulteriori istruzioni su come deve essere eseguito il test e su come interpretare i risultati ottenuti.
\end{itemize}
Per ottenere un buon test è necessario :
\begin{itemize}
	\item che sia ripetibile;
	\item che specifichi l'ambiente di esecuzione;
	\item che indichi input e output richiesti;
	\item che in caso di fallimento fornisca informazioni sul motivo di quest'ultimo tramite una serie di eccezioni che devono essere previste nell scrittura del test.
\end{itemize}


\paragraph{Codice identificativo}
I test vengono identificati tramite un codice così definito:

\begin{center} \textbf{T[tipo][codice identificativo]} \end{center}

Dove il codice identificativo è un numero univoco e il tipo può essere:
\begin{itemize}
	\item \textbf{U}: test di unità;
	\item \textbf{I}: test di integrazione;
	\item \textbf{R}: test di regressione;
	\item \textbf{S}: test di sistema;
	\item \textbf{A}: test di accettazione.
\end{itemize}

\paragraph{Continuous Integration}
I test verranno svolti secondo il workflow della CI (Continuous Integration) che prevede che venga effettuata la build del codice ogni volta che viene eseguito un commit nella repository, durante la quale avviene anche l'esecuzione automatica di tutti i test implementati.\\
Nel caso in cui la build dovesse fallire sarà necessario trovare la causa del fallimento con la massima priorità e in caso non fosse possibile identificarla in poco tempo è necessario ripristinare la repository all'ultimo commit nel quale la build aveva avuto successo.\\
Per evitare quindi di perdere grosse porzioni di lavoro, è buona prassi svolgere commit frequenti, di portata non eccessiva.
La buona implementazione di questa pratica garantisce che il codice all'interno della repository sia sempre funzionante.

\paragraph{Procedimento di verifica del software}
Qui vengono riportate le varie casistiche che posso scaturire durante la fase di test, e il modo in cui comportarsi a segiuto del loro successo o fallimento:
\begin{itemize}
	\item Il programmatore apporta cambiamenti alla repository effettuando un commit;
	\item Viene eseguita la build dell'intera applicazione tramite il processo di CI:
		\begin{itemize}
		\item Se la build ha successo si passa al punto successivo;
		\item Se la build fallisce il programmatore cerca di correggere l'errore nel codice e se non ci riesce ritorna all'ultimo commit in cui il codice è verificato;
		\end{itemize}
	\item Vengono eseguiti i test:
		\begin{itemize}
		\item Se tutti i test hanno successo si passa al punto successivo;
		\item Se almeno un test fallisce il programmatore cerca di correggere l'errore nel codice e se non ci riesce ritorna all'ultimo commit in cui il codice è verificato;
		\end{itemize}
	\item Vengono calcolate in automatico la copertura del codice e le metriche riguardanti la codifica
	\item Vengono registrati i valori nel cruscotto di qualità.
\end{itemize}

\paragraph{Strumenti}
\begin{itemize}
	\item Per l'attività di analisi statica si utilizza SonarCloud, un software configurabile ed integrabile con le Action di Github che permette di controllare la qualità del codice;
	\item Per l'attività di analisi dinamica si utilizza Cypress, un software integrabile con le Action di GitHub che facilità la scrittura dei test;
	\item Per l'implementazione del workflow CI vengono usate le Action messe a disposizione di GitHub, configurabili tramite un file .yml oppure .yaml.
\end{itemize}
\subsubsection{Metriche}
Per mantenere la qualità nella verifica è stata scelta la seguente metrica.
\begin{itemize}
    \item \textbf{MPC12: Code coverage}.
\end{itemize}
\subsection{Validazione\textsubscript{g}}
\subsubsection{Scopo} 
Lo scopo del processo\textsubscript{g} di validazione\textsubscript{g} è quello di determinare se il prodotto\textsubscript{g} finale sia 
conforme con le aspettative preposte e rispetti i requisiti minimi concordati con il 
proponente\textsubscript{g}. Questo processo\textsubscript{g} avviene dopo quello di verifica\textsubscript{g} e sarà il responsabile a 
stabilire se il prodotto\textsubscript{g} è accettabile o ha bisogno di ulteriori verifiche.