\subsubsection{Test}
I test servono per assicurare un certo comportamento delle componenti oppure dell'intera applicazione. Esistono più tipi di test ed ogni tipologia va a controllare aspetti diversi del software, nello specifico:
\begin{itemize}
    \item \textit{Test di unità}:  è il tipo di test che verifica se i singoli moduli funzionano correttamente. L'obiettivo principale del test unitario è identificare, analizzare e correggere i difetti in ciascuna unità isolandola dal sistema. É compito del programmatore che implementa un modulo di scrivere il test di unità corrispondente. Inoltre i test di unità devono rispettare la struttura AAA che prevede di dividere il test in queste 3 parti:
	\begin{itemize}
		\item \textit{Arrange}: si prepara l'oggetto e i prerequisiti del test;
		\item \textit{Act}: si svolge l'effettiva componente da testare;
		\item \textit{Arrange}: si verifica che i risultati ottenuti nella fase di Act corrispondano ai risultati attesi.
	\end{itemize}
    \item \textit{Test di iintegrazione}: è il tipo di test che verifica che i moduli già testati singolarmente si comportino correttamente anche quando interagiscono tra di loro;
    \item \textit{Test di regressione}: è il tipo di test che verifica che dopo l'aggiunta di una nuova feature il codice complessivo non perda di qualità, ovvero che le componenti già integrate continuino a comportarsi nel modo atteso;
    \item \textit{Test di sistema}:  è il tipo di test che verifica l'intero comportamento del sistema. In particolare controlla che i requisiti necessari vengano soddisfatti;
    \item \textit{Test di accettazione}:  è il tipo di test che verifica che l'applicazione nel suo complesso soddisfi pienamente i requisiti dal punto di vista strettamente funzionale. Viene effettuato alla fine dello sviluppo dell'applicazione, quando ha già superato tutti gli altri test, e ne precede il rilascio.
\end{itemize}

\paragraph{Codice identificativo}
I test vengono identificati tramite un codice così definito:

\begin{center} \textbf{T[tipo][codice identificativo]} \end{center}

Dove il codice identificativo è un numero univoco e il tipo può essere:
\begin{itemize}
	\item \textbf{U}: test di unità;
	\item \textbf{I}: test di integrazione;
	\item \textbf{R}: test di regressione;
	\item \textbf{S}: test di sistema;
	\item \textbf{A}: test di accettazione.
\end{itemize}