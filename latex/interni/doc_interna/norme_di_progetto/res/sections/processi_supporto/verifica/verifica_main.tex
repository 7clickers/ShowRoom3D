\subsection{Verifica\textsubscript{g}}
\subsubsection{Scopo}
Lo scopo del processo\textsubscript{g} di verifica\textsubscript{g} è quello di accertare che non siano stati commessi 
errori nello svolgimento delle attività prefissate. Questo processo\textsubscript{g} viene applicato 
costantemente sia durante la stesura della documentazione, che durante lo sviluppo 
del software. 

\subsubsection{Descrizione}
In questa sezione sono descritte le norme necessarie ad applicare il processo\textsubscript{g} di 
verifica\textsubscript{g} in modo efficace.
\begin{itemize}
    \item \textbf{Analisi statica}: questo tipo di analisi va a valutare la 
    documentazione o il software senza bisogno di esecuzione. Si controlla che il 
    prodotto\textsubscript{g} sia corretto rispetto a determinate regole, che soddisfi determinate 
    caratteristiche, che non abbia errori;
    \item \textbf{Analisi dinamica}: questo tipo di analisi invece richiede esecuzione, e, per 
    questo, può essere applicata solo al codice. Si tratta di verificare se il comportamento del 
    prodotto\textsubscript{g} software durante l’esecuzione è corretto, soddisfi i requisiti preposti.
\end{itemize} 
\input{res/sections/processi_supporto/verifica\textsubscript{g}/verifica_documentazione.tex}
\input{res/sections/processi_supporto/verifica\textsubscript{g}/metriche_verifica.tex}