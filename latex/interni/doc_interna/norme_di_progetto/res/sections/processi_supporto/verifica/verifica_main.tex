\subsection{Verifica\textsubscript{g}}
\subsubsection{Scopo}
Lo scopo del processo\textsubscript{g} di verifica\textsubscript{g} è quello di accertare che non siano stati commessi 
errori nello svolgimento delle attività prefissate. Questo processo\textsubscript{g} viene applicato 
costantemente sia durante la stesura della documentazione, che durante lo sviluppo 
del software. 

\subsubsection{Descrizione}
In questa sezione sono descritte le norme necessarie ad applicare il processo\textsubscript{g} di 
verifica\textsubscript{g} in modo efficace.
\begin{itemize}
    \item \textbf{Analisi statica}: questo tipo di analisi va a valutare la 
    documentazione o il software senza bisogno di esecuzione. Si controlla che il 
    prodotto\textsubscript{g} sia corretto rispetto a determinate regole, che soddisfi determinate 
    caratteristiche, che non abbia errori.\\ Vengono inoltre adottati dal gruppo i seguenti metodi di lettura:
    \begin{itemize}
    \item \textbf{Walkthrough}: per attuare questo metodo si eseguono letture ad ampio spettro alla ricerca di errori. Non appena il verificatore trova un errore, sarà compito suo e del redattore di discutere a una possibile soluzione;
    \item \textbf{Inspection}: metodo utile a eseguire una lettura mirata del prodotto in esame, rilevando errori specifici. La ricerca si focalizza quindi su errori noti redigendo una checklist, cioè una lista di controllo che verrà utilizzata dal verificatore.
    \end{itemize} 
    \item \textbf{Analisi dinamica}: questo tipo di analisi invece richiede esecuzione, e, per 
    questo, può essere applicata solo al codice. Si tratta di verificare se il comportamento del 
    prodotto\textsubscript{g} software durante l’esecuzione è corretto, soddisfi i requisiti preposti.
\end{itemize} 
\subsubsection{Verifica della documentazione}
La verifica viene svolta da due verificatori prima del merge con il branch documentation all'apertura di una pull request.
Il primo verificatore sarà quello che esaminerà il documento seguendo una checklist. Il documento verrà esaminato a partire dalle modifiche fatte dopo l'ultima versione verificata, 
deducibili tramite il registro delle modifiche.
\\\\
La verifica di un documento prevede di seguire la seguente checklist:
\begin{itemize}
    \item Lettura esplorativa del testo;
    \item Seconda lettura del testo per capirne meglio i concetti esposti;
    \item Segnalare i concetti poco chiari o errati;
    \item Valutare se il testo è organizzato in maniera opportuna ed eventualmente suggerire un'alternativa migliore;
    \item Valutare se ci sono parti mancanti ed eventualmente segnalarne la mancanza;
    \item Segnalare errori grammaticali;
    \item Segnalare le parti che non rispettano le \textit{Norme di Progetto} riguardanti la stesura del documento;
    \item Segnalare i termini che sono contenuti anche nel \textit{Glossario} a cui non sono stati inseriti i pedici "g";
    \item Segnalare i nomi dei documenti che non sono stati scritti seguendo la convenzione;
    \item Segnalare le parti del registro delle modifiche mancanti o che non sono state compilate correttamente seguendo le \textit{Norme di Progetto};
    \item Segnalare errori nell'intestazione del documento.
\end{itemize}
In caso non ci sia nulla da segnalare il verificatore dovrà spuntare come approvata la Pull Request nella sezione 
dedicata su GitHub\textsubscript{g}. 
A questo punto se un secondo verificatore noterà la necessità di qualche altro cambiamento da apportare lo segnalerà a 
chi ha fatto la stesura, se invece ritiene che il documento è corretto dovrà:
\begin{itemize}
    \item Creare la riga nel registro delle modifiche, nel file changelog\_input.tex, inserendo la nuova versione secondo le regole di versionamento, i nomi dei verificatori e in descrizione "Verifica";
    \item Spuntare come approvata la Pull Request nella sezione dedicata su GitHub\textsubscript{g};
    \item Confermare il merge secondo le \textit{Norme di Progetto} descritte nella sezione dedicata.
\end{itemize}
\subsubsection{Verifica del codice}
L'attività di analisi dinamica del codice si concretizza con lo sviluppo e l'utillizzo dei test. I test servono per assicurare un certo comportamento delle componenti oppure dell'intera applicazione. Esistono più tipi di test ed ogni tipologia va a controllare aspetti diversi del software, nello specifico:
\begin{itemize}
    \item \textit{Test di unità}:  è il tipo di test che verifica se i singoli moduli funzionano correttamente. L'obiettivo principale del test unitario è identificare, analizzare e correggere i difetti in ciascuna unità isolandola dal sistema. É compito del programmatore che implementa un modulo di scrivere il test di unità corrispondente. Inoltre i test di unità devono rispettare la struttura AAA che prevede di dividere il test in queste 3 parti:
	\begin{itemize}
		\item \textit{Arrange}: si prepara l'oggetto e i prerequisiti del test;
		\item \textit{Act}: si svolge l'effettiva componente da testare;
		\item \textit{Arrange}: si verifica che i risultati ottenuti nella fase di Act corrispondano ai risultati attesi.
	\end{itemize}
    \item \textit{Test di integrazione}: è il tipo di test che verifica che i moduli già testati singolarmente si comportino correttamente anche quando interagiscono tra di loro;
    \item \textit{Test di regressione}: è il tipo di test che verifica che dopo l'aggiunta di una nuova feature il codice complessivo non perda di qualità, ovvero che le componenti già integrate continuino a comportarsi nel modo atteso;
    \item \textit{Test di sistema}:  è il tipo di test che verifica l'intero comportamento del sistema. In particolare controlla che i requisiti necessari vengano soddisfatti;
    \item \textit{Test di accettazione}:  è il tipo di test che verifica che l'applicazione nel suo complesso soddisfi pienamente i requisiti dal punto di vista strettamente funzionale. Viene effettuato alla fine dello sviluppo dell'applicazione, quando ha già superato tutti gli altri test, e ne precede il rilascio.
\end{itemize}
Ogni test è caraterrizato dei seguenti parametri:
\begin{itemize}
	\item \textbf{ambiente}: sistema hardware e software sul quale verrà eseguito il test;
	\item \textbf{stato iniziale}: stato iniziale dal quale il test viene eseguito;
	\item \textbf{input}: dati in ingresso immessi;
	\item \textbf{output}: dati in uscita attesi; 
	\item\textbf{istruzioni aggiuntive}: ulteriori istruzioni su come deve essere eseguito il test e su come interpretare i risultati ottenuti.
\end{itemize}
Per ottenere un buon test è necessario :
\begin{itemize}
	\item che sia ripetibile;
	\item che specifichi l'ambiente di esecuzione;
	\item che indichi input e output richiesti;
	\item che in caso di fallimento fornisca informazioni sul motivo di quest'ultimo tramite una serie di eccezioni che devono essere previste nell scrittura del test.
\end{itemize}


\paragraph{Codice identificativo}
I test vengono identificati tramite un codice così definito:

\begin{center} \textbf{T[tipo][codice identificativo]} \end{center}

Dove il codice identificativo è un numero univoco e il tipo può essere:
\begin{itemize}
	\item \textbf{U}: test di unità;
	\item \textbf{I}: test di integrazione;
	\item \textbf{R}: test di regressione;
	\item \textbf{S}: test di sistema;
	\item \textbf{A}: test di accettazione.
\end{itemize}

\paragraph{Continuous Integration}
I test verranno svolti secondo il workflow della CI (Continuous Integration) che prevede che venga effettuata la build del codice ogni volta che viene eseguito un commit nella repository, durante la quale avviene anche l'esecuzione automatica di tutti i test implementati.\\
Nel caso in cui la build dovesse fallire sarà necessario trovare la causa del fallimento con la massima priorità e in caso non fosse possibile identificarla in poco tempo è necessario ripristinare la repository all'ultimo commit nel quale la build aveva avuto successo.\\
Per evitare quindi di perdere grosse porzioni di lavoro, è buona prassi svolgere commit frequenti, di portata non eccessiva.
La buona implementazione di questa pratica garantisce che il codice all'interno della repository sia sempre funzionante.

\paragraph{Procedimento di verifica del software}
Qui vengono riportate le varie casistiche che posso scaturire durante la fase di test, e il modo in cui comportarsi a segiuto del loro successo o fallimento:
\begin{itemize}
	\item Il programmatore apporta cambiamenti alla repository effettuando un commit;
	\item Viene eseguita la build dell'intera applicazione tramite il processo di CI:
		\begin{itemize}
		\item Se la build ha successo si passa al punto successivo;
		\item Se la build fallisce il programmatore cerca di correggere l'errore nel codice e se non ci riesce ritorna all'ultimo commit in cui il codice è verificato;
		\end{itemize}
	\item Vengono eseguiti i test:
		\begin{itemize}
		\item Se tutti i test hanno successo si passa al punto successivo;
		\item Se almeno un test fallisce il programmatore cerca di correggere l'errore nel codice e se non ci riesce ritorna all'ultimo commit in cui il codice è verificato;
		\end{itemize}
	\item Vengono calcolate in automatico la copertura del codice e le metriche riguardanti la codifica
	\item Vengono registrati i valori nel cruscotto di qualità.
\end{itemize}

\paragraph{Strumenti}
\begin{itemize}
	\item Per l'attività di analisi statica si utilizza SonarCloud, un software configurabile ed integrabile con le Action di Github che permette di controllare la qualità del codice;
	\item Per l'attività di analisi dinamica si utilizza Cypress, un software integrabile con le Action di GitHub che facilità la scrittura dei test;
	\item Per l'implementazione del workflow CI vengono usate le Action messe a disposizione di GitHub, configurabili tramite un file .yml oppure .yaml.
\end{itemize}
\subsubsection{Metriche}
Per mantenere la qualità nella verifica\textsubscript{g} è stata scelta la seguente metrica.
\begin{itemize}
    \item \textbf{MPC12: Code coverage}.
\end{itemize}