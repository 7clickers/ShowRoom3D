\subsubsection{Verifica della documentazione}
\label{verifica_documentazione} 
La verifica viene svolta da due verificatori prima del merge con il branch documentation all'apertura di una pull request.
Il primo verificatore sarà quello che esaminerà il documento seguendo una checklist. Il documento verrà esaminato a partire dalle modifiche fatte dopo l'ultima versione verificata, 
deducibili tramite il registro delle modifiche.
\\\\
La verifica di un documento prevede di seguire la seguente checklist:
\begin{itemize}
    \item Lettura esplorativa del testo;
    \item Seconda lettura del testo per capirne meglio i concetti esposti;
    \item Segnalare i concetti poco chiari o errati;
    \item Valutare se il testo è organizzato in maniera opportuna ed eventualmente suggerire un'alternativa migliore;
    \item Valutare se ci sono parti mancanti ed eventualmente segnalarne la mancanza;
    \item Segnalare errori grammaticali;
    \item Segnalare le parti che non rispettano le \textit{Norme di Progetto} riguardanti la stesura del documento;
    \item Segnalare i termini che sono contenuti anche nel \textit{Glossario} a cui non sono stati inseriti i pedici "g";
    \item Segnalare i nomi dei documenti che non sono stati scritti seguendo la convenzione;
    \item Segnalare le parti del registro delle modifiche mancanti o che non sono state compilate correttamente seguendo le \textit{Norme di Progetto};
    \item Segnalare errori nell'intestazione del documento.
\end{itemize}
In caso non ci sia nulla da segnalare il verificatore dovrà spuntare come approvata la Pull Request nella sezione 
dedicata su GitHub\textsubscript{g}. 
A questo punto se un secondo verificatore noterà la necessità di qualche altro cambiamento da apportare lo segnalerà a 
chi ha fatto la stesura, se invece ritiene che il documento è corretto dovrà:
\begin{itemize}
    \item Creare la riga nel registro delle modifiche, nel file changelog\_input.tex, inserendo la nuova versione secondo le regole di versionamento, i nomi dei verificatori e in descrizione "Verifica";
    \item Spuntare come approvata la Pull Request nella sezione dedicata su GitHub\textsubscript{g};
    \item Confermare il merge secondo le norme descritte nella sezione \fullref{Pull_Requests}.
\end{itemize}