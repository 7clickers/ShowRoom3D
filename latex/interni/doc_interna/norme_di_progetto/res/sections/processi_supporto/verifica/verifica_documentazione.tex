\subsubsection{Verifica della documentazione}
La verifica viene svolta da due verificatori prima del merge con il branch documentation all'apertura di una pull request.
Consiste nell'esaminare i file prodotti da chi ne ha fatto la stesura e segnalarne la presenza di errori nei concetti esposti.\\
Un verificatore dovrà verificare il documento a partire dalle modifiche fatte dopo l'ultima versione verificata.
Le modifiche da verificare quindi possono essere dedotte dal registro dei cambiamenti presente in ogni documento oltre che dallo storico fornito
da sistema di versionamento utilizzato.
Una volta controllato il documento, il primo verificatore segnalerà eventuali errori e
successivamente dovrà spuntare come approvata la Pull Request nella sezione dedicata su GitHub\textsubscript{g}. \\
A questo punto se un secondo verificatore noterà la necessità di qualche altro cambiamento da apportare, chi dovrà apportare le modifiche farà una pull in locale per
allineare il proprio branch con quello in remoto e continuare con il proprio lavoro.\\
Dopo il push delle modifiche se i file risultano corretti anche dal secondo verificatore esso aggiungerà i nomi dei verificatori all'intestazione modificando il file titlepage\_input.tex
 e creerà la riga nel registro delle modifiche, nel file changelog\_input.tex,  inserendo la nuova versione secondo le regole di versionamento e scrivendo nella colonna Autore sia il suo nome,
che quello del primo verificatore, e in descrizione "Verifica".
Dopo aver eseguito un commit sul ramo da integrare approva la Pull Request e conferma il merge secondo le \textit{Norme di Progetto} descritte nella sezione dedicata.
La verifica di un documento prevede di seguire la seguente checklist da parte di ogni verificatore:
\begin{itemize}
    \item Lettura esplorativa del testo;
    \item Seconda lettura del testo per capirne meglio i concetti esposti;
    \item Segnalare i concetti poco chiari o errati;
    \item Valutare se il testo é organizzato in maniera opportuna ed eventualmente suggerire un'alternativa migliore;
    \item Valutare se ci sono parti mancanti ed eventualmente segnalarne la mancanza;
    \item Segnalare errori grammaticali;
    \item Segnalare le parti che non rispettano le \textit{Norme di Progetto} riguardanti la stesura del documento;
    \item Segnalare i termini che sono contenuti anche nel \textit{Glossario} a cui non sono stati inseriti i pedici "g";
    \item Segnalare i nomi dei documenti che non sono stati scritti seguendo la convenzione;
    \item Segnalare le parti del registro delle modifiche mancanti o che non sono state compilate correttamente seguendo le \textit{Norme di Progetto};
    \item Segnalare errori nell'intestazione del documento;
    \item In caso non ci sia nulla da segnalare:
        \begin{itemize}
            \item se le modifiche da accettare sono state accettate da tutti gli altri verificatori allora compilare il registro delle modifiche 
                e l'intestazione del documento seguendo le \textit{Norme di Progetto} e fare il merge con il branch di destinazione;
            \item se altri verificatori devono ancora verificare il documento limitarsi a dare il proprio consenso alle modifiche nella sezione dedicata su GitHub\textsubscript{g}.
        \end{itemize}  
\end{itemize}