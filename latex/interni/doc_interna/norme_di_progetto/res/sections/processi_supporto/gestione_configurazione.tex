\subsection{Gestione della configurazione}
\subsubsection{Scopo}
La gestione della configurazione è un processo che mira a gestire e controllare i cambiamenti apportati a un prodotto software o a un sistema durante il suo ciclo di vita. 
La gestione della configurazione per la documentazione descrive come vengono identificate, controllate, tracciate e gestite le versioni di un documento.
\subsubsection{Descrizione}
In questa sezione sono riportate tutte le norme relative alla configurazione degli strumenti utilizzati per il tracciamento e l’organizzazione dei documenti e del codice.
\subsubsection{Versionamento}
Il numero di versione permette di capire lo stato in cui si trova un documento.
Un documento può trovarsi nei seguenti stati:
\begin{itemize} 
    \item \textbf{Approvato}: il documento è verificato ed approvato dal responsabile;
    \item \textbf{Verificato}: il documento risulta verificato ma non ancora visionato dal responsabile;
    \item \textbf{In Sviluppo}: sono presenti delle modifiche che non sono state verificate.
\end{itemize}
Il numero di versione ha il formato \textbf{X.Y.Z} dove:
\begin{itemize} 
    \item \textbf{X}: indica una versione approvata dal responsabile, la numerazione parte da 0
    e la prima versione approvata è la 1.0.0;
    \item \textbf{Y}: indica una versione verificata dal verificatore, la numerazione inizia da 0 e si azzera ad ogni incremento di X. La prima versione
    verificata è la 0.1.0;
    \item \textbf{Z}: indica una versione in fase di modifica da parte dei redattori che ne incrementano il numero ad ogni modifica,
    la numerazione parte da 1 e si azzera ad ogni incremento di X o Y. La prima versione modificata è la 0.0.1.
\end{itemize}

\paragraph{Sistemi software utilizzati}
Per il versionamento il gruppo ha deciso di utilizzare un repository GitHub, un servizio di hosting per progetti software che implementa uno strumento di controllo versione distribuito Git.
Per informazioni più approfondite sulla struttura e la gestione del repository si veda sezione xxx.