\subsection{Gestione della configurazione}
\subsubsection{Scopo}
La gestione della configurazione è un processo\textsubscript{g} che mira a gestire e controllare i cambiamenti apportati a un prodotto\textsubscript{g} software o a un sistema durante il suo ciclo di vita\textsubscript{g}. 
La gestione della configurazione per la documentazione descrive come vengono identificate, controllate, tracciate e gestite le versioni di un documento.
\subsubsection{Descrizione}
In questa sezione sono riportate tutte le norme relative alla configurazione degli strumenti utilizzati per il tracciamento e l’organizzazione dei documenti e del codice.
\subsubsection{Versionamento}
Ogni versione del documento è identificata da un codice di versione nel formato \textbf{X.Y.Z} dove:
\begin{itemize} 
    \item \textbf{X}: indica una versione approvata dal responsabile, la numerazione parte da 0
    e la prima versione approvata è la 1.0.0;
    \item \textbf{Y}: indica una versione che è stata sottoposta ad una revisione completa dopo un numero consistente di modifiche. 
    La numerazione inizia da 0 e si azzera ad ogni incremento di X. La prima versione è la 0.1.0;
    \item \textbf{Z}: indica una versione modificata da parte dei redattori del documento, con una conseguente verifica.
    La numerazione parte da 1 e si azzera ad ogni incremento di X o Y. La prima versione modificata è la 0.0.1.
\end{itemize}

\paragraph{Sistemi software utilizzati}
Per il versionamento il gruppo ha deciso di utilizzare un repository\textsubscript{g} GitHub\textsubscript{g}, un servizio di hosting per progetti software che implementa uno strumento di controllo versione distribuito Git\textsubscript{g}.
Per informazioni più approfondite sulla struttura e la gestione del repository\textsubscript{g} si veda sezione \fullref{GitHub}.