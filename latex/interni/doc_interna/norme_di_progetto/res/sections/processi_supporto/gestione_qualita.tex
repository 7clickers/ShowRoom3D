\subsection{Accertamento della qualità}

\subsubsection{Scopo}
Lo scopo del processo\textsubscript{g} di accertamento della qualità è quello di garantire e rispettare i requisiti di qualità preposti.
\subsubsection{Descrizione}
Questo processo\textsubscript{g} è direttamente collegato al documento di \textit{Piano di Qualifica} in cui il gruppo si impegna a mantenere, con l'adozione di metriche concordate e discusse, la qualità nei processi e nei prodotti.
\subsubsection{Attività per il controllo di qualità}
Queste sono le attività che ogni membro deve rispettare per il mantenimento della qualità.
Ogni membro del gruppo deve:
\begin{itemize}
\item Comprendere le attività da svolgere e gli obiettivi da raggiungere;
\item Aver compreso le metriche inserite nel documento di \textit{Piano di Qualifica} e rispettare le normative e gli standard definiti al suo interno;
\item Mantenere le normative inserite nelle \textit{Norme di Progetto};
\item Riconoscere nel documento su cui si sta lavorando valori di cui tenere conto per un'analisi successiva;
\item Utilizzare il sistema di tracciamento delle issue\textsubscript{g} fornito da Github\textsubscript{g} come descritto nel documento di \textit{Norme di Progetto};
\item Attuare miglioramento continuo, ponendosi obiettivi incrementali.
\end{itemize}
\paragraph{Attuazione del controllo di qualità}
Il documento deve essere aggiornato ad ogni aggiunta nel numero di versione centrale ,come descritto nella sezione di \ref{versionamento} Versionamento , in cui si andrà ad aggiornare i grafici e le misure scritte nel documento. L’analisi e il commento di ogni misura inserita nei grafici verrà successivamente fatta in fase di retrospezione dello Sprint.
\subsubsection{Denominazione requisiti nel \textit{Piano di Qualifica}}
 Per la denominazione nel tracciamento dei requisiti nella sezione di \textit{Specifica dei Test} nel documento di \textit{Piano di Qualifica} viene adottata la seguente scrittura:
 \begin{center}\textbf{T[Tipologia di Test]R[Tipologia di Requisito][Numero Identificativo]}\end{center}
 [Tipologia di Test] indica a che tipo di Test si sta riferendo:
 \begin{itemize}
 \item Test di Unità corrisponde a "TU";
 \item Test di Integrità corrisponde a "TI";
 \item Test di Accettazione "TA";
 \item Test di Sistema corrisponde a "TS".
 \end{itemize}
 [Tipologia di Requisito] indica a che tipo di Requisito si sta riferendo:
 \begin{itemize}
 \item Requisito Funzionale corrisponderà a "RF";
 \item Requisito Qualitativo corrisponderà a "RQ";
 \item Requisito di Dominio corrisponderà a "RD";
 \item Requisito Prestazionale corrisponderà a "RP".
 \end{itemize}
 [Numero Indicativo] indica il numero del Requisito descritto nel documento di \textit{Analisi dei Requisiti}.
 
\subsubsection{Denominazione degli obiettivi di qualità nel \textit{Piano di Qualifica}}
Per la denominazione delle metriche di prodotto\textsubscript{g} è stato adottata la seguente scrittura:
\begin{center}\textbf{MPD[Numero]}\end{center}
Il Numero indicato è un numero che incrementa in base al numero di metriche per sezione.
Per la denominazione delle metriche di processo\textsubscript{g} è stato adottata la seguente scrittura:
\begin{center}\textbf{MPC[Numero]}\end{center}
Il Numero indicato è un numero che incremente in base al numero di metriche per sezione.
Viene inoltre inserito il nome della metrica, il valore minimo e il valore ottimo.\\
La descrizione dettagliata di ogni metrica la si può trovare alla sezione \fullref{Metriche_qualita}.
\subsubsection{Metriche} 
Per tracciare la soddisfazione dei requisiti è stata concordata una semplice metrica.
\begin{itemize}
    \item \textbf{MPC11: Metriche soddisfatte}.
\end{itemize}
