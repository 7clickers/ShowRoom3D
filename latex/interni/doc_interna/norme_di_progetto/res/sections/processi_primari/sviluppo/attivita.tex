\subsubsection{Analisi dei requisiti}
L’analisi dei requisiti è un’attività fondamentale che precede la progettazione. 
Si concretizza nel documento \textit{Analisi dei Requisiti}. Lo scopo di quest’attività è:
    \begin{itemize}
        \item Definire le funzionalità che il prodotto andrà ad offrire;
        \item Porre le basi per la fase di progettazione del software;
        \item Fare una stima della mole di lavoro;
        \item Facilitare la verifica. 
    \end{itemize}

\paragraph{Struttura del documento} Il documento è strutturato nel modo seguente:
    \begin{itemize}
        \item \textbf{Introduzione}: contiene una breve descrizione del documento e specifica gli attori dei casi d’uso;
		\item \textbf{Casi d’uso}: vengono specificati i casi d’uso individuati;
		\item \textbf{Requisiti}: sono classificati i requisiti che il prodotto finale dovrà soddisfare.
    \end{itemize}

\paragraph{Casi d’uso} 

I casi d’uso sono identificati seguendo la seguente struttura:

\begin{center}\textbf{UC [Numero caso d’uso].[Numero sottocaso d’uso]-[Titolo caso d’uso]}\end{center}

\noindent Vengono poi indicati:
    \begin{itemize}
        \item \textbf{Diagramma UML}: viene mostrato il diagramma solo per i casi più complessi, in cui è ritenuto necessario per una maggiore comprensione. Per i casi d’uso facoltativi il diagramma è colorato di azzurro (non standard UML) per una maggiore differenziazione;
	    \item \textbf{Attore primario}: utente esterno al sistema che svolge il caso d’uso;
	    \item \textbf{Descrizione}: breve descrizione del caso d’uso;
	    \item \textbf{Precondizioni}: indica la condizione del sistema prima del verificarsi del caso d’uso;
	    \item \textbf{Postcondizioni}: indica la condizione del sistema dopo che si è verificato il caso d’uso;
	    \item \textbf{Scenario principale}: descrive l’interazione tra l’attore primario e il sistema;
	    \item \textbf{Estensioni}: casi d’uso alternativi che possono verificarsi al posto di quello a cui sono collegati. 
    \end{itemize}

\paragraph{Requisiti} 
I requisiti sono classificati secondo la seguente struttura: 

\begin{center}\textbf{R.[Tipologia][Numero seriale]}\end{center}

\noindent con:
    \begin{itemize}
        \item \textbf{Tipologia}: 
            \begin{itemize}
                \item \textbf{F}: funzionale;
                \item \textbf{Q}: qualitativo;
                \item \textbf{D}: di dominio;
                \item \textbf{P}: prestazionale.
            \end{itemize}
	    \item \textbf{Descrizione}: breve descrizione del requisito;
	    \item \textbf{Classificazione}: un requisito può essere facoltativo o obbligatorio;
	    \item \textbf{Fonti}: possono essere: 
            \begin{itemize}
                \item Casi d’uso;
                \item Capitolato;
                \item Decisione interna.
            \end{itemize}
    \end{itemize}

\subsubsection{Progettazione} 
L’attività di progettazione è assegnata ai progettisti. Seguendo quanto indicato 
nell’\textit{Analisi dei Requisiti}, l’obiettivo è quello di progettare l’architettura del sistema, 
prima con la creazione di un Proof of Concept per la Requirments and Tecnology Baseline e poi andando 
nel dettaglio per la Product Baseline.

\paragraph{Requirments and Tecnology Baseline} 
In questa fase vengono fissati i requisiti che il gruppo si impegna a soddisfare, in accordo con il 
proponente; si studiano le tecnologie, i framework e le librerie utili alla realizzazione del 
prodotto finale e si crea di conseguenza un Proof of Concept. I materiali da mostrare sono:
    \begin{itemize}
        \item Tecnologie, framework, librerie utilizzate, motivando la scelta;
        \item Proof of Concept;
        \item \textit{Analisi dei Requisiti}.
    \end{itemize}
La documentazione da mostrare:
    \begin{itemize}
        \item \textit{Piano di Progetto};
        \item \textit{Piano di Qualifica};
        \item \textit{Norme di Progetto};
        \item \textit{Verbali} interni ed esterni.
    \end{itemize}