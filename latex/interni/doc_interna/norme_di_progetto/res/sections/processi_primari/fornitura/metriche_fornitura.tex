\subsubsection{Metriche}
Per perseguire la qualità nel processo di Fornitura si è deciso di adottare le seguenti metriche:
\paragraph{Planned Value (PV)}
Costo pianificato per realizzare le attività di progetto pianificate al momento del calcolo.
Per calcolare questa metrica si utilizzerà la formula:
\begin{equation*}
\text{PV}=\text{B}\textsubscript{tot}*\%\text{ lavoro pianificato}
\end{equation*}
\textit{Valore minimo:} $\ge 0$\\
\textit{Valore ottimo:}	$\le \text{Budget at Completion\textsubscript{g}}$
\paragraph{Actual Cost (AC)}
Costo sostenuto al momento del calcolo.
Per calcolare questa metrica si tiene conto di:
\begin{equation*}
\text{AC}=\text{Totale ore lavorative per incremento}*\text{costo orario}
\end{equation*}
\textit{Valore minimo:} $\ge 0$\\
\textit{Valore ottimo:} $\le \text{EAC}$
\paragraph{Estimated at Completion (EAC)}
Un valore di revisione di costo stimato per la realizzazione del progetto alla data corrente.
\begin{equation*}
\text{EAC}=\text{AC}+\text{ETC}
\end{equation*}
\textit{Valore minimo:} 
$\text{preventivo} -5\% \le \text{EAC}$ \textit{oppure} $\text{preventivo} +5\% \ge \text{EAC}$\\
\textit{Valore ottimo:} Costo preventivato
\paragraph{Earned Value (EV)}
Indica il valore delle attività realizzate nel progetto fino al momento della misurazione.
\begin{equation*}
\text{EV}=\%\text{ lavoro pianificato}*\text{EAC}
\end{equation*}
\textit{Valore minimo:} $\ge 0$\\
\textit{Valore ottimo:} $\le \text{EAC}$
\paragraph{Estimated to Complete (ETC)}
Metrica che produce un valore intero per indicare il numero di attività necessarie al completamento del progetto.\\
\textit{Valore minimo:} $\ge 0$\\
\textit{Valore ottimo:} $\le \text{EAC}$
\paragraph{Cost Variance (CV)}
Indica se il valore del costo attuale del progetto è maggiore, uguale o minore rispetto al costo effettivo.
Questa metrica si misura tramite una semplice formula:
\begin{equation*}
\text{CV}=\text{EV}-\text{AC}
\end{equation*}
\textit{Valore minimo:} $\le -15\%$\\
\textit{Valore ottimo:} $\ge 0\%$
\paragraph{Schedule Variance (SV)}
Valore utilizzato per analizzare se il progetto è in linea, in anticipo o in ritardo rispetto alla schedulazione delle attività di progetto pianificate nella baseline.
\begin{equation*}
\text{SV}=\text{EV}-\text{PV}
\end{equation*}
\textit{Valore minimo:} $\le -15\%$\\
\textit{Valore ottimo:} $\ge 0\%$
\paragraph{Budget Variance (BV)}
Valore che indica se alla data corrente si è speso di più o di meno del budget iniziale.
\begin{equation*}
\text{BV}=\text{EV}-\text{AC}
\end{equation*}
\textit{Valore minimo:} $\le -10\%$\\
\textit{Valore ottimo:} $\ge 0\%$

