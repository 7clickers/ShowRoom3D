\subsubsection{Ruoli}
I componenti del gruppo si suddivideranno nei seguenti ruoli per periodi di circa 2-3 settimane (dipendentemente dalle esigenze del periodo) e al termine del periodo i ruoli verranno risuddivisi. 
Visto che nelle varie fasi di sviluppo del progetto le attività da svolgere variano, non sempre sarà necessario coprire tutti i ruoli.\\
Inoltre sarà necessario tenere traccia delle ore che ogni componente dedica al progetto ed il ruolo associato a quelle ore, in modo da andare a rispettare la tabella degli impegni individuali.\\
I ruoli e le loro competenze sono i seguenti:

\paragraph{Responsabile}
Deve avere la visione d'insieme del progetto e coordinare i membri, inoltre si occupa di rappresentare il gruppo con le interazione esterne (proponente, committente ecc...). Le sue competenze specifiche sono:
\begin{itemize}
	\item ad ogni iterazione\textsubscript{g} c'è un solo responsabile
	\item presentare il diario di bordo in aula
	\item redarre l'ordine del giorno prima di ogni meeting interno del gruppo
	\item suddivide le attività del gruppo in singole issue (ma non le assegna ai membri del gruppo)
	\item in fase di release\textsubscript{g} si occupa di approvare\textsubscript{g} tutti i documenti che necessitano approvazione
\end{itemize}

\paragraph{Analista}
Si occupa di trasformare i bisogni del proponente nelle aspettative che il gruppo deve soddisfare per sviluppare un prodotto professionale. Le sue competenze specifiche sono:
\begin{itemize}
	\item interrogare il proponente riguardo allo scopo del prodotto e le funzionalità che deve avere
	\item studiare le risposte del proponente per identificare i requisiti\textsubscript{g} e redarre l'analisi dei requisiti
\end{itemize}

\paragraph{Amministratore}
Si occupa del funzionamento, mantenimento e sviluppo degli strumenti tecnologici usati dal gruppo. Le sue competenze specifiche sono:
\begin{itemize}
	\item ad ogni iterazione\textsubscript{g} basta un solo amministratore
	\item gestione delle segnalazioni e problemi dei membri del gruppo riguardanti problemi e malfunzionamenti con gli strumenti tecnologici
	\item valuta l'utilizzo di nuove tecnologie e ne fa uno studio preliminare per poter presentare al gruppo i pro e i contro del suo utilizzo
\end{itemize}

\paragraph{Progettista}
Si occupa di scegliere la modalità migliore per soddisfare le aspettative del committente che gli analisti hanno ricavato dall'analisi dei requisiti\textsubscript{g}. Le sue competenze specifiche sono:
\begin{itemize}
	\item scegliere eventuali pattern architetturali da implementare
	\item sviluppare lo schema UML\textsubscript{g} delle classi\textsubscript{g}
\end{itemize}

\paragraph{Programmatore}
Si occupa di implementare le scelte e i modelli fatti dal progettista. Le sue competenze specifiche sono:
\begin{itemize}
	\item scrivere il codice atto a implementare lo schema delle classi
	\item scrivere eventuali test
	\item scrivere la documentazione per la comprensione del codice che scrive
\end{itemize}

\paragraph{Verificatore}
Si occupa di controllare che ogni file che viene caricato in un branch protetto della repository\textsubscript{g} sia conforme alle norme di progetto. Le sue competenze specifiche sono:
\begin{itemize}
	\item controllare i file modificati o aggiunti durante una pull request tra un ramo non protetto e un ramo protetto siano conformi alle norme di progetto e cercano errori di altra natura (ortografici, sintattici, logici, build ecc...).
\end{itemize}