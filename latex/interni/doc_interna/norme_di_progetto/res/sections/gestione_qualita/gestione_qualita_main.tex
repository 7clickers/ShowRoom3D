\subsection{Accertamento della qualità}

\subsubsection{Scopo}
Lo scopo del processo di accertamento della qualità è quello di garantire e rispettare i requisiti di qualità preposti.
\subsubsection{Descrizione}
Questo processo è direttamente collegato al documento di \textit{Piano di qualifica} in cui il gruppo si impegna a mantenere, con l'adozione di metriche concordate e discusse, la qualità nei processi e nei prodotti.
\subsubsection{Attività per il controllo di qualità}
Queste sono le attività che ogni membro deve rispettare per il mantenimento della qualità.\\
Ogni membro del gruppo deve:
\begin{itemize}
\item comprendere le attività da svolgere e gli obiettivi da raggiungere;
\item aver compreso le metriche inserite nel documento di \textit{Piano di qualifica} e rispettare le normative e gli standard definiti al suo interno;
\item mantenere le normative inserite nelle \textit{Norme di progetto};
\item riconoscere nel documento su cui si sta lavorando valori di cui tenere conto per un'analisi successiva;
\item utilizzare il sistema di tracciamento delle issue\textsubscript{g} fornito da Github\textsubscript{g} come descritto nel documento di \textit{Norme di Progetto};
\item attuare miglioramento continuo, ponendosi obiettivi incrementali.
\end{itemize}
\subsubsection{Denominazione degli obiettivi di qualità nel \textit{Piano di qualifica}}
Per la denominazione delle metriche di prodotto è stato adottata la seguente scrittura:
\begin{center}\textbf{MPD[Numero]}\end{center}
Per la denominazione delle metriche di processo è stato adottata la seguente scrittura:
\begin{center}\textbf{MPC[Numero]}\end{center}
Viene inoltre inserito il nome della metrica, il valore minimo e il valore ottimo.\\
La descrizione dettagliata di ogni metrica la si può trovare all'interno delle \textit{Norme di Progetto} nel processo a loro associato.
\subsubsection{Metriche} 
Per tracciare la soddisfazione dei requisiti è stata concordata una semplice metrica.
\paragraph{Metriche soddisfatte}
Viene calcolato questo indice tramite una percentuale per tenere conto delle metriche soddisfatte.\\
Questi sono i valori da noi ritenuti opportuni:\\\\
\textit{Valore minimo:} $ \ge 90\% $\\
\textit{Valore ottimo:} $ 100\% $\\