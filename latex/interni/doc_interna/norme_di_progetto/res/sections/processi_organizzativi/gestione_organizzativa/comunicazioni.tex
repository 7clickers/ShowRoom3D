\subsubsection{Gestione delle comunicazioni}
Le comunicazioni per tutta la durata del progetto potranno essere interne, ovvero tra i componenti del gruppo;
oppure esterne, quindi tra il gruppo e una o più persone esterne al gruppo, in genere il proponente\textsubscript{g} o il committente\textsubscript{g}.

\paragraph{Comunicazioni Interne}
Si svolgono tra i componenti del gruppo tramite diversi canali di comunicazione. 
Ogni canale è dedito a un diverso tipo di comunicazione e comporta delle regole da seguire, per aumentare la 
produttività del gruppo.

\begin{itemize}
    \item \textbf{Telegram}: viene utilizzato per delle comunicazioni brevi;
    \item \textbf{Discord\textsubscript{g}}: sono stati creati diversi canali, che possono essere:
    \begin{itemize}
        \item \textbf{Testuali}: questi si suddividono in canali per argomento o per ruolo. I canali per argomento vengono utilizzati 
        esclusivamente per trattare dell'argomento in questione, mentre i canali per ruolo sono utili per le comunicazioni tra i membri del gruppo che
        sono assegnati allo stesso ruolo nello stesso periodo e hanno bisogno di comunicare tra loro;
        \item \textbf{Vocali}: vengono utilizzati per riunioni brevi, principalmente di aggiornamento.
    \end{itemize}
    \item \textbf{Zoom\textsubscript{g}}: viene utilizzato per le riunioni interne del gruppo, al termine delle quali verrà redatto un \textit{Verbale}.
\end{itemize}

\paragraph{Comunicazioni Esterne}
Le comunicazioni esterne si svolgono tra il gruppo e una persona esterna, generalmente il proponente\textsubscript{g} o il committente\textsubscript{g}. Con il 
committente\textsubscript{g} verranno utilizzati i canali suggeriti da quest'ultimo, mentre con il proponente\textsubscript{g} si è stabilito di utilizzare:
 \begin{itemize}
    \item \textbf{e-mail}: per comunicazioni brevi e immediate, si farà uso dell'indirizzo e-mail del gruppo. Il compito di scrivere le e-mail è assegnato al 
    responsabile, che, dopo una breve verifica\textsubscript{g} del messaggio con il gruppo avrà il compito di inviare la e-mail e notificare il gruppo alla risposta;
    \item \textbf{Google Chat}: per messaggi brevi, sempre scritti dal responsabile;
    \item \textbf{Google Meet}: per riunioni esterne in videochiamata. Queste riunioni potranno essere 
    richieste sia dal gruppo, che dal proponente\textsubscript{g} e al termine verrà redatto un \textit{Verbale}. 
\end{itemize}

\subsubsection{Gestione delle riunioni}
Le riunioni si dividono in interne ed esterne. Per ogni riunione verrà redatto un \textit{Verbale}, per poter tenere traccia 
degli argomenti trattati e delle decisioni prese.

\paragraph{Riunioni Interne}
Una riunione interna si svolge esclusivamente tra i membri del gruppo utilizzando il canale apposito Zoom\textsubscript{g}. Per ogni riunione 
il responsabile sarà incaricato di 
\begin{itemize}
    \item Preparare una scaletta degli argomenti da trattare, che potranno essere poi integrati da 
    eventuali punti di discussione portati dagli altri membri del gruppo;
    \item Fare da moderatore e dirigere la riunione per ottimizzare il tempo.
\end{itemize}

Le riunioni si svolgeranno a cadenza settimanale, cercando di trovare giorni e orari agevoli a tutti i membri del gruppo.

\paragraph{Riunioni Esterne}
Le riunioni esterne si svolgono tra i membri del gruppo e una persona esterna. Possono essere richieste da entrambe le parti 
e tramite i canali di comunicazione stabiliti, (Google Meet con il proponente\textsubscript{g}). Se necessario verrà preparata una presentazione 
da mostrare ai soggetti esterni per ottimizzare i tempi e facilitare la comprensione. Al termine della riunione verrà redatto un 
\textit{Verbale}. 
