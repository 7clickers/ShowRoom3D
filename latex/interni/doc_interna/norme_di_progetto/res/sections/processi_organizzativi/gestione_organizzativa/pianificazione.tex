\subsubsection{Pianificazione}

\paragraph{Ruoli}
I componenti del gruppo si suddivideranno nei seguenti ruoli per periodi di circa 2-3 settimane (dipendentemente dalle esigenze del periodo) e al termine del periodo i ruoli verranno risuddivisi. 
Visto che nelle varie fasi di sviluppo del progetto le attività da svolgere variano, non sempre sarà necessario coprire tutti i ruoli.\\
Inoltre sarà necessario tenere traccia delle ore che ogni componente dedica al progetto ed il ruolo associato a quelle ore, in modo da andare a rispettare la tabella degli impegni individuali.\\
Per questo tracciamento verrà utilizzato un foglio Excel in cui ogni componente del gruppo segnerà le ore di lavoro settimanalmente.
I ruoli e le loro competenze sono i seguenti:

\begin{itemize}
\item \textbf{Responsabile}: deve avere la visione d'insieme del progetto e coordinare i membri, inoltre si occupa di rappresentare il gruppo con le interazione esterne (proponente, committente ecc...). Le sue competenze specifiche sono:
\begin{itemize}
	\item Ad ogni iterazione\textsubscript{g} c'è un solo responsabile;
	\item Presentare il \textit{Diario di Bordo} in aula;
	\item Redarre l'ordine del giorno prima di ogni meeting interno del gruppo;
	\item Suddivide le attività del gruppo in singole issue (ma non le assegna ai membri del gruppo);
	\item In fase di release\textsubscript{g} si occupa di approvare\textsubscript{g} tutti i documenti che necessitano approvazione.
\end{itemize}

\item \textbf{Analista}: si occupa di trasformare i bisogni del proponente nelle aspettative che il gruppo deve soddisfare per sviluppare un prodotto professionale. Le sue competenze specifiche sono:
\begin{itemize}
	\item Interrogare il proponente riguardo allo scopo del prodotto e le funzionalità che deve avere;
	\item Studiare le risposte del proponente per identificare i requisiti\textsubscript{g} e redarre l' \textit{Analisi dei Requisiti}.
\end{itemize}

\item \textbf{Amministratore}: si occupa del funzionamento, mantenimento e sviluppo degli strumenti tecnologici usati dal gruppo. Le sue competenze specifiche sono:
\begin{itemize}
	\item Ad ogni iterazione\textsubscript{g} basta un solo amministratore;
	\item Gestione delle segnalazioni e problemi dei membri del gruppo riguardanti problemi e malfunzionamenti con gli strumenti tecnologici;
	\item Valuta l'utilizzo di nuove tecnologie e ne fa uno studio preliminare per poter presentare al gruppo i pro e i contro del suo utilizzo.
\end{itemize}

\item \textbf{Progettista}: si occupa di scegliere la modalità migliore per soddisfare le aspettative del committente che gli analisti hanno ricavato dall'analisi dei requisiti\textsubscript{g}. Le sue competenze specifiche sono:
\begin{itemize}
	\item Scegliere eventuali pattern architetturali da implementare;
	\item Sviluppare lo schema UML\textsubscript{g} delle classi\textsubscript{g}.
\end{itemize}

\item \textbf{Programmatore}: si occupa di implementare le scelte e i modelli fatti dal progettista. Le sue competenze specifiche sono:
\begin{itemize}
	\item Scrivere il codice atto a implementare lo schema delle classi;
	\item Scrivere eventuali test;
	\item Scrivere la documentazione per la comprensione del codice che scrive.
\end{itemize}

\item \textbf{Verificatore}: si occupa di controllare che ogni file che viene caricato in un branch protetto della repository\textsubscript{g} sia conforme alle norme di progetto. Le sue competenze specifiche sono:
\begin{itemize}
	\item Controllare i file modificati o aggiunti durante una pull request tra un ramo non protetto e un ramo protetto siano conformi alle norme di progetto e cercano errori di altra natura (ortografici, sintattici, logici, build ecc...).
\end{itemize}

\end{itemize}

\paragraph{Metodo di lavoro}