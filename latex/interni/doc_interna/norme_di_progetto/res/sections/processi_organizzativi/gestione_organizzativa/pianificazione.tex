\subsubsection{Pianificazione}

\paragraph{Ruoli}
I componenti del gruppo si suddivideranno nei seguenti ruoli per periodi di circa 2-3 settimane (dipendentemente dalle esigenze del periodo) e al termine del periodo i ruoli verranno risuddivisi. 
Visto che nelle varie fasi di sviluppo del progetto le attività da svolgere variano, non sempre sarà necessario coprire tutti i ruoli.\\
Inoltre sarà necessario tenere traccia delle ore che ogni componente dedica al progetto ed il ruolo associato a quelle ore, in modo da andare a rispettare la tabella degli impegni individuali.\\
Per questo tracciamento verrà utilizzato un foglio Excel\textsubscript{g} in cui ogni componente del gruppo segnerà le ore di lavoro settimanalmente.
I ruoli e le loro competenze sono i seguenti:

\begin{itemize}
\item \textbf{Responsabile}: deve avere la visione d'insieme del progetto e coordinare i membri, inoltre si occupa di rappresentare il gruppo con le interazione esterne (proponente\textsubscript{g}, committente\textsubscript{g} ecc...). Le sue competenze specifiche sono:
\begin{itemize}
	\item Ad ogni iterazione\textsubscript{g} c'è un solo responsabile;
	\item Presentare il \textit{Diario di Bordo} in aula;
	\item Redarre l'ordine del giorno prima di ogni meeting interno del gruppo;
	\item Suddivide le attività del gruppo in singole issue\textsubscript{g} (ma non le assegna ai membri del gruppo);
	\item In fase di release\textsubscript{g} si occupa di approvare\textsubscript{g} tutti i documenti che necessitano approvazione\textsubscript{g}.
\end{itemize}

\item \textbf{Analista}: si occupa di trasformare i bisogni del proponente\textsubscript{g} nelle aspettative che il gruppo deve soddisfare per sviluppare un prodotto\textsubscript{g} professionale. Le sue competenze specifiche sono:
\begin{itemize}
	\item Interrogare il proponente\textsubscript{g} riguardo allo scopo del prodotto\textsubscript{g} e le funzionalità che deve avere;
	\item Studiare le risposte del proponente\textsubscript{g} per identificare i requisiti e redarre l' \textit{Analisi dei Requisiti}.
\end{itemize}

\item \textbf{Amministratore}: si occupa del funzionamento, mantenimento e sviluppo degli strumenti tecnologici usati dal gruppo. Le sue competenze specifiche sono:
\begin{itemize}
	\item Ad ogni iterazione\textsubscript{g} basta un solo amministratore;
	\item Gestione delle segnalazioni e problemi dei membri del gruppo riguardanti problemi e malfunzionamenti con gli strumenti tecnologici;
	\item Valuta l'utilizzo di nuove tecnologie e ne fa uno studio preliminare per poter presentare al gruppo i pro e i contro del suo utilizzo;
	\item Controllo giornaliero delle project board\textsubscript{g} per garantire una buona organizzazione.
\end{itemize}

\item \textbf{Progettista}: si occupa di scegliere la modalità migliore per soddisfare le aspettative del committente\textsubscript{g} che gli analisti hanno ricavato dall'\textit{Analisi dei Requisiti}. Le sue competenze specifiche sono:
\begin{itemize}
	\item Scegliere eventuali pattern architetturali da implementare;
	\item Sviluppare lo schema UML\textsubscript{g} delle classi.
\end{itemize}

\item \textbf{Programmatore}: si occupa di implementare le scelte e i modelli fatti dal progettista. Le sue competenze specifiche sono:
\begin{itemize}
	\item Scrivere il codice atto a implementare lo schema delle classi;
	\item Scrivere eventuali test;
	\item Scrivere la documentazione per la comprensione del codice che scrive.
\end{itemize}

\item \textbf{Verificatore}: si occupa di controllare che ogni file che viene caricato in un branch protetto della repository\textsubscript{g} sia conforme alle \textit{Norme di Progetto}. Le sue competenze specifiche sono:
\begin{itemize}
	\item Controllare i file modificati o aggiunti durante una pull request tra un ramo non protetto e un ramo protetto siano conformi alle \textit{Norme di Progetto} e cercano errori di altra natura (ortografici, sintattici, logici, build ecc...).
\end{itemize}

\end{itemize}

\paragraph{Sprint}
Seguendo il modello Scrum\textsubscript{g}, i compiti derivanti dai processi di sviluppo vengono suddivisi in sprint dalla durata di una settimana. \newline
Questo rende l'avanzamento del prodotto più gestibile e permette al gruppo di produrre risultati in modo più rapido e con una maggiore flessibilità. \newline
Ogni sprint viene gestito da un amministratore che si occupa di creare e organizzare le issue associate allo sprint in modo da rendere il lavoro degli altri membri più semplice e trasparente. \newline
É compito dell'amministratore dello sprint quello di assicurarsi che sia avvenuta la verifica del Piano di Progetto e delle Norme di Progetto prima dell'inizio dello sprint successivo, in modo da avere la documentazione adatta ad affrontare il periodo successivo. 
\paragraph{Metodo di lavoro}
A seguito della revisione di avanzamento RTB il gruppo ha deciso di cambiare il metodo di lavoro, 
per rendere più efficace il lavoro svolto. Si è deciso di passare ad un modello Agile, utilizzando la tecnica degli sprint 
del metodo Scrum. Ogni sprint viene organizzato nel seguente modo:
\begin{itemize}
	\item \textbf{Sprint planning}:
	\begin{itemize}
		\item Si stabiliscono in gruppo quali sono le attività da svolgere durante lo sprint;
		\item Ogni componente del gruppo segnala le ore che può mettere a disposizione tramite un apposito foglio Google;
		\item Il responsabile definisce gli obiettivi dello sprint nel \textit{Piano di Progetto}, aggiornando il backlog, poi 
		assegna i compiti e crea le issue su GitHub in base alla disponibilità dei membri del gruppo.
	\end{itemize}
	\item \textbf{Daily Scrum}: ogni giorno si compila un apposito foglio Google scrivendo quali compiti si 
	andranno a svolgere. Il responsabile controlla l'andamento dello sprint contattando i componenti del gruppo.
	\item \textbf{Sprint review}:  
	\begin{itemize}
		\item Si fa un consuntivo complessivo: ogni membro del gruppo riferisce quello che ha svolto e 
		gli eventuali dubbi che ha riscontrato;
		\item Viene fatta una lista degli obiettivi raggiunti e quelli non raggiunti.
	\end{itemize}
	\item \textbf{Sprint retrospective}: si fa una valutazione di quello che è andato bene durante lo sprint
	 e di quello che è da migliorare, per capire cosa continuare o smettere di fare allo sprint successivo.

\end{itemize}
