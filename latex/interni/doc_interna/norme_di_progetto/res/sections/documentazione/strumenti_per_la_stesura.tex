\subsection{Strumenti per la stesura}
\begin{itemize} 
    \item LaTeX: è un linguaggio di marcatura per la preparazione di testi, basato sul 				  		  programma di composizione tipografica TEX.\\
	Nel branch documentation  si possono trovare i file .pdf prodotti e la cartella “latex”. La cartella latex contiene tre cartelle interne:
	\begin{itemize}
	\item esterni e interni, contengono file .tex di documentazione esterna ed interna come ad esempio i verbali o altra documentazione esterna/interna:
	\begin{itemize}
		\item la cartella config, contiene i file .tex con le parti fisse dei documenti (intestazione,registro delle modifiche,tracciamento dei temi affrontati) che vengono modificati con i dati del documento specifico 
		\item la cartella res/sections, contiene i file .tex con il contenuto vero e proprio (sezioni del documento) che viene redatto in maniera libera dal redattore
		\item un file col nome del documento pdf con estensione .tex che viene compilato per produrre il file pdf
	\end{itemize}
	 \item template, contiene file .tex di base utilizzati secondo necessità per comporre i documenti:
	 \begin{itemize}
	 \item changelox.tex è il file di template che serve per scrivere il registro delle modifiche
	 \item package.tex è il file che contiene tutti gli usepackage\textsubscript{g}
	 \item titlepage.tex è il file di template che contiene la configurazione della pagina iniziale di ogni documento  
	 \item tracking.tex è il file di template che contiene il tracciamento dei temi affrontati nel documento
	 \end{itemize}
	\end{itemize}
\end{itemize}