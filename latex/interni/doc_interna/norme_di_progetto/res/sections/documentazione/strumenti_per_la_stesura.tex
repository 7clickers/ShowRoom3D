\subsubsection{Strumenti per la stesura}
\begin{itemize} 
    \item \textbf{LaTeX\textsubscript{g}}: è un linguaggio di marcatura per la preparazione di testi, basato sul 				  		  programma di composizione tipografica TEX.\\
	Nel branch documentation  si possono trovare i file .pdf prodotti e la cartella “latex\textsubscript{g}”. La cartella latex\textsubscript{g} contiene tre cartelle interne:
	\begin{itemize}
	\item \textbf{esterni e interni}: contengono file .tex di documentazione esterna ed interna come ad esempio i \textit{Verbali} o altra documentazione esterna/interna:
	\begin{itemize}
		\item La cartella config contiene i file .tex con le parti fisse dei documenti (intestazione, registro delle modifiche, tracciamento dei temi affrontati) che vengono modificati con i dati del documento specifico;
		\item La cartella res/sections contiene i file .tex con il contenuto vero e proprio (sezioni del documento) che viene redatto in maniera libera dal redattore;
		\item Un file col nome del documento pdf con estensione .tex che viene compilato per produrre il file pdf.
	\end{itemize}
	 \item \textbf{template}: contiene file .tex di base utilizzati secondo necessità per comporre i documenti:
	 \begin{itemize}
	 \item changelox.tex è il file di template che serve per scrivere il registro delle modifiche;
	 \item package.tex è il file che contiene tutti gli usepackage;
	 \item titlepage.tex è il file di template che contiene la configurazione della pagina iniziale di ogni documento;  
	 \item tracking.tex è il file di template che contiene il tracciamento dei temi affrontati nel documento.
	 \end{itemize}
	\end{itemize}
\end{itemize}