\subsubsection{Struttura generale}
Ogni documento deve presentare le seguenti sezioni nell'ordine in cui vengono presentate:
\begin{itemize} 
    \item \textbf{Intestazione}:
    contiene:
    \begin{itemize} 
        \item Logo compreso di motto;
        \item Indirizzo email di gruppo; 
        \item Titolo;
        \item Tabella contenente le informazioni generali:
        \begin{itemize}
            \item Versione;
            \item Stato;
            \item Uso;
            \item Approvazione: indica il responsabile di progetto che ha approvato il documento; 
            \item Redazione: elenco dei collaboratori che hanno partecipato alla stesura del documento;
            \item Verifica: elenco dei verificatori che hanno verificato il documento;
            \item Distribuzione: elenco delle persone o organizzazioni a cui è destinato il documento.
        \end{itemize}
        \item Breve descrizione del documento .
    \end{itemize}
    \item \textbf{Registro delle modifiche}:
    tabella che identifica ogni versione del documento indicandone:
    \begin{itemize} 
        \item Versione;
        \item Data;
        \item Autore;
        \item Ruolo;
        \item Descrizione.
    \end{itemize}
    \item \textbf{Indice}:
    elenco ordinato dei titoli dei capitoli, ovvero delle varie parti di cui si compone il documento;
    \item \textbf{Contenuto}:
    varia a seconda del tipo di documento.
\end{itemize}

\paragraph{\textit{Verbali}}
Rispettano tutta la struttura generale.
In aggiunta presentano:
\begin{itemize} 
    \item \textbf{Informazioni Generali}
    contengono:
    \begin{itemize} 
        \item Luogo;
        \item Data;
        \item Ora;
        \item Partecipanti.
    \end{itemize}
\item \textbf{Tabella tracciamento temi affrontati}:
tabella che riassume i punti salienti della riunione indicandone:
    \begin{itemize} 
        \item Codice: ha il formato V\textbf{X} \textbf{Y}.\textbf{Z} dove X indica la tipologia di verbale, Y indica il numero di verbale (incrementale rispetto agli altri verbali);
        e Z indica il numero dell'argomento trattato (incrementale rispetto agli altri argomenti del verbale);
        \item Descrizione: breve descrizione di uno specifico argomento trattato.
    \end{itemize}

\end{itemize}