\subsubsection{Documentazione prodotta}
I documenti prodotti si dividono in:
\begin{itemize}
    \item \textbf{Interni}: documenti che contengono informazioni utili principalmente 
    per il gruppo, e che vengono quindi consultati di conseguenza. 
    Come documenti interni ci sono:
    \begin{itemize}
        \item \textbf{Norme di Progetto}: documento che definisce tutte le regole da rispettare per tutta la durata del progetto.
    \end{itemize}
    \item \textbf{Esterni}: documenti che interessano anche al proponente e al committente. Come documenti esterni ci sono:
    \begin{itemize}
        \item \textbf{Glossario}: documento in cui sono raccolti i termini per i quali si ritiene necessaria una spiegazione specifica, al fine di evitare ambiguità;
        \item \textbf{Piano di Progetto}: documento in cui è presente la pianificazione di tutte le attività del progetto, con relativo preventivo dei costi e consultivo finale;
        \item \textbf{Piano di Qualifica}: documento in cui viene valutata la qualità dei processi e le metriche adottate dal gruppo;
        \item \textbf{Analisi dei Requisiti}: documento in cui vengono stabiliti i requisiti che il prodotto finale dovrà rispettare.
    \end{itemize}
\end{itemize}