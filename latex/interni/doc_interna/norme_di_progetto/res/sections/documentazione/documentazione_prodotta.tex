\subsubsection{Documentazione prodotta}
I documenti prodotti si dividono in:
\begin{itemize}
    \item \textbf{Interni}: documenti che contengono informazioni utili principalmente 
    per il gruppo, e che vengono quindi consultati di conseguenza. 
    Come documenti interni ci sono:
    \begin{itemize}
        \item \textbf{Norme di Progetto}: documento che definisce tutte le regole da rispettare per tutta la durata del progetto.
    \end{itemize}
    \paragraph{\textit{Diario di Bordo}}
Il \textit{Diario di Bordo} è un documento informale ad uso esterno e permette di interagire con il Professore (committente) in modo da aggiornarlo sullo stato di avanzamento del progetto settimanalmente ed è usato 
anche per chiedere chiarimenti sui temi in cui riscontriamo dubbi o domande.
Fino al termine delle lezioni il \textit{Diario di Bordo} veniva redatto settimanalmente dal responsibile che presentava in classe prima dello svolgimento 
della lezione.
Dal termine delle lezioni il \textit{Diario di Bordo} viene redatto settimanalmente dal responsabile e caricato in una cartella denominata diari\_di\_bordo presente nel drive di gruppo.
\\\\
La cartella diari\_di\_bordo si trova al link: \href{https://drive.google.com/drive/u/1/folders/1a0kAZuOSsDEp_AaQUb6OQJ4XQyZ--RCN}{\\https:\/\/drive.google.com\/drive\/u\/1\/folders\/1a0kAZuOSsDEp\_AaQUb6OQJ4XQyZ\-\-RCN}
\\\\
È stato fornito l'accesso in lettura alla cartella al Professore che viene notificato tramite mail (deve contenere il link al \textit{Diario di Bordo} di interesse) al caricamento di un \textit{Diario di Bordo}.
Il formato dei diari di bordo condivisi nella cartella è .pdf.
Inoltre tutti i diari di bordo una volta redatti vengono inviati anche al canale discord\textsubscript{g} "diari-di-bordo-file" in modo da avere una duplice copia a fronte di 
qualsiasi imprevisto dato che i diari di bordo non vengono inseriti all'interno del repository di gruppo.
\\\\
\textbf{Struttura \textit{Diario di Bordo}} 
\\\\
La struttura del \textit{Diario di Bordo} non rispetta quella indicata nella sezione 3.1.1.2 relativa alla struttura generale della documentazione.
Sono state fissate delle regole per la stesura dei diari di bordo per fornire delle presentazioni il più possibile simili tra loro 
senza troppi vincoli superflui per tali documenti.
\\\\
Il \textbf{font} utilizzato per la stesura del documento è Helvetica Neue con dimensione 22pt.
Il nome del file per ogni diario di bordo deve rispettare la codifica \textbf{AAAA/MM/GG} in modo da permettere l'ordinamento
cronologico utilizzando i filtri di ricerca di Google Drive.
\\\\
Il \textit{Diario di Bordo} si divide in 4 slides che possono essere raggruppate in 3 se l'elenco degli obiettivi raggiunti e quelli futuri ci stanno
in una singola slide.
Le slides che compongono il \textit{Diario di Bordo} sono:
\begin{itemize}
    \item Intestazione; 
    \item Obiettivi raggiunti;
    \item Obiettivi futuri;
    \item Domande.
\end{itemize}
Descrizione slides:
\begin{itemize}
    \item \textbf{Intestazione}: presenta nella parte centrale il logo del gruppo compreso di slogan, in basso centrale la data relativa al \textit{Diario di Bordo} e l'anno accademico;
    \item \textbf{Obiettivi raggiunti}: presenta in alto centrale il titolo "Obiettivi raggiunti", a sinistra un elenco puntato con gli obiettivi raggiunti
    della settimana, in alto a destra il logo del gruppo compreso di slogan ed in basso a destra la data del \textit{Diario di Bordo};
    \item \textbf{Obiettivi futuri}: presenta in alto centrale il titolo "Obiettivi futuri", a sinistra un elenco puntato con gli obiettivi raggiunti
    della settimana ed in alto a destra il logo del gruppo compreso di slogan;
    \item \textbf{Domande}: presenta in alto centrale il titolo "Domande", a sinistra un elenco puntato con le domande da porre al Professore, in alto a destra il logo del gruppo compreso di slogan ed in basso 
    a destra la data del \textit{Diario di Bordo}.
\end{itemize}
    \paragraph{\textit{Verbale}}
Il \textit{Verbale} ha lo scopo di rendicontare ciò che viene detto durante una riunione.
Rispetta la struttura generale.
In aggiunta presenta:
\begin{itemize} 
    \item \textbf{Informazioni Generali}
    contiene:
    \begin{itemize} 
        \item Luogo;
        \item Data;
        \item Ora;
        \item Partecipanti.
    \end{itemize}
\item \textbf{Tabella tracciamento temi affrontati}:
tabella che riassume i punti salienti della riunione indicandone:
    \begin{itemize} 
        \item Codice: ha il formato V\textbf{X} \textbf{Y}.\textbf{Z} dove X indica la tipologia di \textit{Verbale}, Y indica il numero di \textit{Verbale} (incrementale rispetto agli altri verbali);
        e Z indica il numero dell'argomento trattato (incrementale rispetto agli altri argomenti del \textit{Verbale});
        \item Descrizione: breve descrizione di uno specifico argomento trattato.
    \end{itemize}
\end{itemize}
    \paragraph{\textit{Studio di Fattibilità}}
Lo \textit{Studio di Fattibilità} ha lo scopo di valutare i pro ed i contro dei capitolati proposti in modo da scegliere il più vantaggioso al quale candidarsi.
La scelta del capitolato viene fatta sulla base di diverse considerazioni.
Vengono valutati:
\begin{itemize}
    \item \textbf{Risorse}: in termini di budget, tempo e personale;
    \item \textbf{Previe Conoscenze}: un capitolato potrebbe rivelarsi più o meno difficile da realizzare a seconda delle conoscenze del contesto applicativo; 
    \item \textbf{Guadagno}: un capitolato potrebbe venire scartato per scarso guadagno netto al termine del lavoro commissionato (non è il nostro caso ma 
andrebbe tenuto in considerazione in ambito lavorativo);
    \item \textbf{Aspetti psicologici}: avendo la possibilità di scegliere tra capitolati (fatto che capita raramente in un reale ambito lavorativo) di pari interesse siamo portati a scegliere il contesto applicativo che ci entusiasma maggiormente.
    Questo aspetto potrebbe avere un impatto positivo sulla produttività del gruppo.
\end{itemize}

Il documento tiene in considerazione tutti i capitolati proposti per non scartare un capitolato per le sensazioni soggettive dei componenti del gruppo.
Non tenendo in considerazione ogni capitolato, infatti, si potrebbe scartare un capitolato di forte interesse senza rendersene conto.
    \item \textbf{Esterni}: documenti che interessano anche al proponente e al committente. Come documenti esterni ci sono:
    \begin{itemize}
        \item \textbf{Glossario}: documento in cui sono raccolti i termini per i quali si ritiene necessaria una spiegazione specifica, al fine di evitare ambiguità;
        \item \textbf{Piano di Progetto}: documento in cui è presente la pianificazione di tutte le attività del progetto, con relativo preventivo dei costi e consultivo finale;
        \item \textbf{Piano di Qualifica}: documento in cui viene valutata la qualità dei processi e le metriche adottate dal gruppo;
        \item \textbf{Analisi dei Requisiti}: documento in cui vengono stabiliti i requisiti che il prodotto finale dovrà rispettare.
    \end{itemize}
    \documentclass[10pt]{article}

\usepackage{geometry}
\usepackage{fancyhdr,graphicx}
\usepackage{hyperref}
\usepackage{eurosym}

\geometry{a4paper,top=2.5cm,bottom=2.5cm,left=2cm,right=2cm}

\fancypagestyle{firstpage}{%
  \fancyhf{}% Clear header/footer
  \renewcommand{\headrulewidth}{0pt}%
}

\fancypagestyle{otherpages}{%
  \fancyhf{}% Clear header/footer
  \renewcommand{\headrulewidth}{1pt}%
}

\AtBeginDocument{\thispagestyle{firstpage}}
\pagestyle{otherpages}

\setlength{\parindent}{0pt}
\setlength{\parskip}{1ex}

\begin{document}

\noindent\begin{minipage}{0.5\textwidth}% adapt widths of minipages to your needs
\includegraphics[width=9cm]{images/logo.jpeg}
\end{minipage}%
\hfill%
\begin{minipage}{4cm}
\includegraphics[width=2.3cm]{images/uni.png}
\end{minipage}

\bigskip\bigskip

\begin{tabular}{ @{} l  }
  Gruppo \textit{Seven Clickers} \\ 
  E-mail: \textit{\href{mailto:7clickersgroup@gmail.com}{7clickersgroup@gmail.com} }\\ 
  Corso di Ingegneria del Software AA 2022/2023 \\
  17 Marzo 2023
\end{tabular}

\bigskip
\hfill
\begin{tabular}{ l @{} }
Prof. Vardanega Tullio\\
Prof. Cardin Riccardo\\
Università degli Studi di Padova\\
Dipartimento di Matematica\\
Via Trieste, 63\\
35121 Padova
\end{tabular}

\bigskip

Egregio Prof. Vardanega Tullio,\\
Egregio Prof. Cardin Riccardo,\\

\bigskip

Con la presente il gruppo \textit{Seven Clickers} intende comunicarVi la partecipazione al secondo passaggio della revisione di avanzamento RTB,
al fine di esporvi l’avanzamento dello sviluppo del progetto, denominato:

\begin{center}
  \textbf{ShowRoom3D}
\end{center}

proposto dall’azienda \textbf{Sanmarco Informatica}.


Si allegano i seguenti documenti di interesse:
\begin{itemize}
  \item \textit{\textit{Studio di Fattibilità}}
  \item \textit{Glossario} vx.x.x
  \item \textit{Analisi dei Requisiti} v2.0.0
  \item \textit{Norme di Progetto} v1.0.0
  \item \textit{Piano di Progetto} v2.0.0
  \item \textit{Piano di Qualifica} v1.0.0
\end{itemize}

Inoltre sono allegati anche i verbali esterni ed interni:

\begin{itemize}
  \item \textit{Verbale} esterno del 25-10-2022
  \item \textit{Verbale} esterno del 17-11-2022
  \item \textit{Verbale} esterno del 11-01-2023
  \item \textit{Verbale} esterno del 18-01-2023
  \item \textit{Verbale} esterno del 17-02-2023
  \item \textit{Verbale} interno del 19-10-2022
  \item \textit{Verbale} interno del 25-10-2022
  \item \textit{Verbale} interno del 26-10-2022
  \item \textit{Verbale} interno del 04-11-2022
  \item \textit{Verbale} interno del 09-11-2022
  \item \textit{Verbale} interno del 16-11-2022
  \item \textit{Verbale} interno del 23-11-2022
  \item \textit{Verbale} interno del 01-12-2022
  \item \textit{Verbale} interno del 07-12-2022
  \item \textit{Verbale} interno del 14-12-2022
  \item \textit{Verbale} interno del 04-01-2023
  \item \textit{Verbale} interno del 25-01-2023
  \item \textit{Verbale} interno del 01-02-2023
  \item \textit{Verbale} interno del 08-02-2023
  \item \textit{Verbale} interno del 24-02-2023
  \item \textit{Verbale} interno del 28-02-2023
\end{itemize}

Il gruppo stima di consegnare il prodotto\textsubscript{g} entro il 03-05-2023 con un preventivo di 13975\euro{}  come
specificato nel documento \textit{Piano di Progetto} v2.0.0.


Cordiali saluti,

\vspace{15pt}

\hfill
\begin{tabular}{ l @{} }
Rino Sincic\\
\textit{Responsabile di Progetto}\\
\includegraphics[width=2.3cm]{images/Rino_Sincic_firma.png}
\end{tabular}

\end{document}
    \paragraph{\textit{Piano di Progetto}}

Il documento piano di progetto ha lo scopo di aiutare il gruppo nella gestione delle risorse a disposizione per portare a termine il progetto entro la data decisa.
Il documento ha anche la funzione di monitorare l'avanzamento del progetto in modo da poter applicare miglioramenti continui e azioni correttive
basandosi sull'esperienza ottenuta da pianificazioni precedenti.
\\\\
\textbf{Struttura piano di progetto}
\\\\
La struttura del piano di progetto segue la struttura generale.
In aggiunta il contenuto del documento si compone di:
\begin{enumerate}
    \item Analisi dei rischi;
    \item Pianificazione;
    \item Preventivo.
\end{enumerate}
\textbf{NOTA:} ogni tabella o immagine all'interno del documento verrá indicata anche nell'indice del documento stesso in modo che sia rintracciabile con 
facilitá.
\\\\
\textbf{Descrizione delle sezioni:}
\\\\
\textbf{Analisi dei rischi:} vengono riportati in forma tabellare i rischi a cui si va incontro aggiudicandosi il capitolato.
Ogni rischio fa riferimento ad una categoria di rischi precisa e viene indicato con un nome,una probabilitá che si verifichio,
un grado che indica l'impatto negativo che puó
comportare es una breve descrizione su come affrontare il rischio nel momento in cui dovesse presentarsi.
\\\\
\textbf{Pianificazione:} la sezione dedicata alle pianificazioni ha lo scopo di indicare l'inizio e la fine di un periodo di pianificazione e 
di fornire la suddivisione e l'organizzazione delle attivitá all'interno di tale periodo.
Viene fornito un elenco con descrizione delle attivitá che si andranno a svolgere e viene diviso il periodo di pianificazione in sottoperiodi 
ciascuno con un inizio,una fine ed indicando cosa verrá fatto.
Infine é presente un diagramma di Gantt che illustra graficamente l'ordine temporale delle attivitá da svolgere tenendo conto di 
eventuali margini temporali dovuti ad imprevisti vari.
\\\\
\textbf{Preventivo:} include una sezione iniziale in cui si vanno ad indicare in forma tabellare le risore umane disponibili all'inizio del progetto e 
come queste risorse andranno suddivise. Successivamente viene fornito,sempre in forma tabellare,il costo totale calcolato in base alle ore 
di impegno dei componenti del gruppo ed al ruolo che dovranno ricoprire (ogni ruolo ha un costo orario).
Le tabelle riportate produrranno come risultato la conclusione del preventivo che comprende il costo totale calcolato e la data di fine progetto.
É presente una sezione (dettaglio periodi) che prevede i preventivi di ciascun periodo di pianificazione.
In questa sezione sono fornite le tabelle con le ore che ciascun componente del gruppo dovrá svolgere con relativi ruoli associati ed una tabella con i costi
derivati da tali ruoli.
Nel caso del nostro progetto viene prevista una rotazione dei ruoli a scopo didattico perció ogni componente userá le sue ore di impegno in modo diverso 
a seconda della rotazione dei ruoli.
\end{itemize}






