\paragraph{\textit{Lettera di Presentazione}}
Una \textit{Lettera di Presentazione} serve a manifestare la volontá da parte del gruppo di prendere un impegno con il committente.
L'impegno puó essere la candidatura per un capitolato di interesse o per una revisione di avanzamento.
Dopo aver ricevuto una \textit{Lettera di Presentazione} sta al committente la decisione di accettare o rifiutare l'impegno che ne consegue.
\\\\
\textbf{Struttura \textit{Lettera di Presentazione}}
\\\\
Una \textit{\textit{Lettera di Presentazione}} é formata dai seguenti campi:
\begin{itemize}
    \item Header
    \item Mittente e data
    \item Destinatari
    \item Contenuto
    \item Riferimenti a documenti
    \item Conclusioni e saluti
    \item Nome,cognome,firma responsabile
\end{itemize}
\textbf{Header} contiene il logo del gruppo (con slogan) a sinistra ed il logo dell'Universitá di Padova a destra.
\\\\
\textbf{Mittente e data} contiene il nome,email del gruppo seguito dal nome del corso e dalla data di invio della lettera al destinatario.
\\\\
\textbf{Destinatari} contiene i destinatari della lettera compresi di titoli e luogo della loro sede.
\\\\
\textbf{Contenuto} contiene il contenuto della lettera scritto in modo formale dichiarando lo scopo della lettera.
\\\\
\textbf{Riferimenti a documenti} contiene un elenco puntato di tutti i documenti (e loro versioni) a cui si vuole sottoporre l'attenzione dei destinatari. 
\\\\
\textbf{Conclusioni} contiene le considerazioni finali seguite dai saluti rivolti al destinatario.
\\\\
\textbf{Nome,cognome,firma responsabile} contiene il nome,cognome,titolo e firma digitale del responsabile.
\\\\
Di seguito viene riportata un'immagine che illustra visivamente i campi appena descritti...
\begin{figure}[htbp]
    \centering
    \fbox{\includegraphics[scale=0.68]{../../../template/images/esempio\_lettera\_di\_presentazione.png}}
    \caption{Esempio di \textit{Lettera di Presentazione}}
\end{figure}
\pagebreak