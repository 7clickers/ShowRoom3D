\paragraph{\textit{Norme di Progetto}}
\textit{Norme di Progetto} è un documento che definisce le regole e gli standard che devono essere seguiti durante il processo di progettazione e 
sviluppo del prodotto. Le \textit{Norme di Progetto} includono le linee guida e le specifiche per la progettazione, i requisiti di qualità,
i requisiti di test e l'usabilità oltre ad includere la documentazione tecnica, le procedure 
e le politiche applicate al progetto. Questo documento può anche fornire una roadmap per lo sviluppo del prodotto e la gestione del cambiamento 
e le responsabilità del team.
\\\\
\textbf{Struttura \textit{Norme di Progetto}}
\\\\
La struttura di \textit{Norme di Progetto} segue la struttura generale.
In aggiunta il contenuto del documento si compone di:
\begin{itemize}
    \item Introduzione;
    \item Processi primari;
    \item Processi di supporto;
    \item Processi organizzativi;
    \item Standard di qualità ISO/IEC 9126;
    \item Standard di qualità ISO/IEC 12207:1995;
\end{itemize}
\noindent Descrizione delle varie sezioni:
\begin{itemize}
\item \textbf{Introduzione:} comprende una descrizione dello scopo del documento e dello scopo del prodotto;

\item \textbf{Processi primari:} i processi primari di sviluppo software sono un insieme di procedure che devono essere seguite per progettare, sviluppare, distribuire e gestire un software. 
Questi processi sono considerati come la base di tutte le attività di sviluppo software e sono spesso chiamati "ciclo di vita del software". 
I processi primari inclusi nel nostro progetto sono:
\begin{itemize}
    \item Fornitura: viene descritto il modo con il quale il gruppo si interfaccia con il proponente;
    \item Sviluppo: noto anche come "ciclo di vita del software";
        \begin{enumerate}
            \item Analisi dei requisiti: è il primo passo per lo sviluppo di un software. Questo processo consiste nell'analizzare le esigenze del cliente e nello specificare i requisiti funzionali e non funzionali del software;
        \end{enumerate}
    \item Progettazione: consiste nell'identificare e specificare le componenti di un sistema software, includendo l'architettura, le interfacce, i modelli di dati e i flussi di lavoro;
    \item Implementazione (codifica): consiste nella codifica, nella compilazione e nella verifica del codice sorgente.
\end{itemize}
\item \textbf{Processi di supporto:} il processo di supporto software è una metodologia utilizzata per la gestione dei problemi tecnici legati al software. 
Consiste nell'utilizzo di strumenti, tecniche e procedure che aiutano un team a risolvere i problemi. 
Il processo di supporto software si concentra sulla gestione delle richieste di assistenza, la diagnosi dei problemi, la risoluzione dei problemi.
I processi di supporto inclusi nel nostro progetto sono:
\begin{itemize}
    \item Documentazione: è un insieme di documenti che descrivono in dettaglio il progetto software, comprese le sue funzionalità, i requisiti, l'architettura, il design, i processi di sviluppo e i risultati. 
    Questa documentazione serve come riferimento per lo sviluppo del software e aiuta a garantire la coerenza e la completezza del progetto. 
    La documentazione del progetto software è un elemento critico per la pianificazione, lo sviluppo e il mantenimento del software e deve essere costantemente aggiornata durante tutto il ciclo di vita del progetto.
    \item Gestione della configurazione: la gestione della configurazione è un processo che mira a gestire e controllare i cambiamenti apportati a un prodotto software o a un sistema durante il suo ciclo di vita. 
    La documentazione della gestione della configurazione descrive come vengono identificate, controllate, tracciate e gestite le versioni dei componenti del prodotto software;
    \item Verifica: la verifica della documentazione del progetto software è un processo che mira a verificare l'accuratezza e la completezza della documentazione del progetto;
    \item Validazione: processo che mira a confermare che la documentazione rappresenti effettivamente il progetto software e che soddisfi i requisiti e le specifiche del progetto;
\end{itemize}
\item \textbf{Processi organizzativi:} i processi organizzativi software sono un insieme di procedure e regole che definiscono come un'organizzazione deve gestire la progettazione, lo sviluppo, l'implementazione e la manutenzione di un software.
Comprendono le linee guida che devono essere seguite, le procedure e i processi da seguire, i requisiti da rispettare e le regole da rispettare.
I processi organizzativi inclusi nel nostro progetto sono:
\begin{itemize}
    \item Gestione organizzativa: riguarda l'organizzazione all'interno del gruppo ovvero la ripartizione dei compiti e le attivitá che ogni figura professionale è tenuta a svolgere fino al 
    completamento del progetto;
    \item Gestione infrastrutture: cruciale per garantire la disponibilità, l'affidabilità e la sicurezza delle risorse necessarie per lo sviluppo e l'esecuzione del software.
\end{itemize}
\item \textbf{Standard di qualità ISO/IEC 9126:} La norma ISO/IEC 9126 definisce uno standard di qualità per il software. 
Si basa su sei criteri di qualità: affidabilità, usabilità, efficienza, mantenibilità, conformità e funzionalità. 
Ogni criterio è suddiviso in sottocriteri che possono aiutare gli sviluppatori a valutare e migliorare la qualità del loro software. 
Il documento offre anche consigli generali su come misurare e monitorare questi criteri di qualità.

\item \textbf{Standard di qualità ISO/IEC 12207:1995:} ISO/IEC 12207:1995 è uno standard internazionale rilasciato dall'International Organization 
for Standardization (ISO) e dall'International Electrotechnical Commission (IEC) nel 1995. 
Lo standard definisce un processo di sviluppo del software che può essere utilizzato da tutti i team di sviluppo software per garantire 
un prodotto di qualità. Il processo di sviluppo può essere applicato a qualsiasi tipo di progetto software, indipendentemente dal 
linguaggio di programmazione, dalle dimensioni, dai requisiti o dall'ambiente. 
Lo standard fornisce una struttura coerente per l'organizzazione, la pianificazione e l'implementazione di un progetto, 
inoltre, definisce i processi necessari per assicurare che il prodotto consegnato soddisfi i requisiti di qualità stabiliti dal cliente.
\end{itemize}