\paragraph{\textit{Studio di Fattibilità}}
Lo \textit{Studio di Fattibilità} ha lo scopo di valutare i pro ed i contro dei capitolati proposti in modo da scegliere il più vantaggioso al quale candidarsi.
La scelta del capitolato viene fatta sulla base di diverse considerazioni.
Vengono valutati:
\begin{itemize}
    \item \textbf{Risorse}: in termini di budget, tempo e personale;
    \item \textbf{Previe Conoscenze}: un capitolato potrebbe rivelarsi più o meno difficile da realizzare a seconda delle conoscenze del contesto applicativo; 
    \item \textbf{Guadagno}: un capitolato potrebbe venire scartato per scarso guadagno netto al termine del lavoro commissionato (non è il nostro caso ma 
andrebbe tenuto in considerazione in ambito lavorativo);
    \item \textbf{Aspetti psicologici}: avendo la possibilità di scegliere tra capitolati (fatto che capita raramente in un reale ambito lavorativo) di pari interesse siamo portati a scegliere il contesto applicativo che ci entusiasma maggiormente.
    Questo aspetto potrebbe avere un impatto positivo sulla produttività del gruppo.
\end{itemize}

Il documento tiene in considerazione tutti i capitolati proposti per non scartare un capitolato per le sensazioni soggettive dei componenti del gruppo.
Non tenendo in considerazione ogni capitolato, infatti, si potrebbe scartare un capitolato di forte interesse senza rendersene conto.