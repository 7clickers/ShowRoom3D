\paragraph{\textit{Studio di Fattibilitá}}
Lo \textit{Studio di Fattibilitá} ha lo scopo di valutare i pro ed i contro dei capitolati proposti in modo da scegliere il piú vantaggioso al quale candidarsi.
La scelta del capitolato viene fatta sulla base di diverse considerazioni.
Un capitolato potrebbe essere vantaggioso per alcuni aspetti ma svantaggioso per altri.
Inoltre potrebbero presentarsi capitolati irrealizzabili per mancanza di risorse o per scarse conoscenze del contesto applicativo.
Oltre a risorse in termini di budget,tempo e personale un capitolato potrebbe venire scartato per scarso guadagno netto al termine del lavoro commissionato (non é il nostro caso ma 
andrebbe tenuto in considerazione in ambito lavorativo).
Avendo la possibilitá di scegliere tra capitolati (fatto che capita raramente in un reale ambito lavorativo) di pari interesse siamo portati a scegliere il contesto applicativo che ci entusiasma maggiormente.
Questo aspetto per lo piú psicologico potrebbe avere un impatto positivo sulla produttivitá del gruppo.
Il documento tiene in considerazione tutti i capitolati proposti per non scartare un capitolato per le sensazioni soggettive dei componenti del gruppo.
Non tenendo in considerazione ogni capitolato,infatti,si potrebbe scartare un capitolato di forte interesse senza rendersene conto.