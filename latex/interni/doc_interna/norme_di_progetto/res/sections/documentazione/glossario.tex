\paragraph{\textit{Glossario}}
All'interno della documentazione si possono trovare dei termini che possono risultare ambigui a seconda del contesto, o non conosciuti dagli utilizzatori.\\\
Per ovviare ad incomprensioni si è deciso di stilare un elenco di termini di interesse accompagnati da una descrizione del loro significato.\\
I termini presenti all'interno dei documenti che necessitano di una descrizione vengono indicati con il pedice 'g' come nell'esempio seguente: termine.
É quindi possibile consultare il \textit{Glossario} per reperire tale descrizione.
\\
Ogni componente del gruppo all'inserimento di un termine ritenuto ambiguo deve preoccuparsi di aggiornare il \textit{Glossario}.

Per aggiornare il \textit{Glossario} si devono inserire i nuovi termini nel file .tex nella cartella corrispondente all'iniziale del termine situata al percorso
\textbf{\textit{latex/esterni/doc\_esterna/Glossario/res/sections/alphabet/}}.
É stato creato uno script che scansiona un documento di interesse per inserire in automatico il pedice sui termini contenuti nel \textit{Glossario}.

Il componente del gruppo che inserisce all'interno del \textit{Glossario} un nuovo termine deve aggiungere nel file .tex il segnaposto \%parola\% dopo la subsection (senza spazi) che racchiude il termine per permetterne il riconoscimento da parte dello script.
\\
Il \textit{Glossario} ordina i termini in ordine alfabetico in modo da permetterne una facile e veloce ricerca.