\subsubsection{Convenzioni}
Le convenzioni di seguito riportate vengono applicate a tutti i documenti.
Esse rendono i documenti stilati omogenei tra loro contribuendo a rendere il progetto professionale.

\paragraph{Versionamento}
Il numero di versione permette di capire lo stato in cui si trova un documento.
Un documento può trovarsi nei seguenti stati:
\begin{itemize} 
    \item \textbf{Approvato}: il documento è verificato ed approvato dal responsabile;
    \item \textbf{Verificato}: il documento risulta verificato ma non ancora visionato dal responsabile;
    \item \textbf{In Sviluppo}: sono presenti delle modifiche che non sono state verificate.
\end{itemize}
Il numero di versione ha il formato \textbf{X.Y.Z} dove:
\begin{itemize} 
    \item \textbf{X}: indica una versione approvata dal responsabile, la numerazione parte da 0
    e la prima versione approvata è la 1.0.0;
    \item \textbf{Y}: indica una versione verificata dal verificatore, la numerazione inizia da 0 e si azzera ad ogni incremento di X. La prima versione
    verificata è la 0.1.0;
    \item \textbf{Z}: indica una versione in fase di modifica da parte dei redattori che ne incrementano il numero ad ogni modifica,
    la numerazione parte da 1 e si azzera ad ogni incremento di X o Y. La prima versione modificata è la 0.0.1.
\end{itemize}

\paragraph{Date}
Le date devono rispettare il seguente formato: \textbf{dd-mm-yyyy}
All'interno delle tabelle il formato deve essere il seguente: \textbf{dd-mm-yy}
\paragraph{Nomi di persona}
All'interno dei documenti i nomi di persona rispetteranno l'ordine nome seguito dal cognome della persona menzionata.

\paragraph{Elenchi puntati}
Gli elenchi puntati devono rispettare le seguenti regole:
\begin{itemize} 
    \item Ogni elemento dell'elenco deve iniziare con la lettera maiuscola;
    \item Ogni elemento dell'elenco deve terminare con ";" ad eccezione dell'ultimo elemento
    che deve terminare con "."; 
    \item Dopo i due punti la frase deve iniziare con la lettera minuscola.
\end{itemize}

\paragraph{Stile del testo}
\begin{itemize} 
    \item \textbf{Grassetto}: stile utilizzato per i titoli delle sezioni e per i primi termini degli elenchi puntati;
    \item \textbf{Corsivo}: viene utilizzato per citare il nome dei documenti, ad esempio \textit{Piano di Progetto}. 
\end{itemize}

\paragraph{Immagini}
Le immagini sono raccolte nella cartella “”, e sono inserite sempre con una didascalia descrittiva posizionata sotto l’immagine.

\paragraph{Tabelle} 
Le tabelle sono provviste di didascalia descrittiva posizionata sotto alla tabella. La tabella contenente il registro delle modifiche è l’unica che fa da eccezione a questa regola.

\paragraph{\textit{Glossario}}
All'interno del documento si possono trovare dei termini che possono risultare ambigui a seconda del contesto, o non conosciuti dagli utilizzatori.\\\
Per ovviare ad errori di incomprensione che possono portare a problemi di vario genere e rallentamenti si è deciso di stilare un elenco di termini 
di interesse accompagnati da una descrizione dettagliata del loro significato.\\
I termini presenti all'interno del \textit{Glossario} vengono indicati con il pedice 'g' come nell'esempio seguente: termine\textsubscript{g}.
Il \textit{Glossario} ordina i termini in ordine alfabetico in modo da permetterne una facile e veloce ricerca.
Ogni componente del gruppo all'inserimento di un termine ritenuto ambiguo deve preoccuparsi di aggiornare il \textit{Glossario} in modo da mantenerlo sempre aggiornato.