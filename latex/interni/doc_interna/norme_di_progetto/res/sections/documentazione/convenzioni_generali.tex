\subsubsection{Convenzioni generali}
Le convenzioni di seguito riportate vengono applicate a tutti i documenti.
Esse rendono i documenti stilati omogenei tra loro contribuendo a rendere il progetto professionale.

\paragraph{Versionamento}
Il numero di versione permette di capire lo stato in cui si trova un documento.
Un documento può trovarsi nei seguenti stati:
\begin{itemize} 
    \item \textbf{Approvato}:Il documento è verificato ed approvato dal Responsabile di progetto
    \item \textbf{Verificato}:Il documento risulta verificato ma non ancora visionato dal Responsabile di progetto
    \item \textbf{In Sviluppo}:Sono presenti delle modifiche che non sono state verificate
\end{itemize}
Il numero di versione ha il formato \textbf{X.Y.Z} dove:
\begin{itemize} 
    \item \textbf{X} indica una versione approvata dal Responsabile di progetto,la numerazione parte da 0
    e la prima versione approvata è la 1.0.0
    \item \textbf{Y} indica una versione verificata dal Verificatore,la numerazione inizia da 0 e si azzera ad ogni incremento di X.La prima versione
    verificata è la 0.1.0
    \item \textbf{Z} indica una versione in fase di modifica da parte dei redattori che ne incrementano il numero ad ogni modifica,
    la numerazione parte da 1 e si azzera ad ogni incremento di X o Y.La prima versione modificata è la 0.0.1
\end{itemize}

\paragraph{Struttura Generale}
Ogni documento deve presentare le seguenti sezioni nell'ordine in cui vengono presentate:
\begin{itemize} 
    \item \textbf{Intestazione}:
    Contiene:
    \begin{itemize} 
        \item Logo compreso di motto
        \item Indirizzo email di gruppo 
        \item Titolo
        \item Tabella contenente le informazioni generali
        \begin{itemize}
            \item Versione
            \item Stato
            \item Uso
            \item Approvazione: indica il responsabile di progetto che ha approvato il documento 
            \item Redazione: elenco dei collaboratori che hanno partecipato alla stesura del documento
            \item Verifica: elenco dei verificatori che hanno verificato il documento
            \item Distribuzione: elenco delle persone o organizzazioni a cui è destinato il documento
        \end{itemize}
        \item Breve descrizione del documento 
    \end{itemize}
    \item \textbf{Registro delle modifiche}:
    Tabella che identifica ogni versione del documento indicandone:
    \begin{itemize} 
        \item Versione
        \item Data
        \item Autore
        \item Ruolo
        \item Descrizione
    \end{itemize}
    \item \textbf{Indice}:
    Elenco ordinato dei titoli dei capitoli, ovvero delle varie parti di cui si compone il documento.
    \item \textbf{Contenuto}:
    Varia a seconda del tipo di documento.
\end{itemize}

\paragraph{Verbali}
Rispettano tutta la struttura generale.
In aggiunta presentano:
\begin{itemize} 
    \item \textbf{Informazioni Generali}
    Contengono:
    \begin{itemize} 
        \item Luogo
        \item Data
        \item Ora
        \item Partecipanti
    \end{itemize}
\item \textbf{Tabella tracciamento temi affrontati}:
tabella che riassume i punti salienti della riunione indicandone
    \begin{itemize} 
        \item Codice: ha il formato V\textbf{X} \textbf{Y}.\textbf{Z} dove X indica la tipologia di verbale,Y indica il numero di verbale (incrementale rispetto agli altri verbali) 
        e Z indica il numero dell'argomento trattato (incrementale rispetto agli altri argomenti del verbale) 
        \item Descrizione: breve descrizione di uno specifico argomento trattato
    \end{itemize}

\end{itemize}

\paragraph{Date}
Le date devono rispettare il seguente formato: \textbf{dd-mm-yyyy}
All'interno delle tabelle il formato deve essere il seguente: \textbf{dd-mm-yy}
\paragraph{Nomi di persona}
All'interno dei documenti i nomi di persona rispetteranno l'ordine nome seguito dal cognome della persona menzionata.

\paragraph{Elenchi puntati}
Gli elenchi puntati devono rispettare le seguenti regole:
\begin{itemize} 
    \item Ogni elemento dell'elenco deve iniziare con la lettera maiuscola;
    \item Ogni elemento dell'elenco deve terminare con ";" ad eccezione dell'ultimo elemento
    che deve terminare con "."; 
    \item Dopo i due punti la frase deve iniziare con la lettera minuscola.
\end{itemize}

\paragraph{Stile del testo}
\begin{itemize} 
    \item \textbf{Grassetto}: stile utilizzato per i titoli delle sezioni e per i primi termini degli elenchi puntati;
    \item \textbf{Corsivo}: viene utilizzato per citare il nome dei documenti, ad esempio \textit{Piano di Progetto}. 
\end{itemize}
