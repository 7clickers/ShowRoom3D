\subsection{Verifica}
La verifica viene svolta da due verificatori prima del merge con il branch documentation.
Consiste nell'esaminare i file prodotti da chi ne ha fatto la stesura e segnalarne la non validità o 
la presenza di errori nei concetti esposti.\\
Un verificatore dovrà verificare il documento a partire dalle modifiche fatte dopo l'ultima versione verificata.
Le modifiche da verificare quindi possono essere dedotte dal registro dei cambiamenti presente in ogni documento.
Una volta controllato il documento, il primo verificatore segnalerà eventuali errori e
successivamente dovrà spuntare come approvata la Pull Request nella sezione dedicata su GitHub\textsubscript{g}. \\
A questo punto se un secondo verificatore noterà la necessità di qualche altro cambiamento da apportare, chi dovrà apportare le modifiche farà una pull in locale per
allineare il proprio branch con quello in remoto e continuare con il proprio lavoro.\\
Dopo il push delle modifiche se i file risultano corretti anche dal secondo verificatore esso aggiungerà i nomi dei verificatori all'intestazione modificando il file titlepage\_input.tex
 e creerà la riga nel registro delle modifiche, nel file changelog\_input.tex,  inserendo la nuova versione secondo le regole di versionamento e scrivendo nella colonna Autore sia il suo nome,
che quello del primo verificatore, e in descrizione "Verifica".
Dopo aver eseguito un commit sul ramo da integrare approva la Pull Request e conferma il merge secondo le norme di progetto descritte nella sezione dedicata.