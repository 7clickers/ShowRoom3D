\subsection{Processi organizzativi}
Le attività e i compiti in un processo organizzativo sono di responsabilità dell'organizzazione che utilizza il processo. L'organizzazione garantisce che il processo sia attivo e funzionante.
I processi organizzativi definiti nello standard sono i seguenti.

\subsubsection{Gestione organizzativa}
Il processo di gestione organizzativa ha lo scopo di organizzare e monitorare i processi per il raggiungimento dei loro obiettivi. Il processo è stabilito da una organizzazione per assicurare la consistente applicazione di pratiche per l'uso dall'organizzazione e nei progetti.
Le attività del processo sono le seguenti:
\begin{itemize}
\item Inizio e definizione dello scopo;
\item Pianificazione;
\item Esecuzione e controllo;
\item Revisione e valutazione;
\item Chiusura.
\end{itemize}
\subsubsection{Gestione delle infrastrutture}
Il processo di gestione delle infrastrutture è un processo per stabilire e mantenere l'infrastruttura necessaria per qualsiasi altro processo. L'infrastruttura può includere hardware, software, strumenti, tecniche, standard e strutture per lo sviluppo, il funzionamento o la manutenzione.
Le attività del processo sono le seguenti:
\begin{itemize}
\item Implementazione del processo;
\item Realizzazione dell'infrastruttura;
\item Manutenzione dell'infrastruttura.
\end{itemize}

\subsubsection{Miglioramento del processo}
Il processo di miglioramento è un processo per stabilire, valutare, misurare, controllare e migliorare un processo del ciclo di vita del software.
Le attività del processo sono le seguenti:
\begin{itemize}
\item Stabilimento del processo;
\item Valutazione del processo;
\item Miglioramento del processo.
\end{itemize}

\subsubsection{Formazione del personale}
Il processo di formazione del personale è un processo per fornire e mantenere personale qualificato. L'acquisizione, la fornitura, lo sviluppo, il funzionamento o la manutenzione di prodotti software dipende in gran parte da personale esperto e qualificato.
Le attività del processo sono le seguenti:
\begin{itemize}
\item Implementazione del processo;
\item Sviluppo materiale didattico;
\item Realizzazione piano formativo.
\end{itemize}