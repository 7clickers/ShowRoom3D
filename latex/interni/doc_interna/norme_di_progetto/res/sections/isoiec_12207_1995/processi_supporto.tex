\subsection{Processi di supporto}
I processi di supporto aiutano le attività di tutti gli altri processi dell'organizzazione a garantire il successo e la qualità del progetto. Questi processi possono essere attivati da un processo primario o da un altro processo di supporto.
Le attività e i compiti in un processo di supporto sono di responsabilità dell'organizzazione che esegue tale processo. Questa organizzazione garantisce che il processo sia in atto e funzionante.
I processi di supporto definiti dallo standard sono i seguenti.
\subsubsection{Documentation process}
Il processo di documentazione è un processo per la registrazione delle informazioni prodotte da un processo o attività del ciclo di vita. Il processo contiene l'insieme delle attività che pianificano, progettano, sviluppano, producono, modificano, distribuiscono e mantengono quei documenti necessari a tutti gli interessati.
Le attività del processo sono le seguenti:
\begin{itemize}
\item Implementazione di processo
\item Produzione
\item Progettazione e sviluppo
\item Manutenzione
\end{itemize}
\subsubsection{Configuration management process}
Il processo di gestione della configurazione è un processo di applicazione di procedure amministrative e tecniche durante tutto il ciclo di vita del software per mantenere l'integrità di tutti i componenti della configurazione e di renderli accessibili a chi ne ha diritto.
Le attività del processo sono le seguenti:
\begin{itemize}
\item Implementazione del processo
\item Identificazione della configurazione
\item Controllo della configurazione
\item Valutazione dello stato di configurazione
\item Gestione del rilascio e distribuzione
\end{itemize}
\subsubsection{Quality assurance process}
Il processo di garanzia della qualità è un processo per fornire un'adeguata garanzia che i prodotti ed i processi software nel ciclo di vita del progetto siano conformi ai requisiti specificati e aderiscano ai piani stabiliti. Per essere imparziale, la garanzia della qualità deve avere libertà organizzativa e autorità da parte delle persone direttamente responsabili dello sviluppo del prodotto software o dell'esecuzione del processo nel progetto.
Le attività del processo sono le seguenti:
\begin{itemize}
\item Implementazione del processo
\item Accertamento di prodotto
\item Accertamento di processo
\item Accertamento della qualità di sistema
\end{itemize}
\subsubsection{Verification process}
Il processo di verifica è un processo per determinare se i prodotti software di un'attività soddisfano i requisiti o le condizioni loro imposti nelle attività precedenti.
Le attività del processo sono le seguenti:
\begin{itemize}
\item Implementanzione del processo
\item Verifica
\end{itemize}
\subsubsection{Validation process}
Il processo di validazione è un processo per determinare se i requisiti e il sistema finale soddisfano l'uso specifico previsto.
Le attività del processo sono le seguenti:
\begin{itemize}
\item Implementazione del processo
\item Validazione
\end{itemize}

\subsubsection{Joint review process}
Il processo di revisione congiunta è un processo per valutare lo stato ed i prodotti di un'attività di un progetto in modo appropriato. Le revisioni congiunte avvengono sia a livello di gestione del progetto che a livello tecnico.
\begin{itemize}
\item Implementazione del processo
\item Revisioni della gestione del progetto
\item Revisioni tecniche
\end{itemize}

\subsubsection{Audit process}
Il processo di Audit ha lo scopo di determinare in maniera indipendente la conformità di prodotti e processi selezionati ai requisiti, piani e accordi.
Le attività del processo sono le seguenti:
\begin{itemize}
\item Implementazione del processo
\item Audit
\end{itemize}

\subsubsection{Problem resolution process}
Il Processo di risoluzione dei problemi è un processo per l'analisi e la risoluzione dei problemi, indipendentemente dalla loro natura o origine, che vengono scoperti durante l'esecuzione di processi di sviluppo, funzionamento, manutenzione o altri. L'obiettivo è fornire un mezzo tempestivo, responsabile e documentato per garantire che tutti i problemi scoperti siano analizzati e risolti e che le tendenze siano riconosciute.
\begin{itemize}
\item Implementazione del processo
\item Risoluzione del problema
\end{itemize}

