\subsection{Processi primari}
Questi processi sono essenziali in quanto un progetto è detto tale se e solo se è attivo almeno un processo\textsubscript{g} primario.
I processi di supporto definiti dallo standard sono i seguenti.
\subsubsection{Acquisizione}
Il processo\textsubscript{g} di acquisizione contiene le attività ed i compiti dell'acquirente.
Le attività del processo\textsubscript{g} sono le seguenti:
\begin{itemize}
\item Iniziazione
\item Preparazione della richiesta di proposta;
\item Preparazione e aggiornamento del contratto;
\item Monitoraggio dei fornitori;
\item Accettazione e completamento.
\end{itemize}

\subsubsection{Fornitura}
Il processo\textsubscript{g} di fornitura contiene le attività e le mansioni del fornitore. Il processo\textsubscript{g} può essere.
Le attività del processo\textsubscript{g} sono le seguenti:
\begin{itemize}
\item Iniziazione;
\item Preparazione alla risposta;
\item Contratto;
\item Pianificazione;
\item Esecuzione e controllo;
\item Revisione e valutazione;
\item Rilascio e completamento.
\end{itemize}

\subsubsection{Sviluppo}
Il processo\textsubscript{g} di sviluppo contiene le attività e i compiti dello sviluppatore.
Il processo\textsubscript{g} contiene le attività per l'analisi dei requisiti, la progettazione, la codifica, l'integrazione, il test, l'installazione e l'accettazione relative al sistema se previsto dal contratto.
Le attività del processo\textsubscript{g} sono le seguenti:
\begin{itemize}
\item Implementazione dei processi;
\item Analisi dei requisiti di sistema;
\item Progettazione architettura di sistema;
\item Analisi requisiti software;
\item Progettazione architettura software;
\item Progettazione dettagliata del software;
\item Codifica e test del software;
\item Integrazione software;
\item Test di qualificazione del software;
\item Integrazione del sistema;
\item Test di qualificazione del sistema; 
\item Installazione software;
\item Supporto per l'accettazione del software.
\end{itemize}

\subsubsection{Gestione operativa}
Il processo\textsubscript{g} di gestione operativa contiene le attività e i compiti dell'operatore. Il processo\textsubscript{g} riguarda il funzionamento del prodotto\textsubscript{g} software e il supporto operativo agli utenti. 
Le attività del processo\textsubscript{g} sono le seguenti:
\begin{itemize}
\item Implementazione dei processi;
\item Collaudo operativo;
\item Operazione di sistema;
\item Supporto all'utente.
\end{itemize}

\subsubsection{Manutenzione}
Il processo\textsubscript{g} di manutenzione contiene le attività e i compiti del manutentore. Questo processo\textsubscript{g} viene attivato quando il prodotto\textsubscript{g} software subisce modifiche al codice e alla documentazione associata a causa di un problema o necessità di miglioramento o adattamento.
Le attività del processo\textsubscript{g} sono le seguenti:
\begin{itemize}
\item Implementazione di processi;
\item Analisi dei problemi e delle modifiche;
\item Implementanzione della modifica;
\item Revisione/Accettazione della manuntenzione;
\item Migrazione;
\item Ritiro del software.
\end{itemize}


