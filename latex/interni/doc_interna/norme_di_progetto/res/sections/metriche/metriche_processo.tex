\subsection{Metriche per la qualità di processo\textsubscript{g}}
In questa sezione sono descritte le metriche di qualità di processo\textsubscript{g} che il gruppo intende adottare.

\subsubsection{MPC01: Planned Value (PV)}
Costo (in \euro) pianificato per realizzare le attività di progetto pianificate al momento del calcolo.
Per calcolare questa metrica si utilizzerà la formula:
\begin{equation*}
\text{PV}=\text{B}\textsubscript{tot} * \%\text{ lavoro pianificato}
\end{equation*}
\noindent Per eseguire il calcolo:
\begin{itemize}
\item la \% di lavoro pianificato viene calcolata dividendo le ore pianificate per le ore pianificate totali;
\item B\textsubscript{tot} indica il budget totale preventivato.
\end{itemize}
\textit{Valore minimo:} $\ge 0$ \euro\\
\textit{Valore ottimo:}	$\le \text{Budget at Completion\textsubscript{g}}$
\subsubsection{MPC02: Actual Cost (AC)}
Costo (in \euro) sostenuto al momento del calcolo.
Per calcolare questa metrica si tiene conto di:
\begin{equation*}
\text{AC}=\sum (\text{Totale ore lavorative per incremento}*\text{costo orario})
\end{equation*}
\noindent Per eseguire il calcolo:
\begin{itemize}
\item il costo orario è un costo fisso per ruolo ad ora;
\item Totale ore lavorative per incremento va a contare per incremento le ore effettive di lavoro per ruolo.
\end{itemize}
\textit{Valore minimo:} $\ge 0$ \euro\\
\textit{Valore ottimo:} $\le \text{EAC}$
\subsubsection{MPC03: Estimated at Completion (EAC)}
Un valore (in \euro) di revisione di costo stimato per la realizzazione del progetto alla data corrente.
\begin{equation*}
\text{EAC}=\text{AC}+\text{ETC}
\end{equation*}
\textit{Valore minimo:} 
$ \text{EAC} \le \text{preventivo} -8\% $ \textit{oppure} $ \text{EAC} \ge \text{preventivo} +5\% $\\
\textit{Valore ottimo:} Costo preventivato
\subsubsection{MPC04: Earned Value (EV)}
Indica il valore (in \euro) delle attività realizzate nel progetto fino al momento della misurazione.
\begin{equation*}
\text{EV}=\text{B\textsubscript{tot} * \%\text{ lavoro effettivo}}
\end{equation*}
\noindent Per eseguire il calcolo:
\begin{itemize}
\item la \% lavoro effettivo è calcolata dividendo le ore effettive per le ore totali preventivate.
\item B\textsubscript{tot} indica il budget totale preventivato.
\end{itemize}
\textit{Valore minimo:} $\ge 0$ \euro\\
\textit{Valore ottimo:} $\le \text{EAC}$
\subsubsection{MPC05: Estimated to Complete (ETC)}
Metrica che produce un valore (in \euro) per indicare il numero di attività necessarie al completamento del progetto.\\
\begin{equation*}
\text{ETC}=B\textsubscript{tot}-\text{EV}
\end{equation*}
\noindent Per eseguire il calcolo:
\begin{itemize}
\item B\textsubscript{tot} indica il budget totale preventivato.
\end{itemize}
\textit{Valore minimo:} $\ge 0$ \euro\\
\textit{Valore ottimo:} $\le \text{EAC}$
\subsubsection{MPC06: Cost Variance (CV)}
Valore (in \euro)che indica se il valore del costo attuale del progetto è maggiore, uguale o minore rispetto al costo effettivo. Se CV $\ge 0$ significa che si sta producendo con maggior efficienza risparmiando sul costo rispetto a quanto pianificato, viceversa se negativo.
\begin{equation*}
\text{CV}=\text{EV}-\text{AC}
\end{equation*}
\textit{Valore minimo:} CV = 0 \euro\\
\textit{Valore ottimo:} CV $\ge 0$ \euro
\subsubsection{MPC07: Schedule Variance (SV)}
Variazione percentuale utilizzata per analizzare se il progetto è in linea, in anticipo o in ritardo rispetto alla schedulazione delle attività di progetto pianificate nella baseline.
\begin{equation*}
\text{SV}=\frac{\text{EV}-\text{PV}}{\text{EV}}*100
\end{equation*}
\textit{Valore minimo:} $\ge -15\%$\\
\textit{Valore ottimo:} 0\%
\subsubsection{MPC08: Budget Variance (BV)}
Valore (in \euro) che indica se alla data corrente si è speso di più o di meno del budget iniziale. Un valore di BV $\ge 0$ significa che il progetto sta spendendo il proprio budget con minor velocità di quanto pianificato, viceversa se negativo.
\begin{equation*}
\text{BV}=\text{PV}-\text{AC}
\end{equation*}
\textit{Valore minimo:} BV = 0 \euro\\
\textit{Valore ottimo:} BV $\ge 0 $ \euro

\subsubsection{MPC09: Requirements Stability Index (RSI)}
Indice che traccia la variazione dei requisiti nell'arco del progetto.\\
Si calcola tramite la seguente formula:
\begin{equation*}
\text{RSI}=1-\frac{\text{numero requisiti cambiati}+\text{numero requisiti eliminati}+\text{numero requisiti aggiunti}}{\text{numero totale dei requisiti iniziali}}*100
\end{equation*}
\textit{Valore minimo:} $70\%$\\
\textit{Valore ottimo:} $100\%$

\subsubsection{MPC10: Indice di Gulpease}
Si tratta dell'indice di leggibilità di un testo tarato sulla lingua italiana.
I risultati sono compresi tra 0 e 100, dove il valore "100" indica la leggibilità più alta e "0" la leggibilità più bassa. Ai seguenti valori si associano i seguenti significati:
\begin{itemize}
\item inferiore a 80 sono difficili da leggere per chi ha la licenza elementare
\item inferiore a 60 sono difficili da leggere per chi ha la licenza media
\item inferiore a 40 sono difficili da leggere per chi ha un diploma superiore
\end{itemize}
Viene adottata la seguente formula per calcolarlo:
\begin{equation*}
89+\frac{300*(\text{numero delle frasi})-10*(\text{numero delle lettere})}{\text{numero delle parole}}
\end{equation*}\\
Questi sono i valori da noi ritenuti opportuni:\\
\textit{Valore minimo}: $ \ge 50 $\\ 
\textit{Valore ottimo}: $ \ge 80 $\\

\subsubsection{MPC11: Metriche soddisfatte}
Viene calcolato questo indice tramite una percentuale per tenere conto delle metriche soddisfatte. Una metrica si dice "soddisfatta" se raggiunge almeno il valore minimo imposto sul \textit{\textit{Piano di Qualifica}}.\\
Questi sono i valori da noi ritenuti opportuni:\\\\
\textit{Valore minimo:} $ \ge 80\% $\\
\textit{Valore ottimo:} $ 100\% $\\

\subsubsection{MPC12: Code coverage}
Viene definita come la percentuale di linee di codice del progetto che sono state eseguite dai test dopo un'esecuzione.\\
Questi sono i valori da noi ritenuti opportuni:\\
\textit{Valore minimo}: $ \ge 70\% $\\
\textit{Valore ottimo}: $ \ge 90\% - 100\% $\\

\subsubsection{MPC13: Rischi non previsti}
Il valore è identificato con un numero intero e indica il numero di rischi non previsti durante il corso del progetto.\\
Questi sono i valori da noi ritenuti opportuni:\\
\textit{Valore minimo}: $ \ge 0$\\
\textit{Valore ottimo}: 0\\