\subsection{Metriche per la qualità di processo}
In questa sezione sono descritte le metriche di qualità di processo che il gruppo intende adottare.

\subsubsection{MPC01: Planned Value (PV)}
Costo pianificato per realizzare le attività di progetto pianificate al momento del calcolo.
Per calcolare questa metrica si utilizzerà la formula:
\begin{equation*}
\text{PV}=\text{B}\textsubscript{tot}*\%\text{ lavoro pianificato}
\end{equation*}
\textit{Valore minimo:} $\ge 0$\\
\textit{Valore ottimo:}	$\le \text{Budget at Completion\textsubscript{g}}$
\subsubsection{MPC02: Actual Cost (AC)}
Costo sostenuto al momento del calcolo.
Per calcolare questa metrica si tiene conto di:
\begin{equation*}
\text{AC}=\text{Totale ore lavorative per incremento}*\text{costo orario}
\end{equation*}
\textit{Valore minimo:} $\ge 0$\\
\textit{Valore ottimo:} $\le \text{EAC}$
\subsubsection{MPC03: Estimated at Completion (EAC)}
Un valore di revisione di costo stimato per la realizzazione del progetto alla data corrente.
\begin{equation*}
\text{EAC}=\text{AC}+\text{ETC}
\end{equation*}
\textit{Valore minimo:} 
$\text{preventivo} -5\% \le \text{EAC}$ \textit{oppure} $\text{preventivo} +5\% \ge \text{EAC}$\\
\textit{Valore ottimo:} Costo preventivato
\subsubsection{MPC04: Earned Value (EV)}
Indica il valore delle attività realizzate nel progetto fino al momento della misurazione.
\begin{equation*}
\text{EV}=\%\text{ lavoro pianificato}*\text{EAC}
\end{equation*}
\textit{Valore minimo:} $\ge 0$\\
\textit{Valore ottimo:} $\le \text{EAC}$
\subsubsection{MPC05: Estimated to Complete (ETC)}
Metrica che produce un valore intero per indicare il numero di attività necessarie al completamento del progetto.\\
\textit{Valore minimo:} $\ge 0$\\
\textit{Valore ottimo:} $\le \text{EAC}$
\subsubsection{MPC06: Cost Variance (CV)}
Indica se il valore del costo attuale del progetto è maggiore, uguale o minore rispetto al costo effettivo.
Questa metrica si misura tramite una semplice formula:
\begin{equation*}
\text{CV}=\text{EV}-\text{AC}
\end{equation*}
\textit{Valore minimo:} $\le -15\%$\\
\textit{Valore ottimo:} $\ge 0\%$
\subsubsection{MPC07: Schedule Variance (SV)}
Valore utilizzato per analizzare se il progetto è in linea, in anticipo o in ritardo rispetto alla schedulazione delle attività di progetto pianificate nella baseline.
\begin{equation*}
\text{SV}=\text{EV}-\text{PV}
\end{equation*}
\textit{Valore minimo:} $\le -15\%$\\
\textit{Valore ottimo:} $\ge 0\%$
\subsubsection{MPC08: Budget Variance (BV)}
Valore che indica se alla data corrente si è speso di più o di meno del budget iniziale.
\begin{equation*}
\text{BV}=\text{EV}-\text{AC}
\end{equation*}
\textit{Valore minimo:} $\le -10\%$\\
\textit{Valore ottimo:} $\ge 0\%$

\subsubsection{MPC09: Defect Density}
Valore percentuale che tiene conto dei difetti per ogni modulo.
Si calcola tramite la seguente formula:\\
\begin{equation*}
\text{Defect Density}=\frac{\text{numero totale dei difetti}}{\text{grandezza del modulo in LOC}}*100
\end{equation*}
dove:
\begin{itemize}
\item \textit{LOC} = acronimo di Line of Code, linee di codice sorgente in un modulo.
\end{itemize}
\textit{Valore minimo:} $\le 5\%$\\
\textit{Valore ottimo:} $\le 2,5\%$

\subsubsection{MPC10: Indice di Gulpease}
Si tratta dell'indice di leggibilità di un testo tarato sulla lingua italiana.
I risultati sono compresi tra 0 e 100, dove il valore "100" indica la leggibilità più alta e "0" la leggibilità più bassa. Ai seguenti valori si associano i seguenti significati:
\begin{itemize}
\item inferiore a 80 sono difficili da leggere per chi ha la licenza elementare
\item inferiore a 60 sono difficili da leggere per chi ha la licenza media
\item inferiore a 40 sono difficili da leggere per chi ha un diploma superiore
\end{itemize}
Viene adottata la seguente formula per calcolarlo:
\begin{equation*}
89+\frac{300*(\text{numero delle frasi})-10*(\text{numero delle lettere})}{\text{numero delle parole}}
\end{equation*}\\
Questi sono i valori da noi ritenuti opportuni:\\
\textit{Valore minimo}: $ \ge 50 $\\ 
\textit{Valore ottimo}: $ \ge 80 $\\

\subsubsection{MPC11: Metriche soddisfatte}
Viene calcolato questo indice tramite una percentuale per tenere conto delle metriche soddisfatte.\\
Questi sono i valori da noi ritenuti opportuni:\\\\
\textit{Valore minimo:} $ \ge 90\% $\\
\textit{Valore ottimo:} $ 100\% $\\

\subsubsection{MPC12: Code coverage}
Viene definita come la percentuale di linee di codice del progetto che sono state eseguite dai test dopo un'esecuzione.\\
Questi sono i valori da noi ritenuti opportuni:\\
\textit{Valore minimo}: $ \ge 70\% $\\
\textit{Valore ottimo}: $ \ge 90\% - 100\% $\\

\subsubsection{MPC13: Rischi non previsti}
Il valore è identificato con un numero intero e indica il numero di rischi non previsti durante il corso del progetto.\\
Questi sono i valori da noi ritenuti opportuni:\\
\textit{Valore minimo}: $ \ge 0$\\
\textit{Valore ottimo}: 0\\