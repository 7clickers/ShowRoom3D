\subsection{Modello di qualità}
Il modello definisce le caratteristiche di qualità. A ciascuna di esse sono associate sotto-caratteristiche. La tabella successiva descrive le caratteristiche e le sue sotto-caratteristiche del software proposti dal modello. 
Il modello presenta le seguenti caratteristiche:

\subsubsection{Funzionalità}
Si intende quella capacità di un prodotto software di determinare le esigenze richieste tramite delle funzioni, sotto precise condizioni.
Le sue sotto-caratteristiche sono rispettivamente:
\begin{itemize}
\item Appropriatezza: rappresenta la capacità di fornire un appropriato insieme di funzioni che permettano agli utenti di svolgere determinate task e di raggiungere gli obiettivi prefissati;
\item Accuratezza: rappresenta la capacità del software di fornire i risultati o gli effetti attesi con il livello di precisione richiesta;
\item Interoperabilità: rappresenta la capacità del software di interagire con uno o più sistemi specificati;
\item Conformità: rappresenta la capacità del software di aderire a standard, convenzioni e regolamenti di carattere legale o prescrizioni simili che abbiano attinenza con la funzionalità;
\item Sicurezza: rappresenta la capacità del software di proteggere le informazioni ed i dati in modo che, persone o sistemi non autorizzati, non possano accedervi e quindi non possano leggerli o modificarli.
\end{itemize}

\subsubsection{Affidabilità}
Si intende quella capacità di un prodotto software di mantenere il livello di prestazione quando utilizzato sotto certe condizioni specifiche.
Le sue sotto-caratteristiche sono rispettivamente:
\begin{itemize}
\item Maturità: rappresenta la capacità del software di evitare che si verifichino errori o siano prodotti risultati non corretti in fase di esecuzione;
\item Tolleranza agli errori: rappresenta la capacità del software  di mantenere il livello di prestazioni in caso di errori nel software o di violazione delle interfacce specificate;
\item Recuperabilità: rappresenta la capacità del software di ripristinare il livello di prestazioni e di recuperare i dati direttamente coinvolti in caso di errori o malfunzionamenti;
\item Aderenza: rappresenta la capacità del software di aderire a standard, convenzioni e regole relative all'affidabilità.
\end{itemize}

\subsubsection{Efficienza}
Si intende quella capacità di un prodotto software di realizzare le funzioni richieste nel minor tempo possibile ed utilizzando nel miglior modo le risorse necessarie.
Le sue sotto-caratteristiche sono rispettivamente:
\begin{itemize}
\item Comportamento rispetto al tempo: rappresenta la capacità del software di fornire appropriati tempi di risposta, tempi di elaborazione e quantità di lavoro eseguendo le funzionalità previste sotto determinate condizioni di utilizzo;
\item Utilizzo delle risorse: rappresenta la capacità del software di utilizzare un appropriato numero e tipo di risorse in maniera adeguata;
\item Conformità: rappresenta la capacità del software di aderire a standard e convenzioni relative all'efficienza.
\end{itemize}

\subsubsection{Usabilità}
Si intende quella capacità di un prodotto software di essere comprensibile e appreso dall'utente, quando usato sotto condizioni specificate.
Le sue sotto-caratteristiche sono rispettivamente:
\begin{itemize}
\item Comprensibilità: rappresenta la capacità del software di permettere all'utente di capire le sue funzionalità e come poterla utilizzare con successo per svolgere particolari task in determinate condizioni di utilizzo;
\item Apprendibilità: rappresenta la capacità del software di permettere all'utente di imparare l'applicazione;
\item Operabilità: rappresenta la capacità del software di permettere all'utente di utilizzarlo e di controllarlo;
\item Attrattività: rappresenta la capacità del software di risultare "attraente" per l'utente;
\item Conformità: rappresenta la capacità del software di aderire a standard,convenzioni o regole relative all'usabilità.
\end{itemize}

\subsubsection{Manutenibilità}
Si intende quella capacità di un prodotto software di essere modificato, includendo correzioni, miglioramenti o adattamenti.
Le sue sotto-caratteristiche sono rispettivamente:
\begin{itemize}
\item Analizzabilità: rappresenta la capacità del software di poter effettuare la diagnosi sul software ed individuare le cause di errori o malfunzionamenti;
\item Modificabilità: rappresenta la capacità del software di consentire lo sviluppo di modifiche al software originale. Si tratta quindi di modifiche al codice o alla progettazione ed alla sua documentazione;
\item Stabilità: rappresenta la capacità del software di evitare effetti non desiderati a seguito di modifiche al software;
\item Testabilità: rappresenta la capacità del software di consentire la verifica e validazione del software modificato, cioè di eseguire i test.
\end{itemize}

\subsubsection{Portabilità}
Si intende quella capacità di un prodotto software di poter essere trasportato da un ambiente di lavoro ad un altro. 
Le sue sotto-caratteristiche sono rispettivamente:
\begin{itemize}
\item Adattabilità: rappresenta la capacità del software di essere adattato a differenti ambienti senza richiedere azioni specifiche diverse da quelle previste dal software per tali attività;
\item Installabilità: rappresenta la capacità del software di essere installato in un determinato ambiente;
\item Conformità: rappresenta la capacità del software di aderire a standard,convenzioni o regole relative alla portabilità;
\item Sostituibilità: rappresenta la capacità del software di sostituire un altro software specifico indipendente, per lo stesso scopo e nello stesso ambiente.
\end{itemize}