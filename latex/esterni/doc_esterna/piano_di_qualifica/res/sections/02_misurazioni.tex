\section{Resoconto delle attività di verifica\textsubscript{g}}
\subsection{Verifica della qualità dei processi\textsubscript{g}}
In questa sezione vengono riportati i risultati dell'attività di verifica\textsubscript{g} effettuata relativa alla qualità del processo\textsubscript{g}.\\
Per calcolare le seguenti misure abbiamo utilizzato le formule e le nozioni descritte nel documento di \textit{Norme di Progetto} e i dati redatti nel documento di \textit{Piano di Progetto}.\\
La metrica di Code Coverage non è stata testata nello sviluppo del PoC\textsubscript{g} in quanto verrà eseguita successivamente in seguito ai Test nel codice del progetto.\\
\begin{longtable}{ 
		>{\centering}M{0.45\textwidth} 
		>{\centering}M{0.17\textwidth}
		>{\centering}M{0.25\textwidth} 
		}
	\rowcolorhead
	\headertitle{Metrica} &
	\centering \headertitle{Valore} &	
	\headertitle{Esito} 
	\endfirsthead	
	\endhead
	
	Planned Value & 8784,29 \euro & Superato\tabularnewline
	Actual Cost & 7460 \euro & Superato\tabularnewline
	Estimated at Completion & 12944,93 \euro & Superato\tabularnewline
	Earned Value & 8490,07 \euro & Superato\tabularnewline
	Estimated to Complete & 5484,93 \euro & Superato\tabularnewline
	Cost Variance& 1030,07 \euro & Superato\tabularnewline
	Schedule Variance & -3,46\% & Superato\tabularnewline
	Budget Variance & 1324,29 \euro & Superato\tabularnewline
	Requirement Stability Index& 100\% & Superato\tabularnewline
	Code coverage & - & Non testato\tabularnewline
	Rischi non previsti & 0 & Superato\tabularnewline
	Metriche soddisfatte & 90\% & Superato\tabularnewline
\end{longtable}
\noindent Qui vengono riportati i grafici per varie metriche di processo\textsubscript{g} significative nei tre periodi identificati.
Nei primi tre grafici viene rappresentato di quanto differisce ciascun valore da quello precedente per periodo.
\paragraph{Planned Value}
\begin{center}
\begin{figure}[H]
  \centering
  \renewcommand{\thefigure}{1}
  \includegraphics[width=10cm]{./res/images/PVGraph.png}
  \caption{Planned Value}
  \label{fig:Grafico Planned Value}
\end{figure}
\end{center}
\pagebreak
\paragraph{Actual Cost}
\begin{center}
\begin{figure}[H]
  \centering
  \renewcommand{\thefigure}{2}
  \includegraphics[width=10cm]{./res/images/ACGraph.png}
  \caption{Actual Cost}
  \label{fig:Grafico Actual Cost}
\end{figure}
\end{center}
\paragraph{Earned Value}
\begin{center}
\begin{figure}[H]
  \centering
  \renewcommand{\thefigure}{3}
  \includegraphics[width=10cm]{./res/images/EVGraph.png}
  \caption{Earned Value}
  \label{fig:Grafico Earned Value}
\end{figure}
\end{center}
\pagebreak
\paragraph{Cost Variance}
\begin{center}
\begin{figure}[H]
  \centering
  \renewcommand{\thefigure}{4}
  \includegraphics[width=10cm]{./res/images/CVGraph.png}
  \caption{Cost Variance}
  \label{fig:Grafico Cost Variance}
\end{figure}
\end{center}

\paragraph{Schedule Variance}
\begin{center}
\begin{figure}[H]
  \centering
  \renewcommand{\thefigure}{5}
  \includegraphics[width=10cm]{./res/images/SVGraph.png}
  \caption{Schedule Variance}
  \label{fig:Grafico Schedule Variance}
\end{figure}
\end{center}
\pagebreak
\paragraph{Budget Variance}
\begin{center}
\begin{figure}[H]
  \centering
  \renewcommand{\thefigure}{6}
  \includegraphics[width=10cm]{./res/images/BVGraph.png}
  \caption{Budget Variance}
  \label{fig:Grafico Budget Variance}
\end{figure}
\end{center}

\subsubsection{Indice di Gulpease}
\noindent Nella seguente tabella vengono riportati gli indici di Gulpease calcolati sulle ultime versioni dei seguenti documenti.\\
Per calcolare i seguenti valori non sono stati considerati: i changelog, la pagina di introduzione del documento, l'indice, tabelle con valori, intestazioni a piè di pagina, captions e la sezione di "Informazioni generali" nei verbali. Sono state incluse invece le colonne di tabelle contenenti descrizioni significative e gli elenchi puntati che contenevano frasi significative. 
\begin{longtable}{ 
		>{\centering}M{0.45\textwidth} 
		>{\centering}M{0.17\textwidth}
		>{\centering}M{0.25\textwidth} 
		}
	\rowcolorhead
	\headertitle{Documento} &
	\centering \headertitle{Valore} &	
	\headertitle{Esito} 
	\endfirsthead	
	\endhead
	
	\textit{Analisi dei Requisiti} & 80 & Superato\tabularnewline
	\textit{Glossario} & 72 & Superato\tabularnewline
	\textit{Norme di Progetto} & 74 & Superato\tabularnewline
	\textit{Piano di Progetto} & 63 & Superato\tabularnewline
	\textit{Piano di Qualifica} & 76 & Superato\tabularnewline
	\textit{Studio di Fattibilità} & 80 & Superato\tabularnewline
	VE 22-10-25 & 82 & Superato\tabularnewline
	VE 22-10-26 & 84 & Superato\tabularnewline
	VE 22-11-17	& 75 & Superato\tabularnewline
	VE 23-01-11	& 57 & Superato\tabularnewline
	VE 23-01-18	& 69 & Superato\tabularnewline
	VE 23-02-17 & 70 & Superato\tabularnewline
	VE 23-04-05 & 60 & Superato\tabularnewline
	VE 23-04-14 & 63 & Superato\tabularnewline
	VE 23-05-12 & 64 & Superato\tabularnewline
	VI 22-10-25 & 74 & Superato\tabularnewline
	VI 22-10-26 & 79 & Superato\tabularnewline
	VI 22-11-04 & 63 & Superato\tabularnewline
	VI 22-11-09 & 88 & Superato\tabularnewline
	VI 22-11-16 & 63 & Superato\tabularnewline
	VI 22-11-23 & 72 & Superato\tabularnewline
	VI 22-12-01 & 60 & Superato\tabularnewline
	VI 22-12-07 & 78 & Superato\tabularnewline
	VI 22-12-14 & 69 & Superato\tabularnewline
	VI 23-01-04 & 63 & Superato\tabularnewline
	VI 23-01-25 & 68 & Superato\tabularnewline
	VI 23-02-01 & 65 & Superato\tabularnewline
	VI 23-02-08 & 58 & Superato\tabularnewline
	VI 23-02-24 & 68 & Superato\tabularnewline
	VI 23-02-28 & 65 & Superato\tabularnewline
	VI 23-03-16 & 67 & Superato\tabularnewline
	VI 23-04-23 & 70 & Superato\tabularnewline
	VI 23-05-05 & 67 & Superato\tabularnewline
	
\end{longtable}

\begin{center}
\begin{figure}[H]
  \centering
  \renewcommand{\thefigure}{7}
  \includegraphics[width=10cm]{./res/images/GulpeaseGen.png}
  \caption{Indice di Gulpease dei documenti}
  \label{fig:Indice di Gulpease dei documenti}
\end{figure}
\end{center}
\pagebreak
\noindent Qui ora vengono riportati i grafici che analizzano l'andamento dell'indice di Gulpease di documenti in continua evoluzione.
\paragraph{Indice di Gulpease - \textit{Analisi dei Requisiti}}
\begin{center}
\begin{figure}[H]
  \centering
  \renewcommand{\thefigure}{8}
  \includegraphics[width=10cm]{./res/images/AdRGraph.png}
  \caption{Indice di Gulpease - \textit{Analisi dei Requisiti}}
  \label{fig:Indice di Gulpease - Analisi dei Requisiti}
\end{figure}
\end{center}
\pagebreak
\paragraph{Indice di Gulpease - \textit{Norme di Progetto}}
\begin{center}
\begin{figure}[H]
  \centering
  \renewcommand{\thefigure}{9}
  \includegraphics[width=10cm]{./res/images/NdPGraph.png}
  \caption{Indice di Gulpease - \textit{Norme di Progetto}}
  \label{fig:Indice di Gulpease - Norme di Progetto}
\end{figure}
\end{center}
\pagebreak
\paragraph{Indice di Gulpease - \textit{Piano di Progetto}}
\begin{center}
\begin{figure}[H]
  \centering
  \renewcommand{\thefigure}{10}
  \includegraphics[width=10cm]{./res/images/PdPGraph.png}
  \caption{Indice di Gulpease - \textit{Piano di Progetto}}
  \label{fig:Indice di Gulpease - Piano di Progetto}
\end{figure}
\end{center}
\pagebreak
\paragraph{Indice di Gulpease - \textit{Piano di Qualifica}}
\begin{center}
\begin{figure}[H]
  \centering
  \renewcommand{\thefigure}{11}
  \includegraphics[width=10cm]{./res/images/PdQGraph.png}
  \caption{Indice di Gulpease - \textit{Piano di Qualifica}}
  \label{fig:Indice di Gulpease - Piano di Qualifica}
\end{figure}
\end{center}

\subsection{Verifica della qualità dei prodotti\textsubscript{g}}
In questa sezione vengono riportati i risultati dell'attività di verifica\textsubscript{g} effettuata relativa alla qualità del prodotto\textsubscript{g}, nel contesto del PoC\textsubscript{g}.\\
La metrica di Percentuale di requisiti soddisfatti non è stata testata in quanto non siamo ancora nella fase di Codifica.

\begin{longtable}{ 
		>{\centering}M{0.45\textwidth} 
		>{\centering}M{0.17\textwidth}
		>{\centering}M{0.25\textwidth} 
		}
	\rowcolorhead
	\headertitle{Metrica} &
	\centering \headertitle{Valore} &	
	\headertitle{Esito} 
	\endfirsthead	
	\endhead
	Percentuale di requisiti soddisfatti& - & Non testato\tabularnewline
	Densità di fallimenti durante l'esecuzione& 0\% & Superato\tabularnewline
	Tempo medio di risposta & minore di 1s & Superato\tabularnewline
	Tempo di caricamento& 7 secondi & Superato\tabularnewline
	Facilità di apprendimento& 90 secondi & Superato\tabularnewline
	Densità dei commenti & 15,93\% & Superato\tabularnewline
	Browser supportati & 100\% & Superato\tabularnewline
\end{longtable}
\noindent Nella seguente tabella è stata calcolata la Complessità Ciclomatica del PoC\textsubscript{g}.
\begin{longtable}{ 
		>{\centering}M{0.40\textwidth} 
		>{\centering}M{0.15\textwidth}
		>{\centering}M{0.20\textwidth}
		}
	\rowcolorhead
	\headertitle{Modulo} &
	\headertitle{Valore} &
	\headertitle{Esito} 	
	\endfirsthead	
	\endhead
	Player.js & 2 & Superato\tabularnewline
	Light.js & 1 & Superato\tabularnewline
	Product.js & 1 & Superato\tabularnewline
	Raycasting.js & 6 & Superato\tabularnewline
	Renderer.js & 2 & Superato\tabularnewline
	Showroom\_poc.js & 1 & Superato\tabularnewline
	Source\_loader.js & 2 & Superato\tabularnewline
	UI\_listeners.js & 1 & Superato\tabularnewline
	
\end{longtable}

Il PoC\textsubscript{g} è stato testato sui seguenti browser:
\begin{longtable}{ 
		>{\centering}M{0.45\textwidth} 
		>{\centering}M{0.25\textwidth} 
		}
	\rowcolorhead
	\headertitle{Browser} &
	\headertitle{Esito} 
	\endfirsthead	
	\endhead
	
	Google Chrome (versione $ \ge 110 $) & Supportato\tabularnewline
	Microsoft Edge (versione $ \ge 110 $) & Supportato\tabularnewline
	Mozilla Firefox (versione $ \ge 109 $) & Supportato\tabularnewline
	Safari (versione $ \ge 16 $) & Supportato\tabularnewline
	Opera (versione $ \ge 95 $) & Supportato\tabularnewline

\end{longtable}

\section{Specifica dei Test}
\begin{itemize}
\item Test di unità: vengono stabiliti durante la progettazione e servono per verificare le singole unità software;
\item Test di integrazione: vengono stabiliti durante la progettazione e servono per integrare il funzionamento di più unità;
\item Test di accettazione: vengono effettuati insieme al proponente\textsubscript{g} durante la fase di collaudo;
\item Test di sistema: vengono stabiliti durante l'\textit{Analisi dei Requisiti} e servono per accertare la copertura dei requisiti software definiti nel documento di \textit{Analisi dei Requisiti}.
\end{itemize}
Gli acronimi utilizzati in questo documento per identificare i test sono specificati dettagliatamente nel documento di \textit{Norme di Progetto}.
In questa sezione vengono utilizzate le seguenti sigle per lo stato di ogni test:
\begin{itemize}
\item \textbf{S}: test superato
\item \textbf{N}: test non implementato
\end{itemize}

\subsection{Test di unità}
Per garantire il funzionamento dei singoli componenti che possono lavorare in maniera autonoma senza bisogno di dipendenze abbiamo effettuato questi test. Non abbiamo ritenuto necessario testare componenti o parti di componenti che:
\begin{itemize}
	\item Utilizzavano funzionalità già assodate e documentate;
	\item Componenti che raggruppavano altri componenti.
\end{itemize} 
\begin{longtable}{
		>{\centering}M{0.20\textwidth}
		>{\centering}M{0.35\textwidth}	 
		>{\centering}M{0.20\textwidth} 
		}
	\rowcolorhead
	\headertitle{Test} &
	\centering \headertitle{Descrizione} &	
	\headertitle{Stato} 
	\endfirsthead	
	\endhead
TU1 & Si verifichi che la posizione del player cambi quando utilizza i tasti di movimento & S\tabularnewline
TU2 & Si verifichi che la rotazione del player cambi quando muove la visuale col mouse & S\tabularnewline
TU3 & Si verifichi che avvenga il cambiamento colore ad un oggetto interagibile & S\tabularnewline
TU4 & Si verifichi la funzionalità dell’aggiunta all’array che rappresenta gli oggetti nel carrello e l’incremento del costo totale & S\tabularnewline
TU5 & Si verifichi che la funzionalità di aggiungere un oggetto al carrello funzioni solo con oggetti validi & S\tabularnewline
TU6 & Si verifichi la possibilità di utilizzare i selettori Redux per reperire l’array logico del carrello e il costo totale & S\tabularnewline
TU7 & Si verifichi la funzionalità della rimozione totale dall’array del carrello & S\tabularnewline
TU8 & Si verifichi che la funzionalità di rimozione totale dell’array carrello funzioni anche se il carrello è vuoto & S\tabularnewline
TU9 & Si verifichi la funzionalità di eliminazione singola dal carrello & S\tabularnewline
TU10 & Si verifichi che l’eliminazione singola funziona solo con oggetti realmente presenti nell’array carrello & S\tabularnewline
TU11 & Si verifichi che il raycaster funzioni & S\tabularnewline
TU12 & Si verifichi che funzioni la possibilità di aggiungere quantità di un oggetto selezionato & S\tabularnewline
TU13 & Si verifichi che funzioni la possibilità di diminuire quantità di un oggetto selezionato & S\tabularnewline
\end{longtable}

\subsection{Test di integrazione}
Per assicurare che ogni componente lavori correttamente con gli altri componenti, il gruppo ha deciso di effettuare i seguenti Test di Integrazione.
\begin{longtable}{
		>{\centering}M{0.20\textwidth}
		>{\centering}M{0.35\textwidth}	 
		>{\centering}M{0.20\textwidth} 
		}
	\rowcolorhead
	\headertitle{Test} &
	\centering \headertitle{Descrizione} &	
	\headertitle{Stato} 
	\endfirsthead	
	\endhead
TI1 & Si verifichi che la modifica di cambiamento colore venga apportata ad un oggetto esempio nel file JSON contenente i prodotti interagibili & S\tabularnewline
TI2 & Si verifichi che un elemento nel carrello , se eliminato singolarmente, scompaia dal componente Cart e anche dall’array logico del carrello & S\tabularnewline
TI3 & Si verifichi che il bottone “Remove all” funzioni rimuovendo sia da Cart che da cartSlice tutti gli oggetti & S\tabularnewline
TI4 & Si verifichi che il bottone “Remove all” funzioni anche se il carrello è vuoto & S\tabularnewline
TI5 & Si verifichi che il cambio di colore nella Sidebar venga salvato & S\tabularnewline
TI6 & Si verifichi che il colore ritorni a standard se viene deselezionata una modifica di colore & S\tabularnewline
TI7 & Si verifichi che venga aggiunto un oggetto al carrello dalla sidebar & S\tabularnewline
TI8 & Si verifichi che venga aggiunta la quantità selezionata di un oggetto già presente nel carrello & S\tabularnewline
\end{longtable}
\subsection{Test di accettazione}
Questi test verranno stabiliti durante la fase di Collaudo.

\subsection{Test di sistema}
Per assicurare che vengano rispettati i requisiti concordati nel documento di \textit{Analisi dei Requisiti}, vengono eseguiti i seguenti test di sistema.
\begin{longtable}{
		>{\centering}M{0.20\textwidth}
		>{\centering}M{0.35\textwidth}	 
		>{\centering}M{0.20\textwidth} 
		}
	\rowcolorhead
	\headertitle{Test} &
	\centering \headertitle{Descrizione} &	
	\headertitle{Stato} 
	\endfirsthead	
	\endhead
TSRF1 & Si verifica\textsubscript{g} che l'utente possa aggiungere, l'oggetto con cui sta interagendo, nel carrello & S \tabularnewline
TSRF2 & Si verifica\textsubscript{g} che l'utente possa visualizzare il contenuto del carrello & S \tabularnewline
TSRF2.1 & Si verifica\textsubscript{g} che l'utente possa visualizzare la lista degli oggetti presenti nel carrello & S \tabularnewline
TSRF2.1.1 & Si verifica\textsubscript{g} che l'utente possa interagire con un oggetto nel carrello & S \tabularnewline
TSRF2.1.1.1 & Si verifica\textsubscript{g} che l'utente possa visualizzare la caratteristica del nome di ogni oggetto presente nella lista degli oggetti presenti nel carrello & S \tabularnewline
TSRF2.1.1.2 & Si verifica\textsubscript{g} che l'utente possa visualizzare la caratteristica del costo di ogni oggetto presente nella lista degli oggetti presenti nel carrello & S \tabularnewline
TSRF2.1.1.3 & Si verifica\textsubscript{g} che l'utente possa visualizzare la caratteristica della quantità di ogni oggetto presente nella lista degli oggetti presenti nel carrello & S \tabularnewline
TSRF2.2 & Si verifica\textsubscript{g} che l'utente possa visualizzare il costo totale degli oggetti che ha inserito nel carrello & S \tabularnewline
TSRF3 & Si verifica\textsubscript{g} che l'utente abbia la possibilità di rimuovere tutti gli oggetti dal carrello & S \tabularnewline
TSRF4 & Si verifica\textsubscript{g} che l'utente abbia la possibilità di rimuovere un singolo oggetto dal carrello & S \tabularnewline
TSRF5 & Si verifica\textsubscript{g} che l'utente possa muoversi in maniera direzionale & S \tabularnewline
TSRF5.1 & Si verifica\textsubscript{g} che l'utente possa compiere movimenti direzionali nell'asse X & S \tabularnewline
TSRF5.2 & Si verifica\textsubscript{g} che l'utente possa compiere movimenti direzionali nell'asse Y & S \tabularnewline
TSRF5.3 & Si verifica\textsubscript{g} che l'utente possa compiere movimenti direzionali nell'asse Z & S \tabularnewline
TSRF6 & Si verifica\textsubscript{g} che l'utente possa compiere spostamenti di camera & S \tabularnewline
TSRF6.1 & Si verifica\textsubscript{g} che l'utente possa compiere spostamenti di camera nell'asse X & S \tabularnewline
TSRF6.2 & Si verifica\textsubscript{g} che l'utente possa compiere spostamenti di camera nell'asse Y & S \tabularnewline
TSRF7 & Si verifica\textsubscript{g} che l'utente possa modificare la combinazione dei colori di un oggetto & S \tabularnewline
TSRF8 & Si verifica\textsubscript{g} che l'utente venga notificato in caso non fosse possibile modificare un oggetto & S \tabularnewline
TSRF9 & Si verifica\textsubscript{g} che l'utente possa visualizzare la lista degli oggetti della stanza in cui si trova & N \tabularnewline
TSRF9.1 & Si verifica\textsubscript{g} che l'utente possa visualizzare un singolo oggetto nella lista degli oggetti della stanza in cui si trova & N \tabularnewline
TSRF9.1.1 & Si verifica\textsubscript{g} che l'utente possa visualizzare la caratteristica del nome di ogni oggetto della lista degli oggetti della stanza in cui si trova & N \tabularnewline
TSRF10 & Si verifica\textsubscript{g} che l'utente possa visualizzare tutti i dettagli di un oggetto selezionato & S \tabularnewline
TSRF11 & Si verifica\textsubscript{g} che l'utente abbia la possibilità di riposizionarsi vicino ad un oggetto nella stanza in cui si trova & N \tabularnewline
TSRF12 & Si verifica\textsubscript{g} che l'utente possa riposizionarsi in una stanza da lui selezionata & N \tabularnewline
TSRF13 & Si verifica\textsubscript{g} che l'utente venga notificato in caso il riposizionamento in una stanza non sia possibile & N \tabularnewline
TSRF14 & Si verifica\textsubscript{g} che l'utente venga notificato in caso il riposizionamento in prossimità di un oggetto selezionato non sia concesso & N \tabularnewline
TSRF15 & Si verifica\textsubscript{g} che l'utente possa visualizzare la lista delle stanze & N \tabularnewline
TSRF15.1 & Si verifica\textsubscript{g} che l'utente possa visualizzare una singola stanza dalla lista delle stanze & N \tabularnewline
TSRF15.1.1 & Si verifica\textsubscript{g} che l'utente possa visualizzare la caratteristica del nome di ogni stanza dalla lista delle stanze & N \tabularnewline
TSRF15.1.2 & Si verifica\textsubscript{g} che l'utente possa visualizzare la caratteristica della tipologia di oggetti presenti in ogni stanza nella lista delle stanze & N \tabularnewline
TSRF16 & Si verifica\textsubscript{g} che l'utente possa riposizionare un oggetto presente nella stanza in cui si trova & N \tabularnewline
TSRF17 & Si verifica\textsubscript{g} che l'utente non possa riposizionare un oggetto in una coordinata non legittima & N \tabularnewline
TSRF18 & Si verifica\textsubscript{g} che l'utente sia in grado ad illuminare l'ambiente davanti a lui & S \tabularnewline
TSRF19 & Si verifica\textsubscript{g} che l'utente venga notificato se il contenuto del carrello è vuoto & S \tabularnewline
TSRF20 & Si verifica\textsubscript{g} che l'utente possa visualizzare un oggetto illuminato & S \tabularnewline
\end{longtable}

\subsection{Tracciamento dei test}
\subsubsection{Test di Sistema - Requisiti}
\begin{longtable}{
		>{\centering}M{0.25\textwidth}
		>{\centering}M{0.25\textwidth}	 
		}
	\rowcolorhead
	\headertitle{Test di sistema} &
	\headertitle{Requisiti}
	\endfirsthead	
	\endhead
TSRF1 & RF1\tabularnewline
TSRF2 & RF2\tabularnewline
TSRF2.1 & RF2.1\tabularnewline
TSRF2.1.1 & RF2.1.1\tabularnewline
TSRF2.1.1.1 & RF2.1.1.1\tabularnewline
TSRF2.1.1.2 & RF2.1.1.2\tabularnewline
TSRF2.1.1.3 & RF2.1.1.3\tabularnewline
TSRF2.2 & RF2.2\tabularnewline
TSRF3 & RF3\tabularnewline
TSRF4 & RF4\tabularnewline
TSRF5 & RF5\tabularnewline
TSRF5.1 & RF5.1\tabularnewline
TSRF5.2 & RF5.2\tabularnewline
TSRF5.3 & RF5.3\tabularnewline
TSRF6 & RF6\tabularnewline
TSRF7 & RF7\tabularnewline
TSRF8 & RF8\tabularnewline
TSRF9 & RF9\tabularnewline
TSRF9.1 & RF9.1\tabularnewline
TSRF9.1.1 & RF9.1.1\tabularnewline
TSRF10 & RF10\tabularnewline
TSRF11 & RF11\tabularnewline
TSRF12 & RF12\tabularnewline
TSRF13 & RF13\tabularnewline
TSRF14 & RF14\tabularnewline
TSRF15 & RF15\tabularnewline
TSRF15.1 & RF5.1\tabularnewline
TSRF15.1.1 & RF15.1.1\tabularnewline
TSRF15.1.2 & RF15.1.2\tabularnewline
TSRF16 & RF16\tabularnewline
TSRF17 & RF17\tabularnewline
TSRF18 & RF18\tabularnewline
TSRF19 & RF19\tabularnewline
TSRF20 & RF20\tabularnewline

\end{longtable}

\subsubsection{Test di Unità - File di Test}
\begin{longtable}{
		>{\centering}M{0.10\textwidth}
		>{\centering}M{0.30\textwidth}
		>{\centering}M{0.45\textwidth}	 
		}
	\rowcolorhead
	\headertitle{Test} &
	\centering \headertitle{File di test} &	
	\centering \headertitle{Nome test} 
	\endfirsthead	
	\endhead
TU1 & \_\_test\_\_/playerSlice.test.js & it(’should set the player position’)\{ … \}\tabularnewline
TU2 & \_\_test\_\_/playerSlice.test.js & it(’should set the player rotation’)\{ … \}\tabularnewline
TU3 & \_\_test\_\_/productSlice.test.js & it('Should change color to the product in example')\{ … \}\tabularnewline
TU4 & \_\_test\_\_/cartSlice.test.js & it('Should add an item to the array (that represents the logic of the cart) of the slice and increment the totalCost’)\{ … \}\tabularnewline
TU5 & \_\_test\_\_/cartSlice.test.js & it('Should not be able to add this item’) \{ … \}\tabularnewline
TU6 & \_\_test\_\_/cartSlice.test.js & it('Should be able to use the cartItems selector and the totalCost selector’)\{ … \}\tabularnewline
TU7 & \_\_test\_\_/cartSlice.test.js & it('Should remove all the item in the cart')\{ … \}\tabularnewline
TU8 & \_\_test\_\_/cartSlice.test.js & it('Should remove all the items even if the cart is empty’)\{ … \}\tabularnewline
TU9 & \_\_test\_\_/cartSlice.test.js & it('Should remove the item in example’)\{ … \}\tabularnewline
TU10 & \_\_test\_\_/cartSlice.test.js & it('Should be able to work even if the item provided do not exist in the cart’)\{ … \}\tabularnewline
TU11 & \_\_test\_\_/useRaycasterLogic.test.js & it('should update intersects when mouse moves’)\{ … \}\tabularnewline
TU12 & \_\_test\_\_/productSidebar.test.js & it('Should add more then one quantity of the item selected’)\{ … \}\tabularnewline
TU13 & \_\_test\_\_/productSidebar.test.js & it('Should decrease of one the quantity of the item’)\{ … \}\tabularnewline
\end{longtable}

\subsubsection{Test di Integrità - File di Test}
\begin{longtable}{
		>{\centering}M{0.10\textwidth}
		>{\centering}M{0.30\textwidth}
		>{\centering}M{0.45\textwidth}	 
		}
	\rowcolorhead
	\headertitle{Test} &
	\centering \headertitle{File di test} &	
	\centering \headertitle{Nome test} 
	\endfirsthead	
	\endhead
TI1 & \_\_test\_\_/productSlice.test.js & it('Should change color to the product with the ID=1 imported from the JSON file’)\{ … \} \tabularnewline
TI2 & \_\_test\_\_/cart.test.js & it('Should stop rendering item in cart when clicking "X" button’)\{ … \}\tabularnewline
TI3 & \_\_test\_\_/cart.test.js & it('Should stop rendering all items in cart when clicking "Remove all" button’)\{ … \}\tabularnewline
TI4 & \_\_test\_\_/cart.test.js & it('Should try to remove the items and work even if cart is empty’)\{ … \}\tabularnewline
TI5	& \_\_test\_\_/productSidebar.test.js & it('Should change the variable selectedColor to the product in example’)\{ … \}\tabularnewline
TI6 & \_\_test\_\_/productSidebar.test.js & it('Should remove the selected color if the button with the current color is pressed, putting the standard color to the item’)\{ … \}\tabularnewline
TI7 & \_\_test\_\_/productSidebar.test.js & it('Should add an item to the cart’)\{ … \}\tabularnewline
TI8 & \_\_test\_\_/productSidebar.test.js & it('Should add more than one quantity of the item to the cart, testing also if it adds the quantity to the item that is already in the cart’)\{ … \}\tabularnewline
\end{longtable}