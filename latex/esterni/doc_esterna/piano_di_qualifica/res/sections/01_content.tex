\section{Introduzione}
\subsection{Scopo del documento}
Questo documento è stato creato dal gruppo Seven Clickers per descrivere degli standard fissati e dei metodi utilizzati al fine di garantire la qualità dei prodotti e dei processi.
In questo documento vengono tracciati periodicamente i risultati ottenuti che verranno analizzati tramite misurazioni permettendoci di correggere eventuali problematiche.

\subsection{Scopo del capitolato}
Il capitolato su cui noi Seven Clickers lavoriamo nasce da una proposta dell'azienda SanMarco Informatica per evitare sprechi dovuti all'utilizzo di uno ShowRoom tradizionale proponendo uno ShowRoom 3D con un ambientazione ugualmente o più coinvolgente.

\subsection{Riferimenti}
\subsubsection{Riferimenti normativi}
Da inserire Norme di Progetto ultima versione ...

\subsubsection{Riferimenti informativi}
\begin{itemize}
	\item Materiale didattico Ingegneria del Software - T02 Processi di ciclo di vita: \url{https://www.math.unipd.it/~tullio/IS-1/2022/Dispense/T02.pdf}
	\item Materiale didattico Ingegneria del Software - T08 Qualità di prodotto: \url{https://www.math.unipd.it/~tullio/IS-1/2022/Dispense/T08.pdf}
	\item Materiale didattico Ingegneria del Software - T09 Qualità di processo: \url{https://www.math.unipd.it/~tullio/IS-1/2022/Dispense/T09.pdf}
	\item Indice di Gulpease: \url{https://it.wikipedia.org/wiki/Indice_Gulpease}
	\item Da inserire futuri riferimenti...
\end{itemize}

\section{Qualità del processo}
Da completare...

\subsection{Obiettivi di qualità del processo}
Da completare...

\subsection{Metriche utilizzate}
Da completare...

\section{Qualità del prodotto}
Il gruppo ha deciso di utilizzare lo standard \textbf{ISO/IEC 9126} selezionando le qualità necessarie per l'intero ciclo di vita del progetto traendo delle metriche per il mantenimento di queste qualità

\pagebreak

\subsection{Obiettivi di qualità del prodotto}
\textbf{Documenti}
\begin{longtable}{ 
		>{\centering}M{0.20\textwidth} 
		>{\centering}M{0.50\textwidth}
		>{\centering}M{0.17\textwidth} 
		}
	\rowcolorhead
	\headertitle{Obiettivo} &
	\centering \headertitle{Descrizione} &	
	\headertitle{Metriche} 
	\endfirsthead	
	\endhead
	
	Comprensione dei testi & I documenti prodotti devono essere leggibili e comprensibili a lettori con licenza media. & ??\tabularnewline	
\end{longtable}

\noindent\textbf{Software}
\begin{longtable}{ 
		>{\centering}M{0.20\textwidth} 
		>{\centering}M{0.50\textwidth}
		>{\centering}M{0.17\textwidth} 
		}
	\rowcolorhead
	\headertitle{Obiettivo} &
	\centering \headertitle{Descrizione} &	
	\headertitle{Metriche} 
	\endfirsthead	
	\endhead
	
	?? & ?? & ??\tabularnewline
	?? & ?? & ??\tabularnewline
	?? & ?? & ??\tabularnewline
\end{longtable}

\subsection{Metriche utilizzate}
Da completare...
