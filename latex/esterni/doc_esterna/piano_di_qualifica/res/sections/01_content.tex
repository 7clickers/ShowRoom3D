\section{Introduzione}
\subsection{Scopo del documento}
Questo documento è stato creato dal gruppo Seven Clickers per descrivere degli standard fissati e dei metodi utilizzati al fine di garantire la qualità dei prodotti e dei processi.
In questo documento vengono tracciati periodicamente i risultati ottenuti che verranno analizzati tramite misurazioni permettendoci di correggere eventuali problematiche.

\subsection{Scopo del capitolato}
Il capitolato su cui noi Seven Clickers lavoriamo nasce da una proposta dell'azienda SanMarco Informatica per evitare sprechi dovuti all'utilizzo di uno ShowRoom tradizionale proponendo uno ShowRoom 3D con un ambientazione ugualmente o più coinvolgente.

\subsection{Riferimenti}
\subsubsection{Riferimenti normativi}
Da inserire Norme di Progetto ultima versione ...

\subsubsection{Riferimenti informativi}
\begin{itemize}
	\item Materiale didattico Ingegneria del Software - T02 Processi di ciclo di vita: \url{https://www.math.unipd.it/~tullio/IS-1/2022/Dispense/T02.pdf}
	\item Materiale didattico Ingegneria del Software - T08 Qualità di prodotto: \url{https://www.math.unipd.it/~tullio/IS-1/2022/Dispense/T08.pdf}
	\item Materiale didattico Ingegneria del Software - T09 Qualità di processo: \url{https://www.math.unipd.it/~tullio/IS-1/2022/Dispense/T09.pdf}
	\item Indice di Gulpease: \url{https://it.wikipedia.org/wiki/Indice_Gulpease}
	\item Complessità ciclomatica: \url{https://it.wikipedia.org/wiki/Complessità_ciclomatica}
	\item Code coverage: \url{https://en.wikipedia.org/wiki/Code_coverage}
	\item Line of Code: \url{https://en.wikipedia.org/wiki/Source_lines_of_code}	
	\item Da inserire futuri riferimenti...
\end{itemize}

\section{Qualità del processo}
Da completare...

\subsection{Obiettivi di qualità del processo}
Da completare...

\subsection{Metriche utilizzate}
Da completare...

\section{Qualità del prodotto}
Il gruppo ha deciso di utilizzare lo standard \textbf{ISO/IEC 9126} selezionando le qualità necessarie per l'intero ciclo di vita del progetto selezionando delle metriche per il mantenimento di queste qualità

\pagebreak

\subsection{Obiettivi di qualità del prodotto}
\textbf{Documenti}
\begin{longtable}{ 
		>{\centering}M{0.20\textwidth} 
		>{\centering}M{0.50\textwidth}
		>{\centering}M{0.17\textwidth} 
		}
	\rowcolorhead
	\headertitle{Obiettivo} &
	\centering \headertitle{Descrizione} &	
	\headertitle{Metriche} 
	\endfirsthead	
	\endhead
	
	Comprensione dei testi & I documenti prodotti devono essere leggibili e comprensibili a lettori con licenza media. & MPD01\tabularnewline	
\end{longtable}

\noindent\textbf{Software}
\begin{longtable}{ 
		>{\centering}M{0.15	\textwidth} 
		>{\centering}M{0.50\textwidth}
		>{\centering}M{0.23\textwidth} 
		}
	\rowcolorhead
	\headertitle{Obiettivo} &
	\centering \headertitle{Descrizione} &	
	\headertitle{Metriche} 
	\endfirsthead	
	\endhead
	
	Funzionalità & Garantire con accuratezza e conformità le funzionalità poste nel documento di Analisi dei Requisiti & MPD02\tabularnewline
	Affidabilità & Capacità del prodotto di svolgere le funzionalità implementate & MPD03\tabularnewline
	Efficienza & Mantenere una velocità di esecuzione del prodotto relativamente alle risorse utilizzate & MPD04\tabularnewline
	Usabilità & Capacità del prodotto di essere utilizzato dall'utente & MPD05, MPD06\tabularnewline
	Manutenibilità & Capacità di modificare il prodotto nel tempo & MPD07, MPD08, MPD09\tabularnewline
	Portabilità & Capacità di funzionare in diversi ambienti di esecuzione & MPD10\tabularnewline
\end{longtable}

\subsection{Metriche utilizzate}
\textbf{MPD01 - Indice di Gulpease:}\\
Indice di leggibilità di un testo tarato sulla lingua italiana.
\begin{equation}
89+\frac{300*(\text{numero delle frasi})-10*(\text{numero delle lettere})}{\text{numero delle parole}}
\end{equation}\\
\textit{Valore minimo}:$$ \geq 50 $$ 
\textit{Valore ottimo}:$$ \geq 80 $$\\

\noindent\textbf{MPD02 - Percentuale requisiti soddisfatti}\\
Per questa metrica si fa riferimento al documento di \textit{Analisi dei Requisiti}.\\
\textit{Valore minimo}: 100\% dei requisiti obbligatori\\
\textit{Valore ottimo}: 100\% di tutti i requisiti\\

\noindent\textbf{MPD03 - Densità di fallimenti durante l'esecuzione}\\
Si intende la percentuale di failure o di esecuzioni non andate a buon fine di determinate azioni.\\
\textit{Valore minimo}: 20\%\\
\textit{Valore ottimo}: 10\%\\

\noindent\textbf{MPD04 - Tempo medio di risposta}\\
Metrica inerente alla velocità di risposta del prodotto in relazione delle risorse a disposizione.\\
\textit{Valore minimo}: 4 secondi\\
\textit{Valore ottimo}: 2 secondi\\

\noindent\textbf{MPD05 - Complessità ciclomatica}\\
Metrica utilizzata per misurare la complessità di un programma. Calcolata sul grafo dei cammini linearmente indipendenti percorsi dal software ed i nodi presenti, cioè i punti decisionali del programma.\\
\begin{equation}
v(G) = e - n + 2p
\end{equation}\\
dove:
\begin{itemize}
	\item \textit{v(G)} = complessità ciclomatica del grafo G
	\item \textit{e} = il numero di archi nel grafo
	\item \textit{n} = il numero di nodi nel grafo
	\item \textit{p} = il numero di componenti connesse
\end{itemize}
\textit{Valore minimo}:$$ \leq 10 $$\\
\textit{Valore ottimo}:$$ \leq 4 $$\\

\noindent\textbf{MPD06 - Facilità di apprendimento}\\
Misura l'intuibilità e la facilità di utilizzo del programma.\\
\textit{Valore minimo}: 5 minuti\\
\textit{Valore ottimo}: 2 minuti\\

\noindent\textbf{MPD07 - Line of Code(LOC)}\\
Questa metrica misura il numero di linee di codice di un modulo escluso i commenti e le linee vuote.\\
\textit{Valore minimo}: ??\\
\textit{Valore ottimo}: ??\\

\noindent\textbf{MPD08 - Densità dei commenti}\\
Misura la percentuale delle righe di commento sul totale delle righe di codice presenti in un modulo.\\
\textit{Valore minimo}: 20\%\\
\textit{Valore ottimo}: 10\%\\

\noindent\textbf{MPD09 - Code Coverage}\\
Viene definita come la percentuale di codice attraversato dai test rispetto al totale del code base.\\
\textit{Valore minimo}: $$ \geq 70\% $$\\
\textit{Valore ottimo}: $$ \geq 100\% $$\\

\noindent\textbf{MPD10 - Browser Supportati}\\
Si calcola una percentuale dei browser supportati dal prodotto software.\\	
\textit{Valore minimo}: 80\%\\
\textit{Valore ottimo}: 100\%\\
