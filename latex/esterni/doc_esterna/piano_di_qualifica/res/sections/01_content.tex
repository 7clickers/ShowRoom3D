\section{Introduzione}
\subsection{Scopo del documento}
Questo documento è stato creato dal gruppo Seven Clickers per descrivere degli standard fissati e dei metodi utilizzati al fine di garantire la qualità dei prodotti e dei processi.
In questo documento vengono tracciati periodicamente i risultati ottenuti che verranno analizzati tramite misurazioni permettendoci di correggere eventuali problematiche.

\subsection{Scopo del capitolato}
Il capitolato su cui noi Seven Clickers lavoriamo nasce da una proposta dell'azienda SanMarco Informatica per evitare sprechi dovuti all'utilizzo di uno ShowRoom tradizionale proponendo uno ShowRoom 3D con un ambientazione ugualmente o più coinvolgente.

\subsection{\textit{Glossario}}
In questo documento sono state segnate con il pedice "g" tutte le parole che, secondo noi, necessitano di una loro definizione più accurata nel documento di \textit{Glossario}.

\subsection{Riferimenti}
\subsubsection{Riferimenti normativi}
\begin{itemize}
\item \textit{Norme di Progetto}.
\end{itemize}

\subsubsection{Riferimenti informativi}
\begin{itemize}
	\item Materiale didattico Ingegneria del Software - T02 Processi di ciclo di vita\textsubscript{g}: \url{https://www.math.unipd.it/~tullio/IS-1/2022/Dispense/T02.pdf}
	\item Materiale didattico Ingegneria del Software - T08 Qualità di prodotto\textsubscript{g}: \url{https://www.math.unipd.it/~tullio/IS-1/2022/Dispense/T08.pdf}
	\item Materiale didattico Ingegneria del Software - T09 Qualità di processo\textsubscript{g}: \url{https://www.math.unipd.it/~tullio/IS-1/2022/Dispense/T09.pdf}
	\item Indice di Gulpease: \url{https://it.wikipedia.org/wiki/Indice_Gulpease}
	\item Complessità ciclomatica: \url{https://www.math.unipd.it/~tullio/IS-1/2022/Dispense/T12.pdf}
	\item Code coverage: \url{https://www.math.unipd.it/~tullio/IS-1/2022/Dispense/T12.pdf}	
	\item Lo standard ISO/IEC 12207:1995 : \url{https://www.math.unipd.it/~tullio/IS-1/2009/Approfondimenti/ISO_12207-1995.pdf}
	\item Riferimento per alcune metriche di processo\textsubscript{g}: \url{https://it.wikipedia.org/wiki/Metriche_di_progetto}
	\item Requirements Stability Index (RSI): \\ \url{https://shiyamtj.wordpress.com/2018/09/26/requirement-stability-index/}
\end{itemize}

\section{Qualità del processo\textsubscript{g}}
Per mantenere la qualità dei processi il gruppo ha deciso di utilizzare lo standard \textbf{ISO/IEC 12207:1995} scegliendo i processi più adatti al nostro progetto, adeguandoli e semplificandoli in base alle necessità del progetto.

\subsection{Obiettivi di qualità del processo\textsubscript{g}}
Nelle seguenti tabelle vengono identificati i processi, una loro breve descrizione e le metriche a loro associate. 
\subsubsection{Processi primari}
\begin{longtable}{ 
		>{\centering}M{0.20\textwidth} 
		>{\centering}M{0.50\textwidth}
		>{\centering}M{0.17\textwidth} 
		}
	\rowcolorhead
	\headertitle{Processo\textsubscript{g}} &
	\centering \headertitle{Descrizione} &	
	\headertitle{Metriche} 
	\endfirsthead
	\endhead
	
	Fornitura & Processo\textsubscript{g} dedito alla determinazione delle procedure e delle risorse necessarie per gestire e garantire il progetto. & MPC01, MPC02, MPC03, MP04, MPC05, MPC06, MPC07, MPC08\tabularnewline
	Sviluppo & Processo\textsubscript{g} contenente le attività relative alle sviluppo del progetto & MPC09\tabularnewline
\end{longtable}

\subsubsection{Processi di supporto}
\begin{longtable}{ 
		>{\centering}M{0.20\textwidth} 
		>{\centering}M{0.50\textwidth}
		>{\centering}M{0.17\textwidth} 
		}
	\rowcolorhead
	\headertitle{Processo\textsubscript{g}} &
	\centering \headertitle{Descrizione} &	
	\headertitle{Metriche} 
	\endfirsthead
	\endhead
	
	Documentazione & Processo\textsubscript{g} dedicato al controllo dei documenti prodotti. I documenti prodotti devono essere leggibili e comprensibili a lettori con licenza media. & MPC10\tabularnewline
	Accertamento della qualità & Processo\textsubscript{g} che garantisce la conformità dei processi e dei prodotti ai requisiti specificati e ai loro piani & MPC11\tabularnewline
	Verifica\textsubscript{g} & Processo\textsubscript{g} che determina se le condizioni o i requisiti di un prodotto\textsubscript{g} sono soddisfatti. Questo processo\textsubscript{g} include analisi,revisione e test & MPC12\tabularnewline
\end{longtable}


\subsubsection{Processi organizzativi}
\begin{longtable}{ 
		>{\centering}M{0.20\textwidth} 
		>{\centering}M{0.50\textwidth}
		>{\centering}M{0.17\textwidth} 
		}
	\rowcolorhead
	\headertitle{Processo\textsubscript{g}} &
	\centering \headertitle{Descrizione} &	
	\headertitle{Metriche} 
	\endfirsthead
	\endhead
	
	Gestione organizzativa & Processo\textsubscript{g} che organizza,monitora e controlla le prestazioni di un processo\textsubscript{g} & MPC13\tabularnewline	
\end{longtable}

\subsection{Metriche utilizzate}
\begin{longtable}{
		>{\centering}M{0.17\textwidth}
		>{\centering}M{0.20\textwidth}	 
		>{\centering}M{0.23\textwidth}
		>{\centering}M{0.24\textwidth} 
		}
	\rowcolorhead
	\headertitle{ID} &
	\centering \headertitle{Metrica} &	
	\headertitle{Valore minimo} &
	\headertitle{Valore ottimo} 
	\endfirsthead	
	\endhead
MPC01 & Planned Value (PV) & $ \ge 0 $ \euro & $ \le \text{Budget at Completion\textsubscript{g}} $ \tabularnewline
MPC02 & Actual Cost (AC) & $ \ge 0 $ \euro & $ \le \text{EAC} $\tabularnewline
MPC03 & Earned Value (EV) & $ \ge 0 $ \euro & $ \le \text{EAC} $ \tabularnewline
MPC04 & Estimated at Completion (EAC) & \text{EAC}$ \le\text{preventivo -8\%}$  \text{EAC}$\ge preventivo +5\%$   & Costo preventivato \tabularnewline
MPC05 & Estimated to Complete (ETC) & $ \ge 0 $ \euro & $ \le \text{EAC} $ \tabularnewline
MPC06 & Cost Variance (CV) & $ \ge 0$ \euro &  0 \euro \tabularnewline
MPC07 & Schedule Variance (SV) & $ \ge -15\% $ & $ 0\% $ \tabularnewline
MPC08 & Budget Variance (BV) & $ \ge 0 $ \euro & 0 \euro \tabularnewline
MPC09 & Requirements Stability Index (RSI) & 70\% & 100\%\tabularnewline
MPC10 & Indice di Gulpease &  $ \ge 50 $ & $ \ge 80 $\tabularnewline
MPC11 & Metriche soddisfatte & $ \ge 80\% $ & 100\% \tabularnewline
MPC12 & Code Coverage & $ \ge 70\% $  & $ \ge 90-100\% $\tabularnewline
MPC13 & Rischi non previsti & $\ge 0$ & 0 \tabularnewline
\end{longtable}

\section{Qualità del prodotto\textsubscript{g}}
Il gruppo ha deciso di utilizzare lo standard \textbf{ISO/IEC 9126} selezionando le qualità necessarie per l'intero ciclo di vita\textsubscript{g} del progetto selezionando delle metriche per il loro mantenimento.

\subsection{Obiettivi di qualità del prodotto\textsubscript{g}}
Nelle seguenti tabelle vengono identificati gli obiettivi di qualità, una loro breve descrizione e le metriche a loro associate.
\subsubsection{Software}
\begin{longtable}{ 
		>{\centering}M{0.20\textwidth} 
		>{\centering}M{0.50\textwidth}
		>{\centering}M{0.17\textwidth} 
		}
	\rowcolorhead
	\headertitle{Obiettivo} &
	\centering \headertitle{Descrizione} &	
	\headertitle{Metriche} 
	\endfirsthead	
	\endhead
	
	Funzionalità & Garantire con accuratezza e conformità le funzionalità poste nel documento di \textit{Analisi dei Requisiti} & MPD01\tabularnewline
	Affidabilità & Capacità del prodotto\textsubscript{g} di svolgere le funzionalità implementate & MPD02\tabularnewline
	Efficienza & Mantenere una velocità di esecuzione del prodotto\textsubscript{g} relativamente alle risorse utilizzate & MPD03,MPD04\tabularnewline
	Usabilità & Capacità del prodotto\textsubscript{g} di essere utilizzato dall'utente & MPD05\tabularnewline
	Manutenibilità & Capacità di modificare il prodotto\textsubscript{g} nel tempo & MPD06, MPD07\tabularnewline
	Portabilità & Capacità di funzionare in diversi ambienti di esecuzione & MPD08\tabularnewline
\end{longtable}

\subsection{Metriche utilizzate}
\begin{longtable}{
		>{\centering}M{0.17\textwidth}
		>{\centering}M{0.20\textwidth}	 
		>{\centering}M{0.23\textwidth}
		>{\centering}M{0.24\textwidth} 
		}
	\rowcolorhead
	\headertitle{ID} &
	\centering \headertitle{Metrica} &	
	\headertitle{Valore minimo} &
	\headertitle{Valore ottimo} 
	\endfirsthead	
	\endhead
MPD01 & Percentuale requisiti soddisfatti & 100\% requisiti obbligatori & 100\% tutti requisiti \tabularnewline
MPD02 & Densità fallimenti durante l'esecuzione & 20\% & 10\% \tabularnewline
MPD03 & Tempo medio di risposta & 4 secondi & 2 secondi \tabularnewline
MPD04 & Tempo di caricamento & 15 secondi & 10 secondi \tabularnewline
MPD05 & Facilità di apprendimento & 5 minuti & 2 minuti \tabularnewline
MPD06 & Complessità ciclomatica & $ \le 10 $ &  $ \le 4 $ \tabularnewline
MPD07 & Densità dei commenti & 20\% & 10\% \tabularnewline
MPD08 & Browser Supportati & 80\% & 100\% \tabularnewline
\end{longtable}

\section{Specifica dei Test}
\begin{itemize}
\item Test di unità: vengono stabiliti durante la progettazione e servono per verificare le singole unità software;
\item Test di integrazione: vengono stabiliti durante la progettazione e servono per integrare il funzionamento di più unità;
\item Test di accettazione: vengono effettuati insieme al proponente\textsubscript{g} durante la fase di collaudo;
\item Test di sistema: vengono stabiliti durante l'\textit{Analisi dei Requisiti} e servono per accertare la copertura dei requisiti software definiti nel documento di \textit{Analisi dei Requisiti}.
\end{itemize}
Gli acronimi utilizzati in questo documento per identificare i test sono specificati dettagliatamente nel documento di \textit{Norme di Progetto}.
In questa sezione vengono utilizzate le seguenti sigle per lo stato di ogni test:
\begin{itemize}
\item \textbf{S}: test superato
\item \textbf{N}: test non implementato
\end{itemize}

\subsection{Test di unità}
Questi test verranno stabiliti durante la Progettazione.

\subsection{Test di integrità}
Questi test verranno stabiliti durante la Progettazione.

\subsection{Test di accettazione}
Questi test verranno stabiliti durante la fase di Collaudo.

\subsection{Test di sistema}
Per assicurare che vengano rispettati i requisiti concordati nel documento di \textit{Analisi dei Requisiti}, vengono eseguiti i seguenti test di sistema.
\begin{longtable}{
		>{\centering}M{0.20\textwidth}
		>{\centering}M{0.35\textwidth}	 
		>{\centering}M{0.20\textwidth} 
		}
	\rowcolorhead
	\headertitle{Test} &
	\centering \headertitle{Descrizione} &	
	\headertitle{Stato} 
	\endfirsthead	
	\endhead
TSRF1 & Si verifica\textsubscript{g} che l'utente possa aggiungere, l'oggetto con cui sta interagendo, nel carrello & N \tabularnewline
TSRF2 & Si verifica\textsubscript{g} che l'utente possa visualizzare il contenuto del carrello & N \tabularnewline
TSRF2.1 & Si verifica\textsubscript{g} che l'utente possa visualizzare la lista degli oggetti presenti nel carrello & N \tabularnewline
TSRF2.1.1 & Si verifica\textsubscript{g} che l'utente possa interagire con un oggetto nel carrello & N \tabularnewline
TSRF2.1.1.1 & Si verifica\textsubscript{g} che l'utente possa visualizzare la caratteristica del nome di ogni oggetto presente nella lista degli oggetti presenti nel carrello & N \tabularnewline
TSRF2.1.1.2 & Si verifica\textsubscript{g} che l'utente possa visualizzare la caratteristica del costo di ogni oggetto presente nella lista degli oggetti presenti nel carrello & N \tabularnewline
TSRF2.1.1.3 & Si verifica\textsubscript{g} che l'utente possa visualizzare la caratteristica della quantità di ogni oggetto presente nella lista degli oggetti presenti nel carrello & N \tabularnewline
TSRF2.2 & Si verifica\textsubscript{g} che l'utente possa visualizzare il costo totale degli oggetti che ha inserito nel carrello & N \tabularnewline
TSRF3 & Si verifica\textsubscript{g} che l'utente abbia la possibilità di rimuovere tutti gli oggetti dal carrello & N \tabularnewline
TSRF4 & Si verifica\textsubscript{g} che l'utente abbia la possibilità di rimuovere un singolo oggetto dal carrello & N \tabularnewline
TSRF5 & Si verifica\textsubscript{g} che l'utente possa muoversi in maniera direzionale & N \tabularnewline
TSRF5.1 & Si verifica\textsubscript{g} che l'utente possa compiere movimenti direzionali nell'asse X & N \tabularnewline
TSRF5.2 & Si verifica\textsubscript{g} che l'utente possa compiere movimenti direzionali nell'asse Y & N \tabularnewline
TSRF5.3 & Si verifica\textsubscript{g} che l'utente possa compiere movimenti direzionali nell'asse Z & N \tabularnewline
TSRF6 & Si verifica\textsubscript{g} che l'utente possa compiere spostamenti di camera & N \tabularnewline
TSRF6.1 & Si verifica\textsubscript{g} che l'utente possa compiere spostamenti di camera nell'asse X & N \tabularnewline
TSRF6.2 & Si verifica\textsubscript{g} che l'utente possa compiere spostamenti di camera nell'asse Y & N \tabularnewline
TSRF7 & Si verifica\textsubscript{g} che l'utente possa modificare la combinazione dei colori di un oggetto & N \tabularnewline
TSRF8 & Si verifica\textsubscript{g} che l'utente venga notificato in caso non fosse possibile modificare un oggetto & N \tabularnewline
TSRF9 & Si verifica\textsubscript{g} che l'utente possa visualizzare la lista degli oggetti della stanza in cui si trova & N \tabularnewline
TSRF9.1 & Si verifica\textsubscript{g} che l'utente possa visualizzare un singolo oggetto nella lista degli oggetti della stanza in cui si trova & N \tabularnewline
TSRF9.1.1 & Si verifica\textsubscript{g} che l'utente possa visualizzare la caratteristica del nome di ogni oggetto della lista degli oggetti della stanza in cui si trova & N \tabularnewline
TSRF10 & Si verifica\textsubscript{g} che l'utente possa visualizzare tutti i dettagli di un oggetto selezionato & N \tabularnewline
TSRF11 & Si verifica\textsubscript{g} che l'utente abbia la possibilità di riposizionarsi vicino ad un oggetto nella stanza in cui si trova & N \tabularnewline
TSRF12 & Si verifica\textsubscript{g} che l'utente possa riposizionarsi in una stanza da lui selezionata & N \tabularnewline
TSRF13 & Si verifica\textsubscript{g} che l'utente venga notificato in caso il riposizionamento in una stanza non sia possibile & N \tabularnewline
TSRF14 & Si verifica\textsubscript{g} che l'utente venga notificato in caso il riposizionamento in prossimità di un oggetto selezionato non sia concesso & N \tabularnewline
TSRF15 & Si verifica\textsubscript{g} che l'utente possa visualizzare la lista delle stanze & N \tabularnewline
TSRF15.1 & Si verifica\textsubscript{g} che l'utente possa visualizzare una singola stanza dalla lista delle stanze & N \tabularnewline
TSRF15.1.1 & Si verifica\textsubscript{g} che l'utente possa visualizzare la caratteristica del nome di ogni stanza dalla lista delle stanze & N \tabularnewline
TSRF15.1.2 & Si verifica\textsubscript{g} che l'utente possa visualizzare la caratteristica della tipologia di oggetti presenti in ogni stanza nella lista delle stanze & N \tabularnewline
TSRF16 & Si verifica\textsubscript{g} che l'utente possa riposizionare un oggetto presente nella stanza in cui si trova & N \tabularnewline
TSRF17 & Si verifica\textsubscript{g} che l'utente non possa riposizionare un oggetto in una coordinata non legittima & N \tabularnewline
TSRF18 & Si verifica\textsubscript{g} che l'utente sia in grado ad illuminare l'ambiente davanti a lui & N \tabularnewline
TSRF19 & Si verifica\textsubscript{g} che l'utente venga notificato se il contenuto del carrello è vuoto & N \tabularnewline
TSRF20 & Si verifica\textsubscript{g} che l'utente possa visualizzare un oggetto illuminato & N \tabularnewline
\end{longtable}

\subsection{Tracciamento dei test}
\subsubsection{Test di Sistema - Requisiti}
\begin{longtable}{
		>{\centering}M{0.25\textwidth}
		>{\centering}M{0.25\textwidth}	 
		}
	\rowcolorhead
	\headertitle{Test di sistema} &
	\headertitle{Requisiti}
	\endfirsthead	
	\endhead
TSRF1 & RF1\tabularnewline
TSRF2 & RF2\tabularnewline
TSRF2.1 & RF2.1\tabularnewline
TSRF2.1.1 & RF2.1.1\tabularnewline
TSRF2.1.1.1 & RF2.1.1.1\tabularnewline
TSRF2.1.1.2 & RF2.1.1.2\tabularnewline
TSRF2.1.1.3 & RF2.1.1.3\tabularnewline
TSRF2.2 & RF2.2\tabularnewline
TSRF3 & RF3\tabularnewline
TSRF4 & RF4\tabularnewline
TSRF5 & RF5\tabularnewline
TSRF5.1 & RF5.1\tabularnewline
TSRF5.2 & RF5.2\tabularnewline
TSRF5.3 & RF5.3\tabularnewline
TSRF6 & RF6\tabularnewline
TSRF7 & RF7\tabularnewline
TSRF8 & RF8\tabularnewline
TSRF9 & RF9\tabularnewline
TSRF9.1 & RF9.1\tabularnewline
TSRF9.1.1 & RF9.1.1\tabularnewline
TSRF10 & RF10\tabularnewline
TSRF11 & RF11\tabularnewline
TSRF12 & RF12\tabularnewline
TSRF13 & RF13\tabularnewline
TSRF14 & RF14\tabularnewline
TSRF15 & RF15\tabularnewline
TSRF15.1 & RF5.1\tabularnewline
TSRF15.1.1 & RF15.1.1\tabularnewline
TSRF15.1.2 & RF15.1.2\tabularnewline
TSRF16 & RF16\tabularnewline
TSRF17 & RF17\tabularnewline
TSRF18 & RF18\tabularnewline
TSRF19 & RF19\tabularnewline
TSRF20 & RF20\tabularnewline

\end{longtable}

