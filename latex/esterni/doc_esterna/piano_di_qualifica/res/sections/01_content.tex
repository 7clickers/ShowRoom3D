\section{Introduzione}
\subsection{Scopo del documento}
Questo documento è stato creato dal gruppo Seven Clickers per descrivere degli standard fissati e dei metodi utilizzati al fine di garantire la qualità dei prodotti e dei processi.
In questo documento vengono tracciati periodicamente i risultati ottenuti che verranno analizzati tramite misurazioni permettendoci di correggere eventuali problematiche.

\subsection{Scopo del capitolato}
Il capitolato su cui noi Seven Clickers lavoriamo nasce da una proposta dell'azienda SanMarco Informatica per evitare sprechi dovuti all'utilizzo di uno ShowRoom tradizionale proponendo uno ShowRoom 3D con un ambientazione ugualmente o più coinvolgente.

\subsection{Riferimenti}
\subsubsection{Riferimenti normativi}
Per i dettagli completi sulle metriche si fa riferimento all'ultima versione delle \textit{Norme di Progetto}.

\subsubsection{Riferimenti informativi}
\begin{itemize}
	\item Materiale didattico Ingegneria del Software - T02 Processi di ciclo di vita: \url{https://www.math.unipd.it/~tullio/IS-1/2022/Dispense/T02.pdf}
	\item Materiale didattico Ingegneria del Software - T08 Qualità di prodotto: \url{https://www.math.unipd.it/~tullio/IS-1/2022/Dispense/T08.pdf}
	\item Materiale didattico Ingegneria del Software - T09 Qualità di processo: \url{https://www.math.unipd.it/~tullio/IS-1/2022/Dispense/T09.pdf}
	\item Indice di Gulpease: \url{https://it.wikipedia.org/wiki/Indice_Gulpease}
	\item Complessità ciclomatica: \url{https://www.math.unipd.it/~tullio/IS-1/2022/Dispense/T12.pdf}
	\item Code coverage: \url{https://www.math.unipd.it/~tullio/IS-1/2022/Dispense/T12.pdf}	
	\item Lo standard ISO/IEC 12207:1995 : \url{https://www.math.unipd.it/~tullio/IS-1/2009/Approfondimenti/ISO_12207-1995.pdf}
	\item Riferimento per alcune metriche di processo: \url{https://it.wikipedia.org/wiki/Metriche_di_progetto}
	\item Requirements Stability Index (RSI): \\ \url{https://shiyamtj.wordpress.com/2018/09/26/requirement-stability-index/}
	\item Defect Density: \url{https://www.softwaretestinghelp.com/defect-density/}
\end{itemize}

\section{Qualità del processo}
Per mantenere la qualità dei processi il gruppo ha deciso di utilizzare lo standard \textbf{ISO/IEC 12207:1995} scegliendo i processi più adatti al nostro progetto, adeguandoli e semplificandoli in base alle necessità del progetto.

\subsection{Obiettivi di qualità del processo}
\subsubsection{Processi primari}
\begin{longtable}{ 
		>{\centering}M{0.20\textwidth} 
		>{\centering}M{0.50\textwidth}
		>{\centering}M{0.17\textwidth} 
		}
	\rowcolorhead
	\headertitle{Processo} &
	\centering \headertitle{Descrizione} &	
	\headertitle{Metriche} 
	\endfirsthead
	\endhead
	
	Fornitura & Processo dedito alla determinazione delle procedure e delle risorse necessarie per gestire e garantire il progetto. & MPC01, MPC02, MPC03, MP04, MPC05, MPC06, MPC07, MPC08\tabularnewline
	Sviluppo & Processo contenente le attività relative alle sviluppo del progetto & MPC09\tabularnewline	
\end{longtable}

\subsubsection{Processi di supporto}
\begin{longtable}{ 
		>{\centering}M{0.20\textwidth} 
		>{\centering}M{0.50\textwidth}
		>{\centering}M{0.17\textwidth} 
		}
	\rowcolorhead
	\headertitle{Processo} &
	\centering \headertitle{Descrizione} &	
	\headertitle{Metriche} 
	\endfirsthead
	\endhead
	
	Documentazione & Processo dedicato al controllo dei documenti prodotti. I documenti prodotti devono essere leggibili e comprensibili a lettori con licenza media. & MPC10\tabularnewline
	Accertamento della qualità & Processo che garantisce la conformità dei processi e dei prodotti ai requisiti specificati e ai loro piani & MPC11\tabularnewline
	Verifica & Processo che determina se le condizioni o i requisiti di un prodotto sono soddisfatti. Questo processo include analisi,revisione e test & MPC12\tabularnewline	
\end{longtable}

\subsubsection{Processi organizzativi}
\begin{longtable}{ 
		>{\centering}M{0.20\textwidth} 
		>{\centering}M{0.50\textwidth}
		>{\centering}M{0.17\textwidth} 
		}
	\rowcolorhead
	\headertitle{Processo} &
	\centering \headertitle{Descrizione} &	
	\headertitle{Metriche} 
	\endfirsthead
	\endhead
	
	Gestione organizzativa & Processo che organizza,monitora e controlla le prestazioni di un processo & MPC13\tabularnewline	
\end{longtable}

\subsection{Metriche utilizzate}
\begin{longtable}{
		>{\centering}M{0.17\textwidth}
		>{\centering}M{0.20\textwidth}	 
		>{\centering}M{0.23\textwidth}
		>{\centering}M{0.24\textwidth} 
		}
	\rowcolorhead
	\headertitle{ID} &
	\centering \headertitle{Metrica} &	
	\headertitle{Valore minimo} &
	\headertitle{Valore ottimo} 
	\endfirsthead	
	\endhead
MPC01 & Planned Value (PV) & $ \ge 0 $ \euro & $ \le \text{Budget at Completion} $ \tabularnewline
MPC02 & Actual Cost (AC) & $ \ge 0 $ \euro & $ \le \text{EAC} $\tabularnewline
MPC03 & Earned Value (EV) & $ \ge 0 $ \euro & $ \le \text{EAC} $ \tabularnewline
MPC04 & Estimated at Completion (EAC) & preventivo -5\% $ \le \text{EAC} $ preventivo +5\% $ \ge \text{EAC} $   & Costo preventivato \tabularnewline
MPC05 & Estimated to Complete (ETC) & $ \ge 0 $ \euro & $ \le \text{EAC} $ \tabularnewline
MPC06 & Cost Variance (CV) & $ \ge 0$ \euro &  0 \euro \tabularnewline
MPC07 & Schedule Variance (SV) & $ \ge -15\% $ & $ 0\% $ \tabularnewline
MPC08 & Budget Variance (BV) & $ \ge 0 $ \euro & 0 \euro \tabularnewline
MPC09 & Defect Density & $\le 5\%$ & $\le 2,5\%$ \tabularnewline
MPC10 & Indice di Gulpease &  $ \ge 50 $ & $ \ge 80 $\tabularnewline
MPC11 & Metriche soddisfatte & $ \ge 80\% $ & 100\% \tabularnewline
MPC12 & Code Coverage & $ \ge 70\% $  & $ \ge 90-100\% $\tabularnewline
MPC13 & Rischi non previsti & $\ge 0$ & 0 \tabularnewline
\end{longtable}

\section{Qualità del prodotto}
Il gruppo ha deciso di utilizzare lo standard \textbf{ISO/IEC 9126} selezionando le qualità necessarie per l'intero ciclo di vita del progetto selezionando delle metriche per il loro mantenimento.

\subsection{Obiettivi di qualità del prodotto}
\subsubsection{Software}
\begin{longtable}{ 
		>{\centering}M{0.20\textwidth} 
		>{\centering}M{0.50\textwidth}
		>{\centering}M{0.17\textwidth} 
		}
	\rowcolorhead
	\headertitle{Obiettivo} &
	\centering \headertitle{Descrizione} &	
	\headertitle{Metriche} 
	\endfirsthead	
	\endhead
	
	Funzionalità & Garantire con accuratezza e conformità le funzionalità poste nel documento di \textit{Analisi dei Requisiti} & MPD01, MPD02\tabularnewline
	Affidabilità & Capacità del prodotto di svolgere le funzionalità implementate & MPD03\tabularnewline
	Efficienza & Mantenere una velocità di esecuzione del prodotto relativamente alle risorse utilizzate & MPD04,MPD05\tabularnewline
	Usabilità & Capacità del prodotto di essere utilizzato dall'utente & MPD06\tabularnewline
	Manutenibilità & Capacità di modificare il prodotto nel tempo & MPD07, MPD08\tabularnewline
	Portabilità & Capacità di funzionare in diversi ambienti di esecuzione & MPD09\tabularnewline
\end{longtable}

\subsection{Metriche utilizzate}
\begin{longtable}{
		>{\centering}M{0.17\textwidth}
		>{\centering}M{0.20\textwidth}	 
		>{\centering}M{0.23\textwidth}
		>{\centering}M{0.24\textwidth} 
		}
	\rowcolorhead
	\headertitle{ID} &
	\centering \headertitle{Metrica} &	
	\headertitle{Valore minimo} &
	\headertitle{Valore ottimo} 
	\endfirsthead	
	\endhead
MPD01 & Percentuale requisiti soddisfatti & 100\% requisiti obbligatori & 100\% tutti requisiti \tabularnewline
MPD02 & Requirements Stability Index (RSI) & 70\% & 100\%\tabularnewline
MPD03 & Densità fallimenti durante l'esecuzione & 20\% & 10\% \tabularnewline
MPD04 & Tempo medio di risposta & 4 secondi & 2 secondi \tabularnewline
MPD05 & Tempo di caricamento & 15 secondi & 10 secondi \tabularnewline
MPD06 & Facilità di apprendimento & 5 minuti & 2 minuti \tabularnewline
MPD07 & Complessità ciclomatica & $ \le 10 $ &  $ \le 4 $ \tabularnewline
MPD08 & Densità dei commenti & 20\% & 10\% \tabularnewline
MPD09 & Browser Supportati & 80\% & 100\% \tabularnewline
\end{longtable}
\pagebreak