\section{Modello di sviluppo attuale}
\subsection{Modello agile}
In seguito alle osservazioni del prof. Vardanega durante la revisione di Requirements and Technology Baseline, il gruppo ha tenuto un incontro interno e ha optato per apportare significative modifiche al nostro approccio di sviluppo. Dopo un'analisi approfondita, infatti, il piano di lavoro è stato completamente rivisto per adeguarsi ai criteri di progetto concordati con il proponente. \\
Abbiamo pertanto scelto di passare da un modello incrementale a uno agile. Questo approccio si adatta meglio alle nostre necessità, considerando le difficoltà incontrate nel pianificare accuratamente a causa delle tecnologie ancora poco familiari.\\

Tenendo conto dell'inesperienza del gruppo, si terrà presente la possibilità che:
\begin{itemize}
\item Alcuni cicli possano richiedere più tempo del previsto, influenzando la tempistica complessiva del progetto;
\item Le risorse possano essere allocate in modo inefficiente, riducendo la produttività del gruppo;
\item La comunicazione interna possa essere inadeguata, causando incomprensioni e ritardi nelle attività.
\end{itemize}

L'utilizzo del modello Agile\textsubscript{g} e del ciclo Scrum\textsubscript{g}, anche senza esperienza pregressa, può portare a diversi vantaggi per il gruppo, tra cui:
\begin{itemize}
\item Il modello Agile\textsubscript{g} offre maggiore flessibilità per adattarsi rapidamente ai cambiamenti nei requisiti del progetto e ai problemi emergenti.
\item Il ciclo Scrum\textsubscript{g} promuove una collaborazione e comunicazione migliorate tra i membri del gruppo, incoraggiando il lavoro di squadra e la risoluzione congiunta dei problemi.
\item I cicli regolari di Scrum\textsubscript{g} permettono un costante monitoraggio del progresso e l'adattamento della pianificazione e delle priorità alle esigenze del progetto.
\item La suddivisione del progetto in cicli brevi (Sprint) favorisce lo sviluppo incrementale, consentendo di concentrarsi su piccoli obiettivi raggiungibili e di rivedere periodicamente i risultati.
\end{itemize}

I cicli Scrum\textsubscript{g} potranno avere frequenza settimanale o durata massima di due settimane, a seconda della disponibilità dei membri del gruppo.
Ogni ciclo inizia con un incontro di gruppo che include:
\begin{itemize}
\item Valutazione del progresso complessivo del progetto;
\item Identificazione di nuove opportunità o sfide emerse nel corso del ciclo precedente;
\item Revisione delle priorità e delle scadenze per il ciclo successivo;
\item Discussione sulle strategie per migliorare la collaborazione e l'efficienza del gruppo.
\end{itemize}
il primo Sprint inizierà immediatamente dopo la revisione di Requirements and Technology Baseline.


