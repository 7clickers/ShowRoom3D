\section{Consuntivo}
%introduzione
\subsection{Introduzione}
Questa sezione riporta i dati raccolti durante il progetto riguardo le ripartizioni dei ruoli e le ore impiegate da parte di tutti i componenti del gruppo e viene posta in confronto alle previsioni pianificate nella sezione di consuntivo.

%qui verrà inserito il consuntivo finale con tutte le ore del progetto.

\subsection{Dettaglio periodi}
In questa sezione viene mostrato l'effettivo utilizzo di ore per ogni componente del gruppo, calcolato alla fine di ogni fase del progetto.

\subsubsection{Analisi Preliminare}
Qui di seguito riportiamo la suddivisione dei ruoli e le ore di lavoro effettive impiegate nel periodo di Analisi Preliminare:

\begin{longtable}{ 
	>{\centering}M{0.15\textwidth} 
	>{\centering}M{0.075\textwidth}
	>{\centering}M{0.085\textwidth}
	>{\centering}M{0.075\textwidth}
	>{\centering}M{0.075\textwidth}
	>{\centering}M{0.095\textwidth}
	>{\centering}M{0.075\textwidth}
	>{\centering}M{0.085\textwidth}
	>{\centering\arraybackslash}M{0.06\textwidth} 
	}
	\rowcolorhead
	\centering \headertitle{Ruolo} &
	\headertitle{Marco} &	
	\headertitle{Giacomo} &
	\headertitle{Elena} &
	\headertitle{Mirko} &
	\headertitle{Tommaso} &
	\headertitle{Rino} &
	\headertitle{Gabriele} &
	\headertitle{Totale}
	\endfirsthead	
	\endhead
	
	Responsabile & 10 (-1) & - & 8 & - & 8 & - & - & 25 \tabularnewline
	Amministratore & - & -  & - & - & 9 (-1) & 9 & - & 17 \tabularnewline
	Analista & 17 (-2)& 7 (-1)  & 15(-2) & 12 & - & 14(-1) & 12 & 71 \tabularnewline
	Progettista & - & -  & - & - & - & - & - & 0 \tabularnewline
	Programmatore & - & - & - & - & - & - & - & 0 \tabularnewline
	Verificatore & - & 15 (+1) & - & 13 & 11 (+1) & - & 13 (+1) & 55 \tabularnewline
	\rowcolorhead \textcolor{white}{\textbf{Totale}} & \textcolor{white}{\textbf{24}} &\textcolor{white}{\textbf{22}} & \textcolor{white}{\textbf{21}} & \textcolor{white}   {\textbf{25}} & 	\textcolor{white}{\textbf{28}} & \textcolor{white}{\textbf{22}} & \textcolor{white}{\textbf{26}} & 	\textcolor{white}{\textbf{168}}\\
		\captionline\caption{Rendiconto effettivo della distribuzione delle ore nel periodo di Analisi Preliminare}
\end{longtable}

I costi effettivi del periodo sono i seguenti:

\begin{longtable}{
	>{\centering}M{0.25\textwidth} 
	>{\centering}M{0.10\textwidth}
	>{\centering\arraybackslash}M{0.15\textwidth} 
	}
	\rowcolorhead
	\centering \headertitle{Ruolo} &
	\headertitle{Ore} &	
	\headertitle{Costo}
	\endfirsthead
	\endhead
	
	Responsabile & 26(-1) & 780(-30)€  \tabularnewline
	Amministratore & 18(-1)  & 360(-20)€  \tabularnewline
	Analista & 77(-6)  & 1925(-150)€  \tabularnewline
	Progettista & 0  & 0  \tabularnewline
	Programmatore & 0 & 0  \tabularnewline
	Verificatore & 52(+3) & 780(+45)€ \tabularnewline
	\rowcolorhead \textcolor{white}{\textbf{Totale Consuntivo}} &\textcolor{white}{\textbf{168}}& \textcolor{white}{\textbf{3090€}}	\\
	\rowcolorhead \textcolor{white}{\textbf{Totale Preventivo}} &\textcolor{white}{\textbf{175}}& \textcolor{white}{\textbf{3245€}}	\\
	\rowcolorhead \textcolor{white}{\textbf{Differenza}} &\textcolor{white}{\textbf{-7}}& \textcolor{white}{\textbf{-155€}}	\\
	\captionline\caption{Prospetto costi nel periodo di Analisi Preliminare} 

\end{longtable}

\paragraph{Resoconto}
\begin{itemize}
	\item \textbf{Analista (-6 ore)}: a causa dell'inesperienza non è stato calcolato accuratamente il 
	tempo necessario per svolgere le attività degli analisti;
	\item \textbf{Verificatore (+3 ore)}: le attività dei verificatori si sono rivelate più lunghe in termini di tempi di lavoro. 
	Per questo motivo sono state impiegate più ore del previsto.  
\end{itemize}
In generale il gruppo ha impiegato meno ore di quelle preventivate, non riuscendo a 
terminare tutte le attività previste. Sono stati sottovalutati i tempi di verifica\textsubscript{g}, che risultano essere più lunghi 
del previsto. Il bilancio è positivo, ma è necessaria una ripianificazione per completare le attività rimaste indietro.


\subsubsection{Progettazione Proof of Concept\textsubscript{g}}
Qui di seguito riportiamo la suddivisione dei ruoli e le ore di lavoro effettive impiegate per la fase di Progettazione Proof of Concept\textsubscript{g}:

\begin{longtable}{ 
	>{\centering}M{0.20\textwidth} 
	>{\centering}M{0.06\textwidth}
	>{\centering}M{0.10\textwidth}
	>{\centering}M{0.05\textwidth}
	>{\centering}M{0.06\textwidth}
	>{\centering}M{0.10\textwidth}
	>{\centering}M{0.05\textwidth}
	>{\centering}M{0.09\textwidth}
	>{\centering\arraybackslash}M{0.06\textwidth} 
	}
	\rowcolorhead
	\centering \headertitle{Ruolo} &
	\headertitle{Marco} &	
	\headertitle{Giacomo} &
	\headertitle{Elena} &
	\headertitle{Mirko} &
	\headertitle{Tommaso} &
	\headertitle{Rino} &
	\headertitle{Gabriele} &
	\headertitle{Totale}
	\endfirsthead	
	\endhead
	
	Responsabile & - & - & - & 12 & - & - & - & 12 \tabularnewline
	Amministratore & - & -  & - & - & - & - & - & 0 \tabularnewline
	Analista & -  & -  & - & - & 12 & - & - & 12 \tabularnewline
	Progettista & 6 (+1) & 3  (-2) & - & - & 3 & 6 & 7 & 27 \tabularnewline
	Programmatore & 5 & 15 & 12 (+1) & 8 & - & 16 & 13 (-2) & 68 \tabularnewline
	Verificatore & 14 (-2) & - & 12 (-2) & - & 5 & - & - & 27 \tabularnewline
	\rowcolorhead \textcolor{white}{\textbf{Totale}} & \textcolor{white}{\textbf{24}} &\textcolor{white}{\textbf{19}} & \textcolor{white}{\textbf{23}} & \textcolor{white}{\textbf{20}} & 	\textcolor{white}{\textbf{20}} & \textcolor{white}{\textbf{22}} & \textcolor{white}{\textbf{18}} & 	\textcolor{white}{\textbf{146}}\\
	\captionline\caption{Rendiconto effettivo della distribuzione delle ore nel periodo di Progettazione Proof of Concept\textsubscript{g}}
\end{longtable}

I costi effettivi del periodo sono i seguenti:

\begin{longtable}{ 
		>{\centering}M{0.25\textwidth} 
		>{\centering}M{0.10\textwidth}
		>{\centering\arraybackslash}M{0.15\textwidth} 
		}
	\rowcolorhead
	\headertitle{Ruolo} &
	\headertitle{Ore} &
	\headertitle{Costo} 
	\endfirsthead	
	\endhead
	
	Responsabile & 12  & 360\euro\tabularnewline
	Amministratore & 0 & 0\euro \tabularnewline
	Analista & 12 & 300\euro \tabularnewline
	Progettista & 27 (-1) & 700(-25)\euro \tabularnewline
	Programmatore & 68 (-1) & 1035(-15)\euro \tabularnewline
	Verificatore & 27 (-4) & 465(-60)\euro \tabularnewline
	\rowcolorhead \textcolor{white}{\textbf{Totale Consuntivo}} & \textcolor{white}{\textbf{146}} & \textcolor{white}{\textbf{2760\euro}}\\
	\rowcolorhead \textcolor{white}{\textbf{Totale Preventivo}} & \textcolor{white}{\textbf{152}} & \textcolor{white}{\textbf{2860\euro}}\\
	\rowcolorhead \textcolor{white}{\textbf{Differenza}} & \textcolor{white}{\textbf{-6}} & \textcolor{white}{\textbf{-100\euro}}\\
	\captionline\caption{Prospetto costi nel periodo di Progettazione Proof of Concept\textsubscript{g}} 
\end{longtable}

\paragraph{Resoconto}
In questo periodo il gruppo non è riuscito a rispettare pienamente quanto previsto dal preventivo. Si 
è verificato un rallentamento nei ritmi di lavoro, che, se pur preso in considerazione nel preventivo, ha causato comunque 
una riduzione delle ore complessive di lavoro e della spesa totale. Le cause di questo rallentamento sono state individuate in:
\begin{itemize}
	\item Presenza di festività;
	\item Impegni personali;
	\item Alcuni componenti del gruppo sono stati impegnati più 
	del necessario a causa della sessione di esami.
\end{itemize}
Il gruppo è comunque riuscito a completare tutte le attività previste e a svolgere un incontro 
con il proponente\textsubscript{g}, ottenendo un feedback sul lavoro svolto e ulteriori indicazioni da seguire.
Per migliorare l'accuratezza dei preventivi si è deciso di tenere in maggiore considerazione gli impegni di ogni 
componente del gruppo e l'impatto che essi hanno di conseguenza sul lavoro cooperativo.


\subsubsection{Codifica Proof of Concept\textsubscript{g}}
Qui di seguito riportiamo la suddivisione dei ruoli e le ore di lavoro effettive impiegate per la fase di Codifica Proof of Concept\textsubscript{g}:

\begin{longtable}{ 
	>{\centering}M{0.20\textwidth} 
	>{\centering}M{0.06\textwidth}
	>{\centering}M{0.10\textwidth}
	>{\centering}M{0.05\textwidth}
	>{\centering}M{0.06\textwidth}
	>{\centering}M{0.10\textwidth}
	>{\centering}M{0.05\textwidth}
	>{\centering}M{0.09\textwidth}
	>{\centering\arraybackslash}M{0.06\textwidth} 
	}
	\rowcolorhead
	\centering \headertitle{Ruolo} &
	\headertitle{Marco} &	
	\headertitle{Giacomo} &
	\headertitle{Elena} &
	\headertitle{Mirko} &
	\headertitle{Tommaso} &
	\headertitle{Rino} &
	\headertitle{Gabriele} &
	\headertitle{Totale}
	\endfirsthead	
	\endhead
	
	Responsabile & - & - & - & 4 & - & - & 4 & 8 \tabularnewline
	Amministratore & - & -  & - & - & - & - & - & 0 \tabularnewline
	Analista & -  & -  & - & - & 8 (-1) & - & - & 7 \tabularnewline
	Progettista & - & 3 & - & - & - & 2 & 3 (-1) & 7 \tabularnewline
	Programmatore & - & 8 (+2) & - & 5 & - & 13 (+1) & 7 & 36 \tabularnewline
	Verificatore & 12 (-1) & - & 13 & 5 & 4(-1) & - & - & 32 \tabularnewline
	\rowcolorhead \textcolor{white}{\textbf{Totale}} & \textcolor{white}{\textbf{11}} &\textcolor{white}{\textbf{13}} & \textcolor{white}{\textbf{13}} & \textcolor{white}{\textbf{14}} & 	\textcolor{white}{\textbf{10}} & \textcolor{white}{\textbf{16}} & \textcolor{white}{\textbf{13}} & 	\textcolor{white}{\textbf{90}}\\
	\captionline\caption{Rendiconto effettivo della distribuzione delle ore nel periodo di Codifica Proof of Concept\textsubscript{g}}
\end{longtable}

I costi effettivi del periodo sono i seguenti:

\begin{longtable}{ 
		>{\centering}M{0.25\textwidth} 
		>{\centering}M{0.10\textwidth}
		>{\centering\arraybackslash}M{0.15\textwidth} 
		}
	\rowcolorhead
	\headertitle{Ruolo} &
	\headertitle{Ore} &
	\headertitle{Costo} 
	\endfirsthead	
	\endhead
	
	Responsabile & 8  & 240\euro\tabularnewline
	Amministratore & 0 & 0\euro \tabularnewline
	Analista & 7 (-1) & 200(-25)\euro \tabularnewline
	Progettista & 7 (-1) & 200(-25)\euro \tabularnewline
	Programmatore & 36 (+3) & 495(+45)\euro \tabularnewline
	Verificatore & 32 (-2) & 510(-30)\euro \tabularnewline
	\rowcolorhead \textcolor{white}{\textbf{Totale Consuntivo}} & \textcolor{white}{\textbf{90}} & \textcolor{white}{\textbf{1610\euro}}\\
	\rowcolorhead \textcolor{white}{\textbf{Totale Preventivo}} & \textcolor{white}{\textbf{91}} & \textcolor{white}{\textbf{1645\euro}}\\
	\rowcolorhead \textcolor{white}{\textbf{Differenza}} & \textcolor{white}{\textbf{-1}} & \textcolor{white}{\textbf{-35\euro}}\\
	\captionline\caption{Prospetto costi nel periodo di Codifica Proof of Concept\textsubscript{g}} 
\end{longtable}

\paragraph{Resoconto}
\begin{itemize}
	\item \textbf{Programmatore (+3 ore)}: nella codifica del Proof of Concept\textsubscript{g} sono state necessarie più 
	ore di lavoro del previsto a causa della poca familiarità che il gruppo aveva con le tecnologie scelte;
	\item \textbf{Verificatore (-2 ore)}: le attività dei verificatori hanno richiesto meno tempo del previsto.  
\end{itemize}
Nonostante alcune discrepanze nei singoli ruoli, in totale le ore effettive risultano in linea con 
quelle preventivate, con una differenza di costi di 35\euro.
% ====================================================== SPRINT 1
\subsubsection{Sprint 1}
Qui di seguito riportiamo la suddivisione dei ruoli e le ore di lavoro effettive impiegate per lo Sprint\textsubscript{g} 1:

\begin{longtable}{ 
	>{\centering}M{0.20\textwidth} 
	>{\centering}M{0.06\textwidth}
	>{\centering}M{0.10\textwidth}
	>{\centering}M{0.05\textwidth}
	>{\centering}M{0.06\textwidth}
	>{\centering}M{0.10\textwidth}
	>{\centering}M{0.05\textwidth}
	>{\centering}M{0.09\textwidth}
	>{\centering\arraybackslash}M{0.06\textwidth} 
	}
	\rowcolorhead
	\centering \headertitle{Ruolo} &
	\headertitle{Marco} &	
	\headertitle{Giacomo} &
	\headertitle{Elena} &
	\headertitle{Mirko} &
	\headertitle{Tommaso} &
	\headertitle{Rino} &
	\headertitle{Gabriele} &
	\headertitle{Totale}
	\endfirsthead	
	\endhead
	
	Responsabile & - & - & - & - & - & 5 (-3) & - & 2 \tabularnewline
	Amministratore & - & - & - & - & 2 (-1) & - & - & 1 \tabularnewline
	Analista & - & - & - & - & - & - & - & 0 \tabularnewline
	Progettista & - & - & - & 3 (-1) & 5 (-2) & - & - & 5 \tabularnewline
	Programmatore & 5 (-2) & - & 5 (-3) & - & - & - & - & 5 \tabularnewline
	Verificatore & - & 5 & - & - & - & - & 5 & 10 \tabularnewline
	\rowcolorhead \textcolor{white}{\textbf{Totale}} & \textcolor{white}{\textbf{3}} &\textcolor{white}{\textbf{5}} & \textcolor{white}{\textbf{2}} & \textcolor{white}{\textbf{2}} & 	\textcolor{white}{\textbf{4}} & \textcolor{white}{\textbf{2}} & \textcolor{white}{\textbf{5}} & \textcolor{white}{\textbf{23}}\\
	\captionline\caption{Distribuzione ruoli-ore nel periodo di Sprint 1}
\end{longtable}

I costi effettivi del periodo sono i seguenti:

\begin{longtable}{ 
		>{\centering}M{0.25\textwidth} 
		>{\centering}M{0.10\textwidth}
		>{\centering\arraybackslash}M{0.15\textwidth} 
		}
	\rowcolorhead
	\headertitle{Ruolo} &
	\headertitle{Ore} &
	\headertitle{Costo} 
	\endfirsthead	
	\endhead
	
	Responsabile & 5 (-3)  & 60 (-90)\euro\tabularnewline
	Amministratore & 2 (-1) & 40 (-20)\euro \tabularnewline
	Analista & 0 & 0\euro \tabularnewline
	Progettista & 8 (-3) & 125 (-75)\euro \tabularnewline
	Programmatore & 10 (-5) & 150 (-75)\euro \tabularnewline
	Verificatore & 10 & 150\euro \tabularnewline
	\rowcolorhead \textcolor{white}{\textbf{Totale Consuntivo}} & \textcolor{white}{\textbf{23}} & \textcolor{white}{\textbf{490\euro}}\\
	\rowcolorhead \textcolor{white}{\textbf{Totale Preventivo}} & \textcolor{white}{\textbf{35}} & \textcolor{white}{\textbf{690\euro}}\\
	\rowcolorhead \textcolor{white}{\textbf{Differenza}} & \textcolor{white}{\textbf{-12}} & \textcolor{white}{\textbf{-200\euro}}\\
	\captionline\caption{Prospetto costi nel periodo di Sprint\textsubscript{g} 1} 
\end{longtable}

\paragraph{Resoconto}
\begin{itemize}
	\item \textbf{Programmatore (-5 ore)}: trattandosi principalmente di attività correttive sulla documentazione sono
	state richieste meno ore del previsto.  
	\item \textbf{Progettista (-3 ore)}: trattandosi principalmente di attività correttive sulla documentazione sono
	state richieste meno ore del previsto. 
	\item \textbf{Responsabile (-3 ore)}: trattandosi principalmente di attività correttive sulla documentazione sono
	state richieste meno ore del previsto.  
\end{itemize}
Sono state richieste meno ore del previsto portando una differenza rispetto al costo preventivato di -200\euro.

\paragraph{Retrospezione}
Durante la riunione di gruppo di retrospezione sono emersi i seguenti punti rilevanti:

\begin{itemize}
	\item Tempi di produzione della documentazione migliorabili ulteriormente: 
	piú componenti del gruppo lavoreranno sulla stesura dello stesso documento per velocizzare i tempi; 
	\item Diari di Bordo poco utili: i prossimi Diari di Bordo verranno prodotti tenendo conto dei chiarimenti ricevuti dal Professore;
	\item Tempi di daily meeting troppo lunghi : i punti da chiarire giornalmente che riguardano i singoli componenti del gruppo 
	verranno discussi in privato con il Responsabile di Progetto tramite piattaforma Telegram. 
\end{itemize}
% ====================================================== SPRINT 2
\subsubsection{Sprint 2}
Qui di seguito riportiamo la suddivisione dei ruoli e le ore di lavoro effettive impiegate per lo Sprint\textsubscript{g} 2:

\begin{longtable}{ 
	>{\centering}M{0.20\textwidth} 
	>{\centering}M{0.06\textwidth}
	>{\centering}M{0.10\textwidth}
	>{\centering}M{0.05\textwidth}
	>{\centering}M{0.06\textwidth}
	>{\centering}M{0.10\textwidth}
	>{\centering}M{0.05\textwidth}
	>{\centering}M{0.09\textwidth}
	>{\centering\arraybackslash}M{0.06\textwidth} 
	}
	\rowcolorhead
	\centering \headertitle{Ruolo} &
	\headertitle{Marco} &	
	\headertitle{Giacomo} &
	\headertitle{Elena} &
	\headertitle{Mirko} &
	\headertitle{Tommaso} &
	\headertitle{Rino} &
	\headertitle{Gabriele} &
	\headertitle{Totale}
	\endfirsthead	
	\endhead
	
	Responsabile & - & - & - & 3 (-1) & - & - & - & 2 \tabularnewline
	Amministratore & - & - & - & - & 2 (-1) & - & - & 1 \tabularnewline
	Analista & - & - & - & - & - & - & - & 0 \tabularnewline
	Progettista & - & - & - & - & 4 (-2) & 7 (-3) & - & 6 \tabularnewline
	Programmatore & 4 (-2) & - & 4 (-2) & - & - & - & - & 4 \tabularnewline
	Verificatore & - & 4 & - & - & - & - & 4 & 8 \tabularnewline
	\rowcolorhead \textcolor{white}{\textbf{Totale}} & \textcolor{white}{\textbf{2}} &\textcolor{white}{\textbf{4}} & \textcolor{white}{\textbf{2}} & \textcolor{white}{\textbf{2}} & 	\textcolor{white}{\textbf{3}} & \textcolor{white}{\textbf{4}} & \textcolor{white}{\textbf{4}} & \textcolor{white}{\textbf{21}}\\
	\captionline\caption{Distribuzione ruoli-ore nel periodo di Sprint 2}
\end{longtable}
\pagebreak
I costi effettivi del periodo sono i seguenti:


\begin{longtable}{ 
		>{\centering}M{0.25\textwidth} 
		>{\centering}M{0.10\textwidth}
		>{\centering\arraybackslash}M{0.15\textwidth} 
		}
	\rowcolorhead
	\headertitle{Ruolo} &
	\headertitle{Ore} &
	\headertitle{Costo} 
	\endfirsthead	
	\endhead
	
	Responsabile & 3 (-1)  & 90 (-30)\euro\tabularnewline
	Amministratore & 2 (-1) & 40 (-20)\euro \tabularnewline
	Analista & 0 & 0\euro \tabularnewline
	Progettista & 11 (-5) & 275 (-150)\euro \tabularnewline
	Programmatore & 8 (-4) & 120 (-60)\euro \tabularnewline
	Verificatore & 8 & 120\euro \tabularnewline
	\rowcolorhead \textcolor{white}{\textbf{Totale Consuntivo}} & \textcolor{white}{\textbf{21}} & \textcolor{white}{\textbf{385\euro}}\\
	\rowcolorhead \textcolor{white}{\textbf{Totale Preventivo}} & \textcolor{white}{\textbf{35}} & \textcolor{white}{\textbf{645\euro}}\\
	\rowcolorhead \textcolor{white}{\textbf{Differenza}} & \textcolor{white}{\textbf{-18}} & \textcolor{white}{\textbf{-320\euro}}\\
	\captionline\caption{Prospetto costi nel periodo di Sprint\textsubscript{g} 2} 
\end{longtable}

\paragraph{Resoconto}
\begin{itemize}
	\item \textbf{Progettista (-5 ore)}: sono state preventivate più ore del previsto per assicurarci che la progettazione venisse fatta in 
	modo accurato trattandosi di un'attività particolarmente importante. 
	\item \textbf{Programmatore (-4 ore)}: sono state richieste meno ore del previsto in quanto i progettisti non hanno avuto particolare bisogno 
	di supporto nella richiesta di dettagli riguardanti l'utilizzo delle tecnologie utilizzate. Questo fatto è dovuto alla costante rotazione
	dei ruoli che ha portato l'intero gruppo di sviluppo ad una buona conoscenza delle tecnologie.
	\item \textbf{Verificatore (-4 ore)}: le attività dei verificatori hanno richiesto meno tempo del previsto.  
\end{itemize}
Sono state richieste meno ore del previsto portando una differenza rispetto al costo preventivato di -260\euro.

\paragraph{Retrospezione}
Avendo concentrato le risorse a disposizione sul diagramma delle classi non siamo riusciti a fare la stesura delle norme di codifica.
Inoltre abbiamo scelto di non produrre il diagramma di sequenza in quando non lo riteniamo indispensabile e dobbiamo far fronte
alla scadenza di consegna imminente o almeno limitare i ritardi di consegna.
Il diagramma delle classi non si deve considerare finito in quanto abbiamo previsto di richiedere un colloquio con il Professor Riccardo Cardin per
ricevere un riscontro su quanto prodotto.
\\\\
Le tasks che tornano a far parte del backlog\textsubscript{g} sono:

\begin{longtable}{ 
	>{\centering}M{0.30\textwidth} 
	>{\centering}M{0.10\textwidth}
	}
	\rowcolorhead
	\centering 
	\headertitle{Task} &	
	\headertitle{Priorità}
	\endfirsthead	
	\endhead
	
	Diagramma delle classi & Alta \tabularnewline
	Norme sulla codifica & Media \tabularnewline
	\captionline\caption{Tasks dello Sprint\textsubscript{g} 2 che ritornano a far parte del backlog\textsubscript{g}}
\end{longtable}

\noindent Durante la riunione di gruppo di retrospezione sono emersi i seguenti punti rilevanti:
\\\\
Lo sprint è andato bene e per adesso manteniamo i processi inalterati anche per il prossimo sprint.
% ====================================================== SPRINT 3

\subsubsection{Sprint 3}
Qui di seguito riportiamo la suddivisione dei ruoli e le ore di lavoro effettive impiegate per lo Sprint\textsubscript{g} 3:

\begin{longtable}{ 
	>{\centering}M{0.20\textwidth} 
	>{\centering}M{0.06\textwidth}
	>{\centering}M{0.10\textwidth}
	>{\centering}M{0.05\textwidth}
	>{\centering}M{0.06\textwidth}
	>{\centering}M{0.10\textwidth}
	>{\centering}M{0.05\textwidth}
	>{\centering}M{0.09\textwidth}
	>{\centering\arraybackslash}M{0.06\textwidth} 
	}
	\rowcolorhead
	\centering \headertitle{Ruolo} &
	\headertitle{Marco} &	
	\headertitle{Giacomo} &
	\headertitle{Elena} &
	\headertitle{Mirko} &
	\headertitle{Tommaso} &
	\headertitle{Rino} &
	\headertitle{Gabriele} &
	\headertitle{Totale}
	\endfirsthead	
	\endhead
	
	Responsabile & - & - & - & 3 (-1) & - & - & - & 2 \tabularnewline
	Amministratore & - & - & - & - & 2 & - & - & 2 \tabularnewline
	Analista & - & - & - & - & - & - & - & 0 \tabularnewline
	Progettista & 3 (-1) & - & - & - & 1 & - & - & 3 \tabularnewline
	Programmatore & - & - & 5 & - & - & 5 & - & 10 \tabularnewline
	Verificatore & - & 4 & - & - & - & - & 4 & 8 \tabularnewline
	\rowcolorhead \textcolor{white}{\textbf{Totale}} & \textcolor{white}{\textbf{2}} &\textcolor{white}{\textbf{4}} & \textcolor{white}{\textbf{5}} & \textcolor{white}{\textbf{2}} & 	\textcolor{white}{\textbf{3}} & \textcolor{white}{\textbf{5}} & \textcolor{white}{\textbf{4}} & \textcolor{white}{\textbf{25}}\\
	\captionline\caption{Distribuzione ruoli-ore nel periodo di Sprint 3}
\end{longtable}
\pagebreak
I costi effettivi del periodo sono i seguenti:


\begin{longtable}{ 
		>{\centering}M{0.25\textwidth} 
		>{\centering}M{0.10\textwidth}
		>{\centering\arraybackslash}M{0.15\textwidth} 
		}
	\rowcolorhead
	\headertitle{Ruolo} &
	\headertitle{Ore} &
	\headertitle{Costo} 
	\endfirsthead	
	\endhead
	
	Responsabile & 3 (-1) & 90 (-30)\euro\tabularnewline
	Amministratore & 2 & 40\euro \tabularnewline
	Analista & 0 & 0\euro \tabularnewline
	Progettista & 4 (-1) & 100 (-25)\euro \tabularnewline
	Programmatore & 10 & 150\euro \tabularnewline
	Verificatore & 8 & 120\euro \tabularnewline
	\rowcolorhead \textcolor{white}{\textbf{Totale Consuntivo}} & \textcolor{white}{\textbf{24}} & \textcolor{white}{\textbf{445\euro}}\\
	\rowcolorhead \textcolor{white}{\textbf{Totale Preventivo}} & \textcolor{white}{\textbf{26}} & \textcolor{white}{\textbf{500\euro}}\\
	\rowcolorhead \textcolor{white}{\textbf{Differenza}} & \textcolor{white}{\textbf{-2}} & \textcolor{white}{\textbf{-55\euro}}\\
	\captionline\caption{Prospetto costi nel periodo di Sprint\textsubscript{g} 3} 
\end{longtable}

\paragraph{Resoconto}
Le ore preventivate sono in linea con quelle del consuntivo portando ad una differenza di soli -55\euro.

\paragraph{Retrospezione}
La scelta di definire per prime le norme di codifica ha portato alla produzione di codice organizzato fin da subito e ha permesso al gruppo di lavorare seguendo dei punti di 
riferimento.
\\\\
Le tasks che tornano a far parte del backlog\textsubscript{g} sono:
\\\\
\textbf{Sono state completate tutte le tasks previste dallo sprint}.
\\\\
\noindent Durante la riunione di gruppo di retrospezione sono emersi i seguenti punti rilevanti:
\begin{itemize}
	\item Durante la codifica i programmatori nei prossimi sprint dovranno segnalare subito eventuali difficoltà in modo da avere suggerimenti tempestivi dagli altri 
	programmatori e continuare la scrittura del codice senza aspettare il daily meeting del giorno successivo;
	\item Il gruppo al momento sta lavorando bene con il modello agile adottato e abbiamo notato un aumento della produttività.
\end{itemize}
\pagebreak

% ====================================================== SPRINT 4

\subsubsection{Sprint 4}
Qui di seguito riportiamo la suddivisione dei ruoli e le ore di lavoro effettive impiegate per lo Sprint\textsubscript{g} 4:

\begin{longtable}{ 
	>{\centering}M{0.20\textwidth} 
	>{\centering}M{0.06\textwidth}
	>{\centering}M{0.10\textwidth}
	>{\centering}M{0.05\textwidth}
	>{\centering}M{0.06\textwidth}
	>{\centering}M{0.10\textwidth}
	>{\centering}M{0.05\textwidth}
	>{\centering}M{0.09\textwidth}
	>{\centering\arraybackslash}M{0.06\textwidth} 
	}
	\rowcolorhead
	\centering \headertitle{Ruolo} &
	\headertitle{Marco} &	
	\headertitle{Giacomo} &
	\headertitle{Elena} &
	\headertitle{Mirko} &
	\headertitle{Tommaso} &
	\headertitle{Rino} &
	\headertitle{Gabriele} &
	\headertitle{Totale}
	\endfirsthead	
	\endhead
	
	Responsabile & - & - & - & 2 & - & - & - & 2 \tabularnewline
	Amministratore & - & - & - & - & - & - & - & 0 \tabularnewline
	Analista & - & - & - & - & - & - & - & 0 \tabularnewline
	Progettista & - & - & - & 5 (-1) & - & - & - & 4 \tabularnewline
	Programmatore & - & 6 (-1) & 7 (-2) & - & 5 (-1) & - & 5 & 19 \tabularnewline
	Verificatore & 6 (-1) & - & - & - & - & 6 (-2) & - & 9 \tabularnewline
	\rowcolorhead \textcolor{white}{\textbf{Totale}} & \textcolor{white}{\textbf{5}} &\textcolor{white}{\textbf{5}} & \textcolor{white}{\textbf{5}} & \textcolor{white}{\textbf{6}} & 	\textcolor{white}{\textbf{4}} & \textcolor{white}{\textbf{4}} & \textcolor{white}{\textbf{5}} & \textcolor{white}{\textbf{34}}\\
	\captionline\caption{Distribuzione ruoli-ore nel periodo di Sprint 4}
\end{longtable}
\pagebreak
I costi effettivi del periodo sono i seguenti:


\begin{longtable}{ 
		>{\centering}M{0.25\textwidth} 
		>{\centering}M{0.10\textwidth}
		>{\centering\arraybackslash}M{0.15\textwidth} 
		}
	\rowcolorhead
	\headertitle{Ruolo} &
	\headertitle{Ore} &
	\headertitle{Costo} 
	\endfirsthead	
	\endhead
	
	Responsabile & 2 & 60\euro\tabularnewline
	Amministratore & 0 & 0\euro \tabularnewline
	Analista & 0 & 0\euro \tabularnewline
	Progettista & 5 (-1) & 125 (-25)\euro \tabularnewline
	Programmatore & 23 (-4) & 345 (-60)\euro \tabularnewline
	Verificatore & 12 (-3) & 180 (-45)\euro \tabularnewline
	\rowcolorhead \textcolor{white}{\textbf{Totale Consuntivo}} & \textcolor{white}{\textbf{34}} & \textcolor{white}{\textbf{580\euro}}\\
	\rowcolorhead \textcolor{white}{\textbf{Totale Preventivo}} & \textcolor{white}{\textbf{42}} & \textcolor{white}{\textbf{710\euro}}\\
	\rowcolorhead \textcolor{white}{\textbf{Differenza}} & \textcolor{white}{\textbf{-8}} & \textcolor{white}{\textbf{-130\euro}}\\
	\captionline\caption{Prospetto costi nel periodo di Sprint\textsubscript{g} 4} 
\end{longtable}

\paragraph{Resoconto}
Le ore preventivate sono in linea con quelle del consuntivo portando ad una differenza di soli -130\euro.

\paragraph{Retrospezione}
La conoscenza acquisita dai membri del gruppo durante lo sviluppo del PoC\textsubscript{g} ci ha permesso di procedere rapidamente nell'impostazione delle norme di testing e nello sviluppo del diagramma delle classi. 
\\\\
Le tasks che tornano a far parte del backlog\textsubscript{g} sono:
\\\\
\textbf{Sono state completate tutte le tasks previste dallo sprint}.
\\\\
\noindent Durante la riunione di gruppo di retrospezione sono emersi i seguenti punti rilevanti:
\begin{itemize}
	\item La codifica di test viene svolta molto più velocemente se effettuata da chi ha sviluppato la funzionalità testata data la sua familiarità con il codice, ma potrebbe essere utile confrontarsi con almeno un altro membro per analizzare i casi limite.
\end{itemize}\pagebreak

% ====================================================== SPRINT 5

\subsubsection{Sprint 5}
Qui di seguito riportiamo la suddivisione dei ruoli e le ore di lavoro effettive impiegate per lo Sprint\textsubscript{g} 5:

\begin{longtable}{ 
	>{\centering}M{0.20\textwidth} 
	>{\centering}M{0.06\textwidth}
	>{\centering}M{0.10\textwidth}
	>{\centering}M{0.05\textwidth}
	>{\centering}M{0.06\textwidth}
	>{\centering}M{0.10\textwidth}
	>{\centering}M{0.05\textwidth}
	>{\centering}M{0.09\textwidth}
	>{\centering\arraybackslash}M{0.06\textwidth} 
	}
	\rowcolorhead
	\centering \headertitle{Ruolo} &
	\headertitle{Marco} &	
	\headertitle{Giacomo} &
	\headertitle{Elena} &
	\headertitle{Mirko} &
	\headertitle{Tommaso} &
	\headertitle{Rino} &
	\headertitle{Gabriele} &
	\headertitle{Totale}
	\endfirsthead	
	\endhead
	
	Responsabile & - & - & - & 3 (+2) & - & - & - & 5 \tabularnewline
	Amministratore & - & - & - & - & - & - & - & 0 \tabularnewline
	Analista & - & - & - & - & - & - & - & 0 \tabularnewline
	Progettista & 10 & - & 8 (+3) & - & - & - & - & 21 \tabularnewline
	Programmatore & - & - & - & - & 11 & 12 (-2) & 10 (+3) & 34 \tabularnewline
	Verificatore & - & 10 & - & 6 & - & - & - & 16 \tabularnewline
	\rowcolorhead \textcolor{white}{\textbf{Totale}} & \textcolor{white}{\textbf{10}} &\textcolor{white}{\textbf{10}} & \textcolor{white}{\textbf{11}} & \textcolor{white}{\textbf{11}} & \textcolor{white}{\textbf{11}} & \textcolor{white}{\textbf{10}} & \textcolor{white}{\textbf{13}} & \textcolor{white}{\textbf{76}}\\
	\captionline\caption{Distribuzione ruoli-ore nel periodo di Sprint 5}
\end{longtable}
\pagebreak
I costi effettivi del periodo sono i seguenti:


\begin{longtable}{ 
		>{\centering}M{0.25\textwidth} 
		>{\centering}M{0.10\textwidth}
		>{\centering\arraybackslash}M{0.15\textwidth} 
		}
	\rowcolorhead
	\headertitle{Ruolo} &
	\headertitle{Ore} &
	\headertitle{Costo} 
	\endfirsthead	
	\endhead
	
	Responsabile & 3 (+2) & 90 (+60)\euro\tabularnewline
	Amministratore & 0 & 0\euro \tabularnewline
	Analista & 0 & 0\euro \tabularnewline
	Progettista & 18 (+3) & 450 (+75)\euro \tabularnewline
	Programmatore & 33 (+1) & 495 (+15)\euro \tabularnewline
	Verificatore & 16 & 240\euro \tabularnewline
	\rowcolorhead \textcolor{white}{\textbf{Totale Consuntivo}} & \textcolor{white}{\textbf{76}} & \textcolor{white}{\textbf{1425\euro}}\\
	\rowcolorhead \textcolor{white}{\textbf{Totale Preventivo}} & \textcolor{white}{\textbf{70}} & \textcolor{white}{\textbf{1275\euro}}\\
	\rowcolorhead \textcolor{white}{\textbf{Differenza}} & \textcolor{white}{\textbf{+6}} & \textcolor{white}{\textbf{+150\euro}}\\
	\captionline\caption{Prospetto costi nel periodo di Sprint\textsubscript{g} 5} 
\end{longtable}

\paragraph{Resoconto}
Le ore preventivate sono in linea con quelle del consuntivo portando ad una differenza di soli +150\euro.

\paragraph{Retrospezione}
La conoscenza acquisita dai membri del gruppo durante lo sviluppo del PoC\textsubscript{g} ci ha permesso di procedere rapidamente nell'impostazione delle norme di testing e nello sviluppo del diagramma delle classi. 
\\\\
Le tasks che tornano a far parte del backlog\textsubscript{g} sono:
\\\\
\textbf{Sono state completate tutte le tasks previste dallo sprint}.
\\\\
\noindent Durante la riunione di gruppo di retrospezione sono emersi i seguenti punti rilevanti:
\begin{itemize}
	\item La codifica di test viene svolta molto più velocemente se effettuata da chi ha sviluppato la funzionalità testata data la sua familiarità con il codice, ma potrebbe essere utile confrontarsi con almeno un altro membro per analizzare i casi limite.
\end{itemize}


% ====================================================== SPRINT 6
\subsubsection{Sprint 6}
Qui di seguito riportiamo la suddivisione dei ruoli e le ore di lavoro effettive impiegate per lo Sprint\textsubscript{g} 6:

\begin{longtable}{ 
	>{\centering}M{0.20\textwidth} 
	>{\centering}M{0.06\textwidth}
	>{\centering}M{0.10\textwidth}
	>{\centering}M{0.05\textwidth}
	>{\centering}M{0.06\textwidth}
	>{\centering}M{0.10\textwidth}
	>{\centering}M{0.05\textwidth}
	>{\centering}M{0.09\textwidth}
	>{\centering\arraybackslash}M{0.06\textwidth} 
	}
	\rowcolorhead
	\centering \headertitle{Ruolo} &
	\headertitle{Marco} &	
	\headertitle{Giacomo} &
	\headertitle{Elena} &
	\headertitle{Mirko} &
	\headertitle{Tommaso} &
	\headertitle{Rino} &
	\headertitle{Gabriele} &
	\headertitle{Totale}
	\endfirsthead	
	\endhead
	
	Responsabile & - & - & - & 3 & - & - & - & 3 \tabularnewline
	Amministratore & - & - & - & - & - & - & - & 0 \tabularnewline
	Analista & - & - & - & - & - & - & - & 0 \tabularnewline
	Progettista & - & - & - & - & - & 4 & - & 4 \tabularnewline
	Programmatore & 6(+3) & 7(+4) & 7 & 4(+3) & - & - & - & 34 \tabularnewline
	Verificatore & - & - & - & - & 6 & - & 5 & 11 \tabularnewline
	\rowcolorhead \textcolor{white}{\textbf{Totale}} & \textcolor{white}{\textbf{9}} &\textcolor{white}{\textbf{11}} & \textcolor{white}{\textbf{7}} & \textcolor{white}{\textbf{10}} & 	\textcolor{white}{\textbf{6}} & \textcolor{white}{\textbf{4}} & \textcolor{white}{\textbf{5}} & \textcolor{white}{\textbf{52}}\\
	\captionline\caption{Distribuzione ruoli-ore nel periodo di Sprint 6}
\end{longtable}
\pagebreak
I costi effettivi del periodo sono i seguenti:

\begin{longtable}{ 
		>{\centering}M{0.25\textwidth} 
		>{\centering}M{0.10\textwidth}
		>{\centering\arraybackslash}M{0.15\textwidth} 
		}
	\rowcolorhead
	\headertitle{Ruolo} &
	\headertitle{Ore} &
	\headertitle{Costo} 
	\endfirsthead	
	\endhead
	
	Responsabile & 3 & 90\euro\tabularnewline
	Amministratore & 0 & 0\euro \tabularnewline
	Analista & 0 & 0\euro \tabularnewline
	Progettista & 4 & 100 \euro \tabularnewline
	Programmatore & 24 (+10) & 360 (+150)\euro \tabularnewline
	Verificatore & 11 & 165 \euro \tabularnewline
	\rowcolorhead \textcolor{white}{\textbf{Totale Consuntivo}} & \textcolor{white}{\textbf{52}} & \textcolor{white}{\textbf{865\euro}}\\
	\rowcolorhead \textcolor{white}{\textbf{Totale Preventivo}} & \textcolor{white}{\textbf{42}} & \textcolor{white}{\textbf{715\euro}}\\
	\rowcolorhead \textcolor{white}{\textbf{Differenza}} & \textcolor{white}{\textbf{+10}} & \textcolor{white}{\textbf{+150\euro}}\\
	\captionline\caption{Prospetto costi nel periodo di Sprint\textsubscript{g} 6} 
\end{longtable}

\paragraph{Resoconto}
Le ore preventivate sono in linea con quelle del consuntivo portando ad una differenza di soli +150\euro.

\paragraph{Retrospezione}
La conoscenza acquisita dai membri del gruppo durante lo sviluppo del PoC\textsubscript{g} ci ha permesso di procedere rapidamente nell'impostazione delle norme di testing e nello sviluppo del diagramma delle classi. 
\\\\
Le tasks che tornano a far parte del backlog\textsubscript{g} sono:
\\\\
\textbf{Sono state completate tutte le tasks previste dallo sprint}.
\\\\
\noindent Durante la riunione di gruppo di retrospezione sono emersi i seguenti punti rilevanti:
\begin{itemize}
	\item La codifica di test viene svolta molto più velocemente se effettuata da chi ha sviluppato la funzionalità testata data la sua familiarità con il codice, ma potrebbe essere utile confrontarsi con almeno un altro membro per analizzare i casi limite.
\end{itemize}
\pagebreak

% ====================================================== CONSUNTIVO FINALE
\subsubsection{Consuntivo Finale}
Qui di seguito riportiamo la suddivisione dei ruoli e le ore di lavoro effettive impiegate nel corso di tutta la durata del progetto:

\begin{longtable}{ 
	>{\centering}M{0.20\textwidth} 
	>{\centering}M{0.06\textwidth}
	>{\centering}M{0.10\textwidth}
	>{\centering}M{0.05\textwidth}
	>{\centering}M{0.06\textwidth}
	>{\centering}M{0.10\textwidth}
	>{\centering}M{0.05\textwidth}
	>{\centering}M{0.09\textwidth}
	>{\centering\arraybackslash}M{0.06\textwidth} 
	}
	\rowcolorhead
	\centering \headertitle{Ruolo} &
	\headertitle{Marco} &	
	\headertitle{Giacomo} &
	\headertitle{Elena} &
	\headertitle{Mirko} &
	\headertitle{Tommaso} &
	\headertitle{Rino} &
	\headertitle{Gabriele} &
	\headertitle{Totale}
	\endfirsthead	
	\endhead
	
	Responsabile & 9 & - & 8 & 30 & 10 & 2 & 4 & 63 \tabularnewline
	Amministratore & - & - & - & - & 10 & 9 & - & 19 \tabularnewline
	Analista & 15 & 6 & 13 & 12 & 19 & 13 & 12 & 90 \tabularnewline
	Progettista & 19 & 4 & 11 & 6 & 8 & 16 & 9 & 73 \tabularnewline
	Programmatore & 19 & 41 & 34 & 19 & 15 & 45 & 36 & 209 \tabularnewline
	Verificatore & 28 & 39 & 23 & 24 & 26 & 4 & 32 & 176 \tabularnewline
	\rowcolorhead \textcolor{white}{\textbf{Totale}} & \textcolor{white}{\textbf{90}} &\textcolor{white}{\textbf{90}} & \textcolor{white}{\textbf{89}} & \textcolor{white}{\textbf{91}} & 	\textcolor{white}{\textbf{88}} & \textcolor{white}{\textbf{89}} & \textcolor{white}{\textbf{93}} & \textcolor{white}{\textbf{630}}\\
	\captionline\caption{Distribuzione ruoli-ore complessivi}
\end{longtable}
I costi effettivi del periodo sono i seguenti:

\begin{longtable}{ 
		>{\centering}M{0.25\textwidth} 
		>{\centering}M{0.10\textwidth}
		>{\centering\arraybackslash}M{0.15\textwidth} 
		}
	\rowcolorhead
	\headertitle{Ruolo} &
	\headertitle{Ore} &
	\headertitle{Costo} 
	\endfirsthead	
	\endhead
	
	Responsabile & 63 & 1890\euro\tabularnewline
	Amministratore & 19 & 380\euro \tabularnewline
	Analista & 90 & 2250\euro \tabularnewline
	Progettista & 73 & 1825 \euro \tabularnewline
	Programmatore & 209 & 3135\euro \tabularnewline
	Verificatore & 170 & 2550 \euro \tabularnewline
	\rowcolorhead \textcolor{white}{\textbf{Totale Consuntivo}} & \textcolor{white}{\textbf{630}} & \textcolor{white}{\textbf{12030\euro}}\\
	\rowcolorhead \textcolor{white}{\textbf{Totale Preventivo}} & \textcolor{white}{\textbf{665}} & \textcolor{white}{\textbf{13975\euro}}\\
	\rowcolorhead \textcolor{white}{\textbf{Differenza}} & \textcolor{white}{\textbf{-35}} & \textcolor{white}{\textbf{1945\euro}}\\
	\captionline\caption{Prospetto costi complessivo} 
\end{longtable}

\paragraph{Resoconto}
 La differenza totale tra il budget previsto e il costo effettivo del progetto è di 1945\euro.

\paragraph{Retrospezione}
Le ore preventivate ad inizio progetto sono state rispettate solo parzialmente, non tanto dal punto di vista complessivo (35 ore in meno), ma dal punto di vista del ruolo associato alle ore, infatti c'è una grande divergenza nel ruolo di amministratore. Questo è stato causato parzialmente dal poca conoscenza del ruolo, ma anche dal fatto che spesso i suoi compiti sono stati delegati ad altri ruoli.
Infine i componenti del gruppo non hanno avuto una distribuzione proporzionata delle ore tra i vari ruoli come previsto nel preventivo, invece ogni componente ha avuto la tendenza di specializzarsi in un paio di ruoli, questo è stato causato dalla necessità, in più momenti del progetto, che le persone con un determinato incarico fossero il più compenti possibile in materia. Sebbene abbiamo cercato di evetare queste situazioni il più possibile, essendo coscenti che il progetto dovrebbe dare l'occasione a tutti di coprire ogni ruolo, alla fine si è comunque creato uno sbilanciamento.
