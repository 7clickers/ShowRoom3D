\section{Modello di sviluppo iniziale}
\subsection{Modello incrementale}
Il Gruppo ha deciso di utilizzare il modello incrementale. \\
 Questo modello prevede un determinato numero di rilasci, ognuno di essi fornisce un incremento di funzionalità.
Ogni incremento punterà ad implementare una serie di requisiti,  i più importanti dal punto di vista strategico verranno implementati prima così da poter essere testati più volte.  Per fare ciò è necessaria una minuziosa classificazione dei requisiti così da valutarne la priorità e di conseguenza inserirli nel incremento adatto.\\
L'utilizzo di questo modello di sviluppo porta i seguenti vantaggi:
\begin{itemize}
\item Ogni incremento riduce il rischio di fallimento;
\item Implementando i requisiti più importanti nei primi incrementi si può avere un software funzionante già nelle prime fasi di sviluppo, in modo tale da avere sempre un prodotto\textsubscript{g} da condividere allo stakeholder e poterne mostrare le funzionalità;
\item  Produco valore ad ogni incremento.
\end{itemize}

\subsection{Incrementi individuati}
Di seguito è riportata una tabella con gli incrementi individuati, associati ai rispettivi requisiti 
e casi d'uso, indicati nella tabella con il loro codice identificativo. Per maggiori informazioni consultare
l'\textit{Analisi dei Requisiti}.

\begin{longtable}{ 
	>{\centering}M{0.20\textwidth} 
	>{\centering}M{0.30\textwidth}
	>{\centering}M{0.20\textwidth}
	>{\centering}M{0.20\textwidth}
	}
	\rowcolorhead
	\centering 
	\headertitle{Incremento} &	
	\headertitle{Obiettivo} &
	\headertitle{Requisiti} &
	\headertitle{Casi d'uso} 
	\endfirsthead	
	\endhead
	
	Incremento 0 & 
    Movimenti direzionali e rotazioni camera & 
    RF5 RF5.1, RF5.2, RF5.3, RF6 RF6.1, RF6.2, RF6.3 & UC5. UC5.1, UC5.2, UC5.3, UC6. UC6.1, UC6.2, UC6.3 \tabularnewline
    Incremento 1 & 
    Visualizzazione dei dettagli di un oggetto nella stanza  & 
    RF10 & UC10, UC10.1, UC10.2, UC10.3, UC10.4, UC10.5 \tabularnewline
    Incremento 2 & 
    Visualizzazione contenuto del carrello & 
    RF2, RF2.1, RF2.1.1, RF2.1.1.1, RF2.1.1.2, RF2.1.1.3, RF2.2, RF19 & 
    UC2, UC2.1, UC2.1.1, UC2.1.1.1, UC2.1.1.2, UC2.1.1.3, UC2.2\tabularnewline
    Incremento 3 & 
    Aggiunta di un oggetto al carrello & 
    RF1 & UC1 \tabularnewline
    Incremento 4 & 
    Rimozione di uno o tutti gli oggetti dal carrello & 
    RF3, RF4 & UC3, UC4 \tabularnewline
    Incremento 5 & 
    Personalizzazione della palette colori di un oggetto & 
    RF7, RF8 & UC7, UC8 \tabularnewline
    Incremento 6 & 
    Visualizzazione della lista degli oggetti in ogni stanza & 
    RF9, RF9.1, RF9.1.1 & UC9, UC9.1, UC9.1.1 \tabularnewline
    Incremento 7 & 
    Riposizionamento vicino ad un oggetto nella stanza & 
    RF11, RF14& UC11, UC14\tabularnewline
    Incremento 8 & 
    Visualizzazione della lista delle stanze & 
    RF15, RF15.1, RF15.1.1, RF15.1.2 & UC15, UC15.1, UC15.1.1, UC15.1.2 \tabularnewline
    Incremento 9 & 
    Riposizionamento in una stanza & 
    RF12, RF13& UC12, UC13\tabularnewline
    Incremento 10 & 
    Spostamento di un oggetto nello spazio & 
    RF16, RF17& UC16, UC17\tabularnewline
    Incremento 11 & 
    Illuminazione di un oggetto & 
    RF18, RF20& UC18\tabularnewline
\end{longtable}


