\section{Introduzione}
\subsection{Scopo del documento}
Lo scopo del seguente documento `e quello di illustrare le funzionalit`a fornite dall’applicazione e le
istruzioni per l’utilizzo della stessa. L’utente sar`a quindi a conoscenza dei requisiti minimi necessari
per il corretto funzionamento di ShopChain.
\subsection{Glossario}
I termini utilizzati in questo documento potrebbero generare dubbi riguardo al loro significato, richiedendo pertanto una definizione al fine di evitare ambiguit`a. Tali termini vengono contrassegnati da
una G maiuscola finale a pedice della parola. Alla fine del documento stesso `e possibile reperire tale
Glossario con i termini di interesse.
\subsection{Cos'è Showroom 3D}
Showroom 3D è un sito web che offre un'esperienza di shopping unica per gli appassionati di acquari. Grazie alla tecnologia 3D, gli utenti possono visualizzare e personalizzare in modo interattivo gli articoli per acquari come piante, rocce e decorazioni. Il sito offre una vasta gamma di prodotti di alta qualità e permette agli utenti di scegliere tra diverse opzioni di personalizzazione per soddisfare le loro esigenze. Inoltre, gli utenti possono acquistare direttamente gli articoli selezionati sul sito e riceverli comodamente a casa propria. Showroom 3D rappresenta la soluzione ideale per chi desidera rendere il proprio acquario unico e personalizzato.
\subsection{Autenticazione}
Non è presente nessuna funzione di autenticazione, quindi ogni parte e funzionalità del sito sono accessibili da chiuque senza dover effettuare un login. 
Questo perchè il progetto didattico non aveva come interesse l'unica sezione in cui sarebbe stato necessario utilizzare una pagina di autenticazione, ovvero la parte di checkout del carrello e acquisto.
\subsection{Browser supportati}
Di seguito viene fornito un breve elenco delle versioni minime di tutti i browser sui quali
il funzionamento del nostro prodotto è garantito:
MODIFICARE
\begin{itemize}
	\item Mozilla Firefox, versione 97;
	\item Google Chrome, versione 98;
	\item Micosoft Edge, versione 98;
	\item Brave, versione 1.36.
\end{itemize}
Per rendere effettivo il funzionamento su tali browser, deve essere abilitato JavaScript.
\pagebreak



\section{Panoramica del sito}
screen completo
L'utente che accede al sito si ritrova immediatamente, o dopo qualche secondo di loadinscreen, immerso nell'ambiente 3D, dove potrà compiere tutte le azioni e funzionalità offerte dal sito. 
IMMAGINE DI COME SI PRESENTA IL SITO APPENA APERTO
è uno showroom di oggetti per acquari
(forse un po' rindondante con cos'è showroom 3D)
\subsection{Funzionalità}
Le funzionalità si cui può disporre l'utente sono le seguenti:
\begin{itemize}
\item navigazione in ambiente 3d immersivo;
\item visione di articoli aquistabili;
\item modifica delle caratteristiche di quest'ultimi;
\item gestione degli acqusti tramite il carrello;
\item riposizionamento.
\end{itemize}
\pagebreak

\section{Interfaccia utente}
Oltre all'ambiente 3D l'utente ha a disposizioni diversi strumenti per interagire col sistema e monitorare le sue attività sul sito:
\subsection{Puntatore centrale}
Al centro della schermata è sempre presente un pallino rotondo che indica la direzione in cui sta puntando la "telecamera" dell'utente. Qunado questo pallino incrocia un elemento acquistabile nell'ambientazione comparirà a schermo il nome dell'elemento puntato e il pallino cambierà colore per notificare più chiaramente che si tratta di un oggetto con cui è possibile interagire.
IMMAGINE ELEMENTO PUNTATO CON NOME DELL'OGGETTO SOTTO
\subsection{Item sidebar}
Clickando il tasto sinistro del mouse quando la telecamera è puntata su un'oggetto è possibile aprirne la scheda che si presenta in questo modo:
IMMAGINE SCHEDA PRODOTTO
Come si nota dall'immagine la scheda riporta molte informazioni rigurdo all'oggetto, ovvero:
\begin{itemize}
	\item Nome;
	\item Descrizione;
	\item Dimensioni;
	\item Peso;
	\item Colore;
	\item Quantità: selezionando una delle due freccie di questa riga è possibile aumentare o diminuire la quantità dell'oggetto da inserire nel carrello;
	\item Costo;
	\item Pulsante "Aggiungi al carrello": premendo questo pulsante verranno inseriti nel carrello un numero di oggetti che dipende dalla quantità selezionata e con le caratteristiche scelte.
\end{itemize}

Per chiudere la scheda dell'oggetto e tornare alla normale navigazione della stanza basta premere col mouse al di fuori di essa.
IMMAGINE DELLA SIDEBAR CON EVIDENZIATE LE CARATTERISTICHE
\subsubsection{Personalizzazione}
All'interno della sidebar è possibile trovare alcuni campi modificabili, identficati da un menù a tendina con cui l'utente può effettuare la sua personalizzazione.
Ogni oggetto è modificabile in almeno una caratteristica tra:
\begin{itemize}
	\item Colore;
	\item Taglia;
	\item Texture.
\end{itemize}
Una volta effettuata la modifica questa sarà applica anche nell'ambinte 3D e sarà quindi possibile osservare l'oggetto nelle sue varie configurazioni prima di effettuare l'aquisto una queste.
IMMAGINE: DUE IMMAGINI DELLO STESSO CORALLO, NELLA PRIMA C'è ANCHE LA SIDEBAR E VIENE EVIDENZIATO IL CAMPO DI MOFICA, NELLA SECONDA IMMAGINE SI VEDE L'ELEMENTO MODIIFCATO
\subsection{Carrello}
IMMAGINE DEL CARRELLO
Una volta acquistato un oggetto, quet'ultimo verrà inserito nel carrello (in basso a destra nello schermo), riportandone il nome, le caratteristiche personalizzate e il prezzo unitario.
Il carrrello si presenta quindi come un elenco inizialmente vuoto, che si aggiorna in tempo reale e riporta gli oggetti scelti dal cliente.
In fondo al carrellosi trovano:
\begin{itemize}
	\item Pulsante Checkout
	\item Pulsante Svuota carrello
	\item Prezzo totatle
\end{itemize}
Il carrello è fittizio, cioè non permette di effettuare veramente degli acquisti, infatti non è prevista nessuna funzione di pagamento ed inoltre non è persistente, cioè non salva nessun dato e tra una sessione e l'altra viene completatmente svuotato.

\pagebreak
\section{Navigazione}
All'interno dello showroom sono previsti varie possibilità e limiti per muoversi:
\begin{itemize}
	\item Movimento orizzaontale coi tasti ↑ (avanti) ← (sinistra) → (destra) ↓ (indietro) oppure rispettivamente coi tasti w a s d;
	\item Salto (breve movimento in alto per poi ricadere a terra) premendo la barra spaziatrice;
	\item Teletrasporto (stanze o oggetti, dipende cosa implementeremo);
	\item Collisioni con gli oggetti. Se l'utente prova a muoveri dentro ad un oggetto il movimento viene interrotto;
	\item Collisioni con il limite della mappa. La zona in cui si muove l'utente è delimitata con varii ostacoli che fanno parte dell'ambientazione, se l'utente prova a muoversi al loro interno il movimento viene interrotto
\end{itemize}
IMMAGINI WASD E FRECCETTE
\pagebreak



Da capire se ci saranno:
TORCIA
STANZE


