\section{Motivo della Riunione}
\begin{itemize}
    \item Richiesta di chiarimenti da parte del gruppo SevenClickers sul progetto CAPTCHA: Umano o Sovrumano?
    \begin{itemize}
        \item Tecnologie
        \item Approcci al problema
        \item Criticità della sicurezza
        \item Criteri di valutazione
        \item Requisiti opzionali
    \end{itemize}
\end{itemize}
\section{Resoconto}
\subsection{Richiesta di chiarimenti da parte del gruppo SevenClickers sul progetto CAPTCHA: Umano o Sovrumano?}
\subsubsection{Tecnologie}
\begin{itemize}
    \item html, css, javascript, php
    \item per il backend la scelta è libera e lasciata al gruppo di lavoro
\end{itemize}
\subsubsection{Approcci al problema}
\begin{itemize}
    \item Usare servizi a pagamento : non è permesso dal proponente\textsubscript{g}
    \item Fare un servizio gratuito usando librerie già esistenti
    \item Fare un servizio gratuito originale, senza utilizzare soluzioni pre-esistenti
\end{itemize}
\subsubsection{Criticità della sicurezza}
\begin{itemize}
    \item Attacchi di brute-force
    \item Comunicazione tra client-server
\end{itemize}
\subsubsection{Criteri di valutazione}
\begin{itemize}
    \item Se l'applicazione è composta da librerie pre-esistenti, l'interesse sarà più incentrato su come quest'ultime comunicano tra loro e su eventuali falle di sicurezza nella comunicazione client-server
    \item Se l'applicazione si basa interamente su contenuto originale, l'interesse sarà più incentrato sull'idea alla base dell'applicazione, alla sua implementazione e all'analisi di robustezza verso attacchi brute-force e bot
\end{itemize}
\subsubsection{Requisiti opzionali}
\begin{itemize}
    \item L'opzione del forum serve unicamente per dimostrare la tecnologia CAPTCHA in un altro contesto d'uso con caratteristiche diverse dalla semplice pagina di login, di conseguenza l'azienda non è interessata ad una implementazione particolarmente complessa 
\end{itemize}